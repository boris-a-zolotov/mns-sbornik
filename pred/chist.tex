\ms\abz Дорогие юные математики, коллеги, друзья! Вы держите в руках не просто сборник олимпиадных задач, а скорее книгу, в которой рассказывается о системе математического образования, созданной Фондом поддержки молодых ученых «Время науки».

\ms\abz Проекты Фонда взаимосвязаны и не могут рассматриваться отдельно друг от друга. Одной из главных задач Фонда является привлечение талантливой молодежи к занятиям наукой, в частности, математикой, и включает в себя по этому направлению  структуру дополнительного образования, систему научных семинаров, Турнир юных математиков, олимпиаду «Математика НОН-СТОП», Летнюю профильную математическую школу, Балтийский научно-инженерный конкурс (секция теоретичес-\linebreak кой, прикладной и элементарной математики). Опыт создания этой системы оказался настолько успешным, что многие из мероприятий Фонда распространились в российские регионы. Так, например, Балтийский научно-инженерный конкурс имеет на сегодняшний день 17 региональных представительств.

\ms\abz Однако по порядку. Основой содержательной математической деятельности является семинар. В 1992 году группа  молодых (тогда) петербургских математиков, среди которых были к.\,ф.-м.\,н. С.М.\,Шиморин, к.\,ф.-м.\,н. Д.Г.\,Бенуа, И.А.\,Чистяков и вскоре присоединившиеся к ним\linebreak к.\,ф.-м.\,н. Т.Н. Шилкин, д.\,ф.-м.\,н. С.И.\,Кублановский, А.О.\,Виро, к.\,ф.-м.\,н. А.А.\,Флоринский, Е.А.\,Абакумов стали вести научные семинары для старшеклассников и руководить научными проектами. Так возникла Лаборатория непрерывного математического образовании (ЛНМО) — в те годы молодежный научный коллектив, ставящий перед собой задачу привлечения к занятиям наукой тех школьников, которые обладали незаурядными математическими способностями, но тем не менее по некоторым причинам в большей своей части не имели значительных олимпиадных достижений.

\ms\abz Время, в которое начинала свою деятельность ЛНМО, было непростым, даже скорее трудным для большинства молодых математиков.\linebreak Многие из них уехали из страны, другие сменили сферу деятельности. Количество способных к математике студентов стало снижаться, несмотря на все усилия декана Математико-механического факультета СПбГУ профессора Г.А.\,Леонова — человека, благодаря которому факультет пережил самые тяжелые годы. Математическое сообщество старело, и привлечение к математике школьников стало одной из основных задач.

\ms\abz В 1994 году ЛНМО стала сотрудничать с Математико-механичес-\linebreak ким факультетом СПбГУ. Это сразу дало свои плоды. Так в 1992 году в ЛНМО работало всего 3 математических семинара. С.М.\,Шиморин и И.А.\linebreak Чистяков руководили семинарами по математическому анализу, Д.Г.\,Бе-\linebreak нуа — по алгебре. В 1995 году в Лаборатории было открыто уже 10 научных семинаров. Среди руководителей семинаров этого периода — про-\linebreak фессор М.М.\,Лесохин, профессор Н.А.\,Широков, профессор В.М.\,Нежинс-\linebreak кий, к.\,ф.-м.\,н. В.Л.\,Кобельский, к.\,ф.-м.\,н. В.Ю.\,Добрынин.

\ms\abz Впоследствии ЛНМО стала частным образовательным учреждением, где работа в Системе научных семинаров и спецкурсов является вершиной в процессе получения школьником среднего образования. 

\ms\abz Семинары 90-х годов воспитали первую плеяду учеников, которые впоследствии стали научными руководителями школьников и студентов, закончили Математико-механический факультет СПбГУ, аспирантуру и защитили кандидатские диссертации. Выпускниками ЛНМО защищены 42 кандидатских диссертации, из каждого выпуска ЛНМО до половины выпускников становятся аспирантами, до 7-8 — кандидатами наук.

\ms\abz Сейчас благодаря гранту Фонда Президентских грантов в ЛНМО работает 27 математических семинаров, а всего по разным специальностям — 61. Среди руководителей — д.\,ф.-м.\,н. С.И.\,Кублановский, к.\,ф.-м.\,н. А.В.\,Смоленский, к.\,ф.-м.\,н. Р.А.\,Гученко, к.\,ф.-м.\,н. Ю.А.\,Ильин, к.\,ф.-м.\,н. С.О.\,Иванов, аспиранты А.В.\,Семенов, В.А.\,Соснило, А.А.\,Зайковский   и многие другие. 

\ms\abz В то же время ощущался дефицит научного общения участников научных семинаров с другими школьниками, тоже делающими свои первые шаги в науке. Тем самым получили развитие научные конференции школьников, в числе которых отметим Сахаровские чтения, проводимые лицеем ФТШ в Петербурге (Я.Д.\,Бирман, к.\,х.\,н. Н.М.\,Химин, Д.В.\,Фреде-\linebreak рикс, к.\,ф.-м.\,н. М.Г.\,Иванов, Е.А.\,Нинбург), и конференцию, посвященную памяти академика С.Н.\,Бернштейна. Большую поддержку в это время оказали профессор М.П.\,Юшков и профессор В.С.\,Виденский. Будучи совсем небольшой, эта конференция привлекала к себе тех юных исследователей, которые уже в возрасте 9–11 класса могли потратить на научную работу значительное время. Высокий уровень профессионального научного жюри стал отличительной карточкой этой конференции.

\ms\abz В Москве в области математики лидирующее положение занимали конференция при МЭИ (Московском энергетическом институте) благодаря усилиям А.А.\,Егорова, к.\,ф.-м.\,н. А.П.\,Савина, Ж.М.\,Раббота, к.\,ф.-м.\,н. В.Н.\,Дубровского и к.\,п.\,н. Л.Б.\,Огурэ\scolon конференция Династия–Аван-\linebreak гард, математическое направление которой возглавляет к.\,ф.-м.\,н. Д.В.\linebreak Андреев\scolon конференция «Юниор» (профессора А.Д.\,Модяев, Н.М.\,Леонова, А.В.\,Михалев и Н.А.\,Кудряшев).

\ms\abz Одновременно возникли международные конференции, из числа которых отметим конференцию молодых ученых (IСYS),  первым победителем которой стал Владимир Камоцкий, занимавшийся, в семинаре по гомологической алгебре (руководитель Д.Г.\,Бенуа, ныне известный математик). Вообще, выпускнику научных семинаров Дмитрию Парилову принадлежит абсолютный рекорд этого научного форума: он в 9, 10 и 11 классах был награжден золотой медалью, став абсолютным победителем. Дмитрий Владимирович Парилов защитил кандидатскую диссертацию и успешно работает в России. 

\ms\abz В 1998 году Россия впервые приняла участие в международном конкурсе научных и инженерных достижений учащихся — ISEF, собирающем более полутора тысяч школьников из более чем 60 стран мира. 30 учеников ЛНМО награждены премиями научного жюри, еще 10 — премиями Карла Менгера, присуждаемыми Американским математичес-\linebreak ким обществом.

\ms\abz Пять научных работ отмечены высшими премиями научного жюри, именами петербургских школьников — Сергея Иванова, Евгения Лохару, Евгения Амосова, Артема Викторова и Гаджи Османова — названы Малые планеты Солнечной системы. С.О.\,Иванов и Е.Э.\,Лохару защитили кандидатские диссертации и активно работают в области математики. Сергей Олегович Иванов в 2014 году назван лучшим молодым математиком Санкт-Петербурга, 14 выпускников Лаборатории награждены премиями имени В.А.\,Рохлина.

\ms\abz Камерная научная конференция, посвященная памяти академика С.Н.\,Бернштейна, не могла уже справиться с нарастающим числом работ и была преобразована в конференцию им. академика П.Л.\,Чебышева, родоначальника Петербургской математической школы, а в 2004 году — в Балтийский научно-инженерный конкурс. На первом Балтийском науч-\linebreak но-инженерном конкурсе было представлено 60 проектов\scolon в 2019 году на юбилейном XV конкурсе в отборочных этапах приняли участвие 2059 юных исследователей, а в финале представлены более 500 проектов из более чем 60 регионов России и Белоруссии.

\ms\abz Председателем научного жюри Конкурса является профессор Н.А.\linebreak Широков, секцию математики возглавляет профессор Н.М.\,Нежинский, а подсекции — к.\,ф.-м.\,н. А.В.\,Смоленский и к.\,п.\,н. В.В.\,Крылов. Второй год работает жюри ПОМИ РАН, которое возглавляет профессор А.И.\,Назаров.

\ms\abz Развитие научной и проектной деятельности высветило ряд проблем, без решения которых далее невозможно развивать научные семинары. Для выполнения серьезных математических исследований школьник должен знать важнейшие разделы математики. Так возникла идея проведения летних профильных математических школ, в которых аспиранты и молодые кандидаты наук погружали старшеклассников в разделы математики, которые были совершенно необходимы для решения математических задач. Часто на семинарах летней школы ставились и задачи для исследования, получали значительные результаты в их решении. 

\ms\abz Пытливый подростковый ум нуждается в постоянной подпитке, решении модельных задач, которые построены по следующему принципу. Каждая задача начинается введением в теорию, в котором ученику разъясняются основные определения. При этом первые пункты задачи более или менее известны. По мере продвижения по задаче пункты усложняются, решения последних пунктов не известны автору задачи и часто являются темой самостоятельного научного исследования.

\ms\abz Форма проведения турнира была заимствована из Белорусских турниров юных математиков (к.\,ф.-м.\,н. Б.В.\,Задворный), хотя Петербургс-\linebreak кий турнир отличается более жесткими условиями правил проведения и трудностью задач. В последнее время проводятся петербургские турниры юных математиков для 5–7 классов (средняя и младшая лига), участники которых, по нашему убеждению, способствуют привлечению к занятиям наукой талантливых детей, которые впоследствии станут учениками научных семинаров ЛНМО. Руководитель проекта — И.С.\,Алексеев. Жюри турниров возглавляют к.\,ф.-м.\,н. Ю.А.\,Ильин и профессор В.М.\,Не-\linebreak жинский.

\ms\abz Усилиями европейских математиков по аналогичным правилам\linebreak организован международный турнир, участниками которого становятся победители национальных турниров. Президентом конкурса является профессор Давид Змейков.  

\ms\abz Олимпиада «Математика НОН-СТОП» выполняет ту же функцию, что и младшая и средняя лиги Турнира юных математиков, — привлекать школьников 4–8 классов к решению исследовательских задач. Нам казалось неразумным сразу приступить к решению этой трудной задачи, необходимо было воспитать молодых математиков — составителей  исследовательских задач и сформировать ряд школ, понимающих важность этого проекта. С этими задачами олимпиада прекрасно справилась.

\ms\abz В первые годы олимпиады отбор задач был традиционен, они подбирались из числа задач петербургских, московских, белорусских олимпиад, однако усложнялось правила ее выполнения. Каждая задача олимпиады состояла из трех пунктов, решение первого из которых оценивалась в 3 балла, решение второго — в 6 баллов, а третьего в 9 баллов. В зачет входил лучший пункт по каждой задаче.

\ms\abz Тем самым школьнику необходимо не только решать задачи, но и продумывать стратегию работы на олимпиаде. Многие школьники, которые брались за решение только сложных пунктов, зачастую делали в них ошибки и получали невысокий балл на олимпиаде. Такая же участь ждала тех, кто решал простые пункты, их суммарный балл также был невысок.

\ms\abz Большое значение мы придавали тому факту, что школьники, которые не справились с решениями задач во время олимпиады, дома и в школе могли продолжить их решение. А наиболее глубокие школьники получали приглашение заниматься в научных семинарах и учиться в профильных математических классах, организуемых на основе государ-\linebreak ственно-частного партнерства ЛНМО и ГБОУ СОШ №564.

\ms\abz Тем самым была решена задача по привлечению школьников к занятиям математикой, а также сформировался коллектив студентов–ма-\linebreak тематиков, способных предлагать исследовательские задачи. И с 2016 года олимпиада «Математика НОН-СТОП» приобрела задуманную форму. Каждая задача {\itshape базовых вариантов}, как и прежде, состоит из трех пунктов — последний из которых наиболее сложен, размышления по этому пункту могут привести школьника к решению научной проблемы — а для школьников 7–8 классов появился {\itshape профильный вариант}, для реализации возможности поиска одаренных детей, которые не занимаются в математических кружках, даже если, возможно, учатся в математических школах.

\ms\abz Это дало результат. Многие дети из общеобразовательных школ получили возможность занятий в профильных научных семинарах. Олимпиада поддержана кафедрой математики и информатики АППО (к.\,п.\,н. Е.Ю.\,Лукичева) и вошла в перечень региональных конкурсов интеллектуальной направленности Правительства СПб. В олимпиаде участвуют более полутора тысяч школьников Санкт-Петербурга и Ленинградской области.

\ms\abz Руководитель проекта — Б.А.\,Золотов, победитель олимпиады «Математика НОН-СТОП» 2011 года. Председатель жюри — Д.Г.\,Штукенберг.   

\ms\abz Книга адресована не только школьникам, интересующимся математикой, но и учителям и организаторам математического образования. Руководители научно-исследовательских работ школьников смогут почерпнуть из сборника темы — задачи для исследования. 