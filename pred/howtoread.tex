\noindent\abz Олимпиада <<Математика НОН-СТОП>> проводится \hypertarget{lcme}{ЧОУ ОиДО <<Лаборатория непрерывного математического образования>>} с 2010 года. С 2016 года, с момента основания Фонда поддержки научного и научно-технического творчества молодых ученых <<Время науки>>, право на проведение олимпиады было передано Фонду. В 2018 году поддержку олимпиаде оказал Фонд президентских грантов. 

\aabz За время проведения олимпиады значительно выросло как число ее участников, так и интерес, проявляемый к ней в том числе со стороны образовательных организаций Санкт-Петербурга.

\aabz В 2010--2015 годах составителем условий задач был И.~А.~Чистяков, директор \hyperlink{lcme}{ЛНМО}. С 2016 года условия задач для олимпиады составляют прежде всего Б.~А.~Золотов, выпускник и сотрудник \hyperlink{lcme}{ЧОУ ОиДО <<ЛНМО>>}, и Д.~Г.~Штукенберг, преподаватель ЛНМО и Университета ИТМО, сотрудник Фонда <<Время науки>>.

\aabz Изначально олимпиада включала в себя варианты для 5--8 классов (в отдельные годы присутствовал также 9 класс). В 2016 году в олимпиаде появился профильный вариант для 7--8 классов математических школ: задачи профильного варианта представляют из себя целую исследовательскую проблему, раскрывающуюся перед школьником пункт за пунктом.  Последним нововведением олимпиады стал вариант для четвертого класса, присутствующий с 2017 года.

\aabz С самого начала особенностью олимпиады было разделение каждой базовой задачи на пункты A, B и C, различающиеся сложностью и количеством начисляемых баллов. Исследовательские задачи бьются на большее количество пунктов, которые рекомендуется решать один за другим.

\aabz Первая часть этой книги~--- условия олимпиад 2016--18 годов. Условия приводятся в том виде, в котором они были даны на олимпиаде. 
Повторы задач из условий принципиально не удалялись, чтобы варианты можно было давать детям на занятиях в математических кружках. Задачи прошлых лет позволяют примерно ориентироваться на то, какими будут задания на олимпиаде <<Математика НОН-СТОП>> в ближайшем будущем. Авторы задач — Б.~А.~Золотов, Д.~Г.~Штукенберг, И.~С.~Алексеев.

\aabz Во второй части этой книги представлены решения задач, предложенные самими авторами (рассмотрены в том числе задачи профильных вариантов). Если при разборе задачи возникают трудности или стало интересно ознакомиться с необычными методами решения, стоит смотреть как раз вторую часть книги. Автор разбора 2017–18 — Б.~А.~Золотов, 2016 — Д.~Г.~Штукенберг.

\aabz В третьей части книги представлены избранные, наиболее оригинальные задачи 2011–15 годов. Эти задачи публикуются сразу с решениями. Авторы разборов 2012–15 — Б.~А.~Золотов, А.~В.~Семенов, 2011 — Л.~А.~Бакунец, И.~Г.~Прокофьева, Д.~Г.~Штукенберг.

\aabz Четвертая часть этой книги --- условия задач Санкт-Петербургского турнира юных математиков (СПбТЮМ),  основанного в 2015 году Ильей Александровичем Чистяковым. СПбТЮМ также является состязанием\linebreak Фонда <<Время науки>>. Этот турнир отличается интересными, сложными заданиями --- некоторые из них впоследствии перерастают в научные работы, представляемые на различных конференциях школьников. В этой книге мы собрали все задачи СПбТЮМ с самого его основания.

\aabz Авторы задач СПбТЮМ — К.~М.~Чепуркин, И.~С.~Алексеев.

\thispagestyle{empty}

\aabz Наконец, в пятой части книги мы привели несколько тем для школьной научной работы  в области математики. С одной стороны, эти темы  интересны для науки, а с другой, как нам представляется, посильны детям. Подобные темы могут быть любопытны для школьников, желающих выступить на научных конференциях и конкурсах, например, на Балтийском научно-инженерном конкурсе фонда «Время науки».

\aabz Желаем успехов в работе с математическими задачами, представленными в сборнике, а в дальнейшем --- в научном творчестве!

\defaultstyle
