\noindent\abz Олимпиада «Математика НОН-СТОП» проводится с 2010 года, за это время значительно выросло как число её участников, так и интерес, проявляемый к ней в том числе со стороны образовательных организаций Санкт-Петербурга и известных фондов, которые теперь оказывают поддержку олимпиаде.

\aabz В 2010–2015 годах составителем условий задачи был И.А.\,Чистяков, а олимпиада включала в себя варианты для 5–8 классов из 6–12 задач, поделённых на пункты A, B и C. С 2016 года условия олимпиады составляют Б.А.\,Золотов и Д.Г.\,Штукенберг (Дмитрий Григорьевич — сотрудник ЛНМО, Борис Алексеевич — выпускник и сотрудник ЛНМО).

\aabz Одновременно с этим в олимпиаде появился профильный вариант для 7–8 классов: задачи профильного варианта представляют из себя целую исследовательскую проблему, раскрывающуюся перед решающим пункт за пунктом. Последним нововведением олимпиады стал вариант для четвёртого класса, присутствующий с 2017 года.

\aabz Первая часть этой книги — условия олимпиад, прошедших в 2016–18 годах. Задачи можно давать детям на занятиях в математических кру- жках\scolon они позволяют примерно ориентироваться на то, какими будут задания на олимпиаде «Математика НОН-СТОП» в ближайшем будущем. Авторы задач — Б.А.\,Золотов, Д.Г.\,Штукенберг, И.С.\,Алексеев.

\aabz Следом за условиями задач 2016–18, во второй части этой книги, представлены их решения, предложенные самими авторами задач (ра-\linebreak зобраны в том числе задачи профильных вариантов). Если при разборе задачи возникли трудности или стало интересно ознакомиться с необычными методами решения, стоит смотреть как раз во вторую часть книги. Автор разбора 2017–18 — Б.А.\,Золотов, 2016 — Д.Г.\,Штукенберг.

\aabz В третьей части книги представлены избранные, наиболее оригинальные задачи 2011–15 годов, сразу с решениями. Обычно таких задач оставалось по 3–4 на вариант. Авторы разборов 2012–15 — Б.А.\,Золотов, А.В.\,Семенов, 2011 — Л.А.\,Бакунец, И.Г.\,Прокофьева, Д.Г.\,Штукенберг.

\aabz Четвёртая часть этой книги — условия задач Петербургских турниров юных математиков (СПбТЮМ). СПбТЮМ проводится с 2015 года и отличается интересными, сложными заданиями — некоторые из них впоследствии оказываются научными работами, представляемыми на различных конференциях школьников. В этой книге мы собрали все задачи, бывшие на турнирах с самого его основания. Авторы задач — К.М.\,Чепур-\linebreak кин, И.С.\,Алексеев.

\aabz Наконец в пятой части книги мы привели несколько интересных тем для школьной научной работы: проектам, исследующим подобранные нами вопросы, гарантирован успех на научных конференциях шко-\linebreak льников — например, на Балтийском научно-инженерном конкурсе под эгидой Фонда «Время науки».

\aabz Желаем приятного и познавательного чтения!