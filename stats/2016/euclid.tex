\task{Рекуррентные функции}
\vspace{-0.35cm} \rightline{\it «Нужно просто сесть на сороковой трамвай и посчитать.»}

\ms Будем понимать функцию как чёрный ящик, который по нескольким натуральным числам выдаёт новое натуральное число. Нам понадобятся функции от одного и от двух аргументов --- соответственно, на вход они принимают одно или два числа.

\begin{enumerate}

\item Рассмотрим функцию $f$ от двух аргументов. Известно, что $f(a,b) = f(b,a)$, значение $f(a,0)$ определено однозначно для каждого $a$, и если $a>b$, то $f(a,b) = f(a-b,b)$. Доказать, что в таком случае функция $f$ однозначно определена для всех значений $(a,b)$.

\item Пусть при тех же условиях $f(a,0)=a$. Чему равно $f(a,b)$?

\item Что произойдёт со значением $f(a,b)$, если $f(a,0)$ определять иным образом?

\item Поменяем определение функции $f$. По прежнему $f(a,b) = f(b,a)$ и $f(a,0)$ заранее известно, но теперь $f(a,b) = T(f(a-b,b))$, где $T(x)$ --- некоторая функция от одной переменной. Доказать, что значение $f(a,b)$ снова однозначно определено для всех пар $(a,b)$.

\item Пусть $T(x) = x+k$, где $k$ --- некоторое натуральное число; $f(a,0)$ определено каким-либо образом. Чему в таком случае равно $f(a,b)$?

\item Рассмотрим функцию $h$ от двух аргументов. Про неё известно следующее:
  \subitem (а) $h(ac,b) = h(a,b) \cdot h(c,b)$ для любых $a$, $b$, $c$;
  \subitem (б) $h(a,bc) = h(a,c) \cdot h(a,b)$ для любых $a$, $b$, $c$;
  \subitem (в) $h(a,b) = h(a \bmod b, b)$ при $a>b$;
  \subitem (г) $h(1,t)$, $h(2,t)$, $h(t,1)$, $h(t,2)$ определены однозначно и заранее;
  \subitem (д) $h(p_1,p_2) = T(h(p_2,p_1))$, где $p_1$, $p_2$ --- нечётные простые числа, $p_1<p_2$, $T$~--- некоторая функция от одной переменной.

\ms Доказать, что значение $h(a,b)$ однозначно определено для всех значений $(a,b)$.

\end{enumerate}