\task{$(a,b)$--слоны}
\vspace{-0.35cm} \rightline{\itshape «У какого верблюда три горба?»}

\ms На шахматном поле $(a,b)$--слоном будем называть фигуру, делающую ходы следующего рода: она идёт по диагонали на $a$ клеток, поворачивается на $90^{\circ}$ в какую-либо из сторон и идёт на $b$ клеток по перпендикулярной диагонали. Пусть слон был изначально поставлен в какую-то клетку на бесконечном во все стороны шахматном поле. Будем называть эту клетку {\itshape начальной}.

\begin{enumerate}

\item (а) Cколько возможных ходов из начальной клетки может сделать $(a,b)$--слон, $a \ne b$? (б) Правда ли, что $(a,b)$--слон тождественен $(b,a)$--слону? (в) Можно ли присвоить обычному шахматному слону какие-либо $a$ и $b$?

\item Введите на чёрных клетках шахматного поля раскраску в два цвета такую, что любой ход $(1,0)$--слона меняет цвет клетки, на которой он стоит. При каких значениях $(a,b)$ соответствующий слон может ходить только по клеткам одного из этих двух цветов? Как часто меняется цвет клетки, в которую приходит слон, при остальных значениях $(a,b)$?

\item При каких значениях $t$ возможно $(1,t)$--слоном за несколько ходов дойти до клетки, имеющей с начальной общую вершину? Постройте для таких $t$ алгоритм дохождения, а для остальных докажите невозможность дохождения.

\item Пусть слон некоторым числом ходов может достичь клетки, имеющей с начальной общую вершину. Чему тогда может быть равен $\gcd (a,b)$?

\item Для класса значений $(a,b)$, которые одновременно удовлетворяют условиям, полученным вами в пунктах (2) и (4), проверить, может ли соответствующий $(a,b)$--слон попасть за несколько ходов в клетку, имеющую с начальной общую вершину. Вам требуется построить алгоритм попадания или доказать его невозможность для как можно б\'oльшего числа слонов в классе.

\item {\itshape Слон Безу---Тьюринга.} Доказать, что если $(a,b)$--слон может дойти до клетки, имеющей с начальной общую вершину, на бесконечном шахматном поле, то он может сделать это и на вертикальной / горизонтальной бесконечной ленте ширины $2\cdot(a+b)$, содержащей начальную и финишную клетки.

\item Теперь будем пытаться дойти до клетки, отстоящей от начальной на 2 по диагонали (соседней с соседней). Показать, что это можно сделать любым $(1,t)$--слоном. Чему может быть равен $\gcd (a,b)$, если это можно сделать $(a,b)$--слоном?

\item Описать всех $(a,b)$--слонов, с помощью которых можно дойти до клетки, отстоящей от начальной на 2 по диагонали, и при этом (а) с их помощью нельзя добраться до клетки, имеющей с начальной общую вершину (б) они не являются $(2a,2b)$--слонами, где $(a,b)$--слон может дойти до клетки, имеющей с начальной общую вершину.

\end{enumerate}