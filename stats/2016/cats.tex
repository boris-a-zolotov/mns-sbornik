\task{Кошки на координатной плоскости}
\vspace{-0.35cm} \rightline{\it «A radioactive cat has eighteen half-lives.»}

\ms Начинающий экономист Пабло Огурито спешит на защиту своей научной работы по экономической геометрии. Ему нужно пройти к метро по крайне скользкой дорожке. Чтобы не упасть, Пабло идёт по ней исключительно прямо, из некоторой точки на одном торце дорожки к некоторой точке на другом её торце.

\ms Около метро в обилии водятся кошки, все лоснящиеся и чёрные. В свою очередь, Пабло суеверен, но по-немецки. Он считает, что кошка, перебежавшая перед тобой дорогу слева направо (в том числе, если ты с ней встретился), приносит тебе несчастья, а кошка, перебежавшая дорогу справа налево — удачу (опять же, и в случае встречи с ней).

\ms Кошки в количестве $n$ штук сидят в некоторых точках на местности, Пабло стоит в некоторой точке на торце дорожки. Для каждой из кошек и для Пабло выбрана {\it финишная} точка, в которую направляется соответствующий персонаж в нашей истории. В момент времени $t=0$ всё сообщество из $n$ кошек и одного Пабло начинает равномерное прямолинейное движение с тем расчётом, чтобы в момент времени $t=1$ оказаться каждый в своей финишной точке.

\ms Введём дополнительное обозначение: $[(a,b)..(c,d)]$ --- прямоугольник со сторонами, параллельными осям координат, и двумя противоположными вершинами в точках $(a,b)$ и $(c,d)$.

\ms Вам предлагается ответить на следующие вопросы:

\begin{enumerate}

\item Пусть дорожка представляет из себя прямоугольник $[(-2,-1)..(2,1)]$. П. находится в точке $(2,0)$, $n=1$, единственная кошка сидит в точке $(0,-1)$ и готова направиться в точку $(0,1)$. Какие точки на противоположном конце дороги могут быть финишными для П., если он не хочет, чтобы кошка успела пересечь его путь до него?

\item Теперь дорожка представляет из себя прямоугольник $[(0,0)..(9,3)]$. П. стоит в точке $(9,1.5)$, $n=2$: одна кошка направляется из $(6,3)$ в $(6,0)$, другая — из $(3,0)$ в $(3,3)$. Какие точки могут быть финишными для П., если он хочет, чтобы вторая кошка не успела перебежать ему дорогу? Какие точки могут быть финишными для П., если он также хочет зарядиться удачей от первой кошки?

\item Дорожка — всё ещё $[(0,0)..(9,3)]$, П. находится в точке $(9,1.5)$ и идёт в точку $(0,1.5)$. Известно что есть две кошки — одна вертикально сверху направляется в точку $(3,0)$, другая --- вертикально снизу в точку $(6,3)$. Какие точки могут быть стартовыми для этих двух кошек, если П. хочет, чтобы первая кошка успела перебежать ему дорогу, а вторая --- нет?

\item Дорожка — снова $[(0,0)..(9,3)]$, и П. направляется из точки $(9,1.5)$ в точку $(0,1.5)$. Придумать расстановку на местности четырёх кошек такую, что одновременно (а) стартовая точка каждой кошки есть финишная для какой-то другой кошки (б) финишная точка каждой кошки есть стартовая для какой-то другой кошки (в) никакие две кошки не стартуют из одной точки и не финишируют в одну точку (г) ровно две кошки успевают перебежать дорогу П. справа налево, и ни одна — слева направо.

\item Могут ли в условиях предыдущего пункта три кошки перебежать П. дорогу справа налево?

\item Теперь дорога — прямоугольник $[(0,0)..(12,12)]$, П. направляется из точки $(12,9)$ в точку $(0,3)$. Предъявить такие $a$ и $b$, что кошка, идущая из точки $(x,a)$ в точку $(x,b)$, обязательно перебежит дорогу П. при любом вещественном $x$, $0 \leq x \leq 12$.

\item Выведите формулы для координат $x(t)$ и $y(t)$ кошки, равномерно и прямолинейно направляющейся из точки $(a,b)$ в точку $(c,d)$, в зависимости от времени $t$, $0 \leq t \leq 1$.

\end{enumerate}