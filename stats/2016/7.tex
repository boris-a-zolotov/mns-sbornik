%%%%%%%%%%%%%%%
\task{Деление и остатки}
\begin{itemize}
\itA Две подруги в преддверии олимпиады «Математика НОН-СТОП» решили поупражняться в нахождении наибольших общих делителей и наименьших общих кратных пар чисел. Они взяли два числа $a$ и $b$, и одна подруга получила $\gcd (a,b) = 564$, а другая $\mbox{НОК\,} (a,b) = 2016$. Докажите, что кто-то из них ошибся.

\itB Всё те же подруги теперь тренируются считать остатки от деления числа на число. Первая берёт число $a$, считает его остаток от деления на число $b$, а затем считает остаток от деления этого остатка на число $c$. Вторая утверждает, что процесс можно ускорить, сразу считая остаток от деления $a$ на $c$, без потери правильности ответа. Права ли она? Если да, докажите это, если нет — приведите контрпример.

\itC Даны два числа $b$ и $c$. Доказать, что для любого $a$ выполнено $(a \bmod b) \bmod c = a \bmod c$ тогда и только тогда, когда $b$ делится на $c$.
\end{itemize}
%%%%%%%%%%%%%%%

%%%%%%%%%%%%%%%
\task{Пятница}
\begin{itemize}
\itA По пятницам мама наливает пятерым детям парного коровьего молока. Она делает это в два круга, в первый раз как-то наполняя пять кружек, а во второй раз расходуя горшок до конца так, что уровень молока во всех кружках становится одинаковым. После первого круга в кружках четырёх детей было одинаковое количество молока, а в кружке пятого было на 20\% больше этого количества. Сколько миллилитров составляла эта самая двадцатипроцентная разница, если горшок имеет объём 2 литра, а на втором круге мама использовала 70\% его объёма?

\itB По пятницам в селе Хотчланд устраивались бега быков, но в качестве быков использовались тигры. У каждого тигра на ошейнике написаны два числа: $q$ — во сколько раз этот тигр бегает быстрее человека и $t$ — время в секундах, в течение которого этот тигр бежит с максимальной скоростью, а после этого ложится отдохнуть. Помогите участнику бегов Исинбаю Еленову вывести формулу для расстояния $x$, на котором ему нужно держаться от данного тигра, чтобы не быть съеденным.

\itC У мальчика Дани пять пятниц на неделе, а у Кости — три пятницы на неделе. Сколько существует расстановок пятниц на неделе таких, что ровно два дня будут пятницей и для Дани, и для Кости?
\end{itemize}
%%%%%%%%%%%%%%%

%%%%%%%%%%%%%%%
\task{Эксперименты с клавиатурой}
\begin{itemize}
\itA У братьев А. и Б. на клавиатурах начался {\it дребезг}: у А. все символы набирались пятикратно, а {\tt Backspace} удалял сразу восемь последних напечатанных символов. У Б. символы набирались семь раз подряд, а {\tt Backspace} удалял последние четыре напечатанных символа. При равной скорости нажатия на клавиши, кто из братьев будет печатать быстрее?

\itB Починив дребезг, братья обнаружили другие неполадки: у А. сломалась клавиша {\tt Shift}, а у Б. — {\tt Caps Lock}. До поломки оба печатали с одинаковой скоростью: они нажимали по пять клавиш в секунду. Неисправности повлияли на скорость печати: теперь А. должен нажимать {\tt Caps Lock} перед каждой последовательностью заглавных букв и ещё раз нажимать после неё, а Б. — удерживать {\tt Shift}, чтобы печатать заглавные буквы: при этом его скорость падает до двух букв в секунду, но на нажатие {\tt Shift} время не тратится. Вам нужно придумать строчку, которую А. напечатает минимум в два раза быстрее, чем Б. А есть ли строчка, которую Б. печатает в два раза быстрее, чем А.?

\itC Через год у Б. на клавиатуре осталось две рабочих клавиши, да и печатать он стал медленнее: нажимает лишь четыре клавиши в секунду. Паузы в процессе набора информации не несут; две клавиши одновременно нажимать нельзя. Вам нужно придумать способ без уменьшения русского алфавита (признания каких-нибудь букв равными) набирать любую букву не более чем за секунду либо доказать, что такого способа нет. Если его нет, то какое минимальное количество букв достаточно отождествить в одну, чтобы он появился?
\end{itemize}
%%%%%%%%%%%%%%%

%%%%%%%%%%%%%%%
\task{Факториалы}
Факториалом числа $n$ будем называть произведение всех натуральных чисел от 1 до $n$.

\begin{itemize}
\itA Егор посчитал факториал числа 33 и записал его на бумажку. Его сестра решила пошалить, и стёрла одну из цифр факториала. Получилась запись: \smallskip \\
\centerline{$33!=$ 8'683'317'618'811'886'49$\square$'518'194'401'280'000'000} \smallskip
Помогите Егору восстановить стёртую цифру.

\itB Посчитан факториал числа, не меньшего 12, и из его записи стёрта ровно одна цифра — причём известно, с какой позиции. Докажите, что её всегда можно однозначно восстановить.

\itC Докажите, что при чётном $n$ из произведения $1! \cdot 2! \cdot 3! \cdot \ldots \cdot n!$ можно вычеркнуть один факториал так, что оставшееся произведение будет квадратом целого числа.
\end{itemize}
%%%%%%%%%%%%%%%


%%%%%%%%%%%%%%%
\task{Ох уж эти школьницы!}
\begin{itemize}
\itA В течение четверти Арина получала оценки по математике. Она выписала их в строку, поставила между какими-то знак умножения и посчитала получившееся произведение. У неё получилось 22887. Каков средний балл Арины за четверть?

\itB Ольга придумывает два натуральных числа и вычитает квадрат одного из квадрата другого. В некий момент у неё получилось 3476. Найдите придуманные ей числа.

\itC Девочка Лиана записывает пятизначные числа, переставляет их первую цифру в конец и записывает полученные числа в пару к исходным. Проходящая мимо мама сказала: «Лиана, а зачем ты пишешь пары чисел, сравнимых по модулю 41?» Права ли мама в своём вопросе --- действительно ли числа в парах всегда сравнимы по этому модулю?
\end{itemize}
%%%%%%%%%%%%%%%

%%%%%%%%%%%%%%%
\task{Очень умные муравьи}
\begin{itemize}
\itA На плоскости живут четыре муравья. Могут ли они нарисовать себе на плоскости четыре связные области так, чтобы любые два муравья могли бы общаться друг с другом~— их области имели бы участок общей границы?

\itB Человек соорудил муравьям планету, то есть подвесил в воздух коробку размером $1 \times 1 \times 1$ метр. Через некоторое время он заметил, что муравьи организовывают экспедиции к углам коробки, как будто ходят в горы. Рост муравья — 2мм. Подъёму на холм какой высоты для человека нормального роста соответствует восхождение муравья к углу коробки, если рост нормального человека (по мнению составителя задач, несомненно) равен двум метрам?

\itC Счётное сообщество муравьёв хочет организовать на плоскости треугольную сетку такую, чтобы к каждой вершине прилегало ровно пять треугольников, жители которых могли бы попить чай в этой вершине. По силам ли муравьям это предприятие?
\end{itemize}
%%%%%%%%%%%%%%


%%%%%%%%%%%%%%%
\task{Несправедливый турнир}
Турнир по скоростному распиливанию проходит по усовершенствованной олимпийской системе. Изначально в турнире $2^t$ участников, $t \geq 2$, все в одной группе, потом группа потихоньку дробится: в каждом раунде в каждой группе участники бьются на пары, в которых состязаются в распиливании, и после раунда образуются группа победителей и группа проигравших, где повторяется аналогичный турнир.

\ms Так продолжается, пока все участники не побьются на группы по одному человеку: тогда места участников в итоговой таблице определяются естественным образом. В частности, победитель турнира — участник, выигравший во всех $t$ раундах, а аутсайдер — проигравший во всех $t$ раундах.

\begin{itemize}
\itA Может ли самый слабый участник турнира не оказаться аутсайдером?

\itB Может ли состязающийся сильнее как минимум половины участников турнира и ещё одного человека встретиться в финале с аутсайдером?

\itC Доказать, что для всякого состязающегося слабее половины участников турнира можно подобрать разбиение на пары в раундах так, чтобы в финале он встретился с аутсайдером.
\end{itemize}
%%%%%%%%%%%%%%%

%%%%%%%%%%%%%%%
\task{Дело-то житейское}
\begin{itemize}
\itA За книгу заплатили 200 рублей, и осталось заплатить втрое больше, чем осталось бы заплатить, если бы заплатили половину заплаченного и ещё столько, сколько осталось заплатить. Сколько стоит книга?

\itB В поисках нотного стана для очередной композиции гитарист Полина использует $n$ сайтов. Известно, что на каждом сайте приведены ссылки на некоторые из числа оставшихся $n-1$ сайтов, и всегда по крайней мере одна. Доказать, что на каких-то двух сайтах приведено одинаковое число ссылок.

\itC Мальчик Саша очень любит писать олимпиады. В некоторый месяц в году он выписал подряд без пробелов все числа этого месяца, выделив цветом даты трёх предстоящих олимпиад. Выяснилось, что никакие две олимпиады не проходили в подряд идущие дни, и что все незакрашенные промежутки из цифр имеют одинаковую длину. Доказать, что первая цифра в выписанной Сашей строке была закрашена.
\end{itemize}
%%%%%%%%%%%%%%%

%%%%%%%%%%%%%%%
\task{День, когда Стёпа всё испортил}
\begin{itemize}
\itA Лиза обещала подарить Стёпе диск со всеми сериями «Adventure time», если он сможет досчитать на пальцах до 1023 так, что все числа будут показаны разными комбинациями пальцев. Как Стёпе заполучить желанный диск?

\itB Стёпа сломал весы Всезнамуса, случайно сдвинув стрелку на циферблате, но не поняв, на сколько делений именно. В шкафу у Всезнамуса лежат бутылка воды и кусок циркония. Как Стёпе определить, на сколько делений сдвинута стрелка, и починить весы?

\itC Придя домой, Стёпа уселся за свою любимую компьютерную игру — Portal. Один портал стоит на большой высоте на вертикальной стене. Стёпа вылетает из него, перед самым приземлением ставит под собой второй портал и попадает в него с тем, чтобы вновь вывалиться из первого портала на стене, причём параллельно земле. Как быстро будет нижний портал отдаляться от стены? При прохождении через портал скорость Стёпы не изменяется.
\end{itemize}
%%%%%%%%%%%%%%%


%%%%%%%%%%%%%%%
\task{Хитрый Миша}
\begin{itemize}
\itA Миша подошёл к заведующему городским тиром с предложением открыть экспериментальный тир. Его зона имела бы форму равнобедренного треугольника со стрельбищем в вершине, а в конце дня награждались бы все, достигшие минимальной за этот день суммы расстояний от пулевой дырки до прямых, продолжающих боковые рёбра зоны. Докажите, что Миша тем самым разорит тир.

\itB Маша попросила Мишу вырезать из бумаги шесть развёрток для куба. Вместо этого Миша вырезал шесть крестиков, состоящих из пяти одинаковых тетрадных клеток каждый. Может ли Маша оклеить без наложений хоть какой-нибудь куб крестиками, вырезанными Мишей?

\itC Автор учебника по геометрии попросил Мишу набрать текст учебника. А Миша, как и следовало ожидать, допустил опечатку. В задаче, гласящей: «Отмерьте на одной стороне угла в 60 градусов 20 см, а на другой стороне — $t$ см, и посчитайте расстояние между отмеченными точками», Миша намеренно увеличил $t$ на 4 см, но при этом ответ на задачу остался верным. Чему равно $t$?
\end{itemize}
%%%%%%%%%%%%%%%