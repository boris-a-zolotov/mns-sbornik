\task{Римская десятичная система счисления}
\def\D{_{\text{\footnotesize Д}}} \def\R{_{\text{\footnotesize Р}}}

\begin{flushright} \itshape
	Интеллект человечества поддерживают лишь \\
	те неудобства, которые оно себе создаёт.
\end{flushright}

\ms Давайте добавим в знакомую нам десятичную систему счисления немного Древнего Рима. Значение числа по его записи мы теперь будем восстанавливать так: начиная с самого правого разряда, сравниваем $k$--ую цифру числа с $k+1$--ой~— и если более старшая цифра оказывается не меньше, то мы прибавляем к результату умножения её на соответствующую степень десятки число, которое получено нами при «раскодировании» первых $k$ разрядов\scolon в противном же случае~— вычитаем это число.

\ms Приведём несколько примеров. Записи «742» будет соответствовать число $700 + 40 + 2$, в то время как записи «342»~— число $300 - (40 + 2) = 258$ (так как $3<4$).

\ms Записи «6342» соответствует число $6300 - 42 = 6258$, а записи «2342»~— число $2000 - (300 - (40+2)) = 1742$ (здесь сразу $2<3$ и $3<4$). Записи «55» соответствует число 55.

\ms Чтобы не запутаться, будем обозначать через $S\D$ число, соответствующее строке $S$, если воспринимать её как запись в традиционной десятичной системе счисления, а через $S\R$ — число, соответствующее строке $S$, если воспринимать её как запись в «римской» десятичной системе счисления. Иными словами,
	$$2342\R\ =\ 1742\D\ =\ \text{одна тысяча семьсот сорок два} \in \mathbb N.$$

\begin{enumerate}

\item Какие числа соответствуют следующим записям:
\begin{center}
	$333\R$, $2050\R$, $10001\R$, $404004\R$?
\end{center}

\item Перевести десятичные числа в десятичную римскую систему счисления (знак «минус» в десятичной римской системе счисления не используется!): $91\D$, $150\D$, $-1\D$, $13\D$.

\item Опишите все $S$ такие, что $S\D = S\R$.

\item Предложите алгоритм построения по {\bfseries двузначному} положительному десятичному числу (то есть, имеющему вид $xy\D$) его десятичной римской записи.

\item Пусть $X\D = Y\R = N$. Какая десятичная римская запись будет соответствовать числу $10 \cdot N$? Числу $-N$?

\item Приведите пример числа $N$ такого, что есть две {\bfseries различных} строки $S$, $T$, для которых выполнено условие
	\vspace{-0.2cm}$$S\R = T\R = N.$$

	\vspace{-0.4cm}
\item Может ли у одного числа быть строго больше двух различных десятичных римских записей?

\item Придумайте признаки делимости на 2, на 5, на 3 в десятичной римской системе счисления.

\item Пусть $Y\R = N>0$. Верно ли в \underline{десятичной} римской системе неравенство
$$\frac{Y\D}{4} < N \leq Y\D?$$

\item {\bfseries\itshape А был ли мальчик?} Можно ли вообще считать «десятичную римскую систему» системой счисления? Покажите, что, строго говоря, нет: приведите пример числа $M \in \mathbb N$, которому не соответствует ни одной десятичной римской записи.

\end{enumerate}