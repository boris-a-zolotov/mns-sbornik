\task{Простеющие числа}
\def\lne{\medskip \\ \rule{0.65\textwidth}{0.028cm}}

\noindent Рассмотрим число 12. Среди чисел, меньших, чем 12, взаимно просты с ним следующие:
\vspace{-0.2cm}
$$1, 5, 7, 11.$$

\vspace{-0.2cm}
\noindent Все они, кроме единицы, являются простыми. А вот среди чисел, меньших 10 и взаимно простых с 10, есть, например, 9 — составное число.

\ms Итак, число называется {\itshape простеющим}, если все числа, меньшие его и взаимно простые с ним — 1 или простые. Как мы выяснили, 12 — простеющее число, а 10 — нет.

\begin{enumerate}

\vspace{0.25cm}
\item Приведите примеры других простеющих чисел, кроме 12.
\lne

%%%%%%%%%%%%
\item Перечислите все нечетные простеющие числа.
\item Перечислите все простеющие числа, не делящиеся на 3.
\item Перечислите все простеющие числа, не делящиеся на 5.
\lne
%%%%%%%%%%%%

%%%%%%%%%%%%
\item Докажите, что число вида $p^2 + 1$, где $p$ — простое, не может быть простеющим.
\item Докажите, что если $n > p_1 \cdot p_2$, $p_1$ и $p_2$ — простые числа, и $n$ не делится ни на $p_1$, ни на $p_2$, то оно \\ не может быть простеющим.
\item Докажите, что всякое простеющее число имеет вид $p+1$, где $p$ — какое-то простое.
\lne
%%%%%%%%%%%%

\item Бесконечно ли множество простеющих чисел?
\end{enumerate}