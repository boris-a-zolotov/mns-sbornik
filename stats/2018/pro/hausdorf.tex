\vfill\eject
\task{Расстояние между множествами}
\def\dist{\mathrm{dist}\,} \def\l#1{\limits_{#1}}
\def\Dist{\mathrm{DIST}\,}

\begin{flushright} \itshape
	«Знаете, как на русский язык переводится фамилия „Хаусдорф“? \\
	Домик в деревне!»
\end{flushright}

\ms В данной задаче мы рассматриваем {\bfseries только} конечные множества точек на плоскости, которые всегда будем обозначать буквами $A$, $B$, $C$. Плоскость замечательна тем, что на ней определено расстояние между точками $\dist (x,y)$ — его можно мыслить, как длину кратчайшего отрезка, соединяющего $x$ и $y$. Напомним три основных свойства расстояния между точками:

\vspace{-0.15cm}
\begin{itemize}
	\item $\dist (x,y) = \dist (y,x)$\scolon
	\item $\dist (x,y) = 0$ тогда и только тогда, когда $x=y$\scolon
	\item Для всех $x$, $y$, $z$ выполнено «неравенство треугольника»:
		\vspace{-0.2cm} $$\dist (x,y) + \dist (y,z) \geq \dist (x,z).$$
\end{itemize}

\vspace{-0.4cm}
\ms Также нам понадобится понятие {\itshape наименьшего значения}: если $f(x)$ — вы- ражение, куда можно подставлять разные значения переменной $x$, то
	$$\min_x\,f(x)\text{\ \ —}$$
это наименьшее число, которое может получиться при подстановке чего-либо в выражение $f$. Например, $\min_x x^2 = 0$. Похожим образом обозначается наибольшее значение — $\max_x f(x)$.

\vspace{-0.2cm}
\begin{enumerate}

\item Рассмотрим квадрат $A_1A_2B_1B_2$ со стороной 1. Пусть $M_1 = \{A_1, A_2\}$, $M_2 = \{B_1, B_2\}$. Чему равно число
\vspace{-0.2cm}
$$\max\l{x \in M_1}\ \Bigl(\min\l{y \in M_2}\ \dist (x,y)\Bigr)?$$

\vspace{-0.2cm}
А чему равно
\vspace{-0.2cm}
$$\min\l{y \in M_2}\ \Bigl(\max\l{x \in M_1}\ \dist (x,y)\Bigr)?$$

\vspace{-0.2cm}
Например, для подсчёта первого выражения вам нужно для каждой из точек $A_1$, $A_2$ найти расстояние до ближайшей к ней точки из множества $M_2$, а затем взять наибольшее из этих двух расстояний.

\item Как оказалось, $M_1$ и $M_2$ из предыдущего пункта — пример таких множеств, что указанные нами величины для них не совпадают. Докажите, тем не менее, что для любых двух $A$, $B$ выполнено неравенство:
\vspace{-0.2cm}
$$\max\l{x \in A}\ \Bigl(\min\l{y \in B}\ \dist (x,y)\Bigr)\ \ \le\ \ 
	\min\l{y \in B}\ \Bigl(\max\l{x \in A}\ \dist (x,y)\Bigr).$$

\vspace{-0.45cm}
\item Докажите, что по заранее заданному положительному числу $r$ всегда можно подобрать два множества $A$, $B$ так, что разность двух величин из предыдущего пункта будет равна $r$. Иными словами, эту разность можно сделать сколь угодно большой.

\end{enumerate}

\noindent Также нам потребуется понятие {\itshape окрестности множества}. Пусть $A$ — конечное и состоит из точек на плоскости. Тогда его $\rho$--окрестность — это фигура, являющаяся объединением кругов радиуса $\rho$ с центрами в точках множества $A$. Для иллюстрации этого понятия предлагаем вам изобразить 1--окрестность множества из двух точек, расстояние между которыми равно 2.

\begin{enumerate}
\setcounter{enumi}{3}

\item Приведите пример $A$, $B$ таких, что $A$ целиком лежит в 1--окрестнос- ти $B$, но $B$ не лежит в 1--окрестности $A$.

\item Пусть $A$, $B$ и $C$ — три конечных множества точек на плоскости. Докажите, что если $B$ целиком лежит в $\rho_1$--окрестности $A$, а $C$ целиком\linebreak лежит в $\rho_2$--окрестности $B$, то $C$ целиком лежит в $(\rho_1 + \rho_2)$--окрест- ности множества $A$.

\item Докажите, что для любого $A$ выполнено
\vspace{-0.2cm}
$$\max\l{x \in A}\ \Bigl(\min\l{y \in A}\ \dist (x,y)\Bigr)\ =\ 0.$$

\vspace{-0.35cm}
\item Докажите, что если $\max_{x \in A}\ \bigl(\min_{y \in B}\ \dist (x,y)\bigr) \leq R$, то множество $A$ целиком лежит в $R$--окрестности множества $B$. Докажите обратный факт.

\item Пользуясь предыдущими пунктами, проверьте, что для следующего выражения (вместо $A$ и $B$ можно подставлять конечные множества точек на плоскости) выполнены три свойства расстояния, перечисленные в начале этой задачи:
\vspace{-0.2cm}
$$\Dist (A,B)\ \ =\ \ \max\ \biggl\{\max\l{x \in A}\ \Bigl(\min\l{y \in B}\ \dist (x,y)\Bigr),
	\ \max\l{x \in B}\ \Bigl(\min\l{y \in A}\ \dist (x,y)\Bigr)\biggr\}.$$

\vspace{-0.2cm} В частности, первое свойство следует из того, что выражение никак не меняется при замене $A$ на $B$ и наоборот.
\end{enumerate}

\noindent Поздравляем вас! Только что вы определили {\bfseries расстояние Хаусдорфа} — $\Dist (A,B)$ — между множествами на плоскости. Это незаменимый объект в математике. Предлагаем вам доказать простейший, но очень важный факт про это расстояние:

\begin{enumerate}
\setcounter{enumi}{8}
\item Если $A$ и $B$ являются подмножествами одного и того же круга радиуса $R$, то $\Dist (A, B) \leq 2R$.
\end{enumerate}