\task{Фургончик}
\begin{enumerate}
\itA Длины стен кузова нового фургона, который строит себе мороженщик Саша, в метрах выражаются двумя различными простыми числами. Известно, что если удлинить каждую из стен на 1 метр, площадь фургона увеличится на $15\,\text{м}^2$. Найдите размеры фургона.

\itB В сашином фургоне родилась сороконожка (ее ноги пронумерованы от 1 до 40). Она хочет сделать первый шаг — и переставляет первую ногу. Вторым шагом она переставляет все ноги, номера которых делятся на~2. Третьим — все ноги, номера которых делятся на 3 и которые не были переставлены ранее. Сколько ног ей теперь осталось переставить, чтобы окончательно сдвинуться с места?

\itC Однажды утром мороженщик Саша отправился развозить мороженое на своем новом фургоне. Он обслуживает семь городов — $A$, $B$, $C$, $D$, $E$, $F$ и $G$ — и эти города в каком-то порядке стоят вдоль одного прямого шоссе. Саша выехал из города $A$ и проехал 18 километров до $B$. Потом — 10.5 километров до $C$. Затем — 27 километров до $D$, 15 километров до $E$ и 19.5 километров до $F$. Наконец — 12 километров до $G$. Посмотрев вечером в атлас, Саша к своему удивлению узнал, что от $A$ до $G$ указано расстояние по шоссе, равное \SI{41}{\text{км}}. Докажите, что информация в атласе неверна.
\end{enumerate}