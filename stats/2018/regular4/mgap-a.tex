\task{Ужасный гадкий аккуратный подсчёт}
\begin{enumerate}
\itA Из клетчатой бумаги вырезали прямоугольник размером $4 \times 5$ клеток. Сколько на нём можно найти квадратов? А прямоугольников?

\itB Круг разделён на $n$ секторов одинакового размера. Сколькими способами можно покрасить эти $n$ секторов в $n$ цветов, если две раскраски, получающиеся друг из друга вращением круга, считаются одинаковыми?

\itC Есть шесть цветов — красный, белый, синий, зелёный, чёрный, жёлтый. Нам хочется составить из них всевозможные триколоры (то есть флаги, состоящие из трёх горизонтальных цветных полос, как российский или немецкий). При этом если рядом оказываются две полосы одного цвета, они сливаются в одну, поэтому такой флаг не считается триколором. А вот флаг вроде «белый — синий — белый» — считается. Так сколько же триколоров можно составить?
\end{enumerate}