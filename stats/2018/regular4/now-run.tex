\task{Необходимости и достаточности}
\begin{enumerate}
\itA Длина тела мышки — 10 сантиметров, а кошки — 55 сантиметров. Мышка пробегает 35 своих тел за секунду, а кошка — всего 9 своих тел за секунду. Догонит ли кошка мышку?

\itB На одном болоте живут 100 ужасных Йожинов. Председатель решил, что с этой ситуацией надо наконец разобраться — обезвредить Йожинов и продать их в зоопарк. Известно, что Йожина можно обезвредить, только скинув на него с самолёта порошок. Председателю нужно выбрать, какой самолёт использовать: винтовой или реактивный. \smallskip \\
Винтовой самолёт за один вылет осыпает порошком двух Йожинов, но, чтобы окончательно обезвредить одного Йожина, нужно осыпать его трижды. Реактивный самолёт, в силу своей более высокой скорости, за один вылет осыпает порошком 5 Йожинов, но, так как на каждого Йожина теперь попадает меньше порошка, для обезвреживания его нужно осыпать восемь раз. \smallskip \\
Какой же самолёт эффективнее: какому потребуется меньше вылетов, чтобы обезвредить всех Йожинов?

\itC Несколько велосипедистов отправились в поход. За обедом они в сумме съедают 2 килограмма еды плюс 0.1 кг за каждый килограмм еды, который они везли на себе до этого. Например, если у них было 10 килограммов еды на всех, то на ближайшем обеде они съедят $2 + 0.1 \cdot 10 = 3$ килограмма, а на следующем~— $2 + 0.1 \cdot (10-3) = 2.7$ килограммов. В походе планируется 30 обедов (а велосипедисты не завтракают и не ужинают). Сколько еды им нужно взять с собой, чтобы её хватило на весь поход (и в конце похода не осталось ничего лишнего)?
\end{enumerate}