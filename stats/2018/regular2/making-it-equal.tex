\task{Одновременное вычитание}
\begin{enumerate}
\itA На доске написаны пять чисел, сумма которых делится на три. Разрешается одновременно уменьшать на единицу три из написанных на доске чисел. Всегда ли можно добиться того, чтобы на доске в итоге оказалось пять нулей?

\itB На плоскости расположено несколько точек, каждой из которых приписан {\it вес} — целое число. При этом известно, что сумма весов всех точек равна нулю. Точки можно соединять кривыми, у каждой из которых есть {\it цена}. Если две точки соединены кривой с ценой $w$ ($w$ — целое число), то к весу одной из них прибавляется $w$, а из веса другой вычитается $w$ (куда именно прибавлять, а откуда вычитать, можно решать самому). Докажите, что можно соединить точки кривыми с какими-то ценами так, чтобы веса всех точек оказались нулевыми.

\itC В стране несколько городов, между ними проложены дороги. Для каждой дороги указаны направление (все дороги односторонние) и {\it вес} — натуральное число. Известно, что для каждого города сумма весов входящих в него дорог равна сумме весов исходящих. Докажите, что несколько машин (на номере каждой из которых написано натуральное число) могли проехать каждая по кругу через несколько городов так, что вес каждой дороги оказался равен сумме номеров машин, побывавших на ней.
\end{enumerate}