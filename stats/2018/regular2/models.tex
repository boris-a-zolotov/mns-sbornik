\task{Модельки}
\begin{enumerate}
\itA Вовочка подумал: «Если автомобиль „Жигули“ весит \SI{1200}{\text{кг}}, а мама подарила мне модельку масштаба $1\ :\ 43$, сделанную из тех же материалов, что и полноценная машина, то моделька должна весить $1200 / 43 \approx 28$ килограммов... Однако она ощутимо легче моего кота, про которого мама недавно сказала, что он толстый, потому что преодолел отметку в 6 кило. Где же логика?» \smallskip \\
Помогите Вовочке разобраться — почему моделька на самом деле не так тяжела?

\itB В 1791 году единица длины {\bfseries метр} была определена как одна сорокамиллионная часть Парижского меридиана. А в современном спорте популярно измерение не скорости, а {\itshape темпа} бегуна — сколько минут он тратит на преодоление километра. \smallskip \\
Самый быстрый темп, которого умеет достигать моделька самолёта — $0.54$ мин/км. За сколько часов такая моделька долетит вдоль Парижского меридиана от Северного полюса до Южного и обратно?

\itC Можно ли собрать из шестерёнок такую систему, что в ней найдутся две шестерёнки A и B, и при вращении A в движение приводятся все остальные шестерёнки, а при вращении B — нет?
\end{enumerate}