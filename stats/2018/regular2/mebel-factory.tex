\task{Современная мебельная фабрика}
\begin{enumerate}

\itA Восемь ящиков экспериментального письменного стола расположены по кругу. Каждый ящик может быть открыт или закрыт. Стол устроен так, что можно одновременно изменять состояние пары ящиков, между которыми ровно два других ящика — то есть можно либо открыть два сразу, либо закрыть два сразу, либо открыть один, закрыв другой. \smallskip \\
У стола на витрине открыты два противоположных ящика. Покажите, как за 4 действия закрыть их оба.

\itB В понедельник перед обедом обыкновенный мебельщик Сергей растворил пачку красителя для шкафов в десятилитровом ведре воды. В обед обиженный на начальство фабрики Фёдор в отчаянии вылил из ёмкости 4 литра раствора, долил 4 литра воды и тщательно размешал (чтобы замести следы). \smallskip \\
На следующий день Сергей снова растворил пачку красителя в 10 литрах воды. На этот раз Фёдор вылил из ведра 2 литра раствора, долил 2 литра воды, тщательно размешал — и повторил ту же последовательность действий ещё раз. В какой из дней в ведре осталось больше красителя?

\itC Экспериментальный стул с использованием нанотехнологий (одна из инноваций заключается, например, в том, что у такого стула ровно 720 ножек) падает с лестницы (в качестве испытания, разумеется). Выяснилось, что при падении он потерял в три раза меньше ножек, чем у него бы осталось, потеряй он в три раза меньше ножек, чем у него осталось сейчас. Так сколько же ножек осталось у стула?
\end{enumerate}