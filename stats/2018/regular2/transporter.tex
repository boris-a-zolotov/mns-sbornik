\task{Как провожают транспортёры...}
\noindent Транспортёром будем называть движущуюся ленту, на которой можно перемещать предметы (все видели такую на кассе в «Пятёрочке»\scolon ещё её можно сравнить с траволатором на ст.\,м.\,«Спортивная»).

\begin{enumerate}
\itA Если транспортёр движется со скоростью $v$ м/с, то лежащий на нём питон проезжает мимо неподвижного наблюдателя за 14 секунд. Давайте возьмём питона–детёныша (его длина составляет $\tfrac{3}{4}$ от длины взрослого питона), в шесть раз более медленный транспортёр, а также заставим наблюдателя идти со скоростью $\tfrac{1}{3}v$~м/с навстречу транспортёру. За какое время детёныш питона пронесётся мимо наблюдателя?

\itB Два кубика размером $5 \times 5 \times 5$ см едут по транспортёру, причём расстояние между ними равняется 10 см. С данного транспортёра они попадают на следующий, в два раза более быстрый, и дальше едут по нему. Каково расстояние между ними теперь?

\itC В отдел приёма песка фабрики «Весёлый Песочник» привезли 1'200 кг песка. Из отдела приёма в отдел первичной очистки на фабрике идёт два транспортёра: один переносит 500 граммов песква в секунду, а другой, новый, — 1 кг песка в секунду. Как поделить песок между этими двумя транспортёрами так, чтобы перевезти весь песок из одного отдела в другой за наименьшее время?
\end{enumerate}