\task{Как провожают транспортеры...}
\noindent Транспортером будем называть движущуюся ленту, на которой можно перемещать предметы (все видели такую на кассе в «Пятерочке»; еще ее можно сравнить с траволатором на ст.\,м.\,«Спортивная»).

\begin{enumerate}
\itA Если транспортер движется со скоростью $v$ м/с, то лежащий на нем питон проезжает мимо неподвижного наблюдателя за 14 секунд. Давайте возьмем питона–детеныша (его длина составляет $\tfrac{3}{4}$ от длины взрослого питона), в шесть раз более медленный транспортер, а также заставим наблюдателя идти со скоростью $\tfrac{1}{3}v$~м/с навстречу транспортеру. За какое время детеныш питона пронесется мимо наблюдателя?

\itB Два кубика размером $5 \times 5 \times 5$ см едут по транспортеру, причем расстояние между ними равняется 10 см. С данного транспортера они попадают на следующий, в два раза более быстрый, и дальше едут по нему. Каково расстояние между ними теперь?

\itC В отдел приема песка фабрики «Веселый Песочник» привезли 1`200 кг песка. Из отдела приема в отдел первичной очистки на фабрике идет два транспортера: один переносит 500 граммов песква в секунду, а другой, новый, — 1 кг песка в секунду. Как поделить песок между этими двумя транспортерами так, чтобы перевезти весь песок из одного отдела в другой за наименьшее время?
\end{enumerate}