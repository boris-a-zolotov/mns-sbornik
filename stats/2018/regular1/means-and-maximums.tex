\task{Средние арифметические}
\noindent Напомним: среднее арифметическое набора чисел $a_1 \ldots a_n$ вычисляется по формуле

$$\frac{a_1 + \ldots + a_n}{n}.$$

\begin{enumerate}
\itA Придумайте четыре набора по пять чисел каждый так, чтобы наибольшее из средних арифметических этих наборов было больше,\linebreak чем среднее арифметическое наибольших чисел этих наборов.

\itB В первый класс школы №265 поступило 120 детей ростом соответственно 101, 102, 103 $\ldots$ 220 сантиметров. Завуч хочет распределить их на 4 класса по 30 человек так, чтобы, если в каждом классе взять рост самого низкого ученика, среднее арифметическое полученных четырех чисел было наибольшим. Как ему это сделать?

\itC В первый класс школы №235 поступило 80 детей ростом соответственно 51, 52, 53 $\ldots$ 130 сантиметров. Завуч хочет распределить их на 4 класса по 20 человек так, чтобы, если в каждом классе взять средний арифметический рост его учеников, минимум полученных четырех чисел был наибольшим. Как ему это сделать?
\end{enumerate}