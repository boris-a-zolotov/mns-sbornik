\task{Клиренсы}
\begin{enumerate}

\itA Диаметр колеса велосипеда — 74 см. На высоте центра колеса расположена {\itshape каретка} — узел, вокруг которого крутятся педали. Расстояние от каретки до педали — 175 мм. Каково минимальное расстояние от педали до земли (если на велосипеде едут по прямой, не наклоняясь)? Размерами педалей пренебречь.

\itB Расстояние между соседними ножками стула — 50 см. К ножкам стула прикрепили колесики и стали втаскивать его за веревку по стене многоэтажного дома, который имеет форму куба, так, что стул едет по стене колесиками. Каково должно быть расстояние от сидения стула до земли, чтобы он смог въехать со стены многоэтажки на ее крышу, не поцарапав нижнюю сторону сиденья?

\itC Автобус с диаметром колес 1 метр и колесной базой $10.5$ метров (так называют расстояние между передней осью и задней) стоит на планете Маленького принца, диаметр которой 20 метров. Каким должен быть дорожный просвет (расстояние от пола до земли) у автобуса, чтобы он не царапал днищем грунт?
\end{enumerate}