\task{Где-то я это уже видел}
\begin{enumerate}

\itA Сколько дат в году могли бы оказаться на экране цифровых часов в качестве времени? {\itshape Например, 19 июня~— 19:06, а 27 ноября времени не соответствует.}

\itB Сколько дат в году могли бы появиться на экране цифровых часов, если разрешено использовать сначала месяц, а потом день? {\itshape Например, 27 ноября тогда будет соответствовать время 11:27.}

\def\l{\!\!\!\!}
\itC Автомобильный номер в Ленинградской области имеет вид

\begin{center}\begin{tabular}{|cccccc|c|}
	\hline
	$\times\vphantom{\int\limits_1^1}$\l &
	$\square$\l &
	$\square$\l &
	$\square$\l &
	$\times$\l &
	$\times$\ &
	$47^{\mathrm{\ RUS}}$ \\
	\hline
\end{tabular} \end{center}

\noindent Вместо квадратиков стоят цифры, а вместо крестиков~— буквы русского алфавита, заглавные варианты которых похожи на какие-\linebreak либо буквы английского алфавита (например, такие буквы~— А, Т или У). \smallskip \\
На скольких номерах в Ленинградской области есть ровно одна\linebreak гласная? А ровно две гласных?

\end{enumerate}