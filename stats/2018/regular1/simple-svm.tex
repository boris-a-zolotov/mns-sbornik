\task{Об одной задаче классификации}
\begin{enumerate}

\itA Полосой будем называть часть плоскости, заключённую между двумя параллельными прямыми. Ширина полосы — расстояние между ограничивающими её прямыми. Пусть на плоскости даны два непересекающихся круга. Покажите, как с помощью линейки без делений и циркуля отделить их друг от друга полосой максимальной ширины.

\itB В условиях пункта A отделите полосой максимальной ширины два непересекающихся одинаково ориентированных квадрата на плоскости.

\itC В условиях пункта A отделите полосой максимальной ширины два произвольных квадрата на плоскости.
\end{enumerate}