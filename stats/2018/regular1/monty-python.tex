\task{Летающий цирк}

\vspace{-0.5cm}
\begin{flushright} \it
Если вы скажете слово «матрас», он наденет \\
ведро себе на голову.
\end{flushright}

\begin{enumerate}
\itA Если сказать мистеру Лэмберту слово {\tt «МАТРАС»}, он кричит „Караул!“, снимает перчатки, надевает на голову ведро, встаёт одной ногой в коробку из-под телевизора и поёт два куплета из песни про коня.\smallskip \\
Если сказать мистеру Лэмберту слово {\tt «СТАРТ»}, он кричит „Караул!“, снимает перчатки, встаёт двумя ногами в коробку из-под телевизора и поёт один куплет из песни про коня.\smallskip \\
А что будет, если сказать мистеру Лэмберту слово {\tt «МАРС»}?

\itB У джентльмена есть 34 доллара, и он хочет купить себе шляпу. Продавец называет цену в 120 долларов, но джентльмен должен торговаться. Каждый раз, когда джентльмен торгуется, продавец сбавляет цену до среднего арифметического финансовых возможностей джентльмена и цены, названной на предыдущем шаге, — а джентльмен в это время оглядывает и находит одну долларовую монетку, лежащую на брусчатке. Сможет ли джентльмен когда-нибудь купить себе желанную шляпу?

\itC {\bf Тревор:} «Этот сконфуженный кот стоит 9600 рублей.» \\
{\bf Джереми:} «Кот дешевле, поскольку Тревор в 4 раза преувеличивает каждое число, которое называет. Хоть он только что и сказал про стоимость кота в 2400 рублей, кот на самом деле стоит 150 рублей.» \\
Подсчитайте, во сколько раз Джереми преуменьшает каждое произносимое число, и сколько на самом деле стоит сконфуженный кот.

\end{enumerate}