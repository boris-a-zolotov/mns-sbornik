\task{Без пробуксовки}
\begin{enumerate}

\itA Легковая машина с колёсной базой (так называют расстояние между передней и задней осью) 5 метров повернула переднее левое колесо на $30^\circ$ влево, при этом заднее левое колесо осталось в исходном положении, а правые колёса повернулись так, чтобы машина могла ездить без пробуксовки. С повёрнутыми таким образом колёсами машина стала ездить по окружности. По окружности какого радиуса ездит заднее левое колесо?

\itB Погрузчик в супермаркете с колёсной базой $1.8$ метра повернул переднее левое колесо на $45^\circ$ влево, а заднее левое — на столько же в противоположном направлении. Остальные колёса повернулись так, чтобы погрузчик мог ездить без пробуксовки. С повёрнутыми таким образом колёсами погрузчик стал ездить по окружности. Укажите точку, вокруг которой он ездит.

\itC Расстояние между передней и средней осью трёхосного автобуса — 9 метров, а между средней и задней~— 3 метра. Переднее левое колесо повернулось на $60^\circ$ влево, среднее левое осталось в прямом положении. На сколько градусов и куда нужно повернуться заднему левому колесу, чтобы автобус смог поехать без пробуксовки?
\end{enumerate}