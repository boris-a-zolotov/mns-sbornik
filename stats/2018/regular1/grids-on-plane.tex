\task{Сетки на плоскости}
\begin{enumerate}

\itA Если замостить плоскость равносторонними треугольниками одного размера, то их стороны образуют треугольную сетку (в ее форме также обычно строят карточные домики). В треугольной сетке есть три направления ребер — они соответствуют сторонам складываемых треугольников. Докажите, что любой кратчайший путь по треугольной сетке от одного ее узла к другому использует ребра максимум двух направлений из трех.

\itB В деревне Малые Пауки поставили поперек реки рыболовную сеть с квадратными ячейками, размером $m \times n$ ячеек. Окунь Виталий умеет сгрызать узлы в сетке, но, так как вода мутная, он не видит, какой именно узел грызет, то есть каждый раз выгрызает случайный из узлов. Сколько узлов ему нужно сгрызть, чтобы сетка гарантированно развалилась хотя бы на две части?

\itC Завхоз офисного здания, на третьем этаже которого есть бесконечно длинный и бесконечно широкий коридор, заказал в ООО~„Странные ванные“ бесконечно много четырехугольных кафельных плиток одинаковой формы и размера (при этом четырехугольник не обязан быть ни прямоугольником, ни даже выпуклым). Докажите, что какой бы формы ни были плитки, ими все равно можно покрыть весь пол в коридоре.
\end{enumerate}