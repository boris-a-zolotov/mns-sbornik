\task{Прогрессивное сложение}
\noindent В свободных школах, не имеющих предрассудков, решили складывать числа, просто приписывая их друг к другу. Мы будем обозначать это действие значком $\oplus$: например, $2 \oplus 2 = 22$, $2000 \oplus 2000 = 20002000$. \smallskip\\
В обычной жизни, в каком порядке числа ни складывай, результат остаtтся неизменным: $2+3+5 = 5+3+2$. Однако, если выполнять с числами действие $\oplus$, результат может изменяться в зависимости от порядка чисел: $2 \oplus 3 \oplus 5 = 235 \ne 532 = 5 \oplus 3 \oplus 2$.

\begin{enumerate}
\itA Даны числа 95 и 500. В каком порядке их нужно сложить, чтобы результат получился больше?

\itB Даны три произвольных числа $P$, $Q$, $R$. В каком порядке нужно выполнять с ними действие $\oplus$, чтобы получить наибольший возможный результат?

\itC Определим «прогрессивную разность»: $a \ominus b$ — это такое число $c$, что $b \oplus c = a$ Приведите пример чисел $a$ и $b$ таких, что их разность $a \ominus b$ не определена (нужного числа $c$ не найдtтся).
\end{enumerate}