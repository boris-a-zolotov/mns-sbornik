\task{У магазина}
\begin{enumerate}
\itA Два продавца в магазине, Фёдор и Кирилл, увеличивают все числа, которые называют, в несколько раз. \smallskip\\
{\bf Фёдор:} Кирилл умножает числа, которые произносит, на 144, так что не паникуйте, обсуждая с ним цены. \smallskip\\
{\bf Кирилл:} А ты когда вчера сказал, что учебник истории стоит 43200 рублей, покупатели в обморок упали! \smallskip\\
{\bf Игорь Евгеньевич, директор магазина:} Вы так забавно ссоритесь! Причём если спросить у вас, сколько стоит учебник, вы скажете одну и ту же сумму. \smallskip\\
Так во сколько же раз увеличивают числа продавцы — и сколько стоит учебник по истории?

\itB Злоумышленник пришёл на парковку магазина и затёр по одной цифре на номерном знаке каждой из стоящих там машин. Он не знал, что числа на всех номерах машин в стране делятся на 99. Докажите, что даже после его пакостей можно однозначно восстановить стёртую цифру на номере каждой машины.

\itC Покупатель и продавец в магазине торгуются: покупатель хочет купить товар по одной цене, а продавец~— продать по другой (причём не обязательно продавец называет б\'oльшую сумму\scolon ему важнее не получить больше прибыли, а настоять на своём). Цены, конечно же, целые и неотрицательные. Торг происходит так: если сумма, названная продавцом, больше названной покупателем, то он вычитает из своей цены покупательскую. В противном случае цена, названная продавцом, вычитается из названной покупателем. Так происходит, пока один из участников торга не назовёт нулевую цену. \smallskip\\
(а) Любой ли торг завершится? (б) Покупатель и продавец называют цены, не превосходящие 20. При каких значениях цен торг будет наиболее долгим?
\end{enumerate}