\task{Напрасно называют север крайним}
\begin{enumerate}
\itA Один коротышка с двумя ногами поехал кататься на велосипеде. Но так как на дворе была зима, $-10$ градусов, он отморозил себе одну ногу. Другой коротышка через месяц поехал кататься на велосипеде. Но на дворе по-прежнему была зима, уже $-20$ градусов, и он отморозил себе все имеющиеся ноги (их также было две). \smallskip \\
Сколько ног отморозит себе на 10- и на 20-градусном морозе туристическая группа из 40 коротышек? А их маленький серый кот, у которого ног изначально четыре? А речной рак, у которого восемь ног?

\itB Барон Мюнхгаузен говорит, что обошёл вокруг света (то есть побывал на всех возможных долготах Земного шара) за 40 минут. При этом известно, что он не лжёт. Как такое могло произойти?

\itC Однажды в стране Северной Болоторфии собрались построить дороги. Каждый город решили соединить дорогой с тремя другими городами, самыми близкими к нему. Может ли статься так, что найдутся города $A$ и $B$, для которых, согласно указанному правилу, город $A$ должен быть соединён с городом $B$, а город $B$ с городом $A$ — нет?
\end{enumerate}