\task{Не модельная, а модальная!}
\noindent Пусть есть событие $X$, которое может происходить или не происходить в зависимости от того, какой сегодня день. Например, событие $X = \text{«сегодня суббота»}$ случается раз в семь дней, а событие «сегодня я смотрел на часы» — каждый день. 

\def\sq{\square}
\def\td{\triangledown}
\ms Имеются два символа, $\square$ и $\triangledown$, которые рассказывают что-то о разных событиях. Так, фраза «$\sq X$» означает «начиная с сегодняшнего дня каждый день случается событие $X$». Фраза «$\td X$» означает «в будущем найдётся день, когда случится событие $X$».

\ms Два символа, упомянутых нами, можно комбинировать. Легко понять, что фраза «$\td\sq X$» значит «в будущем найдётся день, начиная с которого ежедневно будет происходить событие $X$». А фраза «$\sq\sq X$» значит то же самое, что и «$\sq X$» (убедитесь в этом сами). Фразы, значащие одно и то же, будем называть эквивалентными.

\begin{enumerate}
\itA Верно ли утверждение «$\sq\td \text{ сегодня суббота}$»? Что вообще значит фраза «$\sq\td X$»?

\itB Докажите, что фразы «$\sq\td\sq X$» и «$\td\sq X$» эквивалентны.

\itC Сколько вообще существует попарно неэквивалентных фраз вида «\_$X$» (вместо подчёркивания стоит последовательность из символов $\sq$ и $\td$)?
\end{enumerate}