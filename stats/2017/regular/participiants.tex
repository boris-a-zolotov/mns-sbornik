\task{Участники «Математики НОН-СТОП»}
\begin{itemize}

\itA Парты в одном из кабинетов, где проходит олимпиада, стоят в три колонки по шесть парт в каждой. За 20 минут до олимпиады в кабинете сидело 8 школьников. Докажите, что из кабинета пока что можно утащить две свободные парты, стоящие друг за другом. А если бы школьников было 9?

\itB Не оставляет никакого сомнения, что некоторые участники нашей олимпиады (в прошлом году, например, их было более 400) знакомы друг с другом. Докажите, что найдутся два участника, имеющие одинаковое количество знакомых среди других участников олимпиады.

\itC Докажите, что среди участников олимпиады «Математика НОН-\linebreak СТОП» найдутся трое, знакомые каждый друг с другом, или трое, не знакомые друг с другом.
\end{itemize}