\task{Игра}
\begin{itemize}

\itA 2017 единиц стоит в ряд, между ними поставлены плюсы. Двое по очереди ставят пары скобок в выражении так, что после каждого хода оно остаётся осмысленным, причём пару скобок нельзя ставить дважды на одни и те же места. Расставив 2016 пар скобок, они считают значение получившегося выражения — если оно чётно, выигрывает второй, иначе первый. Кто победит при правильной игре?

\itB Даны две кучи камней: в одной 23 камня, вторая пока пустая. Также дан мешок с 2017 камнями. Разрешены два типа ходов. Можно брать 1, 2, 3 или 4 камня и перекладывать их из первой кучи во вторую. Также можно перекладывать 1, 2, 3 или 4 камня (если они там есть) из второй кучи в первую — при этом столько же камней, сколько взято, нужно выкинуть из мешка в окно. Играют двое\scolon проигрывает тот, кто выкидывает последний камень из мешка. Кто победит при правильной игре?

\itC Дана куча, в которой $n$ камней. Играют двое\scolon за ход можно убирать из кучи 1, 2, 3, ..., 10, 11, 12 или 14 камней. Выигрывает убравший последний камень. Кто победит при правильной игре? Не забудьте, что ответ должен зависеть от $n$.
\end{itemize}