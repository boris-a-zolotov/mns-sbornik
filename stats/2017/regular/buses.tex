\task{Пути автобуса неисповедимы}
\begin{itemize}

\itA В стране Экляндии несколько городов, некоторые соединены между собой дорогами.
Между городами ходят автобусы. Известно, что дорог столько же, сколько городов.
4 марта из каждого города выехало по автобусу, а 5 марта каждый автобус приехал
в город, соединенный прямой дорогой с его родным. При каком количестве городов
возможно организовать в стране сетку дорог такую, чтобы 5 марта в каждом городе
могло оказаться ровно по одному автобусу?

\itB В стране Двуляндии названия городов начинаются исключительно на буквы П и К. При этом каждая дорога соединяет город на букву П с городом на К. Наконец, городов на П на 16 больше, чем городов на букву К. 4 марта из каждого города выехало по автобусу, а 5 марта каждый автобус приехал в город, соединенный дорогой с его родным. Докажите, что в каком-то из городов теперь более одного автобуса.

\itC На острове Квадрайлэнд четыре города. Перечислите все способы соединить
эти города дорогами так, чтобы автобусы из них могли 4 марта отправиться в путь
и 5 марта оказаться по одному в городе.
\end{itemize}