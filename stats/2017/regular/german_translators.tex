\task{Переводчики с немецкого}
\begin{itemize}

\itA Переводчику нужно перевести несколько рекламных брошюр и несколько газетных заметок. Он подсчитал, что если увеличить в некоторое целое число раз количество имеющихся у него брошюр, то их станет 116. А если увеличить в такое же число раз количество имеющихся газетных заметок, то их станет 217. Сколько же брошюр и сколько заметок предстоит перевести?

\itB Перед коллективом из трёх переводчиков стоит задача перевести 16 журналистских обзоров, 16 художественных текстов и 16 технических. Каждый из них сказал, сколько текстов какой специфики хочет перевести, причём пожелание каждого включало 16 текстов. Более того, в сумме переводчики хотят перевести ровно 16 журналистских, 16 художественных и 16 технических текстов. Докажите, что их начальник может распределить тексты для перевода так, чтобы удовлетворить пожеланиям каждого из переводчиков.

\itC Переводчик работает с текстом на 2-немецком. Текст на 2-немецком характерен тем, что значение может быть заключено не только в словах, но и в сочетаниях из двух подряд идущих слов. При этом всякое отдельное слово и всякое сочетание имеют своё значение. Сколькими способами можно разбить текст из $n$ слов на слова и сочетания?
\end{itemize}