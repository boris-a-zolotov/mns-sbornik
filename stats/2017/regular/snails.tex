\task{Гонки улиток}
\begin{itemize}

\itA Две улитки ползут снизу вверх по столбу высотой 7 метров. Первая за день проползает 5 метров, но за ночь скатывается на 4 метра. Вторая за день преодолевает 3 метра, а за ночь соскальзывает лишь на 1. Какая из улиток быстрее доберется до верха столба?

\itB Дан клетчатый лист $31 \times 31$. В центре каждой клетки сидит по улитке. В полночь каждая улитка переползает на одну из четырех клеток, соседних с ее родной. Докажите, что в какой-то из клеток теперь нет ни одной улитки.

\itC Высоко-высоко на стене сидит улитка. Прямо под ней, у подножия стены — еще одна. Верхняя улитка хочет встретиться с нижней, а нижняя — избежать встречи с верхней. Про каждую улитку известна ее максимальная скорость. Докажите, что у более быстрой улитки из этих двоих всегда есть возможность осуществить свое собственное желание.
\end{itemize}