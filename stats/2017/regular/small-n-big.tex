\task{О числах маленьких и больших}
\begin{itemize}

\itA «Произведение двух чисел – это мелко и ничтожно! — кричал Незнайка. — Вот сумма – это другое дело! Глядите, $1+5 > 1 \cdot 5$ и даже $1+1000 > 1 \cdot 1000$, вот как!» Докажите, тем не менее, что если числа $a \ge 2$,\linebreak $b > 2$, то их сумма строго меньше их произведения.

\itB Единица, стоящая первой в числе 1'000'000 уверена, что при зачёркивании первой цифры числа от него остаётся сущий пустяк. Помогите ей разобраться, существуют ли натуральные числа, которые при зачёркивании первой цифры уменьшаются ровно в (а) 57 раз (б) 58 раз.

\itC Пусть дано составное число $n \geq 4$. Докажите, что $n$ можно представить в виде произведения нескольких (более одного) натуральных чисел, так что их сумма также равна $n$.
\end{itemize}