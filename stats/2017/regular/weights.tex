\task{Взвешивания}

\begin{enumerate}
\itA Кухонные весы врут — число, которое они показывают, на какое-то фиксированное количество граммов больше, чем реально лежащая на них масса. При взвешивании картофеля получилось 1000 граммов, при взвешивании домашнего кота — 4400 граммов. При взвешивании кота вместе с картофелем — 5000 граммов. Чему же равна погрешность весов?

\itB Даны 729 монет, из них одна фальшивая — немного легче настоящих. Найдите её за 6 взвешиваний на двухчашечных весах без гирь.

\itC Весы на рынке умеют показывать суммарную массу лежащих на них предметов. Есть 15 мешков: в 14 настоящие монеты, каждая весом по 20 граммов, и в последнем фальшивые — весом по 25 граммов. Как за одно взвешивание определить, в каком из мешков лежат фальшивые монеты?
\end{enumerate}