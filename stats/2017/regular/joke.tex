\task{Шутка}

\begin{itemize}
\itA Дана 200-этажная башня. Стул с 30 ножками скидывают с ее крыши, и одновременно с этим более легкий стул совсем без ножек отправляют катиться вниз по лестнице внутри башни. Может ли безногий стул достигнуть земли быстрее, чем летящий?

\itB Выписка из дневника автора задач олимпиады «Математика НОН-СТОП»:

\begin{quote}
\itshape Так, вчера я придумал всего четыре задачи... Мне снилось, что сегодня я придумаю в полтора раза больше задач, чем в сумме за день, когда сегодня останется вчера, и за день, для которого сегодня должно было наступить завтра... Однако в день, который будет вчера для завтра и был завтра для вчера, я придумал в три раза больше задач, чем за послезавтрашнее вчера... Что за сны-то такие странные в последнее время?..
\end{quote}

\noindent Если верить сну, сколько задач должен был придумать автор в день, когда он оставил эту заметку?

\itC Автомобиль выехал из Петербурга в Пекин и сломался через 80 километров. На исправление неполадок ушло, правда, всего две минуты. Однако, проехав еще 40 километров, автомобиль вновь сломался, но вновь был отремонтирован за две минуты. Далее перед каждой следующей поломкой автомобиль проезжал вдвое меньше, чем перед предыдущей, но приводился в рабочее состояние за не- изменные две минуты. Доедет ли он в итоге до Пекина?
\end{itemize}