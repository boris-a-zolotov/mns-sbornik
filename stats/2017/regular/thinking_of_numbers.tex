\task{Загадывание чисел}
\begin{itemize}

\itA Ваня и Даня загадали по числу. Сложив эти два числа, мальчики выяснили, что их сумма делится на одно из них. Чему равен наибольший общий делитель чисел, загаданных мальчиками?

\itB Галя и Валя загадали по числу. Оказалось, что загаданные девочками числа взаимно просты. Могут ли остатки от деления этих чисел на 17 оказаться не взаимно простыми? С другой стороны, верно ли, что если остатки $a$ и $b$ от деления на любое число взаимно просты, то и $a$ взаимно просто с $b$?

\itC Болек загадывает число. Лёлек просит его прибавить к загаданному числу сначала 3, потом 4, потом 5, потом 6, и наконец перемножить полученные четыре результата. У Болека получилось 288. Помогите Лёлеку найти загаданное число!
\end{itemize}