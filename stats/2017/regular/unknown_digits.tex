\task{Неизвестные цифры}
\begin{itemize}

\itA Имеет ли данный ребус решение --- то есть, можно ли сопоставить разным буквам разные цифры так, чтобы равенство стало верным:

\begin{center}
	\texttt{М$\cdot$И$\cdot$З$\cdot$А$\cdot$Н$\cdot$Т$\cdot$Р$\cdot$О$\cdot$П  = 
	ХРОМОТА}\,?
\end{center}

\itB Решите ребус (то есть, сопоставьте разным буквам разные цифры, а одинаковым — одинаковые так, чтобы равенство стало верным): 

\begin{center}
	{\texttt{КРЕМ + КРЕМ = ЖЕЛЕ}}\scolon\quad известно, что $\text{\texttt{Л}}=9$.
\end{center}

\itC Учитель написал на доске 10 последовательных чисел. Шаловливый Стёпа, уходя после уроков домой, стёр одно — и тут же забыл, какое. Он помнил только, что сумма оставшихся на доске чисел равна 2017. Какое же конкретно число он стёр?
\end{itemize}