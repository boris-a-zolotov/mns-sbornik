\task{Велопоход}
\begin{itemize}

\itA Девочка въезжает в горку длиной 400 метров со скоростью 10 километров в час. Как долго она будет это делать?

\itB Начинающая Полина едет на велосипеде без остановок со скоростью \SI{15}{\text{км}/\text{ч}}, а опытный Дмитрий Григорьевич — со скоростью\linebreak \SI{34}{\text{км}/\text{ч}}, но остановки на отдых отнимают у него столько же времени, сколько он находится в движении. Кто же в итоге быстрее?

\itC Подъём в горку и спуск с неё имеют одинаковую длину. Степан на гоночном велосипеде въезжает в горку со скоростью \SI{10}{\text{км}/\text{ч}}, а спускается со скоростью \SI{40}{\text{км}/\text{ч}}. А Пётр на тракторе едет с постоянной скоростью \SI{17}{\text{км}/\text{ч}}. Кто из них быстрее преодолеет подъём и спуск?
\end{itemize}