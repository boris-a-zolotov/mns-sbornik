\task{Средства передвижения}
\begin{itemize}

\itA Шины на задних колёсах грузовика изнашиваются после $25 000$ км. пробега, на передних -- после $15 000$ км. Сколько километров может пройти грузовик без замены шин, если в нужный момент поменять местами передние и задние шины? Когда нужно провести замену шин?

\itB Пройдя треть длины моста, Андрей заметил троллейбус, приближающийся к мосту со скоростью $45$ км/ч. Если Андрей побежит назад, то встретится с троллейбусом в начале моста\scolon если вперед, то троллейбус нагонит его в конце моста. С какой скоростью бегает Андрей?

\itC Паша выписал в ряд номера шести трамваев, проехавших мимо него. Известно, что каждый номер, начиная с третьего, равен сумме двух предыдущих, а сумма всех выписанных номеров равна 8032. Установите номер пятого трамвая.
\end{itemize}