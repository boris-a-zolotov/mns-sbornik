\task{Все числа состоят из цифр}
\begin{itemize}

\itA Существует ли такое двузначное число, что если поменять в нём цифры местами, оно станет в три раза больше?

\itB Илья и Алексей разгадывают числовой шифр $XYZ$ из трёх цифр. Им известно, что искомое число делится на 9 и не делится на 10. Кроме того, первые две цифры образуют двузначное число $XY$, которое является квадратом некоторого натурального числа, а две последние цифры образуют двузначное число $YZ$, которое меньше 40. Помогите ребятам разгадать шифр.

\itC Натуральное число $n$ имеет $61$ разряд и состоит из двоек, троек и четверок. При этом двоек на $19$ больше, чем четверок. Найти остаток от деления числа $n$ на 9.
\end{itemize}