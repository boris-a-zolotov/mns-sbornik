\task{Простые, но такие сложные}
\begin{itemize}

\itA Натуральное число называется простым, если оно нацело делится только на себя и на единицу. Найдите все такие простые числа $p$, что числа $p+2$ и $p+4$ тоже простые.

\itB Натуральное число $n$ является произведением двух простых чисел. Каждое из этих простых чисел увеличили на 1. Произведение полученных чисел оказалось на 100 больше, чем $n$. Чему равно число $n$? Найдите все возможные варианты и объясните, почему других нет.

\itC В ряд стоят 50 выключателей. Мимо них проходят 50 электриков~— $k$-ый из них переключает каждый $k$-ый выключатель (включает, если он был выключен, и наоборот). Например, седьмой электрик переключит фонари под номерами 7, 14, 21, 28 и так далее. Какие фонари останутся включенными после прохода электриков?
\end{itemize}