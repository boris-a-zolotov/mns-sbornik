\task{И пусть Бетховен услышит}

\noindent Девочка Лина играет на круговом фортепиано аналог «Лунной сонаты» собственного сочинения. На таком фортепиано клавиши расположены в виде кольца, а исполнитель должен предварительно залезть внутрь этой конструкции. Таким образом, если идти слева направо, после всех 88 клавиш ноты начинаются с начала.
\begin{itemize}

\itA Первую часть сонаты Лина начинает с клавиши под номером один. Сначала она прыгает на один шаг вправо. Затем на две клавиши влево. Потом на три клавиши вправо, четыре клавиши влево, и так далее. На каком шаге Лина первый раз нажмёт на клавишу под номером 45?

\itB Вторую часть сонаты Лина подпевает: ЛЯ, ЛЮ-ЛЯ, ЛЮ-ЛЮ-ЛЮ-ЛЯ, ЛЮ-ЛЮ-ЛЮ-ЛЮ-ЛЮ-ЛЮ-ЛЮ-ЛЮ-ЛЯ,... (перед каждой букой Л добавляется ЛЮ). На каждое ЛЮ или ЛЯ она, начиная с первой клавиши, идёт слева направо и нажимает по одной клавише на фортепиано. Когда первый раз «ЛЯ» Лины будет пропето одновременно\linebreak с нажатием клавиши под номером 48?

\itC Третью часть сонаты Лина играет, нажимая сначала на первую клавишу, потом прыгает на одну клавишу вправо, потом ещё на две клавиши, ещё на три, $\ldots$, ещё на 100. После этого она повторяет такую мелодию ещё 1935 раз. На какую по счёту клавишу она нажмёт последней?
\end{itemize}