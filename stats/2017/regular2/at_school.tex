\task{Необычные школьные будни}
\begin{enumerate}

\itA Школьный уборщик нарисовал тряпкой на полу 5 чисел. Сложив их по два, первоклассники получили $0, 2, 4, 4, 6, 8, 9, 11, 13, 15$. Найдите эти числа.

\itB На доске подряд записаны числа $1,2,3,\dots, 2014, 2015, 2016, 2017.$ Каких цифр при записи этих чисел использовано больше: двоек или единиц? На сколько?

\itC На городскую олимпиаду пришло 859 человек. На всех этажах школы, в которой проводится олимпиада, поровну кабинетов (причём хотя бы по два), а во всех классах поровну двухместных парт (хотя бы по две). На третьем этаже забили тревогу — одному участнику не хватило места, потому что все парты в здании были заняты. Проблему разрешили, посадив участника писать олимпиаду на личный диван директора. Найдите сумму количеств этажей, кабинетов и парт в здании.
\end{enumerate}