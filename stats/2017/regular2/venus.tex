\task{Неземное стихосложение}
\begin{itemize}

\itA Известный венерианский поэт несколько лет назад написал знаменитое незамысловатое стихотворение, начальные строчки которого мы приводим: \\
\begin{tabular}{p{1.7cm}l}
& Два два. \\
& Три два. \\
& Два два два. \\
& Три три. \\
& Пять два. \\
& Три два два.
\end{tabular} \\
Продолжите его, напишите последующие три строчки.

\itB Венерианскому поэту на День рождения подарили большой круглый торт, и он прямым разрезом поделил его пополам. Придумайте форму блюдца такую, что на одно блюдце этой формы нельзя положить полторта, но на два одинаковых блюдца такой формы можно положить целый торт.

\itC В Венерианском литературном обществе состоит 2017 поэтов. Докажите, что среди них найдутся трое, знакомые каждый друг с другом, или трое, не знакомые друг с другом.
\end{itemize}