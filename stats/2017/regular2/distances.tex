\task{О, как мы далеки!}
\begin{itemize}
\def\inkm#1{\SI{#1}{\text{км}}}

\itA На прямой дороге расположены четыре остановки: $A$, $B$, $C$, $D$ (не обязательно в таком порядке). Известно, что расстояние между ос- тановками $A$ и $D$ равно \inkm{1}, между $B$ и $C$ — \inkm{2}, между $B$ и $D$ — \inkm{3}, между $A$ и $B$ — \inkm{4}, а между $C$ и $D$ — \inkm{5}. Чему равно расстояние между остановками $A$ и $C$?

\itB Вдоль прямой аллеи растут четыре дерева. Расстояния между соседними равны 63, 14 и 84 метра соответственно. Сколько деревьев надо ещё посадить, чтобы расстояние между любыми двумя соседними деревьями было одинаковым?

\itC Можно ли на прямой отметить точки $A$, $B$, $C$, $D$ и $E$ так, чтобы расстояния между ними оказались равны: $AB=6$, $BC=7$, $CD=10$, $DE=9$, $AE=12$? Если можно, то покажите как, если нет — объясните, почему.
\end{itemize}