\section{О, как мы далеки!}
\begin{itemize}

\itA На прямой дороге расположены четыре остановки: $A$, $B$, $C$, $D$ (не обязательно в таком порядке). Известно, что расстояние между остановками $A$ и $D$ равно 1 км, между $B$ и $C$ — 2 км, между $B$ и $D$ — 3 км, между $A$ и $B$ — 4 км, а между $C$ и $D$ — 5 км. Чему равно расстояние между остановками $A$ и $C$.

\itB Вдоль прямой аллеи растут четыре дерева. Расстояния между соседними равны 63, 14 и 84 метра соответственно. Сколько деревьев надо ещё посадить, чтобы расстояние между любыми двумя соседними деревьями было одинаковым?

\itC Можно ли на прямой отметить точки $A$, $B$, $C$, $D$ и $E$ так, чтобы расстояния между ними оказались равны: $AB=6$, $BC=7$, $CD=10$, $DE=9$, $AE=12$? Если можно, то покажите как, если нет — объясните, почему.
\end{itemize}