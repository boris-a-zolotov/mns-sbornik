\section{Многонациональные захватчики}
\begin{itemize}

\itA Армии девяти государств вторглись на остров, имеющий форму таблицы $5 \times 5$. Каждая из армий хочет захватить себе по ячейке на этом острове так, чтобы любая из незахваченных ячеек имела бы общую сторону ровно с одной захваченной. Помогите им это сделать.

\itB Армии ста государств вторглись на остров, имеющий форму таблицы $100 \times 100$. Они хотят поделить его между собой так, чтобы клетки в каждой строке и в каждом столбце все принадлежали разным государствам. Потом каждое государство захватило себе по одной клетке. Всегда ли можно поделить между государствами остальные клетки так, чтобы не нарушить поставленное условие?

\itC Армия одного государства вторглась на остров, заселённый аборигенами. Известно, что остров имеет форму таблицы $(2k+1) \times (2k+1)$. Может ли эта армия захватить некоторые клетки острова таким образом, чтобы каждая клетка имела ровно две захваченных, соседних с ней по стороне?
\end{itemize}