\section{Порезать торт на День рождения}
\begin{itemize}

\itA Девочке Глаше на День рождения подарили большой круглый торт. Может ли её непоседливый брат Гоша сделать в нём три непересекающихся прямых разреза так, чтобы нельзя было провести ещё трёх разрезов, которые вместе с исходными образовывали бы замкнутую несамопересекающуюся шестизвенную ломаную?

\itB Девочке Зине на День рождения тоже подарили большой круглый торт, а она прямым разрезом поделила его пополам. Придумайте форму блюдца такую, что на одно блюдце этой формы нельзя положить полторта, но на два одинаковых блюдца такой формы можно положить целый торт.

\itC Мальчику Феде на День рождения подарили торт в форме большого куба. Его верх и бока равномерно политы шоколадной глазурью с кокосовой крошкой. Помогите Феде разделить торт так, чтобы ему и четырём его друзьям досталось поровну объёма торта и поровну глазури.
\end{itemize}