\secklas{8}

\taskno{1}

\begin{itemize}

	\itA Мальчик Дима утверждает, что он может выписать 3 простых числа таких, что разность любых двух из них (из большего вычитают меньшее) — тоже простое число. Прав ли Дима? Может ли он выписать четыре таких простых числа?
	
	\itr Заметим, что среди выписанных простых чисел не может быть больше одного четного, потому что единственное четное простое число — 2.
	
	Аналогично, не может быть больше двух нечетных: если их хотя бы 3, то разность между наибольшим и наименьшим — четное число не меньше 4, то есть, является составной.

	Таким образом, мы получили верхнюю оценку — $1+2=3$ — на количество простых чисел, которые выписал Дима. Поэтому четырех чисел он выписать не мог.
	
	Выписать три числа просто — 2, 5, 7.

	\itB Найдите все натуральные четырехзначные числа, у которых сумма первых трех цифр равна 2, сумма последних трех цифр равна 5, а сумма первой и последней цифр делится на 7.
	
	\itr Сумма первой и последней цифр может быть равна 0, 7 или 14.
	
	Ноль — точно не вариант: тогда у числа имеются ведущие нули. Также сумма первой и последней цифр должна быть не больше $2+5$ (из условия) $=7$.
	
	При этом первая цифра не превосходит 2, последняя — 5. Отсюда единственный доступный вариант для числа — $2005$.
	
	\itC Найдите все натуральные числа, которые можно представить в виде произведения и суммы, пользуясь одинаковыми наборами чисел. Например,
	$$10 = 5+2+1+1+1 = 5 \cdot 2 \cdot 1 \cdot 1 \cdot 1.$$
	
	\itr Все составные числа.
	
	Если число $p$ — простое, то единственный способ представить его в виде произведения — $p \cdot 1 \cdot \ldots \cdot 1$. Сумма чисел этого набора строго больше $p$.
	
	Если число $n = q_1 \cdot q_2$, $q_1, q_2 \ge 2$ составное, то $q_1 \cdot q_2 \ge q_1 + q_2$. Тогда можно добавить несколько единиц так, чтобы $q_1 \cdot q_2 \cdot 1 \cdot \ldots \cdot 1$ было равно $q_1 + q_2 + 1 + \ldots + 1$: единицы не меняют произведения, но увеличивают сумму.

\end{itemize}

\taskno{2}

\begin{itemize}

	\itC У министра финансов в кармане лежат несколько монет. Если он наугад вытащит из кармана 3 монеты, то среди них обязательно найдется монета 1 рубль. Если же он наугад вытащит 4 монеты, то среди них обязательно найдется монета 2 рубля. Министр вытащил наугад из кармана 5 монет. Назовите эти монеты.
	
	\itr Эта задача интересна тем, что она не указывает явно, какие вообще монеты могут быть у министра в кармане — например, она не запрещает монеты достоинством 5000001 рубль, и это надо иметь в виду при решении.
	
	Заметим, однако, что среди вытащенных монет должно быть хотя бы две двухрублевых, потому что иначе (если таких монет не больше одной) возьмем все остальные монеты, их хотя бы 4, и среди них нет двухрублевых, что противоречит условию задачи.
	
	По аналогичным причинам среди вытащенных монет хотя бы три однорублевых. Получается, мы уже наверняка знаем достоинства $2+3 = 5$ — то есть, неожиданно, всех — монет.

\end{itemize}

\taskno{3}

\begin{itemize}

	\itA Чемпион тараканьих бегов Таракан Янычар объявил, что может бежать со скоростью 50\,м/мин. Он как всегда все перепутал: он считает, что в метре 60 сантиметров, а в минуте 100 секунд. С какой скоростью может бежать Янычар?
	
	\itr Будем надеяться на то, что сантиметры и секунды Таракан понимает правильно. Тогда он вводит на свою скорость две поправки: (а) скорость в м/мин увеличивается, потому что минута у Таракана более «тянутая», и за эту минуту он может успеть больше (б) скорость, опять же, увеличивается за счет того, что метры для Таракана более короткие, и он может чаще, чем это надо, прибавлять метр к расстоянию, которое пробежал.

	Обе эти поправки увеличивают скорость таракана в 100/60 раз, поэтому ответ на задачу —
	$$50\ \ :\ \ \ll\frac{100}{60}\rr^2= \frac{50 \cdot 9}{25} =
	18\,(\text{м/мин}).$$

\end{itemize}

\taskno{5}

\begin{itemize}

	\itB Сколько минут осталось до 12 часов дня, если их число в 4 раза\linebreak меньше количества минут, прошедших с 9 часов утра до события, состоявшегося 50 минут тому назад?
	
	\itr Пусть до 12 часов дня осталось $x$ минут. Тогда с 9 часов утра до недавнего события прошло $\tfrac{x}{4}$ минут. С другой стороны, этот же отрезок времени равен
	$$\ll 12 \cdot 60 - x - 50 \rr - 9 \cdot 60 \text{ минут,}$$
	потому как слагаемое в скобках — это время от начала суток до текущего момента, и из этого времени нужно вычесть 9 часов, которые не считаются. Таким образом,
	$$x = 4 \cdot \ll 12 \cdot 60 - x - 50 - 9 \cdot 60 \rr$$
	$$x = 520 - 4x$$
	$$x = 104.$$

\end{itemize}

\taskno{8}

\begin{itemize}

	\itA Я иду от дома до школы 30 минут, а мой брат 40 минут. Через сколько минут я догоню брата, если он вышел на 5 минут раньше меня?
	
	\itr Будем измерять скорость в метрах в минуту. Если $S$ метров — расстояние от дома до школы, то моя скорость — $\tfrac{S}{30}$, а скорость брата — $\tfrac{S}{40}$. Таким образом, мы решаем уравнение
	$$\frac{S}{30} \cdot t = \frac{S}{40} \cdot \ll t+5 \rr.$$

	Заметим, что $S$ не играет никакой роли в уравнении, мгновенно сокращаясь.
	$$\frac{t}{30} = \frac{t+5}{40}.$$
	
	Это уравнение уже просто решить:
	$$t = 15.$$
	
	Ответ — 15 минут. Его можно получить и по-другому: если весь путь до школы брат идет на 10 минут дольше, чем я, то половину пути — как раз на 5 минут дольше. А половина пути занимает у меня 15 минут.
	
	\itB Двое семиклассников пробежали 100 метров. Когда Андрей финишировал, Боре оставалось еще 10 метров. Следующий забег они решили провести так, что Андрей стартовал на 10 метров позади линии старта, с которой стартовал Боря. Во второй раз они бежали с той же скоростью, что и в первый. Кто пришел к финишу раньше?
	
	\itr За время, за которое Андрей смог финишировать стометровку, Боря пробежал только 90 метров — значит, скорость Андрея составляет $\tfrac{10}{9}$ скорости Бори. Во второй раз Андрею предложено пробежать 110 метров, то есть в $\tfrac{11}{10}$ раз больше, чем Боре. Отсюда его время относительно времени Бори будет составлять
	$$\frac{11}{10}\ \ :\ \ \frac{10}{9} = \frac{99}{100} < 1.$$
	
	Значит, Андрей опять финиширует быстрее.

\end{itemize}