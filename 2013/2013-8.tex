\secklas{8}

\taskno{1}

\begin{itemize}

	\itA Мальчик Дима утверждает, что он может выписать 3 простых числа таких, что разность любых двух из них (из большего вычитают меньшее) — тоже простое число. Прав ли Дима? Может ли он выписать четыре таких простых числа?
	
	\itr Заметим, что среди выписанных простых чисел не может быть больше одного чётного, потому что единственное чётное простое число — 2. При этом среди выписанных простых чисел\linebreak должна быть двойка, потому что иначе все разности между выписанными числами будут чётными, а две различные разности обязательно должны найтись — значит, одна из них будет составной.
	
	(В частности, поэтому нельзя выписать 4 простых числа, потому что среди них обязательно найдутся 3 нечётных.)
	
	Выписать три числа просто — 2, 5, 7.

	\itB Найдите все натуральные четырёхзначные числа, у которых сумма первых трёх цифр равна 2, сумма последних трёх цифр равна 5, а сумма первой и последней цифр делится на 7.
	
	\itr Сумма первой и последней цифр может быть равна 0, 7 или 14.
	
	Ноль — точно не вариант: тогда у числа имеются ведущие нули. Также сумма первой и последней цифр должна быть не больше $2+5$ (из условия) $=7$.
	
	При этом первая цифра не превосходит 2, последняя — 5. Отсюда единственный доступный вариант для числа — $2005$.
	
	\itC Найдите все натуральные числа, которые можно представить в виде произведения и суммы, пользуясь одинаковыми наборами чисел. Например,
	$$10 = 5+2+1+1+1 = 5 \cdot 2 \cdot 1 \cdot 1 \cdot 1.$$
	
	\itr Все составные числа.
	
	Если число $p$ — простое, то единственный способ представить его в виде произведения — $p \cdot 1 \cdot \ldots \cdot 1$. Сумма чисел этого набора строго больше $p$.
	
	Если число $n = q_1 \cdot q_2$, $q_1, q_2 \ge 2$ составное, то $q_1 \cdot q_2 \ge q_1 + q_2$. Тогда можно добавить несколько единиц так, чтобы $q_1 \cdot q_2 \cdot 1 \cdot \ldots \cdot 1$ было равно $q_1 + q_2 + 1 + \ldots + 1$: единицы не меняют произведения, но увеличивают сумму.

\end{itemize}

\taskno{2}

\begin{itemize}

	\itC У министра финансов в кармане лежат несколько монет. Если он наугад вытащит из кармана 3 монеты, то среди них обязательно найдётся монета 1 рубль. Если же он наугад вытащит 4 монеты, то среди них обязательно найдётся монета 2 рубля. Министр вытащил наугад из кармана 5 монет. Назовите эти монеты.
	
	\itr Эта задача интересна тем, что она не указывает явно, какие вообще монеты могут быть у министра в кармане — например, она не запрещает монеты достоинством 5000001 рубль, и это надо иметь в виду при решении.
	
	Заметим, однако, что среди вытащенных монет должно быть хотя бы две двухрублёвых, потому что иначе (если таких монет не больше одной) возьмём все остальные монеты, их хотя бы 4, и среди них нет двухрублёвых, что противоречит условию задачи.
	
	По аналогичным причинам среди вытащенных монет хотя бы три однорублёвых. Получается, мы уже наверняка знаем достоинства $2+3 = 5$ — то есть, неожиданно, всех — монет.

\end{itemize}