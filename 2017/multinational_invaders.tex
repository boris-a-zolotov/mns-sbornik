\task{Многонациональные захватчики}
\begin{itemize}

\itA Пусть девять государств захватили отмеченные на рисунке клетки:

\begin{center}
\tikz{
  \filldraw[fill=gray] (0.5,0) rectangle (1.5,0.5);
  \filldraw[fill=gray] (0,1) rectangle (0.5,2.5);
  \filldraw[fill=gray] (1.5,1) rectangle (2,2.5);
  \filldraw[fill=gray] (2,0) rectangle (2.5,0.5);
  \draw[thick] (0,0) rectangle (2.5,2.5);
  \foreach \x in {0.5,1,1.5,2} { \draw (\x,0) -- (\x,2.5); \draw (0,\x) -- (2.5,\x); }
}\end{center}

\itB Пусть 99 государств захватили себе первые 99 клеток верхнего ряда таблицы, а сотое государство~— вторую сверху клетку оставшегося столбца таблицы, где ещё не было ни одного государства.

Единственное государство, которое могло бы заселить оставшуюся клетку верхнего ряда, чтобы не нарушить условие задачи,~— сотое. Однако и оно не может там обосноваться, потому что тогда в соответствующем столбце за ним будет целых две клетки.

\itC При любом $k$ захват клеток, описанный в условии, невозможен.

\begin{center}
\tikz{
  \foreach \x/\y in {0.5/0,0/0.5,1/1.5,1.5/1} {
     \filldraw[fill=gray] (\x,\y) rectangle (\x+0.5,\y+0.5);
     \draw (\x,\y) rectangle (\x+0.5,\y+0.5);
     \draw (\x+0.55,\y+0.55) -- (\x+0.95,\y+0.95);
     \draw (\x+0.95,\y+0.55) -- (\x+0.55,\y+0.95);
  }
  \draw[thick,dashed] (2.5,2.5) -- (3,3);
  \draw[thick] (0,2.5) -- (0,0) -- (2.5,0);
  \draw[thick,dashed] (0,3) -- (0,2.5);
  \draw[thick,dashed] (2.5,0) -- (3,0);
}\end{center}

Будем смотреть на клетки , соседние с главной диагональю. Первые две из них должны быть захвачены, чтобы обеспечить двух захваченных соседей угловой клетки. Следующие две не могут быть захвачены, так как у второй клетки на диагонали уже есть два захваченных соседа. Так как размеры таблицы нечётны, получаем, что две соседних клетки противоположного угла таблицы должны быть не захвачены~— и это будет нарушать условие задачи.
\end{itemize}