\task{Многонациональные захватчики}
\begin{itemize}

\itA {\bf Здесь должна быть картинка.}

\itB Пусть 99 государств захватили себе первые 99 клеток верхнего ряда таблицы, а сотое государство~— вторую сверху клетку оставшегося столбца таблицы, где ещё не было ни одного государства.

Единственное государство, которое могло бы заселить оставшуюся клетку верхнего ряда, чтобы не нарушить условие задачи,~— сотое. Однако и оно не может там обосноваться, потому что тогда в соответствующем столбце за ним будет целых две клетки.

\itC При любом $k$ захват клеток, описанный в условии, невозможен.

{\bf Здесь должна быть картинка.}

Будем смотреть на клетки , соседние с главной диагональю. Первые две из них должны быть захвачены, чтобы обеспечить двух захваченных соседей угловой клетки. Следующие две не могут быть захвачены, так как у второй клетки на диагонали уже есть два захваченных соседа. Так как размеры таблицы нечётны, получаем, что две соседних клетки противоположного угла таблицы должны быть не захвачены~— и это будет нарушать условие задачи.
\end{itemize}