\task{Велопоход}
\begin{itemize}

\itA $$t = \frac{S}{v} = \frac{\SI{400}{\text{м}}}{\SI{10}{\text{км/ч}}}
	= \frac{\SI{0.4}{\text{км}}}{\SI{10}{\text{км/ч}}}
	= \frac{1}{25}\ \text{ч}.$$

Это, в свою очередь, равно $2.4$ минутам.

\itB Остановки занимают половину времени Дмитрия Григорьевича, поэтому его средняя скорость будет в два раза меньше его скорости в движении — и равна \SI{17}{\text{км/ч}}. Это, тем не менее, выше средней скорости Полины, которая равна \SI{15}{\text{км/ч}}. Поэтому Д.\,Г. быстрее

\itC Пусть длина подъёма в горку равна $x$ километров. Тогда время, за которое Степан преодолеет подъём и спуск, в часах равно

$$\frac{x}{10} + \frac{x}{40} = \frac{5x}{40} = \frac{x}{8}.$$

Время же, которое потратит Пётр, равно $\frac{2x}{17}$ — и нам нужно сравнить эти два числа. Посмотрим на их отношение:

$$\frac{x \cdot 17}{8 \cdot 2x} = \frac{17}{16} > 1.$$

То есть, Пётр всё-таки будет ехать дольше.

\end{itemize}