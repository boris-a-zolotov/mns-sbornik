\task{Возводим в степень}
\begin{itemize}

\itA Подойдёт, например, 423 (делится на 9), 424 (делится на 4), 425 (делится на 25).

\itB Укажите наименьшее натуральное число такое, что его половина — квадрат натурального числа, его треть~— куб натурального числа, а его пятая часть — пятая степень натурального числа.

Будем искать это число в виде $2^m3^n5^k$: по условию, эти множители должны в него входить, а лишнего нам не надо. Ясно следующее:

\begin{quote}
$m$ делится на 3 и на 5, но нечётно\scolon \\
$n$ делится на 2 и на 5, но имеет остаток 1 по модулю 3\scolon\\
$k$ делится на 2 и на 3, но имеет остаток 1 по модулю 5.
\end{quote}

Найдём наименьшие подходящие $m$, $n$ и $k$ — это 15, 10 и 6. Ответ: $2^{15} \cdot 3^{10} \cdot 5^6$.

\itC Пусть нам надо придумать цепочку длины $n$. Возьмём $n$ произвольных простых чисел $p_1 \ldots p_n$ — их квадраты являются попарно взаимно простыми.

В силу Китайской теоремы об остатках найдётся достаточно большое число $N$, сравнимое с $i$ по модулю $p_i^2$, $1 \le i \le n$. Искомой цепочкой будет $N-n \ldots N-1$.
\end{itemize}