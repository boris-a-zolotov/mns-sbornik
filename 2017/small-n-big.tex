\task{О числах маленьких и больших}
\begin{itemize}

\itA \label{small-n-big-a} Без ограничения общности будем считать, что $b \geq a \geq 2$. Тогда

\vspace{-0.4cm}
$$a+b\ \stackrel{\text{(1)}}{\leq}\ 
	2 \cdot\max(a,b)\ \stackrel{\text{(2)}}{\leq}\ 
	\min(a,b) \cdot \max(a,b)\ =\ a\cdot b.$$

\noindent Теперь, если оба числа $a$, $b$ строго больше двух, то неравенство (2) становится строгим, а если только одно — то неравенство (1) становится строгим. Что и требовалось.

\itB (а): Пусть $a$ — первая цифра числа $X$.

Чтобы найти число $X$, которое при удалении первой цифры станет в 57 раз меньше, нужно придумать такую цифру $a$, что $a \cdot 10^{\ldots} = 56 \cdot (X - a \cdot 10^{\ldots})$. Для этого, в частности, число $a00\ldots 0$ должно делиться на 56. Число 70000 отлично подойдёт. Получаем ответ:

\vspace{-0.4cm}
$$1250 \cdot 57 = 71250.$$

(б): Чтобы найти ответ в этом пункте, нужно подобрать такую цифру $a$, что $a00\ldots0$ делится на 57. Пусть такая есть: $57 \mid a \cdot 10^k$. 10 взаимно просто с 57, поэтому тогда $57 \mid a\cdot 10^{k-1}$. Продолжая уменьшать степень десятки, пользуясь этим соображением, получим $57 \mid a$. Но ненулевая цифра не может делится на 57 — получаем противоречие.

\itC Отдельно рассмотрим случай $n=4$: $4 = 2 \cdot 2 = 2 + 2$. Если же составное $n$ строго больше четырёх, что его можно представить в виде $a \cdot b$, $a \geq 2$, $b > 2$. 

Из \hyperref[small-n-big-a]{пункта A} мы знаем, что тогда $a+b < a \cdot b = n$. Тогда можно взять $n-a-b$ единиц, и получить

\vspace{-0.4cm}
$$a+b+1+\ldots+1 = a\cdot b \cdot 1 \cdot \ldots \cdot 1 = n.$$
\end{itemize}