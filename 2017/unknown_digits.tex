\task{Неизвестные цифры}
\begin{itemize}

\itA Заметим, что в слове «Мизантроп» 9 различных букв\scolon а также буква Х, встречаясь в этом ребусе, является уже десятой. При этом ни одна из уже перечисленных нами букв не может быть равна нулю — в одном случае произведение в правой части получится нулевым, в другом же в числе ХРОМОТА окажется ведущий ноль.

\itB Между младшим и предпоследним разрядами в этом примере должен был случиться перенос разряда, так как в противном случае Е$+$Е$=$*9, что невозможно из соображений чётности.

Отсюда E$+$E$+$1 должно оканчиваться на 9. Значит, Е равно 4 или 9.

\subitem Если Е равно 9, то М$+$М$=$19, что невозможно (М$\le$9).

\subitem Если Е равно 4, то М$+$М$=$14, тогда М$=$7. Тогда посмотрим на букву Р: Р$+$Р оканчивается на 4.

Тогда либо Р равно либо 2, либо 7. Последний случай невозможен, так как тогда Р совпадает с М. Значит, остаётся ребус вида $2 \cdot \text{К247} = \text{Ж494}$. На значения букв К и Ж не влияют никакие другие части выражения, поэтому можно взять произвольное К от 1 до 4 и Ж$=$$2 \cdot$К — у ребуса будет 4 решения.

\itC Пусть на доске были написаны числа $x_0 \ldots x_9$, $x_i = n+i$, $n$ — какое-то натуральное.

Пусть стёрто число $x_k$. Тогда $2017 = 9n+ 1 + \ldots + 9 - k$. 2017 имеет остаток 1 при делении на 9, значит, и левая часть тоже должна иметь остаток 1. $1 + \ldots + 9 = 45$, это делится на 9 — значит, $k$ имеет остаток 8, и поэтому равно 8.

Значит, сумма 9 последовательных натуральных чисел равна 2015, $9n = 1980$, $n=220$. Поэтому с доски стёрто число 228.
\end{itemize}