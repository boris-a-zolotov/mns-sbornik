\task{Неизвестные цифры}
\begin{itemize}

\itA Заметим, что в слове «Мизантроп» 9 различных букв; а также буква Х, встречаясь в этом ребусе, является уже десятой. При этом ни одна из уже перечисленных нами букв не может быть равна нулю — в одном случае произведение в правой части получится нулевым, в другом же в числе ХРОМОТА окажется ведущий ноль.

\itB Между младшим и предпоследним разрядами в этом примере должен был случиться перенос разряда, так как в противном случае Е$+$Е$=$*9, что невозможно из соображений четности.

Отсюда E$+$E$+$1 должно оканчиваться на 9. Значит, Е равно 4 или 9.

\subitem Если Е равно 9, то М$+$М$=$19, что невозможно (М$\le$9).

\subitem Если Е равно 4, то М$+$М$=$14, тогда М$=$7. Тогда посмотрим на букву Р: Р$+$Р оканчивается на 4.

Тогда либо Р равно либо 2, либо 7. Последний случай невозможен, так как тогда Р совпадает с М. Значит, остается ребус вида $2 \cdot \text{К247} = \text{Ж494}$.

На значения букв К и Ж не влияют никакие другие части выражения, поэтому можно подбирать их отдельно, руководствуясь лишь тем, что $2 \cdot \text{К} = \text{Ж}$.

К не может быть равно 1, потому что тогда $\text{Ж} = 2$, а цифра 2 уже занята. Равно как K не может быть равно 2 и 4: цифра 4 тоже занята. К не может быть больше или равно 5, потому что тогда случится перенос через разряд. Поэтому остается единственный ответ:

\begin{center} \begin{tabular}{cccc}
	3 & 2 & 4 & 7 \\
	3 & 2 & 4 & 7 \\ \hline
	6 & 4 & 9 & 4
\end{tabular} \end{center}

\itC Пусть на доске были написаны числа $x_0 \ldots x_9$, $x_i = n+i$, $n$ — какое-то натуральное.

Пусть стерто число $x_k$. Тогда $2017 = 10n+ 1 + \ldots + 9 - k$. 2017 имеет остаток 7 при делении на 10, значит, и левая часть тоже должна иметь остаток 7. $1 + \ldots + 9 = 45$, имеет остаток 5 — значит, $k$ имеет остаток 8, и поэтому равно 8.

Значит, сумма десяти последовательных натуральных чисел равна 2025, $10n = 1980$, $n=198$. Поэтому с доски стерто число 206.
\end{itemize}