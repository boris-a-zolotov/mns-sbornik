\task{Шутка}

\begin{itemize}
\itA Ответ «нет» в задаче–шутке смотрелся бы странно, поэтому в этом пункте подразумевается ответ «да»: из башни можно откачать воздух и сделать лестницу почти отвесной — тогда стул без ножек сможет двигаться вертикально вниз так же, как и его 30-ногий собрат, но ему не будет мешать сопротивление воздуха, наличествующее снаружи башни; в общем, он окажется быстрее.

\itB Пусть $x_A$ задач было придумано вчера, $x_B$ — сегодня и $x_C$ — завтра. Тогда из условия

$$
\begin{cases}
	x_B = 1.5(x_A+x_C) \\
	x_B = 3x_C \\
	x_A = 4
\end{cases}
$$

Осталось только решить эти уравнения. Из первых двух легко вывести, что $x_A = x_C$, тогда $x_B = 1.5 \cdot 8 = 12$.

\itC Автомобиль не сможет доехать до Пекина, так как за любое конечное время после своего старта он проезжает не более чем $80+40+20+\ldots=160$ километров.
\end{itemize}