\task{Пути автобуса неисповедимы}
\begin{itemize}

\itA Описанная в условии задачи ситуация возможна при любом количестве городов — достаточно соединить все города дорогами по кругу, «хороводом», и 4 марта отправить автобус из каждого города в следующий.

\itB Автобусов, выехавших из городов на П (и, соответственно, приехавших в города на К) больше, чем собственно городов на К. В силу принципа Дирихле, в каком-то городе на К будет больше одного автобуса.

\itC Рассмотрим какой-то из городов, назовем его $A$. Из него 4 марта выехал автобус~--- обозначим город его прибытия через $B$. Автобус из $B$, в свою очередь, мог поехать либо в $A$, либо куда-то еще.

Если он поехал обратно в $A$, то у двух других городов нет иного выхода, кроме как тоже <<обменяться>> своими автобусами. Поэтому дороги могут быть расположены, например, так, чтобы допускать разбиение городов на две пары, города в которых соединены дорогой.

Если же автобус из $B$ поехал дальше, в город $C$, то автобус из $C$ не мог отправиться ни в $A$ (тогда автобусу в четвертом городе придется сидеть на месте, что запрещено), ни куда-либо еще (и в $B$, и в $C$ уже есть прибывшие автобусы).

Таким образом, другой способ расположения дорог~--- такой, при котором есть цикл из четырех дорог, проходящий через все города.

Перечислим способы разбить города на две пары и способы провести циклы через четыре города:

\def\cities{
	\foreach \x in {-0.4,0.4} {
		\foreach \y in {-0.4,0.4} {
			\filldraw (\x, 0.075 cm + \y cm) arc (90:450:0.075); 
	}; }}

\medskip
\begin{center} \tikz{

\begin{scope}[xshift=-2.2cm]
	\cities; 
	\dwt (-0.4,0.4) -- (-0.4,-0.4); 
	\dwt (0.4,-0.4) -- (0.4,0.4); 
\end{scope}

\begin{scope}[xshift=0cm]
	\cities; 
	\dwt (-0.4,0.4) -- (0.4,0.4); 
	\dwt (0.4,-0.4) -- (-0.4,-0.4); 
\end{scope}

\begin{scope}[xshift=2.2cm]
	\cities; 
	\dwt (-0.4,0.4) -- (0.4,-0.4); 
	\dwt (-0.4,-0.4) -- (0.4,0.4); 
\end{scope}

\begin{scope}[xshift=-2.2cm,yshift=-1.5cm]
	\cities; 
	\dwt (0.4,0.4) -- (-0.4,0.4) -- (-0.4,-0.4) -- (0.4,-0.4) -- cycle; 
\end{scope}

\begin{scope}[xshift=0cm,yshift=-1.5cm]
	\cities; 
	\dwt (0.4,-0.4) -- (-0.4,0.4) -- (-0.4,-0.4) -- (0.4,0.4) -- cycle; 
\end{scope}

\begin{scope}[xshift=2.2cm,yshift=-1.5cm]
	\cities; 
	\dwt (-0.4,0.4) -- (0.4,0.4) -- (-0.4,-0.4) -- (0.4,-0.4) -- cycle; 
\end{scope}

}\end{center}

Получилось шесть способов — такие соединения дорогами нам подходят. Также подойдет любая ситуация, которая «надстроена» над перечисленными нами: то есть, взяты все дороги и добавлены какие-то еще.
\end{itemize}
