\task{Пути автобуса неисповедимы}
\begin{itemize}

\itA Описанная в условии задачи ситуация возможна при любом количестве городов — достаточно соединить все города дорогами по кругу, «хороводом», и 4 марта отправить автобус из каждого города в следующий.

\itB Автобусов, выехавших из городов на П (и, соответственно, приехавших в города на К) больше, чем собственно городов на К. В силу принципа Дирихле, в каком-то городе на К будет больше одного автобуса.

\itC Если автобусы смогли разъехаться, то либо четыре города оказались разбиты на пары так, что автобусы из городов пары поменялись местами, либо из четырёх городов собрался цикл, и каждый автобус отправился в следующий город цикла.

{\bf Здесь должна быть картинка.}

Есть три способа разбить четыре города на пары и три способа провести через них цикл длины 4 — такие соединения дорогами нам подходят. Также подойдёт любая ситуация, которая «надстроена» над перечисленными нами: то есть, взять все дороги и добавлены какие-то ещё.
\end{itemize}