\task{Загадывание чисел}
\begin{itemize}

\itA Пусть оказалось, что $a+b$ делится на $b$, где $a$ и $b$ — числа, загаданные мальчиками. Тогда

\vspace{-0.4cm}
$$a+b = k \cdot b,$$

\noindent и, соответственно

\vspace{-0.4cm}
$$a = (k-1) \cdot b.$$

Таким образом, $a$ делится на $b$ — и наибольший общий делитель этих двух чисел равен $b$.

\itB Ответ на первый вопрос — да, конечно: 3 и 20 — взаимно простые числа, а их остатки от деления на 17 совпадают и, разумеется, не взаимно просты.

Чтобы показать, что числа $a$ и $b$ из второго вопроса пункта обязаны быть взаимно простыми, рассмотрим число

\vspace{-0.4cm}
$$\max (a,b) +1.$$

\noindent Остатки при делении чисел $a$ и $b$ на него равны им самим и по условию взаимно просты — значит, $a$ и $b$ взаимно просты.

\itC Наша задача — решить уравненение

\vspace{-0.4cm}
$$(x+3)(x+4)(x+5)(x+6) = 288.$$

\noindent Рассмотрим произведения пары крайних множителей и пары средних множителей:

\vspace{-0.4cm}
$$(x^2 + 9x + 18)(x^2 + 9x + 20) = 288.$$

Иными, словами,

\vspace{-0.4cm}
$$Y(Y+2) = 288.$$

\noindent Разложим число 288 на множители: $288 = 2^5 \cdot 3^2$. Заметим, что есть ровно два способа представить 288 в виде произведения двух чисел, различающихся на 2: $16 \cdot 18$ и $(-18) \cdot (-16)$.

\ms Что же делать Лёлеку? В каждом из этих двух случаев, чтобы найти $x$, нужно решить квадратное уравнение. Это можно сделать, например, представив $-18$ в виде произведения двух чисел, различающихся на 3 — одно из них и будет $x+3$.

\ms Мы надеемся, что ни то, ни другое не представляет для Лёлека никакого труда.

\end{itemize}
