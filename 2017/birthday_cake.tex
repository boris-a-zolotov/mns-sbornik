\task{Порезать торт на День рождения}
\begin{itemize}

\itA Проведём три параллельных разреза через торт. Если они вместе с какими тремя разрезами образуют замкнутую ломаную, то есть звено, идущее от одного крайнего разреза к другому — то есть, пересекающее средний разрез. Таким образом, построить несамопересекающуюся ломаную нельзя.

\itB {\bfseries Здесь должна быть картинка.}

\itC Задача сводится к задаче разрезания квадрата на пять фигур одинаковой площади и с одинаковой длиной пересечения с внешними сторонами квадрата. Для этого разделим каждую сторону квадрата на пять равных по длине отрезков и соединим концы этих отрезков с центром квадрата. Получим 20 треугольников одинаковой площади с одинаковой длиной основания. Искомые фигуры получим, объединяя соседние треугольники по четыре штуки.

Разрезание куба строится из разрезания квадрата просто: достаточно «протащить» разрезание квадрата через куб по вертикали.
\end{itemize}