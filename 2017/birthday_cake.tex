\task{Порезать торт на День рождения}
\begin{itemize}

\itA Проведём три параллельных разреза через торт. Если они вместе с какими тремя разрезами образуют замкнутую ломаную, то есть звено, идущее от одного крайнего разреза к другому — то есть, пересекающее средний разрез. Таким образом, построить несамопересекающуюся ломаную нельзя.

\itB $\phantom{x}$
\vspace{-0.5cm}
\begin{center}\tikz{\draw (0,0) circle [radius = 1.5]; \draw (0,1.5) -- (-0.5,0) -- (0.5,0) -- (0,-1.5)}\end{center}
\vspace{0.2cm}

\itC Задача сводится к задаче разрезания квадрата на пять фигур одинаковой площади и с одинаковой длиной пересечения с внешними сторонами квадрата. Для этого разделим каждую сторону квадрата на пять равных по длине отрезков и соединим концы этих отрезков с центром квадрата. Получим 20 треугольников одинаковой площади с одинаковой длиной основания. Искомые фигуры получим, объединяя соседние треугольники по четыре штуки.

\medskip
\begin{center} \tikz{
	\draw[very thick] (0,0) -- (0,2.5) -- (2.5,2.5) -- (2.5,0) -- cycle;
	\foreach \x in {0,...,4} {
		\draw [dotted] (2.5 cm - 0.5*\x cm, 0)
			-- (0.5*\x cm, 2.5);
	};
	\foreach \x in {1,...,5} {
		\draw [dotted] (0, 2.5 cm - 0.5*\x cm)
			-- (2.5, 0.5*\x cm);
	};
	\draw[thick] (1.25,1.25) -- (2.5,2.5);
	\draw[thick] (1.25,1.25) -- (0.5,2.5);
	\draw[thick] (1.25,1.25) -- (0,1);
	\draw[thick] (1.25,1.25) -- (1,0);
	\draw[thick] (1.25,1.25) -- (2.5,0.5);
} \end{center}

Разрезание куба строится из разрезания квадрата просто: достаточно разрезать куб вертикально по всей его толщине в соответствии с рисунком выше.
\end{itemize}
