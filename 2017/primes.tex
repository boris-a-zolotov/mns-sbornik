\task{Простые, но такие сложные}
\begin{itemize}

\itA Хотя бы одно из чисел $p$, $p+2$, $p+4$ должно делиться на 3~— это можно понять, рассмотрев всевозможные остатки при делении $p$ на 3. Единственное простое число, делящееся на 3,~— это 3.

$p+4$ не может быть равно трём, потому что тогда $p=-1$~— не простое. $p+2$ не может быть равно трём, потому что тогда $p=1$~— не простое. Остаётся единственные ответ~— $p=3$, $p+2 = 5$, $p+4 = 7$. Все эти числа простые.

\itB Пусть $n = p_1 \cdot p_2$. Тогда $n+100 = (p_1 + 1)(p_2 + 1) = n + p_1 + p_2 + 1$. Таким образом, мы ищем простые числа $p_1$ и $p_2$, такие что $p_1 + p_2 = 99$. Сумма двух чисел нечётна~— значит, одно из них обязательно должно быть чётным. Отсюда единственный ответ~— $p_1 = 2$, $p_2 = 97$, так как 2 — единственное простое число.

\itC Рассмотрим выключатель под номером $k$. Какие электрики переключат его? Очевидно, что те, номера которых являются делителями числа $k$. Изначально все выключатели выключенными, поэтому включенными в конце останутся те, номера которых имеют нечётное чило делителей. Известный факт заключается в том, что этому условию удовлетворяют только квадраты натуральных чисел.

Таким образом, включенными останутся выключатели с номерами–полными квадратами.
\end{itemize}