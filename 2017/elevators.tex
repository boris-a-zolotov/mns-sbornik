\task{Пока не пришел лифтер}

\begin{itemize}

\itA Если первый общий этаж для мальчиков — 123-ий, то $\lcm(n,m) = 123$ (так как первый общий этаж как раз и имеет номер, соответствующий наименьшему общему кратному). $123 = 3 \cdot 41$, поэтому $n$ и $m$ могут быть равным 1, 123 или 3, 41.

\itB Без ограничения общности можно считать, что Витя находится на нулевом этаже, а Петя — на первом. Тогда Витин лифт перемещается только по этажам, номера которых делятся на $k+1$. Если $k=0$, Петя остается на месте, и Витя, конечно, может к нему приехать.

Если же $k>0$, то Витя не может приехать на первый этаж (1 не делится на $k+1$), поэтому если Петя просто будет оставаться на месте, он не встретится с Витей.

\itC Заметим, что номер текущего этажа, на котором находится Витя, равен сумме со слагаемыми вида $\pm m$ и $\pm n$. Любая такая сумма делится на $\gcd (n,m)$, а единственный делитель единицы — это она сама. Поэтому $\gcd(n,m) = 1$.
\end{itemize}