\task{Переводчики с немецкого}
\begin{itemize}

\itA Разложим на простые множители числа $116$ и $217$: $116 = 2^2 \cdot 29$, $217 = 7 \cdot 31$. Эти числа взаимно просты, то есть, у них единственный общий множитель — единица. Поэтому переводчику надо перевести 116 брошюр и 217 заметок.

\itB Заметим, что если текстов каждой тематики было бы по одному и каждому переводчику надо было бы перевести ровно один текст, то у начальника не возникло бы проблем. Тогда давайте сведём задачу распределения $3n$ текстов, по $n$ на тему, к задаче распределения $3(n-1)$ текстов, по $n-1$ на тему.

Среди трёх переводчиков найдётся тот, кто указал наименьшее число различных тематик текстов в своём списке желаний. Выделим ему один текст одной из его желаемых тематик. Среди оставшихся двух переводчиков есть тот, у кого тематик поменьше. Он точно хочет себе хотя бы один текст тематики, отличной от той, которую мы уже дали первому переводчику. Выделим ему текст этой тематики.

Остался третий переводчик. Если он хотел себе текст третьей тематики, которая ещё никому не выдана, всё хорошо. Если вдруг он «заказывал» только два различных вида текстов, и это те самые виды, которые уже «отданы» первому и второму переводчикам, то у кого-то из них (предположим, у второго) в списке желаний есть третья тематика. Дадим ему эту самую третью тему, а третьему переводчику — то, что раньше было у второго.

Таким образом мы успешно раздали три текста — раздавая по три текста разных тематик, дойдём до ситуации, когда текстов осталось по одному.

\itC Текст длины 1 бьётся одним способом, текст длины 2 — двумя способами. Теперь рассмотрим последнее слово в тексте из $n$ слов — оно может быть либо самостоятельным, либо частью сочетания. В первом случае нам останется побить на слова и сочетания текст длины $n-1$, во втором — текст длины $n-2$.

Таким образом, ответ на задачу для $n$ равен сумме ответов для $n-1$ и $n-2$. Этому условию и полученным нами начальным данным удовлетворяет последовательность чисел Фибоначчи. Поэтому ответ — $\mathcal{F}_n$.

\end{itemize}