\task{Разрезания и углы}
\begin{itemize}

\item[\bfseries A–B.] {\bfseries Здесь должна быть картинка}

\itC {\bfseries Здесь должна быть картинка.}

Смотреть рисунок: отмеченные углы равны в силу построения и теоремы о накрест лежащих углах. Проведём ещё несколько доволнительных построений: получившийся треугольник — прямоугольный и равнобедренный, то есть угол, равный сумме $A$ и $B$, — $45^\circ$, то есть, равен по величине углу $C$.
\end{itemize}