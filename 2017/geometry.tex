\task{Разрезания и углы}
\begin{itemize}

\itA С помощью линейки можно убедиться, что треугольник из условия задачи является равнобедренным. Давайте повернём его «правильно» и разрежем:

\begin{center} \tikz{
	\draw[very thick] (0,0) -- (-3,0) -- (0,1.5) -- (3,0) -- cycle;
	\draw[thick] (-0.5,0) -- (0,0.75) -- (-7/8,17/16) -- cycle;
	\draw[thick] (0.5,0) -- (0,0.75) -- (7/8,17/16) -- cycle;
	\draw[thick] (0,1.5) -- (0,0.75);
} \end{center}

\itB Будем пользоваться следующим соображением: если вершина треугольника лежит за границами круга, диаметром которого является противолежащая ей сторона, то угол в этой вершине острый.

\begin{center} \tikz{
	\draw[color=gray, thick, dashed]
		(2,2) arc (90:270:2) arc (0:180:1) arc (0:180:1) arc (-90:90:2);
	\draw[thick] (0,-2) -- (0.25,-1.2) -- (2,-2) -- (0.25,-1.2) -- (2,2);
		\draw[thick] (0,-2) -- (-0.25,-1.2) -- (-2,-2) -- (-0.25,-1.2) -- (-2,2);
	\draw[thick] (0,2) -- (-0.25,-1.2) -- (0.25,-1.2) -- cycle;
	\draw[very thick] (-2,-2) -- (2,-2) -- (2,2) -- (-2,2) -- cycle;
} \end{center}

\itC Смотреть рисунок: отмеченные углы равны в силу построения и теоремы о накрест лежащих углах.

\begin{center} \tikz{
	\draw[thick,double] (1.7,0) arc (0:26.565:0.45);
	\draw[thick,double] (3.75-0.55,1.25) arc (180:180+26.565:0.55);
%%%%%%%%
	\draw (0,0) -- (3.75,1.25); \draw (2.5,0) -- (3.75,1.25);
	\draw[rotate around={-18.435:(0,2.5)}]
		(0.3,2.5) -- (0.3,2.2) -- (0,2.2);
	\draw[rotate around={-18.435:(1.875,1.875)}] (1.875,1.75) -- (1.875,2);
	\begin{scope}[rotate around={90:(1.25,0)}]
		\draw[rotate around={-18.435:(1.875,1.875)}] (1.875,1.75) -- (1.875,2);
	\end{scope};
	\draw[very thick] (-1.25,-1.25) -- (3.75,1.25) -- (0,2.5) -- cycle;
%%%%%%%%
	\foreach \x in {0, 1.25, 2.5} {\draw[thick]
		(\x,0) -- ++(1.25,0) -- ++(0,1.25) -- ++(-1.25,0) -- cycle;};
	\draw[thick] (0,1.25) -- (3.75,1.25) -- (3.75,2.5) -- (0,2.5) -- cycle;
	\draw[thick] (0,2.5) -- (-1.25,2.5) -- (-1.25,-1.25) -- (0,-1.25) -- cycle;
	\draw[thick] (0,-1.25) rectangle (1.25,0);
%%%%%%%%
	\draw (0.175,-0.2) node{$A$};
	\draw (1.25+0.175,-0.2) node{$B$};
	\draw (2.5+0.175,-0.2) node{$C$};
%%%%%%%%
	\draw[thick] (0.5,0) arc (0:18.435:0.5);
	\draw[thick] (3.25,1.25) arc (180:180-18.435:0.5);
	\draw[thick] (3.4,1.25) arc (180:180+18.435:0.35);
} \end{center}

Проведём ещё несколько доволнительных построений — получив\linebreak шийся треугольник будет прямоугольным и равнобедренным, то есть угол, равный сумме $A$ и $B$, по величине равен $45^\circ$ и, соответственно, углу $C$.
\end{itemize}