\task{Все числа состоят из цифр}
\begin{itemize}

\itA Запишем условие из задачи: $\overline{xy} = 3 \cdot \overline{yx}$. Это значит то же, что

\vspace{-0.4cm}
\begin{align*}
& 10x + y = 30y + 3x\scolon \\
& 7x = 29y.
\end{align*}

И правая, и левая части равенства должны делить на 29. Это значит, что $x$ делится на 29 — единственная цифра, кратная, 29, это ноль. Разумеется, при $x=0$ решений у данной задачи нет — значит, нет и вообще.


\itB Последовательно будем интерпретировать условие задачи. То, что искомое число не делится на 10, значит, что $Z \ne 0$. То, что число $YZ$ меньше 40, значит, что $Y$ равен 0, 1, 2 или 3. Единственное двузначное число, являющееся квадратом и оканчивающееся на цифры 0–3 — это 81. Таким образом, $X=8$, $Y=1$.

Наконец, для цифры $Z$ остаётся два возможных варианта, чтобы число $XYZ$ делилось на 9 — 0 и 9. Так как мы с самого начала поняли, что $Z \ne 0$, получается $Z = 9$.

Ответ: искомое число — 819.

\itC Попробуем посчитать сумму цифр числа $n$ — по признаку делимости на 9, её остаток будет таким же, как у самого числа $n$. 19 разрядов из 61 занимают двойки — если отбросить эти 19 разрядов, двоек и четвёрок будет поровну. Пусть четвёрки в числе занимают $t$ разрядов. Тогда сумма цифр числа $n$ равна

\vspace{-0.4cm}
$$19 \cdot 2 + 4 \cdot t + 2 \cdot t + 3 \cdot (61 - 19 - 2t) =$$
$$= 38 + 6t + 3 \cdot 42 - 6t = 38 + 42 = 80.$$

Остаток при делении 80 на 9 равен 8 — значит, и число $n$ сравнимо с 8 по модулю 9.

\end{itemize}