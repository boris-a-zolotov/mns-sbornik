\task{Игра}
\begin{itemize}

\itA Это игра–шутка: значение суммы не зависит от расстановки в ней скобок — сумма в любом случае будет равна 2017, то есть нечётна. Отсюда победит первый игрок.

\itB У первого игрока есть выигрышная стратегия. Первым ходом он должен переложить 3 камня из первой кучки во вторую. Затем он должен реагировать на ходы второго игрока следующим образом:

\subitem Если второй перекладывает $x$ камней из первой кучки во вторую, то первый должен переложить $5-x$ камней также из первой кучки во вторую.

\subitem Если же второй перекладывает камни из второй кучки в первую, то первый должен вернуть эти камни обратно во вторую кучку.

Заметим, что после хода первого игрока количество камней во второй кучке всегда имеет остаток 3 от деления на 5, а после хода второго игрока количество камней во второй кучке никогда не имеет такого остатка. Это значит, что второй игрок не может переложить все камни во вторую кучку, вынудив первого сделать ход, при котором ему придётся выкидывать камни из мешка.

То есть, (а) первый всегда может сделать ход, соответствующей придуманной нами стратегии, (б) выкидыванием камней из мешка занимается исключительно второй игрок. Он и проиграет.

\itC Начнём со случая $n=13$. Если первый игрок взял $k$ камней, то второй может взять $13-k$ и победить. Если $n$ не превосходит 14 и не равно 13, то все камни может взять за один ход первый игрок.

Для $n \ge 15$ рассмотрим два случая:

\subitem $n$ делится на 13. Заметим, что после любого хода первого игрока оставшееся количество камней не будет делиться на 13. Зато второй в случае любого хода первого сможет сделать так, что после его хода количество камней, оставшихся в кучке, будет вновь делиться на 13. Для этого на взятие $k$ камней, $1 \le k \le 12$, нужно ответить взятием $13-k$ камней, на взятие 14 камней — 12 камнями, а на взятие 15 камней — 11 камнями. Ноль делится на 13 — кучка может остаться пустой только после хода второго игрока.

\subitem $n$ не делится на 13. Тогда выигрышная стратегия есть у первого игрока. Своим первым ходом он берёт от 1 до 12 камней так, чтобы осталось количество, кратное 13 — а затем играет так, как играл бы второй игрок в предыдущем пункте.

Ответ: если $n$ делится на 13, выигрывает второй игрок\scolon иначе выигрывает первый.
\end{itemize}