\task{И пусть Бетховен услышит}
\begin{itemize}

\itA Заметим, что клавиша с номером 45 находится ровно напротив клавиши с номером~1. Также заметим, что Лина нажимает симметричные клавиши, всё больше отдаляясь от первой: сначала первую справа, потом первую слева, потом вторую справа, потом вторую слева.

Таким образом каждая клавиша окажется нажатой ровно один раз, и клавиша номер 45 будет последней — то есть, нажатой на 88-ом шаге.

\itB Заметим, что номера клавиш, на которых Лина поёт «ЛЯ», — это остатки степеней двойки (начиная с числа 2) при делении на 88. То есть, мы ищем наименьшую степень двойки, имеющую вид $88k+48$. Перебором можно установить, что это 4096. Отсюда на 4095 шаге Лина споёт «ЛЯ», нажимая на 48-ю клавишу.

\itC За одну мелодию Лина охватывает $1+1+2+3+\ldots+100$ $=$ $\frac{100 \cdot 101}{2} + 1$ $=$ $5051$ клавиш. Значит, за всю игру ей будет охвачено $1935 \cdot 5051$ клавиш, и нам нужно найти остаток этого числа при делении на 88, он и даст нам номер последней нажатой клавиши. Остаток числа $1935$ равен $-1$, остаток числа $5051$ — $35$. Таким образом, последней нажатой клавишей будет $88-35=53$--я.
\end{itemize}