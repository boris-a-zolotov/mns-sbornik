\task{Примечательный учебный день}
\begin{itemize}

\itA К высоте $m$ метров дерево подходит, имея $(m-1)!$ ветвей. Соответственно, ответ на задачу — $11!$ веток.

\itB Очевидно, что больше 13 рассадок не бывает: мальчик каждый день обязан сидеть с новой девочкой. 13 же рассадок реализовать просто: нужно взять какую-то рассадку, и каждый день сдвигать девочек относительно мальчиков «по кругу».

\itC Давайте решим задачу в общем случае: есть класс из $4k+2$ человек — придумать $4k+1$ способов рассадить их за парты так, чтобы одна пара не появлялась в двух разных рассадках (больше нельзя по очевидной причине — каждый человек может сидеть не более чем с $4k+1$ другими).

Будем изображать $i$--ую рассадку, ставя число $i$ в клетки таблицы $(4k+2) \times (4k+2)$, из которой выкинута центральная диагональ. Наша задача тогда — расставить числа от 1 до $4k+1$ в клетки таблицы, так чтобы (а) каждое число было написано ровно $2k$ раз (б) встречалось в каждом столбце и каждой строке ровно по одному разу (в) его вхождения в таблицу были бы симметричны относительно центральной диагонали. Построим расстановку.

В первую строку таблицы впишем числа от 1 до $4k+1$ справа налево, а в первый столбец — снизу вверх. Заполняя $i$-ую строку, $1<i<4k-1$, поступим так: зарезервируем самую правую клетку строки, не будем её трогать\scolon в остальные клетки впишем числа от 1 до $4k+1$, сдвинув их на одну клетку влево относительно предыдущей строки. После этого в самую правую клетку запишем число, которое должно было стоять на центральной диагонали. Последнюю строку получим отражением относительно центральной диагонали уже сформированного последнего столбца.

Приведём пример такой таблицы для $k=2$ (в задаче было $k=6$):

\begin{center}
\begin{tabular}{|cccccccccc|}

\hline
  & 9 & 8 & 7 & 6 & 5 & 4 & 3 & 2 & 1 \\
9 &   & 7 & 6 & 5 & 4 & 3 & 2 & 1 & 8 \\
8 & 7 &   & 5 & 4 & 3 & 2 & 1 & 9 & 6 \\
7 & 6 & 5 &   & 3 & 2 & 1 & 9 & 8 & 4 \\
6 & 5 & 4 & 3 &   & 1 & 9 & 8 & 7 & 2 \\
5 & 4 & 3 & 2 & 1 &   & 8 & 7 & 6 & 9 \\
4 & 3 & 2 & 1 & 9 & 8 &   & 6 & 5 & 7 \\
3 & 2 & 1 & 9 & 8 & 7 & 6 &   & 4 & 5 \\
2 & 1 & 9 & 8 & 7 & 6 & 5 & 4 &   & 3 \\
1 & 8 & 6 & 4 & 2 & 9 & 7 & 5 & 3 &  \\ \hline

\end{tabular}
\end{center}

\noindent Несложно убедиться в том, что она обладает нужными нам свойствами.

\end{itemize}
