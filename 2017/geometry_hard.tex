\task{На плоскости}
\begin{itemize}

\itA Единственный способ расположить квадраты согласно условию — вдоль двух сторон квадрата $18 \times 18$ уложить квадратики $1 \times 1$, а оставшееся место занять квадратом $17 \times 17$. Его площадь будет равна 289.

\itB 12 заборов строятся в каком-то порядке, один за другим. Ни один, ни два забора ничего не отгораживают, как их ни поставь. Зато третий забор может, пересекая первый и второй, огородить одну область. И вообще — $n$--ый забор, пересекая все предыдущие, может отгородить $n-2$ новых области. Таким образом, ответ на эту задачу — $1+\ldots+10=55$. Построить пример расстановки заборов, отгораживающей именно это количество областей, несложно.

\itC Возьмём самую длинную сторону четырёхугольника $ABCD$ — пусть это сторона $AB$. Один из углов, прилежащих к ней, должен быть острым (не умаляя общности — угол $DAB$), иначе сторона $CD$ будет длиннее $AB$. Рассмотрим сторону $DA$ и вершину $D$.

Высота $DH$ из точки $D$ на стоону $AB$ упадёт именно что на сторону $AB$, а не на её продолжение, так как иначе

$$|AD|^2 \stackbin[\text{Th.\,Пифагора}]{ }{=} |AH|^2 + |DH|^2
\stackbin[\begin{minipage}{2cm}\scriptsize $AH$ — продол-\\ жение $AB$\end{minipage}]{ }{>}
|AB|^2 + |DH|^2 > |AB|^2.$$

Откуда $AD$ длиннее $AB$, что противоречит изначальному выбору стороны $AB$.
\end{itemize}