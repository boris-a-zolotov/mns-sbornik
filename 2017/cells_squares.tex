\task{Вовочка и клетчатая тетрадь}
\begin{itemize}

\itA У Вовочки есть клетчатая тетрадь (клетки — одинаковые, квадратные), линейка без делений и карандаш. Площадь каждой клетки в тетради~— $100\,\text{мм}^2$. Как Вовочке имеющимися средствами построить квадрат площадью $1000\,\text{мм}^2$?

Заметим, что $10^2 + 30^2 = 1000$. То есть, прямоугольный треугольник с катетами $\SI{10}{\text{мм}}$, $\SI{30}{\text{мм}}$ (который легко нарисовать по клеткам) имеет гипотенузу $\sqrt{1000}\,\text{мм}^2$.

Расположив четыре таких треугольника, как показано на рисунке, получим квадрат со стороной $\sqrt{1000}\,\text{мм}^2$, площадь которого равна \SI{1000}{\text{мм}^2}.

\itB {\bfseries Здесь должна быть картинка}

Смотреть рисунок.

\itC Аналогично пункту B данной задачи мы умеем делить на произвольное количество частей любой отрезок, начало и конец которого — узлы сетки. Также $80 \cdot 25 = 2000$ — поэтому, если мы научимся строить треугольник площадью $\sqrt{2000}\,\text{мм}^2$, мы сможем поделить каждую из его сторон на пять частей и получить 25 квадратов нужной нам площади в \SI{80}{\text{мм}^2}.

Наконец, $2000 = 20^2 + 40^2$\scolon далее аналогично пункту A.
\end{itemize}