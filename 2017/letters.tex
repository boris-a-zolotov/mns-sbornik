\task{Буквы на белом листе}
\begin{itemize}

\def\ltr#1{`#1'}

\itA \ 

\begin{center}\begin{tabular}{lllllll}
	Б $\longrightarrow$ В & \quad & З $\longrightarrow$ В & & О $\longrightarrow$ Ю & & Ш $\longrightarrow$ Щ \\
	Г $\longrightarrow$ Б & & И $\longrightarrow$ Й & \quad & Р $\longrightarrow$ В & & Ь $\longrightarrow$ Б \\
	Г $\longrightarrow$ П & & К $\longrightarrow$ Ж & & С $\longrightarrow$ О & \quad & Ь $\longrightarrow$ Ы \\
	Г $\longrightarrow$ Т & & Л $\longrightarrow$ Д & & Ц $\longrightarrow$ Щ & & Ь $\longrightarrow$ Ъ \\
	Е $\longrightarrow$ В & & У $\longrightarrow$ Х
\end{tabular}\end{center}

Остальные буквы (в их «типографском» начертании) ни во что превратить нельзя. Однако эта задача оставляет большую свободу трактовок, поэтому оценивалась в пользу участника.

\itB Если лист бесконечный, то это буквы \ltr В и \ltr Ф, имеющие в своем составе два «кольца». Если же мы рассматриваем обычный лист бумаги A4, то на нем можно написать букву \ltr Ж, распространив ее до краев листа — и она поделит его на 6 областей.

\itC От двух областей (когда они написаны одна поверх другой) — до 9, когда они пересекаются в двух точках и касаются краев листа.
\end{itemize}