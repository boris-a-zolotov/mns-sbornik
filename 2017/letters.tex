\task{Буквы на белом листе}
\begin{itemize}

\def\ltr#1{`#1'}

\itA \ \\

\begin{center}\begin{tabular}{lllllll}
	Б → В & \quad & Е → В & & О → Ю & & Ш → Щ \\
	Г → Б & & И → Й & \quad & Р → В & & Ь → Б \\
	Г → П & & К → Ж & & С → О & \quad & Ь → Ы \\
	Г → Т & & Л → Д & & Ц → Щ & & Ь → Ъ
\end{tabular}\end{center}

Остальные буквы (в их «типографском» начертании) ни во что превратить нельзя. Однако эта задача оставляет большую свободу трактовок, поэтому оценивалась в пользу участника.

\itB Если лист бесконечный, то это буквы \ltr В и \ltr Ф, имеющие в своём составе два «кольца». Если же мы рассматриваем обычный лист бумаги A4, то на нём можно написать букву \ltr Ж, распространив её до краёв листа — и она поделит его на 6 областей.

\itC От двух областей (когда они написаны одна поверх другой) — до 9, когда они пересекаются в двух точках и касаются краёв листа.
\end{itemize}