\task{Участники «Математики НОН-СТОП»}
\begin{itemize}

\itA Побьём парты на пары стоящих друг за другом, по три пары в ряду. Получится девять пар, а школьников пока всего восемь. Значит, одна пара парт полностью свободна, и её можно утащить.

Если же школьников 9, то посадим по школьнику за 1, 3 и 5 парты каждого ряда — и ничего нельзя будет унести.

\itB Пусть участников всего $N$. Если среди участников есть один, не знакомый ни с кем, то не может быть участника, знакомого со всеми. Если же есть участник, который со всеми знаком, то каждый знаком хоть с кем-то.

Таким образом, либо все участники знакомы не более чем с $(N-2)$ людьми каждый, либо все знакомы не менее чем с одним, но не более чем с $N-1$ людьми каждый. В любом случае на $N$ участников получается $(N-1)$ вариантов, поэтому найдутся двое с одинаковым числом знакомых.

\itC Возьмём шесть участников, нам хватит. Будем соединять красной линией знакомых, а синей линией — незнакомых. Все участники окажутся попарно соединены.

Из каждого участника выходит по пять линий, значит как минимум три из них имеют один цвет. Пусть из данного участника выходит три красных линии — посмотрим на людей, в которых они приходят. Если между ними есть хоть одна красная линия, получается красный треугольник с участником, выбранным нами изначально. Если же между ними все линии синие, то это даёт нам синий треугольник, то есть они попарно незнакомы. Что и требовалось.

\end{itemize}
