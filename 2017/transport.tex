\task{Средства передвижения}
\begin{itemize}

\itA Очевидно, что грузовик проедет на данном наборе шин наибольшее расстояние, если все шины износятся одновременно: в противном случае на какх-то шинах останется неиспользованный ресурс, который мог бы превратиться в преодолённое расстояние.

Будем рассматривать {\itshape износ} шины~— число, линейно растущее с пройденным расстоянием, и обращающееся в единицу, когда шина достигает своего предела.

Один километр для передней шины увеличивает её износ на $\tfrac{1}{15000}$, для задней шины~— на $\tfrac{1}{25000}$. Пусть до замены грузовик проехал $S_1$ километров, а после~— $S_2 километров$. Тогда условие о том, что две пары шин износились одновременно, превратится в

$$
\begin{cases}
	\frac{1}{25000}S_1 + \frac{1}{15000}S_2 = 1\scolon \\
	\frac{1}{15000}S_1 + \frac{1}{25000}S_2 = 1.
\end{cases}
$$

Эти равенства можно преобразовать в

$$
\begin{cases}
	3S_1 + 5S_2 = 75000\scolon \\
	5S_1 + 3S_2 = 75000.
\end{cases}
$$

Получаем $S_1 = S_2 = 9375$, и грузовик сможет проехать 18750 километров.


\itB Если Андрей побежит вперёд, ему придётся преодолеть на треть длины моста большее расстояние, чем если он побежит назад. Соответственно, бежать ему придётся больше на время, требуемое для преодоления трети моста. Из условия задачи, за это время троллейбус должен преодолеть весь мост со скоростью $\SI{45}{\text{км}/\text{ч}}$. Значит, скорость бега Андрея~— $\SI{15}{\text{км}/\text{ч}}$.

\itC Паша выписал в ряд номера шести трамваев, проехавших мимо него. Известно, что каждый номер, начиная с третьего, равен сумме двух предыдущих, а сумма всех выписанных номеров равна 8032. Установите номер пятого трамвая.

Пусть номер первого трамвая — $A$, а второго — $B$. Тогда можно выписать номера остальных трамваев:

$$A+B,\ A+2B,\ 2A+3B,\ 3A+5B.$$

Сложив эти номера, получаем $8A + 12B = 8032$. Нам нужно найти номер пятого трамвая, равный $2A+3B$ — но это ровно четверть от суммы всех номеров!

Отсюда ответ — $\tfrac{8032}{4} = 2008.$
\end{itemize}