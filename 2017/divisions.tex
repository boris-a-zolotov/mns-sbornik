\task{Поделим – посмотрим}
\begin{itemize}

\itA Первые два треугольника, пересекаясь, образуют 6 областей: внешность, два «края» каждого из треугольников и их пересечение. Заметим теперь, что для замкнутой линии количество областей, которые она добавляет к картинке, равно количеству её пересечений с другими, уже имеющимися, линиями. Третий треугольник пересечёт не более восьми линий — он может максимум дважды «входить» и «выходить» из имеющихся двух треугольников. Четвёртый треугольник добавит максимум 12 областей, по тем же причинам.

Таким образом, ответ — $6+8+12=26$ областей, пример легко построить.

\itB Прямая может «входить» в семиугольник и «выходить» из него. При этом изначально она находится снаружи и в конце должна оказаться там же. Каждую сторону семиугольника прямая пересекает не более одного раза, а количество пересечений должно быть чётным. Значит, прямая пересекает максимум шесть сторон, «проходя» через семиугольник трижды.

Получается, она делит семиугольник на максимум на четыре части: до первого пересечения, между первым и вторым, между вторым и третьим, после третьего пересечения.

\itC Два одинаково ориентированных квадрата пересекаются максимум в двух точках (если не совпадают). Это значит, что $k$--ый нарисованный квадрат добавляет на картинку не более $2(k-1)$ новых областей. Отсюда ответ на задачу — $2+2+4+\ldots+2\cdot 14$ $=$ $2+2\cdot105$ $=$ $312$.

Изобразить 15 попарно пересекающихся квадратов несложно — достаточно взять один и 14 раз немного сдвинуть его по диагонали.
\end{itemize}