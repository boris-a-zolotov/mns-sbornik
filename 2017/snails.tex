\task{Гонки улиток}
\begin{itemize}

\itA Улитки доползут до верха одновременно — каждая за три дня.

\itB Покрасим клетки листа в белый и чёрный, как на шахматной доске. Чёрных и белых клеток будет разное количество (всё-таки площадь листа нечётна), и при этом улитка переползает с белой клетки на чёрную, а с чёрной — на белую. Поэтому улиткам, стартовавшим в клетках цвета, которого больше, не хватит клеток цвета, которого меньше. 

\itC Пусть более быстрая улитка — верхняя. Тогда план её действий таков: спуститься вертикально вниз в точку, где сидела другая улитка, а затем догнать её по её же пути.

Пусть более быстрая улитка — нижняя. План её действий — поползти перпендикулярно от стены. Кратчайший путь от начального положения верхней улитки до точки, где находится нижняя улитка, всегда будет длиннее расстояния, пройденного нижней улиткой — поэтому более медленная верхняя не сможет её догнать.
\end{itemize}