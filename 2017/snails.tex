\task{Гонки улиток}
\begin{itemize}

\itA Улитки доползут до верха одновременно — каждая за три дня.

\itB Покрасим клетки листа в белый и черный, как на шахматной доске. Черных и белых клеток будет разное количество (все-таки площадь листа нечетна), и при этом улитка переползает с белой клетки на черную, а с черной — на белую. Поэтому улиткам, стартовавшим в клетках цвета, которого больше, не хватит клеток цвета, которого меньше. 

\itC Пусть более быстрая улитка — верхняя. Тогда план ее действий таков: спуститься вертикально вниз в точку, где сидела другая улитка, а затем догнать ее по ее же пути.

Пусть более быстрая улитка — нижняя. План ее действий — поползти перпендикулярно от стены. Кратчайший путь от начального положения верхней улитки до точки, где находится нижняя улитка, всегда будет длиннее расстояния, пройденного нижней улиткой — поэтому более медленная верхняя не сможет ее догнать.
\end{itemize}