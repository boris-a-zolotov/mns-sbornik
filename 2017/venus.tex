\task{Неземное стихосложение}
\begin{itemize}

\itA
Поэт в своём стихотворении занимается расложением последовательных чисел в сумму простых, меньших данного числа (например, число 4 он не использовал, придумывая разложение для шести). Записываются простые числа в порядке убывания. Поэтому следующие три строчки будут иметь вид: \\
\begin{tabular}{p{1.7cm}l}
& Два два два два. \\
& Три три два два. \\
& Пять три.
\end{tabular}


\itB $\phantom{x}$
\vspace{-0.5cm}
\begin{center}\tikz{\draw (0,0) circle [radius = 1.5]; \draw (0,1.5) -- (-0.5,0) -- (0.5,0) -- (0,-1.5)}\end{center}
\vspace{0.2cm}

\itC Будем соединять знакомых поэтов белой ниткой, а незнакомых — чёрной ниткой. Рассмотрим поэта Васю — к нему привязаны 2016 ниток, значит среди них уж точно есть три нитки одного цвета. Пусть это белые нитки. Рассмотрим поэтов, находящихся на других концах этих ниток.

Если какие-то два из этих трёх пожтов знакомы, то образуется белый треугольник из них и поэта Васи. Если же они все попарно незнакомы друг с другом — то нам не хуже, это тоже вариант из условия задачи.
\end{itemize}