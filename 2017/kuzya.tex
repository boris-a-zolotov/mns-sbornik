\task{Обаятельный домовенок}
\begin{itemize}

\itA $6-4=2$, отсюда Кузя догоняет издателей со скоростью 2 статьи в день. На то, чтобы нагнать 40 статей, у него уйдет 20 дней.

\itB Площадь квадратика $2 \times 2$ в 4 раза больше площади квадратика $1 \times 1$, поэтому на него уходит в 4 раза больше чернил. Значит, на том же картридже Кузя сможет напечатать $10000 \div 4 = 2500$ квадратиков $2 \times 2$.

\itC Среди гвоздиков почти каждого горизонтального ряда как минимум на двух должны быть сделаны повороты: ведь нитка входит и выходит из этого ряда. Однако найдутся два горизонтальных ряда гвоздиков, где нитка начинается или кончается — поэтому количество поворотов нитки может быть оценено сверху числом
$$2 \cdot 5 + 2 =12.$$

\vspace{-0.25cm}
Протянуть нитку, сделав 12 поворотов, просто: можно, например, стартовать из верхней левой клетки и пойти до конца направо, потом, сделав два поворота, спуститься на ряд вниз и пойти налево — и так далее, смотреть рисунок.

\begin{center} \tikz{

\foreach \x in {-3,...,3} {
	\foreach \y in {-3,...,3} {
		\filldraw (0.6 * \x cm, 0.6 * \y cm + 0.1 cm) arc (90:450:0.1); 
	}; 
	\dwt (-1.8,0.6*\x cm) -- (1.8,0.6*\x cm); 
}; 

\foreach \y in {-3,-1,1} {
	\dwt (-1.8,0.6*\y cm) -- (-1.8,0.6*\y cm +0.6 cm); 
	\dwt (1.8,0.6*\y cm +1.2cm) -- (1.8,0.6*\y cm +0.6 cm); 
}

} \end{center}

\end{itemize}