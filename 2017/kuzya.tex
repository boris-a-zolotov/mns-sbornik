\task{Обаятельный домовёнок}
\begin{itemize}

\itA $6-4=2$, отсюда Кузя догоняет издателей со скоростью 2 статьи в день. На то, чтобы нагнать 40 статей, у него уйдёт 20 дней.

\itB Площадь квадратика $2 \times 2$ в 4 раза больше площади квадратика $1 \times 1$, поэтому на него уходит в 4 раза больше чернил. Значит, на том же картридже Кузя сможет напечатать $10000 \div 4 = 2500$ квадратиков $2 \times 2$.

\itC Среди гвоздиков почти каждого горизонтального ряда как минимум на двух должны быть сделаны повороты: ведь нитка входит и выходит из этого ряда. Однако найдутся два горизонтальных ряда гвоздиков, где нитка начинается или кончается — поэтому количество поворотов нитки может быть оценено сверху числом $2 \cdot 5 + 2 =12$.

Протянуть нитку, сделав 12 поворотов, просто: можно, например, стартовать из верхней левой клетки и пойти до конца направо, потом, сделав два поворота, спуститься на ряд вниз и пойти налево — и так далее.
\end{itemize}