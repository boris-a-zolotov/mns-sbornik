\task{О, как мы далеки!}
\begin{itemize}

\itA Не умаляя общности, пусть остановка $A$ находится левее $D$. Тогда остановка $C$ — либо слева от $A$, либо справа от $D$.

Если остановка $C$ слева от $A$, то она находится на расстоянии 4 от неё. Но тогда нам некуда поставить $B$, так чтобы $AB=4$, $BC=2$. Отсюда $C$ находится справа, и $AC=6$.

\itB Сажая новое дерево, мы бьём промежуток между прежде соседними деревьями на два меньших промежутка. Если мы добились того, что расстояние между соседними деревьями стало равным $d$, то $d$ является делителем 63, 84 и 14.

$\text{НОД}\,(63,84,14)=7$, поэтому деревья можно посадить каждые 7 метров (и большего расстояния между соседними добиться нельзя). Отсюда ответ —
$$(63/7-1)+(84/7-1)+(14/7-1) = 20\ \text{деревьев}.$$

\def\disbetw#1{\ \ \stackrel{#1}{\longleftrightarrow}\ \ }

\itC Да, точки можно расставить:
$$A \disbetw 3 D \disbetw 3 B \disbetw 6 E \disbetw 1 C.$$
\end{itemize}