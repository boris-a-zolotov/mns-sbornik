\task{Без нулей}

\begin{enumerate}

\item $\phantom{x}$

\def\ra{\ \ \longrightarrow\ \ }
\def\X{\text{\ttfamily X}}
\begin{center}\begin{tabular}{lll}
	$110 \ra \X\X$ & \quad & $1\X17 \ra 2017$ \\
	$2202 \ra 21\X2$ & & $\X\X\X\X \ra 11110$ \\
	$500000 \ra 49999\X$ & & $512 \ra 512$
\end{tabular}\end{center}

\item Значение числа $\overline{c_nc_{n-1}\ldots c_0}$ восстанавливается по его записи в модифицированной системе счисления так же, как по десятичной записи: $\sum c_k10^k$, где $\X$ интерпретируется как 10.

Пусть $N = \overline{c_nc_{n-1}\ldots c_0} = \overline{s_ns_{n-1}\ldots s_0}$. Давайте формально вычтем эти два выражения друг из друга. Получим
$$0 = \sum\limits_{k=0}^n (c_k-s_k) \cdot 10^k, \qquad -9 \le c_k-s_k \le 9.$$

Пусть старший разряд, в котором различаются записи числа $N$, имеет номер $M$. Тогда
\begin{align*}
	& c_M-s_M = - \sum\limits_{0 \le k < M} (c_k-s_k) \cdot 10^k \\
	& 10^M \le \left| c_M-s_M \right| \le
		\sum\limits_{0 \le k < M} \left| c_k-s_k \right| \cdot 10^k \le \\
	& \le 9 \cdot (1 + 10 + \ldots + 10^{M-1}) = 10^M-1.
\end{align*}

Получили противоречие.

\item Опишите алгоритм перевода чисел из десятичной системы в модифицированную и обратно.

Перевести из модифицированной в десятичную совсем просто: надо все разряды, в которых стоят $\X$, заменить на нули — и сложить получившееся число в столбик с числом, которое состоит в основном из нулей, а единицы стоят в разрядах, предшествующих тем, где в исходном числе стояли $\X$:

\begin{center}
$9\X\X7 =$ \begin{tabular}{ccccc}
	   & 9 & 0 & 0 & 7 \\
	   & 1 & 1 & 0 & 0 \\ \hline
	1 & 0 & 1 & 0 & 7 \\
\end{tabular} \end{center}

Для перевода из десятичной в модифицированную будем пользоваться тем соображением, что
$$\frame{$\vphantom{\sum}\,k\,$}000\ldots00
	= \frame{$\vphantom{\sum}\,k-1\,$}999\ldots9\X.$$

\def\rdn{\Bigl|}
Будем «читать» число справа налево и заменять блоки нулей, согласно установленному нами правилу:
\begin{align*}
	& 10100560075\rdn\Bigr. \\
	& 101005600\rdn75\Bigr. \\
	& 1010055\rdn9\X75\Bigr.
\end{align*}

При этом слева от нас могут появляться новые нули, но справа нулей точно не остаётся:
\begin{align*}
	& 10100\rdn559\X75\Bigr. \\
	& 100\rdn9\X559\X75\Bigr.
\end{align*}

Наконец, если первая цифра в числе — ноль, то от неё просто избавимся, уменьшив количество разрядов:
$$\rdn9\X9\X559\X75\Bigr.$$

\item Запись $\X$, несомненно, короче, чем 10 — но почему запись в модифицированной системе счисления не бывает длиннее?

С одной стороны, допустимые вклады каждого разряда в число стали не меньше, так что надобности в большем количестве разрядов просто неоткуда взяться. С другой стороны, у нас есть процедура построения записи и доказательство её единственности — заметим, что мы нигде не изменяем количество разрядов в числе, разве что в самом конце, избавляясь от ведущего нуля.

\item Идея за сложением и умножением в столбик, несомненно, будет стоять всё та же\scolon различия будут заключаться в некоторых технических деталях:

\def\cif{\text{\itshape цифра}}
\def\per{\text{\itshape перенос}}
\subitem $\cif + \X = \cif$ и перенос единицы в следующий разряд\scolon
\subitem $\per + \X = \per$ и перенос единицы в следующий разряд\scolon
\subitem $5 + 5 = \X$, без переноса в следующий разряд\scolon
\subitem $\X + \X = \X$ и перенос единицы в следующий разряд\scolon
\subitem $\per + \X + \X = \per$ и перенос двойки в следующий разряд\scolon
\subitem $\cif \cdot \X = \X$ и перенос $(\cif-1)$ в следующий разряд.

\item Можно, например, домножать на $-1$ только старший разряд, записывая, например, знакомое нам $-15949 = -20000 + 4501 = -2\,44\X1$. Тогда ноль будет записываться как $-1\,\X$.

\item Признак делимости на степени двойки будет таким же, как в десятичной системе счисления: всё потому, что число, в котором отбросили последние $k$ разрядов, будет делиться на $10^k$, а, значит, и на $2^k$.

Признак делимости на степени пятёрки аналогичен предыдущему.

Признак делимости на 3 и 9 — число сравнимо по модулю 3 (9) со своей суммой цифр — по-прежнему работает:
$$\overline{c_nc_{n-1}\ldots c_0} = \sum c_k10^k \equiv \sum c_k \pmod 3.$$

Признак делимости на 11 — число сравнимо по модулю 11 со своей знакочередующейся суммой цифр — также остаётся тем же:
$$\overline{c_nc_{n-1}\ldots c_0} = \sum c_k10^k \equiv \sum c_k(-1)^k \pmod{11}.$$

\end{enumerate}