\task{Плавучий зоопарк}

\begin{itemize}
\itA {Здесь должна быть картинка.}

\itB Конечно же, минимальное количество углов у пересечения — 3. Если мы найдём максимальное количество и приведём пример, когда оно достигается, то все промежуточные количества углов будет несложно получить.

Верхняя оценка на количество углов — 24: каждая из шести сторон шестиугольника могла бы пересекать каждую из четырёх сторон четырёхугольника. В свою очередь, пример, когда пересечение фигур — 6 четырёхугольников, легко построить.

\itC Если $m$ и $n$ чётны, то ответ — от трёх до $mn$ углов (смотреть предыдущий пункт). Если $m$ чётно, $n$ нечётно, то каждая из $m$ сторон $m$-угольника пересекает не более $n-1$ стороны $n$-угольника, так как количество точек пересечения любой прямой с любым многоугольником чётно: прямая должна «входить» и «выходить» из многоугольника. Пример, когда пересечение имеет $m(n-1)$ углов, строится по аналогии с первым пунктом.

При нечётных $m$ и $n$ ответ — $(m-1)(n-1)$, пример приводится аналогично. Больше углов пересечение не может иметь по следующей причине: $m$ угольник не может пересечь $n$-угольник «слева направо» больше $m-1$ раза, и каждое пересечение будет давать {\it в среднем} не более $n-1$ угла.
\end{itemize}