\task{Взвешивания}

\begin{enumerate}
\itA Пусть картофель весит $P$ граммов, а кот — $K$ граммов. Пусть погрешность составляет $M$ граммов.

Тогда $P+M=1000$, $K+M=4400$, $P+K+M=5000$. Отсюда $P=600$ (вычтем из третьего равенства второе), $K=4000$ (вычтем из третьего первое), $M=400$.

\itB Поделим 729 монет на три равных кучки. Положим две из них на весы — если одна из них окажется легче другой, то в ней находится фальшивая монета. Если они равны по весу, то фальшивая монета находится в оставшейся трети.

Таким образом, за один ход мы умеем уменьшать количество «подозреваемых» монет втрое. $729 = 3^6$, поэтому через шесть ходов останется одна монета, которая может быть фальшивой — она и окажется фальшивой.

\itC Выложим на весы одну монету из первого мешка, две монеты из второго мешка,~$\ldots$ 15 монет из 15-го мешка. Если бы все монеты были настоящими, их суммарная масса была бы равна $20 \cdot (1+ \ldots + 15)$ граммов. По факту мы получим б\'oльшую массу — она будет отличаться от приведённой нами ранее на $5\cdot($№ мешка с фальшивыми монетами) граммов. Так мы и выясним, где фальшивки.
\end{enumerate}