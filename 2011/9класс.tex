\documentclass[12pt]{amsart}
%
% amsart -- стандартный стиль для оформления математических статей, весьма 
% удобен и во многих случаях лучше пользоваться именно им.
%

\usepackage[T2A]{fontenc}
\usepackage[cp1251]{inputenc}
\usepackage[russian]{babel}
%
% Это нужно для стандартной русификации. 
%

\usepackage{amsmath,amssymb}
\usepackage{graphicx}
% Это для вставки графики при компиляции с помощью pdfLaTeX

%\usepackage[dvips]{graphicx}
% Это для вставки графики при компиляции в PostScript

\theoremstyle{definition}
\newtheorem{defin}{Определение}
% Окружение типа "определение"

\newtheorem{claim}{Предложение}

\theoremstyle{remark}
\newtheorem{remark}{Замечание}
% Окружение типа "замечание"

\theoremstyle{plain}
\newtheorem{thm}{Теорема}
% Окружение типа "теорема"
\newtheorem*{mainthm}{Основная теорема}
% * означает, что мы определения не нумеруем автоматически
\newtheorem{lemma}{Лемма}
\newtheorem{corollary}{Следствие}


\begin{document}

1A. Если сложить числа последовательно, то получится четное число -- действительно, если $n$ --- количество отрицательных чисел, то вся сумма равна
$$
82-2n,
$$
то есть четному числу. Пусть $k$ --- номер шага суммирования, на котором впервые получилось $41-n$. Тогда первые $k$ чисел и остальные $82-k$ образуют две искомые группы.

1B.
$$
\begin{array}{l}
\text{Первое расположение:}\ 18:36:54:72:90 \\
\text{Второе расположение:}\ 9:18:27:36:45.
\end{array}
$$

1C. Ответ "`можно"', например так:
$$
\begin{array}{l}
1562437\to 2437156\to 7124356\to\\
\to 4712356\to 2347156\to7123456 \\
\to 4567123\to 1234567.
\end{array}
$$

2A. Для пары целых положительных чисел, очевидно, из неравенства
$$
x>y^2+1
$$
следует неравенство
$$
x>y,
$$
и уж тем более
$$
y<x^2+1,
$$
поэтому любая симпатичная пара является приятной. Однако приятная пара отнюдь не обязана быть симпатичной --- примером несимпатичной приятной пары будет
$$
x=1,\ y=1.
$$

2B. Заметим, что 
\begin{equation}
\label{fr} 9!=2^7\cdot 3^4\cdot 5\cdot7.
\end{equation}

1) Одно из слагаемых в левой части равенства должно быть равно нулю. Если бы это было второе слагаемое, то в ненулевых членах суммарно оставалось бы 9 различных букв, и, следовательно, хотя бы одна из них была равна 5. Но, поскольку в (\ref{fr}) есть 5 только в первой степени, один член был бы кратен 5, а другой --- нет. Следовательно, В*А*С*Я=0.

2) Итак, имеем Л*Ю*Д*А=Л*Ю*Б*О*В*Ь. Заметим, что никакие буквы из этих, кроме Л и Ю, не могут быть равны 5 или 7, поскольку в противном случае одна часть равенства была бы кратна 5 или 7 соответственно, а другая --- нет. Далее, произведение любых четырех различных цифр не меньших 2 больше $9^2$, поэтому одна из последних четырех букв в правой части должна быть равна 1. Для остальных букв остались теперь значения 2,3,4,6,8,9. Заметим, что в разложении \ref{fr} присутствует 2 в нечетной степени, но правая и левая части должны делиться на одинаковую степень 2, таким образом исключается значение $8=2^3$ или значение 2 (значение 6 не может быть исключено, поскольку то же верно для 3). В обоих случаях нетрудно подобрать значения Д и А (с точностью до замены их местами, что, разумеется, не существенно). В первом случае Д+А=4+9=13, во втором --- Д+А=8+9=17.

Ответ: 17 и 13.

2C. Пусть $A,B,C,D$ вершины квадрата, следующие по часовой стрелке. Пусть $a,b,c,d$ --- соответствующие числа ($a$ соответствуюет $A$ и т.д.). Пусть
$$\begin{array}{l}
d_{ab}=\text{НОД}(A,B),\\
d_{bc}=\text{НОД}(B,C),\\
d_{cd}=\text{НОД}(C,D),\\
d_{da}=\text{НОД}(D,A).
\end{array}
$$
Нетрудно проверить, что все они попарно взаимно просты. Таким образом наименьшие четыре числа, удовлетворяющие требованию задачи, есть попарные произведения первых четырех простых чисел:
$$
\begin{array}{l}
a=10,
b=15,
c=21,
d=14.
\end{array}
$$

3B. Для удобства обозначим момент времени, когда Гена прошел треть всего пути $T_0$, а когда две трети --- $T_1$. Поскольку Вова за $T_1-T_0$ прошел $\frac{1}{2}$ всего пути, а Гена --- $\frac{1}{3}$, Вова шел быстрее и, поскольку в момент $T_1$ он уже обогнал Гену, в Полной Жути он окажется раньше, хотя из Светлого Пути он вышел позже. То же соображение позволяет заключить, что скорость Вовы в $\frac{3}{2}$ раза больше скорости Гены.

3C. Заметим, что
$$
a+b+c+d+e+f=2x.
$$
Но сумма первых 6ти натуральных чисел нечетна, поэтому среди $a,b,c,d,e,f$ должно присутствовать число, большее 6.
$$
\begin{array}{l}
a=1,\\
b=3,\\
c=7,\\
d=2,\\
e=4,\\
f=5
\end{array}
$$
подходят, при этом
$$
x=11.
$$

4A. Пусть
$$\begin{array}{l}
A=3a+4b+5c,\\
B=9a+b+4c,
\end{array}$$
тогда
$$
4B-A=11(3a+c),
$$
то есть $4B-A$ кратно 11, значит если одно из слагаемых кратно 11, то и второе --- тоже.


4B. Перемножим данные числа, получим
$$
aeD\cdot dCc\cdot bAB\cdot (-Bec)\cdot (-aCA)\cdot (-bdD)=-a^2b^2c^2d^2e^2A^2B^2C^2D^2,
$$
из чего видно, что произведение отрицательно. Таким образом, хотя бы одно из предложенных чисел должно быть отрицательным и, поскольку всего чисел четное число, хотя бы одно --- положительным.

4C. Преобразуем выражение:
$$
2001a^2-40ab-b^2=14\cdot(143a^2-3ab)-(a-b)^2.
$$
Отсюда видно, что  исходное выражение кратно 14 тогда и только тогда, когда $(a-b)^2$ делется на 14. Но 14 не содержит полных квадратов в разложении на множители, поэтому это верно если и только если $a-b$ делится на 14. Так как $a,b$ --- натуральные и $a>b$, то $a=15$ и $b=1$ --- наименьшие $a$ и $b$, при которых это выполняется. Таким образом ответ 16.

5А. Одна из точек $K$ и $M$ находится не дальше от прямой $AB$, чем вторая (см. рисунок). Пусть это будет точка $M$. Проведем через нее отрезок $ML$ так, чтобы $ML$ был параллелен $AB$ и точка $L$ лежала на $PK$.


Треугольники $APB$ и $LPM$ подобны, поэтому
$$
\frac{LM}{AB}=\frac{PM}{AP},
$$
из чего следует, что $AP>PM$. Случай, когда $K$ расположена не выше $M$, аналогичен.

5B. Проведем построения, как показано на рисунке:

Здесь $AB'C'D'$ --- прямоугольник, полученный из $ABCD$ растяжением из точки $A$ в два раза. Теперь заметим, что $D'M'$ параллельно $PM$, а, значит, угол $M'D'B$ равен углу $MPB$. Но угол $M'D'B$, очвидно, равен углу $MAN$. Ч.т.д.

5C. Внутри треугольника $BMC$ построим точку $M'$ так, чтобы угол $M'BC$ был равен углу $M'CB$ и равен $15^{\circ}$. Тогда угол $MBM'$ равен $60^{\circ}$ и следовательно треугольник $BMM'$ равносторонний и угол $BMM'$ равен $60^{\circ}$. Далее, поскольку углы $BM'C$, $BM'M$ и $MM'C$ дополняют друг друга до $360^{\circ}$, имеем: угол $MM'C$ равен $150^{\circ}$. Значит, треугольник $MM'C$ равен треугольнику $AMB$ и угол $BMC$ равен $75^{\circ}$.

6A. На рисунке показано, как это можно сделать (разные куски проволоки отмечены разными цветами)

6B. Пусть $\alpha$, $\beta$ и $\gamma$ --- величины углов в правильном условии задачи, тогда, если Дима удвоил первые два угла, то
$$
\begin{cases}
\alpha+\beta+\gamma=180^{\circ},\\
4\alpha+4\beta+\gamma=360^{\circ},
\end{cases}
$$
откуда $\alpha+\beta=60^{\circ}$, а значит $\gamma=120^{\circ}$, то есть $120^{\circ}$ --- наибольший угол в правильном условии.

6C. Ответ "`можно"', см. соответствующий рисунок.

7A. Можно, например, так:
$$
\begin{array}{llll}
1 & 1 & 1 & 1\\
0 & -1 & -1 & 0\\
0 & -1 & -1 & 0\\
1 & 1 & 1 & 1
\end{array}
$$

7B. Да, может. Во всех случаях, кроме того, когда Дима умножил на 5, это следует из того, что $2011-9$ кратно 7 ( чтобы получить 2011, получаем 9, а потом прибавляем 7, то есть 5+2, нужное количество раз). В случае, когда Дима нажатием красной кнопки получил 10, надо заметить, что $2011-16$ кратно 5, то есть надо получить еще 6=2+2+2 и далее прибавить 5 нужное количество раз.

7C. Простым перебором показывается, что последовательность, начинающаяся с 9, самая длинная.

8А. Если обозначить ошибку весов за $-a\,\text{кг}$, реальный вес портфеля Гены за $x\,\text{кг}$, а портфеля Вовы за $y\,\text{кг}$, то
$$
\begin{cases}
x+y-a=6,\\
x+y-2a=5,
\end{cases}
$$
откуда $a=1$. Следовательно, реальный вес портфелей $4\,\text{кг}$ у Гены и $3\,\text{кг}$ у Вовы.

8В. Вот фрагмент русско-Ам-Ямского словаря:
$$
\begin{array}{ll}
\text{мышка} & \text{ту}\\
\text{кошка} & \text{ля}\\
\text{пошла} & \text{ям}\\
\text{ночью} & \text{ам}\\
\text{гулять} & \text{му}\\
\text{видит} & \text{бу}\\
\text{поймать} & \text{гу}
\end{array}
$$

8C. Разлинуем наш квадрат $10\times 10$ на клетки $1\times 1$. Отрезки, соединяющие узлы -- вершины клеток, назовем ребрами. Заметим, что вершины прямоугольников, на которые был разрезан квадрат, являются узлами.

1) Свяжем площадь прямоугольника с количеством ребер, лежащих внутри него: пусть внутри лежит $k$ ребер, тогда площадь будет равна $k+1$, если у прямоугольника есть сторона длиной $1$, и не больше $k$ в противном случае.

2) Внутри исходного квадрата $10\times 10$ лежит $180$ ребер. Поскольку суммарная длина разреза равна $100$, количество ребер, лежащих внутри всех прямоугольников, равно $80$.

3) Квадрат $10\times 10$ имеет площадь $100$. Пусть у всех прямоугольков найдется сторона длиной $1$. Тогда по пункту $1)$ площадь каждого прямоугольника равна количеству лежащих внутри ребер "`плюс"' $1$, соответственно, суммарная площадь равна количеству ребер, лежащих внутри всех прямоугольников, плюс количество пямоугольников, то есть $80+20=100$. Если же найдется прямоугольник, все стороны которого длиннее $1$, площадь его будет не больше количества лежащих внутри ребер, и суммарная площадь будет меньше $100$, что противоречит здравому смыслу.


\begin{figure}[h]
\begin{center}
\includegraphics[width=9cm]{5A.jpg}
% При компиляции pdfLaTeX'ом допустимы рисунки в формате jpeg или pdf
% \includegraphics[width=6cm{diagr.eps}
% При компиляции в PostScript допустимы рисунки в формате eps
\caption{задача 5А}
\end{center}
\end{figure}


\begin{figure}[h]
\begin{center}
\includegraphics[width=9cm]{95B.jpg}
% При компиляции pdfLaTeX'ом допустимы рисунки в формате jpeg или pdf
% \includegraphics[width=6cm{diagr.eps}
% При компиляции в PostScript допустимы рисунки в формате eps
\caption{задача 5B}
\end{center}
\end{figure}


\begin{figure}[h]
\begin{center}
\includegraphics[width=9cm]{95C.jpg}
% При компиляции pdfLaTeX'ом допустимы рисунки в формате jpeg или pdf
% \includegraphics[width=6cm{diagr.eps}
% При компиляции в PostScript допустимы рисунки в формате eps
\caption{задача 5С}
\end{center}
\end{figure}


\begin{figure}[h]
\begin{center}
\includegraphics[width=9cm]{96A.jpg}
% При компиляции pdfLaTeX'ом допустимы рисунки в формате jpeg или pdf
% \includegraphics[width=6cm{diagr.eps}
% При компиляции в PostScript допустимы рисунки в формате eps
\caption{задача 6А}
\end{center}
\end{figure}


\begin{figure}[h]
\begin{center}
\includegraphics[width=9cm]{96C.jpg}
% При компиляции pdfLaTeX'ом допустимы рисунки в формате jpeg или pdf
% \includegraphics[width=6cm{diagr.eps}
% При компиляции в PostScript допустимы рисунки в формате eps
\caption{задача 6С}
\end{center}
\end{figure}

\end{document}
