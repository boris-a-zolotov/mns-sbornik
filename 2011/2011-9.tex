\secklas{9}

\taskno{1}

\begin{itemize}
\itA Даны 82 числа, каждое из которых равно 1 или $-1$. Можно ли их разбить на две группы так, 
чтобы суммы чисел, входящих в каждую группу, были равны?

\itr Обозначим за $S_k$ сумму первых $k$ чисел, и положим $S_0 = 0$. 
$S_k$ отличается от $S_{k+1}$ в точности на 1 (на каждом шаге мы прибавляем или 
вычитаем 1), поэтому все целые числа между $0$ и $S_{82}$ достигаются на каком-то шаге.

Обозначим за $n$ число минус единиц среди 82 чисел. Тогда $S_{82} = 82-2n$
и $\frac{S_{82}}{2} = 41-n$. При этом, поскольку $41-n$ целое, и поскольку справедливо 
$|82-2n|=2\cdot|41-n|$, то число $41-n$ находится между 0 и $S_{82}$, то есть 
достижимо на каком-то шаге $k$: $S_k = 41-n$.

Возьмем первые $k$ чисел в первую группу, и оставшиеся $82-k$ чисел во вторую;  это и будет
примером разбиения, требуемым в условии. И, как было показано, данное разбиение может быть 
построено при любых заданных числах.
%Ответ: да, всегда можно разбить на две такие группы.
\end{itemize}

\taskno{2}

\begin{itemize}
\itC В вершинах квадрата поставлены натуральные числа, большие 1, причем пары чисел, 
стоящие на концах сторон, не взаимно просты, а на концах диагоналей --- взаимно просты. 
Найти такие четыре числа, удовлетворяющие этому условию,
чтобы наибольшее из них было наименьшим из возможных.

\itr 
Пусть $a,b,c,d$ --- числа, записанные в вершинах квадрата в порядке обхода вершин по часовой стрелке. 

\begin{center}
\tikz{
  \draw (0,0) node[left] {$a$} -- (0,2) node[left] {$b$} -- (2,2) node[right] {$c$} -- (2,0) node[right] {$d$} -- cycle; 
}
\end{center}

Заметим, что так как $a$ не взаимно просто с $b$ и $d$, то можно разложить эти числа в 
следующие произведения:
$a = pqx$, $b = py$, $d = qt$, причем $p>1, q>1$ и $\gcd(py,qt)=1$. 

Применив эти рассуждения к каждой вершине квадрата, получаем, что каждое из чисел является
составным, причем эти числа имеют как минимум четыре различных простых сомножителя, 
так как $\gcd(py,qt)=1$. 

Поэтому нам нужно взять первые четыре простых числа (2, 3, 5 и 7) --- любые другие б\'ольшие простые числа
только увеличат значения в вершинах --- и с помощью небольшого перебора
выбрать из их шести попарных произведений четыре:
$$
\begin{array}{l}
a=5 \cdot 2,\quad
b=5 \cdot 3,\quad
c=3 \cdot 7,\quad
d=2 \cdot 7
\end{array}
$$
\end{itemize}

\taskno{3}

\begin{itemize}
\itC Найдите наименьшее число $x$, для которого выполняются равенства:
$x = a+b+c = d+e+f$, где $a$, $b$, $c$, $d$, $e$ и $f$ --- попарно различные натуральные числа.

\itr Заметим, что $a+b+c+d+e+f=2x$. Но сумма первых шести натуральных чисел равна 21, нечетному числу.
Уменьшить сумму мы не можем, значит, ее надо увеличить хотя бы до ближайшего четного --- до 22; 
то есть $x \ge 11$. Теперь заметим, что $x \le 11$, поскольку
$1 + 3 +7 = 2 + 4 + 5 = 11$.

Ответ: $x=11$.
\end{itemize}

\taskno{6}
\begin{itemize}
\itB Переписывая из задачника по геометрии условие одной задачи, Дима по невнимательности, 
свойственной некоторым школьникам, завысил вдвое величины двух углов треугольника и уменьшил 
вдвое величину третьего. Тем не менее, он смог построить треугольник с новыми углами. Найдите 
самый большой из углов треугольника, описанного в правильном условии задачи.

\itr
 Пусть $\alpha$, $\beta$ и $\gamma$ --- величины углов в правильном условии задачи, тогда, если 
Дима удвоил первые два угла и уменьшил в два раза третий, то
$$
\begin{cases}
\alpha+\beta+\gamma=180^{\circ},\\
2\alpha+2\beta+\frac{\gamma}{2}=180^{\circ},
\end{cases}
$$
то есть
$$
\begin{cases}
\alpha+\beta+\gamma=180^{\circ},\\
4\alpha+4\beta+\gamma=360^{\circ},
\end{cases}
$$
откуда $3(\alpha+\beta)=180^{\circ}$, значит, $\alpha+\beta=60^\circ$ и $\gamma=120^{\circ}$.
Таким образом, треугольник в условии задачи --- тупоугольный, с наибольшим углом 
$120^{\circ}$.

\end{itemize}
