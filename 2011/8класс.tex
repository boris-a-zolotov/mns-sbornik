\documentclass[12pt]{amsart}
%
% amsart -- стандартный стиль для оформления математических статей, весьма 
% удобен и во многих случаях лучше пользоваться именно им.
%

\usepackage[T2A]{fontenc}
\usepackage[cp1251]{inputenc}
\usepackage[russian]{babel}
%
% Это нужно для стандартной русификации. 
%

\usepackage{amsmath,amssymb}
\usepackage{graphicx}
% Это для вставки графики при компиляции с помощью pdfLaTeX

%\usepackage[dvips]{graphicx}
% Это для вставки графики при компиляции в PostScript

\theoremstyle{definition}
\newtheorem{defin}{Определение}
% Окружение типа "определение"

\newtheorem{claim}{Предложение}

\theoremstyle{remark}
\newtheorem{remark}{Замечание}
% Окружение типа "замечание"

\theoremstyle{plain}
\newtheorem*{thm}{Результат}
% Окружение типа "теорема"
\newtheorem*{mainthm}{Основная теорема}
% * означает, что мы определения не нумеруем автоматически
\newtheorem{lemma}{Лемма}
\newtheorem{corollary}{Следствие}


\begin{document}

1А. Так как семь карандашей дороже восьми тетрадей, карандаш дороже тетради. Следовательно восемь карандашей дороже девяти тетрадей.

1В. Для начала надо заметить, что последняя цифра степени тройки может быть равна только 1, 3, 9 или 7: более того, эти значения чередуются именно в этом порядке. Это видно прямой проверкой:
$$
\begin{array}{l}
3^0=1,\\
3^1=3,\\
3^2=9,\\
3^3=27,
\end{array}
$$
а далее последние цифры снова повторяются. Теперь будем рассуждать по индукции:

База легко проверяется.

Переход. Пусть для $3^n$ утверждение верно:
$$
3^n=10^ka_k+10^{k-1}a_{k-1}+...+10a_1+a_0,\quad a_1\ \text{четное}.
$$
тогда, поскольку
$$
3^{n+1}=3^n\cdot 3,
$$
последние две цифры $3^{n+1}$ получаются из последних двух цифр $3\cdot (10a_1+a_0)$. Но в разряде десятков у $3a_0$ всегда стоит четное число (достаточно перебрать все возможные значения $a_0$), как и в разряде десятков у $30a_1$ (потому что $3a_1$ четно). Поэтому и в разряде десятков $3\cdot (10a_1+a_0)$ стоит четное число, что завершает индукционный переход.

1C. См. задачу 2В 9го класса.

2А. Спичками предложеных размеров можно выложить куб со стороной 1м --- это связано с тем, что длина, ширина и высота спички делит соответственно длину, ширину и высоту куба нацело. Поэтому количество спичек, которые можно изготовить из данного куба, равно отношению объемов:
$$
\frac{1000^3\text{мм}^3}{50\cdot 2\cdot 2\text{мм}^3}=5000000,
$$
то есть 5 миллионов.

2В. Возможны три варианта:

1) Никакие три точки не лежат на одной прямой. Тогда прямых ровно столько, сколькими способами можно выбрать пару точек из восьми различных. Это число легко подсчитать --- для каждой из восьми точек существует семь вариантов дополнить ее до пары; каждую пару мы посчитаем таким образом два раза, поэтому результат
$$
\frac{8\cdot 7}{2}=28.
$$

2) Три точки, лежащие по одну сторону относительно одной из проведенных осей перегиба (обозначим ее $l$), лежат на одной прямой. Тогда и четвертая лежит на этой же прямой, а также четыре точки, лежащие по другую сторону от $l$, опять же лежат на одной прямой --- это очевидные соображения симметрии. Если проводить подсчет как в первом случае, мы посчитаем 10 лишних прямых, поэтому ответ будет 18.

3) Три точки лежат на одной прямой, причем прямая эта проходит через пересечение осей перегиба. Аналогично случаю два мы получаем, что четвертая точка должна лежать на этой же прямой, а остальные четыре --- на другой прямой. Из тех же соображений ответ 18.

2С. Утверждается, что квадрат с вершинами в точках $(-1/2,1/2)$, $(1/2,3/2)$, $(3/2,1/2)$ и $(1/2,-1/2)$ будет иметь максимальную площадь. Его площадь равна 2. Пусть дан любой квадрат $ABCD$, удовлетворяющий условию задачи. Его площадь равна квадрату расстояния между прямыми $AB$ и $CD$. Прямые $AB$ и $CD$ пересекают один и тот же квадрат координатной сетки (квадрат, вершины которого --- точки с целыми координатами), не содержащий точек с целыми координатами. Любой отрезок внутри квадрата не длиннее его диагонали, а поскольку $AB$ и $CD$ пересекают один и тот же квадрат координатной сетки, расстояние между ними не длиннее какого-то отрезка внутри этого квадрата, а, значит, и не длиннее диагонали этого квадрата, то есть не длиннее $\sqrt{2}$. Следовательно площадь $ABCD$ не больше 2. Таким образом площадь квадрата, заявленного в начале решения, действительно максимальна.

Ответ 2.

3А. Оранжевому попрыгунчику следует совершить в направлении $OA$ два больших прыжка, а потом три маленьких в обратном направлении.

3В. Очевидно, все члены второй последовательности больше первого члена первой, то есть больше 5. Так же все члены первой последовательности, кроме первого, больше первого члена второй, то есть больше 6. Все члены последовательностей, кроме первых, получаются без участия 2, поэтому все члены первой последовательности, кроме первого, четны (ведь все простые числа, большие 2, нечетны), а все члены второй последовательности, кроме первого, нечетны. Поэтому никакой член первой последовательности не может быть равен члену второй последовательности.

3С. См. задачу 2А 9го класса.

4А. Пусть нашлись такие нечетные $a$, $b$, $c$ и $d$, что
$$
a=bc+d.
$$
Но в правой части равенства стоит четное число, а в левой --- нечетное. Значит, равенство невозможно, то есть ответ "`не может"'.

4B. Ответ "`да"'. Чтобы показать это достаточно доказать, что от любого набора можно перейти к набору $\{1,1,1,1,1\}$. Рассмотрим для начала набор, в котором ровно два отрицательных числа; поскольку порядок не имеет значения, без умаления общности он выглядит так: $\{-1,-1,1,1,1\}$. Сделаем следующие преобразования:
$$
\{-1,-1,1,1,1\}\to \{1,-1,-1,-1,1\}\to \{1,1,1,1,1\}.
$$
Итак, в случае двух отрицательных чисел мы знаем, что делать. Случай трех отрицательных очевиден. Все остальные случаи сводятся к случаю двух или трех отрицательных:
$$\begin{array}{l}
\{-1,1,1,1,1\}\to \{1,-1,-1,1,1\} \\
\{-1,-1,-1,-1,1\}\to \{-1,-1,1,1,-1\} \\
\{-1,-1,-1,-1,-1\}\to  \{-1,-1,1,1,1\}.
\end{array}$$

4С. См. задачу 7С 9го класса.

5A. Обозначим величины углов так, как показано на рисунке. Поскольку сумма внутренних углов треугольника всегда равна $180^{\circ}$, имеем:
$$
a+b+x=a+c+y.
$$
Но $b<c$, поскольку $AC<AB$, значит $x>y$, то есть угол $AEB$ больше угла $AEC$.

5В. См. задачу 5С 9го класса.

5С. Сделаем дополнительные построения так, как показано на рисунке ($N'M'$ перпендикулярно $P'Q'$). Запишем равенство, которое требуется доказать:
$$\begin{array}{l}
(AN+NO+OP+PA)+(CQ+QO+OM+MC)=\\
=(BN+NO+OQ+QB)+(DP+PO+OM+MD).
\end{array}$$
Сделав возможные сокращения, получаем:
$$
AN+PA+CQ+MC=BN+QB+DP+MD.
$$
Подставим удобные выражения для каждой величины в равенстве:
$$\begin{array}{l}
(AN'-NN')+(AP'+PP')+(CQ'+Q'Q)+(CM'-MM')=\\
=(BN'+NN')+(BQ'-QQ')+(DP'-PP')+(DM'+M'M).
\end{array}$$
Перегруппируем слагаемые:
$$\begin{array}{l}
(AN'+AP'+CQ'+CM')+(PP'+QQ'-NN'-MM')=\\
=(BN'+BQ'+DP'+DM')+(NN'+MM'-QQ'-PP').
\end{array}$$
Нетрудно проверить, что первая скобка в левой части равна первой скобке в правой части. Сокращая их, приводя подобные и сокращая на двойку получаем:
$$
PP'+QQ'=NN'+MM'.
$$
Теперь надо заметить равенство углов $NON'$, $MOM'$, $POP'$ и $QOQ'$. Обозначим величину этих углов $\alpha$. Домножим обе части полученного равенства на $ctg\alpha$. В результате получим:
$$
P'O+Q'O=N'O+M'O.
$$
Последнее равенство, очевидно, верно. Поскольку все преобразования были равносильными, верно и первое равенство, то есть то, что требовалось доказать.

6А. Ответ "`верно"'. Если $X$ --- конечное множество, через $|X|$ будем обозначать количество содержащихся в нем элементов, если $X$ и $Y$ два множества, то через $X\cap Y$ будем обозначать множество, содержащее только те элементы, которые есть и в $X$, и в $Y$. Заметим, что $X\cap Y=Y\cap X$. Будем обозначать множество всех блондинок $W$, всех голубоглазых $B$, а все население --- $P$. Теперь условие задачи переписывается так:
$$
\frac{|W\cap B|}{|B|}>\frac{|W|}{|P|}.
$$
Домножая обе части неравенства на $\frac{|B|}{|W|}$, что
$$
\frac{|W\cap B|}{|W|}>\frac{|B|}{|P|},
$$
то есть то, что требовалось.

6В. Рассмотрим выражение
$$
10k+9.
$$
Подставляя 0, 1, 2, ..., 123 вместо $k$ мы получим данные в задаче для примеров числа. Теперь, представляя $10=1+9$, мы получим данныке в задаче для примеров выражения:
$$
10k+9=9k+(k+9).
$$

6C. Любое число, старшая цифра которого больше всех остальных, является старшим --- например, 21111, 54321. Сосчитаем теперь количество всех старших чисел с цифрами 1, 2 и 3. Если все цифры числа равны 1, то оно не является старшим, поэтому хотябы одна цифра не меньше 2. Перечислим все старшие числа с цифрами 1 и 2:
$$\begin{array}{l}
21111,\\
21211,\\
22111,\\
22121,\\
22211,\\
22221.
\end{array}
$$
Если же в цифрах встречаются тройки, то они могут находиться только на тех местах, где находятся двойки у вышеперечисленных. Остальные же цифры могут быть равными произвольно 1 и 2. Таким образом первому из перечисленных чисел соответствует 16 чисел, второму и третьему 8, четвертому и пятому 4 и последнему 2, всего 28. Значит, всего чисел 28+6=34.

7А. Может. Если цифры числа, задуманного Димой, обозначить (в порядке от старшей к младшей) за $a_3$, $a_2$ и $a_1$, причем $a_3\geq a_1$, то разность будет иметь вид
$$
100a_3+10a_2+a_1-(100a_1+10a_2+a_3)=100(a_3-a_1)-(a_3-a_1)=99(a_3-a_1),
$$
а последняя цифра будет $10-(a_3-a_1)$. Значит Вова модет посчитать, чему равно $(a_3-a_1)$, а значит и всю разность.

7В. Если бы Незнайка получил 2310, перемножив несколько цифр, число 2310 можно было бы разложить на множители, каждый из которых меньше 10. Но
$$
2310=2\cdot 3\cdot5\cdot 11,
$$
из чего следует, что это невозможно.

7C. Рассмотрим шестизначное число $10^5a_5+10^4a_4+...+a_0$. Это число делится на $p$ тогда и только тогда, когда
\begin{equation}
\label{a} 10^4a_4+10^3a_3...+a_0=-10^5a_5\ (mod\ p).
\end{equation}
Эффект, демострируемый в условии задачи, выполняется тогда и только тогда, когда
$$
10^5a_4+10^4a_3+...+10a_0=-a_5\ (mod\ p).
$$
Но из \ref{a} следует, что
$$
10^5a_4+10^4a_3+...+10a_0=-10^6a_5\ (mod\ p),
$$
таким образом необходимо и достаточно выполнения следующего условия:
$$
10^6=1\ (mod\ p).
$$
Это несложно проверяется для указаных в задаче простых $p$ --- например для $7$ это делается так (все равенства $mod\ 7$):
$$
\begin{array}{l}
10=3;\\
10^2=3^2=9=2;\\
10^6=(10^2)^3=2^3=8=1.
\end{array}
$$
Аналогично для 11 (отрицательные числа ничем не хуже положительных):
$$\begin{array}{l}
10=-1;\\
10^6=(-1)^6=1.
\end{array}
$$

8А. Мальчики стоят в следующем порядке:

1) Миша

2) Вова

3) Гена

4) Дима

5) Гриша

Соответственно бюллетеня не достанется Грише.

8В. См. задачу 8А 9го класса.

8C. Приведем вариант раскраски, при которой Дима не вычеркнет все белые клетки. Укажем, какие клетки должны остаться белыми: рассмотрим прямоугольник $n\times (n+1)$, получившийся из исходного квадрата вычеркиванием первой строки. В этом прямоугольнике, начиная с верхней левой клетки, будем отмечать клетки в шахматном порядке. Вот эти клетки и должны остаться белыми. Теперь докажем корректность раскраски:

1) Почему Дима не вычеркнет все белые клетки? Действительно, предположим, например, что Дима вычеркнул уже все $n-1$ строк --- невычеркнутыми остались две случайные строки. Когда Дима вычеркнет $n-1$ столбцов, невычеркнутыми останутся четыре случайные клетки в этих строках, причем эти клетки лежат в двух невычеркнутых столбцах, по две в каждом столбце. Таким образом, чтобы Дима не мог вычеркнуть все белые клетки необходимо и достаточно, чтобы среди любых четырех клеток, расположенных как указано выше, нашлась хотя бы одна белая. Нетрудно убедиться, что предложенная раскраска удовлетворяет этому требованию.

2) Почему Вовочка сможет оставить такое количество белых клеток? В нашей раскраске $\frac{n(n+1)}{2}$ белых клеток. Значит достаточно доказать, что
$$
(n+1)^2-3n\geq \frac{n(n+1)}{2}.
$$
Если раскрыть скобки и привести подобные, получим, что это неравенство равносильно
$$
(n-1)^2\geq 0,
$$
что выполняется при всех $n$. Ч. т. д.!


\begin{figure}[h]
\begin{center}
\includegraphics[width=16cm]{85A.jpg}
% При компиляции pdfLaTeX'ом допустимы рисунки в формате jpeg или pdf
% \includegraphics[width=6cm{diagr.eps}
% При компиляции в PostScript допустимы рисунки в формате eps
\end{center}
\caption{задача 5А}
\end{figure}


\begin{figure}[h]
\begin{center}
\includegraphics[width=16cm]{85C.jpg}
% При компиляции pdfLaTeX'ом допустимы рисунки в формате jpeg или pdf
% \includegraphics[width=6cm{diagr.eps}
% При компиляции в PostScript допустимы рисунки в формате eps
\end{center}
\caption{задача 5C}
\end{figure}

\end{document}
