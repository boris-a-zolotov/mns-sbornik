\documentclass[12pt]{amsart}
%
% amsart -- стандартный стиль для оформления математических статей, весьма 
% удобен и во многих случаях лучше пользоваться именно им.
%

\usepackage[T2A]{fontenc}
\usepackage[cp1251]{inputenc}
\usepackage[russian]{babel}
%
% Это нужно для стандартной русификации. 
%

\usepackage{amsmath,amssymb}
\usepackage{graphicx}
% Это для вставки графики при компиляции с помощью pdfLaTeX

%\usepackage[dvips]{graphicx}
% Это для вставки графики при компиляции в PostScript

\theoremstyle{definition}
\newtheorem{defin}{Определение}
% Окружение типа "определение"

\newtheorem{claim}{Предложение}

\theoremstyle{remark}
\newtheorem{remark}{Замечание}
% Окружение типа "замечание"

\theoremstyle{plain}
\newtheorem*{thm}{Результат}
% Окружение типа "теорема"
\newtheorem*{mainthm}{Основная теорема}
% * означает, что мы определения не нумеруем автоматически
\newtheorem{lemma}{Лемма}
\newtheorem{corollary}{Следствие}


\begin{document}

1A. Оранжевому попрыгунчику следует совершить три больших прыжка в направлении $OA$ и шесть маленьких прыжков в обратном направлении.

1В. Пусть $a$ --- цифра, стоящая в младшем разряде, в десятичной записи числа $n$.

1) $a=9$. Тогда $n(n+1)$ заканчивается нулем и Вовочка должен был использовать 0.

2) $a<9$. Тогда последняя цифра числа $n(n+1)$ --- это последняя цифра числа $a(a+1)$. Простым перебором показывается, что она может быть равна только 0, 2 или 6 --- именно тем цифрам, которые не использовал Вовочка.

1С. Пусть мы записали максимальное количество таких десятизачных чисел. Среди наших чисел не более одного со старшей цифрой $1$, не более одного со старшей цифрой $2$ и т.д. Значит всего записано не более девяти чисел. На самом деле можно записать ровно девять, например, так:
$$\begin{array}{l}
1234567890\\
2345678901\\
3456789012\\
4567890123\\
5678901234\\
6789012345\\
7890123456\\
8901234567\\
9012345678.
\end{array}$$
Значит ответ 9.

2А. Произведение всех натуральных чисел от 1 до 23 делится на
$$
2\cdot5\cdot 10\cdot20=2000,
$$
а, значит, делится на 1000. Но это означает, что последние три цифры этого произведения равны 0.

2В. Заметим, для того, чтобы представить в виде суммы или в виде произведения, требуется представить в виде суммы как минимум двух слагаемых или множителей соответственно. Утверждается, что это возможно тогда и только тогда, когда число составное. Действительно, рассмотрим сначала $n$ --- простое число. Тогда его можно разложить на множетели только тиаким способом:
$$
n=n\cdot1\cdot1\cdot ...\cdot1.
$$
Но такое разложение, конечно, не удовлетворяет условию задачи. Пусть теперь $n=ab$ --- составное число, $a>1$ и $b>1$. Тогда
$$
n=a+b+\underbrace{1+1+...+1}_{n-a-b\ \text{раз}}=a\cdot b\cdot\underbrace{1\cdot1\cdot...\cdot 1}_{n-a-b\ \text{раз}}.
$$

2C. См. задачу 1В 8го класса.

3А. См. задачу 4А 8го класса.

3В. См. задачу 4А 9го класса.

3С. См. зажачу 4С 9го класса.

4А. Из того, что девять карандашей дороже десяти тетрадей следует, что карандаш дороже тетради (ведь карандашей меньше). Значит, десять карандашей тем более дороже одиннадцати тетрадей.

4В. См. задачу 2А 8го класса.

4С. Идея в том, что, произнося название трехзначного числа, мы говорим одно слово, обозначающее старшую цифру, и название числа, образуемого десятками и единицами. Таким образом, досчитав до четырехсот девяносто девяти включительно, Гриша произнес название всех чисел меньших 100 пять раз и четыреста раз произнесли название сотен --- по одному слову каждый раз, начиная со ста. В назнания всех десятков входит 162 слова, значит Гриша произнес к этому моменту
$$
162\cdot 5+400=1210.
$$
Начиная с этого момента Гриша произнес слово "`пятьсот"' 65 раз, а так же еще 104 слова. Значит всего Гриша произнес 1379 слов.

5А. См. задачу 6А 8го класса.

5В. Нет, не может быть. Действительно, если бы любые два из них внесли меньше трети стоимости покупки, то любой из них уж точно внес бы меньше трети стоимости. Значит, к примеру, Полено и Птица вместе внесли меньше трети, Михалыч и Леня Голубков --- меньше трети и Абрам Романов тоже меньше трети, поэтому в сумме они внесли меньше стоимости покупки, что невозможно.

5С. Очевидно, надо выкинуть ноль, иначе произведение чисел в одной группе будет обнуляться, а в другой --- нет. Также надо выкинуть 5 и 7. Действительно, если мы не выкинули, например, 5, то произведение чисел из той группы, куда попало 5, будет кратно 5, а произведение чисел другой группы --- нет. Для 7 рассуждения аналогичные. Пусть оставшиеся числа можно разбить на две группы, пусть $S$ --- произведение чисел в одной из них. Пусть $n$ --- максимальная степень двойки, являющая делителем $S$, то есть $2^n$ делит $S$, а $2^{n+1}$ не делит. Тогда максимальная степень двойки, делящая произведение чисел в другой группе, равна $7-n$. Но $7-n\neq n$, потому что $2n\neq 7$, ведь 7 нечетное, поэтому произведения не могут быть равны. Следовательно надо выкинуть еще, как минимум, одно число. Например, можно выкинуть еще 2, тогда остальные числа разобьются на две группы:

1) 1, 3, 4, 6

2) 8, 9.

Итак, ответ 4.



\end{document}
