\secklas{5}

\taskno{1}
\begin{itemize}
\itC У Димы есть прибор с четырьмя кнопками – двумя красными и двумя синими. Нажатие 
одной из красных кнопок приводит к тому, что число на табло прибора умножается на 2, 
а другой – умножается на 5. Нажатие же одной из синих кнопок приводит к тому, что к числу 
на табло прибора прибавляется 2, а другой – прибавляется 5. Вначале на табло прибора 
было записано число 2. Может ли мальчик Дима, нажимая эти кнопки, получить из него число 
2011, если, недолго думая, Дима нажал синюю кнопку? А красную кнопку?

\itr Да, может. Во всех случаях, кроме того, когда Дима умножил число на 5, это 
следует из того, что $2002 = 2011-9$ кратно 7: сперва получим $9$, а потом прибавляем
7 (получаемое как $2+5$) нужное количество раз.

В случае, когда Дима нажатием красной кнопки получил 10, надо заметить, что $1995 = 2011-16$ 
кратно 5, то есть надо получить $6$, прибавив $2$ два раза, и далее прибавить 5 нужное 
количество раз.
\end{itemize}

\taskno{2}
\begin{itemize}
\itC Гена и Вова вышли в разное время из деревни Светлый Путь в деревню Полная Жуть, каждый с некоторой 
постоянной скоростью. Когда Гена прошёл треть всего пути, Вова прошёл четверть всего 
пути. Когда же Вове оставалось пройти четверть пути, Гене ещё оставалось пройти треть 
всего пути. Кто из ребят шёл быстрее и во сколько раз? Кто из них раньше вышел из деревни
Светлый Путь, и кто из них раньше придёт в деревню Полная Жуть?

\itr Для удобства обозначим момент времени когда Гена прошел треть всего пути за $T_0$, 
а когда две трети --- за $T_1$. Поскольку Вова за $T_1-T_0$ прошел 
$(1-\frac{1}{4})-(\frac{1}{4}) = \frac{1}{2}$ всего пути, 
а Гена --- $\frac{1}{3}$, Вова шел быстрее и, поскольку в момент $T_1$ он уже обогнал Гену,
в Полной Жути он окажется раньше, хотя из Светлого Пути он вышел позже. 
То же соображение позволяет заключить, что скорость Вовы в $\frac{3}{2}$ раза больше
скорости Гены.
\end{itemize}

\taskno{4}
\begin{itemize}
\itC Лидер партии «В здоровом теле - здоровый дух» взял с собой в командировку 
кусок мыла. Вечером седьмого дня командировки он отметил, что размеры куска 
сократились в два раза. Хватит ли ему мыла на оставшиеся два дня? Предполагается, 
что партиец каждый день использует одинаковое количество мыла.

\itr Если размеры куска мыла уменьшились в два раза, то объем уменьшился в 
$2\cdot 2\cdot 2=8$ раз. Таким образом, к вечеру седьмого дня мыло уменьшилось 
в восемь раз, и, следовательно, от всего мыла было использовано $\frac{7}{8}$. 
Значит, в день использовалась $\frac{1}{8}$ часть мыла. Таким образом, мыла 
осталось всего на один день. Ответ: нет, не хватит.

\end{itemize}

