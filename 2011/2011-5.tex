\secklas{5}

\taskno{1}
\begin{itemize}
\itC У Димы есть прибор с четырьмя кнопками --- двумя красными и двумя синими. Нажатие 
одной из красных кнопок приводит к тому, что число на табло прибора умножается на 2, 
а другой --- умножается на 5. Нажатие же одной из синих кнопок приводит к тому, что к числу 
на табло прибора прибавляется 2, а другой --- прибавляется 5. Вначале на табло прибора 
было записано число 2. Может ли мальчик Дима, нажимая эти кнопки, получить из него число 
2011, если, недолго думая, Дима нажал синюю кнопку? А красную кнопку?

\itr Да, может. Во всех случаях, когда Дима не нажимал вторую красную кнопку (то есть, не
умножал число на 5), сперва можно получить $9$, а затем прибавлять по 7 ($2+5$) нужное
количество раз. Так как $2011-9 = 2002 = 286 \cdot 7$, то через 286 прибавлений семерки
мы действительно получим 2011.

Если же Дима нажатием первой красной кнопки получил 10, то тогда можно заметить, что 
$1995 = 2011-16 = 399 \cdot 5$, то есть надо сперва получить $16$, прибавив к 10 на экране
три раза по $2$, а затем получить 2011, прибавив 5 нужное количество раз.
\end{itemize}

\taskno{2}
\begin{itemize}
\itC Гена и Вова вышли в разное время из деревни Светлый Путь в деревню Полная Жуть, каждый с некоторой 
постоянной скоростью. Когда Гена прошел треть всего пути, Вова прошел четверть всего 
пути. Когда же Вове оставалось пройти четверть пути, Гене еще оставалось пройти треть 
всего пути. Кто из ребят шел быстрее и во сколько раз? Кто из них раньше вышел из деревни
Светлый Путь, и кто из них раньше придет в деревню Полная Жуть?

\itr Для удобства обозначим момент времени, когда Гена прошел треть всего пути за $T_0$, 
а когда две трети --- за $T_1$. Поскольку Вова за $T_1-T_0$ прошел 
$(1-\frac{1}{4})-(\frac{1}{4}) = \frac{1}{2}$ всего пути, 
а Гена --- $\frac{1}{3}$, Вова шел быстрее. Поскольку в момент времени $T_0$ 
Вова прошел меньше Гены (четверть пути вместо трети), он вышел позже. Однако, 
в момент $T_1$ он уже обогнал Гену, потому в Полной Жути он окажется раньше.
То же соображение позволяет заключить, что скорость Вовы в $\frac{3}{2}$ раза больше
скорости Гены.
\end{itemize}

\taskno{4}
\begin{itemize}
\itC Лидер партии «В здоровом теле --- здоровый дух» взял с собой в командировку 
кусок мыла. Вечером седьмого дня командировки он отметил, что размеры куска 
сократились в два раза. Хватит ли ему мыла на оставшиеся два дня? Предполагается, 
что партиец каждый день использует одинаковое количество мыла.

\itr Если размеры куска мыла уменьшились в два раза, то объем уменьшился в 
$2\cdot 2\cdot 2=8$ раз. Таким образом, к вечеру седьмого дня мыло уменьшилось 
в восемь раз, и, следовательно, от всего мыла было использовано $\frac{7}{8}$. 
Значит, в день использовалась $\frac{1}{8}$ часть мыла. Таким образом, мыла 
осталось всего на один день. Ответ: нет, не хватит.

\end{itemize}

