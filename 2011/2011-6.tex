\secklas{6}

\taskno{3}

\begin{itemize}

\itC Про натуральные числа $a$ и $b$ ($a>b$) известно, что число $2001a^2-40аb-b^2$ делится на 14.
Какое наименьшее значение может принимать выражение $a^2-b^2$?

\itr Преобразуем выражение:
$$
2001a^2-40ab-b^2=14\cdot(143a^2-3ab)-(a-b)^2.
$$
Отсюда видно, что  исходное выражение кратно 14 тогда и только тогда, когда $(a-b)^2$ 
делится на 14. Но 14 не содержит полных квадратов в разложении на множители, поэтому 
это верно если и только если $a-b$ делится на 14. Так как $a,b$ --- натуральные и $a>b$, 
то $a=15$ и $b=1$ --- наименьшие $a$ и $b$, при которых это выполняется. 
Ответ: 224.

\end{itemize}
