\documentclass[12pt]{amsart}
%
% amsart -- стандартный стиль для оформления математических статей, весьма 
% удобен и во многих случаях лучше пользоваться именно им.
%

\usepackage[T2A]{fontenc}
\usepackage[cp1251]{inputenc}
\usepackage[russian]{babel}
%
% Это нужно для стандартной русификации. 
%

\usepackage{amsmath,amssymb}
\usepackage{graphicx}
% Это для вставки графики при компиляции с помощью pdfLaTeX

%\usepackage[dvips]{graphicx}
% Это для вставки графики при компиляции в PostScript

\theoremstyle{definition}
\newtheorem{defin}{Определение}
% Окружение типа "определение"

\newtheorem{claim}{Предложение}

\theoremstyle{remark}
\newtheorem{remark}{Замечание}
% Окружение типа "замечание"

\theoremstyle{plain}
\newtheorem*{thm}{Результат}
% Окружение типа "теорема"
\newtheorem*{mainthm}{Основная теорема}
% * означает, что мы определения не нумеруем автоматически
\newtheorem{lemma}{Лемма}
\newtheorem{corollary}{Следствие}


\begin{document}

1А. Чтобы разрезать шестиметровую доску на одинаковые куски по 2 метра каждый, необходимо сделать два распила. Значит, чтобы разрезать 300 шестиметровых досок на одинаковые куски по 2 метра каждый, требуется сделать 600 распилов. Аналогично, чтобы разрезать восьмиметровую доску на такие же куски, делается 3 распила, а чтобы разрезать 200 таких досок --- 600 распилов. Следовательно в обоих случаях станку требуется сделать 600 распилов, которые он делает за один час. Ответ: 1ч.

1В. Обозначим это число за $x$. Тогда Мальвина попросила получить число $4x+15$, но Буратинов получил $15x+4$. Числа оказались равными, то есть
$$
4x+15=15x+4.
$$
Решаем уравнение и находим, что $x=1$.

1C. См. задачу 7В 9го класса.

2А. В Димином примере частное, делимое, делитель и остаток --- нечетные числа. Это невозможно (см. задачу 4А 8го класса).

2В. См. задачу 1В 6го класса.

2С. См. задачу 3В 9го класса.

3А. a) Да, можно:
$$
3\cdot 33+3:3=100.
$$

б) Нет, нельзя.

3В. См. задачу 3В 8го класса.

3С. Произведение будет четно, коль скоро хотя бы одна из сумм будет четна. Среди чисел 1, 2, ..., 7 всего 3 четных и 4 нечетных. Если все суммы нечетные, то к каждому нечетному числу прибавлялось четное. Но этого не может быть, потому что нечетных больше, чем четных. Следовательно, хотя бы одна сумма четна, а поэтому четно и все произведение.

4А. Самый левый Матроскин, затем Федор, затем Печкин, затем Шарик.

4В. Если размеры куска мыла уменьшились в два раза, то объем уменьшился в $2\cdot2\cdot2=8$ раз. Таким образом, к вечеру седьмого дня мыло уменьшилось в восемь раз, и, следовательно, от всего мыла было использовано $\frac{7}{8}$. Значит, в день использовалась $\frac{1}{8}$ часть мыла, таким образом, мыла осталось всего на один день. Ответ: нет, не хватит.

4С. См. задачу 2В 8го класса.

5А. "`Пифагоровы штаны во все стороны равны"'.

5B. Заметим, что имеет место равенство
$$
(a+c)+(b+d)=(a+d)+(b+c).
$$
Из предложенных чисел только две пары удовлетворяют этому условию: $(13,22)$ и $(15,20)$. Тогда
$$
a+b=16.
$$
Но
$$
a+b+c+d=35.
$$
Из этого легко находится
$$
c+d=19.
$$

5С. См. задачу 8В 9го класса.

\end{document}
