\secklas{8}

\taskno{2}

\begin{itemize}

\itB Будем мыслить плоскость как лист бумаги. На нём начертили две перпендикулярные прямые, 
согнули лист по одной прямой, затем по другой и прокололи в двух местах. После этого лист 
разогнули и через каждые две получившиеся проколами точки провели прямую. Сколько прямых при 
этом могло получиться?

\itr Возможны три варианта:

\begin{enumerate}
\item Никакие три точки не лежат на одной прямой. Тогда прямых ровно столько, сколькими 
способами можно выбрать пару точек из восьми различных. Это число легко подсчитать~--- 
для каждой из восьми точек существует семь вариантов дополнить ее до пары; каждую пару 
мы посчитаем таким образом два раза, поэтому результат
$$
\frac{8\cdot 7}{2}=28.
$$

\item Три точки, лежащие по одну сторону относительно одной из проведенных осей перегиба 
(обозначим ее $l$), лежат на одной прямой. Тогда и четвертая лежит на этой же прямой, 
а также четыре точки, лежащие по другую сторону от $l$, опять же лежат на одной прямой~--- 
это очевидные соображения симметрии. Если проводить подсчет как в первом случае, мы 
посчитаем 10 лишних прямых, поэтому ответ будет 18.

\item Три точки лежат на одной прямой, причем прямая эта проходит через пересечение осей 
перегиба. Аналогично случаю {\bfseries 2} мы получаем, что четвертая точка должна лежать на этой же 
прямой, а остальные четыре --- на другой прямой. Из тех же соображений ответ 18.
\end{enumerate}

\itC На координатной плоскости отмечены точки с целыми координатами. Какую 
наибольшую площадь может иметь квадрат, не содержащий ни одной отмеченной точки?

\itr Квадрат с вершинами в точках $(-0.5,0.5)$, $(0.5,1.5)$, $(1.5,0.5)$ и $(0.5,-0.5)$ 
будет иметь площадь, равную 2.

\begin{center}\tikz{
	\draw[dotted] (-1,-1) grid (2,2);
	\foreach \x in {-1,0,1,2} {
		\foreach \y in {-1,0,1,2} {
			\fill (\x,\y) circle[radius=0.03];
		}
	}
	\draw[thick] (-0.5,0.5) -- (0.5,1.5) -- (1.5,0.5) -- (0.5,-0.5) -- cycle;
}\end{center}

Покажем, что эта площадь максимальна.
В самом деле, рассмотрим некоторый квадрат $ABCD$ со стороной $d$, удовлетворяющий условию.
Впишем в этот квадрат круг --- его радиус будет $\frac{d}{2}$, и он тоже не содержит
отмеченных точек. 

\begin{center}\tikz{
	\draw[dotted] (-0.5,0) -- (3,0);
	\draw[dotted] (-0.5,2) -- (3,2);
	\draw[dotted] (0,-0.5) -- (0,3);
	\draw[dotted] (2,-0.5) -- (2,3);
	\foreach \x in {0,2} {
		\foreach \y in {0,2} {
			\fill (\x,\y) circle[radius=0.03];
		}
	}
	\draw[thick,dashed] (0.4,1) circle[radius=0.8];
%	\draw (0.4,1) node[above] {$P$} circle[radius=0.03];
	\draw[thick,rotate around={15:(0.4,1)}] (0.4-1.1314,1) node[left]{$A$}-- (0.4,1+1.1314) node[above]{$B$}
                                             -- (0.4+1.1314,1) node[right]{$C$}-- (0.4,1-1.1314) node[below]{$D$}
                                             -- cycle;
}\end{center}

Рассмотрим данный круг сам по себе, и рассмотрим центр этого 
круга, $P$. Эта точка находится в некотором
единичном квадрате, образованном соседними отмеченными точками.

\begin{center}\tikz{
	\draw[dotted] (-0.5,0) -- (3,0);
	\draw[dotted] (-0.5,2) -- (3,2);
	\draw[dotted] (0,-0.5) -- (0,3);
	\draw[dotted] (2,-0.5) -- (2,3);
	\foreach \x in {0,2} {
		\foreach \y in {0,2} {
			\fill (\x,\y) circle[radius=0.03];
		}
	}
	\draw[thick] (0.4,1) circle[radius=0.8];
	\draw (0.4,1) node[above] {$P$} circle[radius=0.03];
%	\draw[thick,rotate around={15:(0.4,1)}] (0.4-1.1314,1) node[left]{$A$}-- (0.4,1+1.1314) node[above]{$B$}
%                                             -- (0.4+1.1314,1) node[right]{$C$}-- (0.4,1-1.1314) node[below]{$D$}
%                                             -- cycle;
}\end{center}

Ясно, что расстояние от любой точки единичного квадрата до ближайшей отмеченной 
точки не превышает $\frac{\sqrt{2}}{2}$. А поскольку радиус круга $\frac{d}{2}$ не 
превышает расстояние от $P$ до ближайшей отмеченной точки (круг не содержит отмеченных точек), 
то и $$\frac{d}{2} \le \frac{\sqrt{2}}{2}.$$
Поэтому и площадь квадрата $ABCD$, равная $d^2$, не превышает 2.
\end{itemize}

\taskno{3}

\begin{itemize}
\itC Пара целых чисел $x$ и $y$ в пределах первого десятка называется симпатичной, если $x>y^2+1$, 
и называется  приятной, если $y<x^2+1$. Существуют ли приятные, но несимпатичные пары? 
А симпатичные, но неприятные?

\itr Для пары целых положительных чисел, очевидно, из неравенства
$x>y^2+1$ следует неравенство $x>y$, и уж тем более $y<x^2+1$,
поэтому любая симпатичная пара является приятной. Однако приятная пара отнюдь 
не обязана быть симпатичной~--- примером несимпатичной приятной пары будет
$x=1$, $y=1$.
\end{itemize}

\taskno{4}

\begin{itemize}
\itB Рассматриваются наборы из 5 чисел, каждое из которых равно $1$ или $-1$. Разрешается в 
каждом наборе изменять знак одновременно у трёх чисел. Можно ли с помощью этой операции 
от любого заданного набора перейти к любому другому из этих наборов?

\itr Да, возможно. Чтобы показать это, достаточно доказать, что от любого набора можно 
перейти к набору $(1,1,1,1,1)$. Выполнив же соответствующие операции смены знака в 
обратном порядке, мы можем перейти от набора $(1,1,1,1,1)$ к любому другому набору.

Рассмотрим для начала набор, в котором ровно два 
отрицательных числа; поскольку порядок не имеет значения, без умаления общности он 
выглядит так: $(-1,-1,1,1,1)$. Сделаем следующие преобразования:
$$
(\underline{-1},-1,\underline{1,1},1)\to (1,\underline{-1,-1,-1},1) \to (1,1,1,1,1)
$$
Итак, в случае двух отрицательных чисел мы знаем, что делать. Случай трёх отрицательных 
очевиден. Все остальные случаи сводятся к случаю двух или трех отрицательных:
$$\begin{array}{rcl}
(\underline{-1,1,1},1,1) &\to & (1,-1,-1,1,1) \\
(-1,-1,\underline{-1,-1,1}) & \to & (-1,-1,1,1,-1) \\
(-1,-1,\underline{-1,-1,-1}) & \to & (-1,-1,1,1,1)
\end{array}$$

%\itC Построим числовую последовательность по правилу: на нулевом шаге задаётся некоторое 
%число; если на каком-либо шаге получено число $a$, то на следующем шаге получается:
%\begin{itemize}
%\item число $а/2$, если $а$ чётно;
%\item число $3а+1$, если а нечётно и не равно 1;
%\item построение завершается, если $а =1$. 
%\end{itemize}
%
%Какова наибольшая длина последовательности получится, если на нулевом шаге задать однозначное число?
%
%\itr Простым перебором показывается, что последовательность, начинающаяся с 9, самая длинная.
\end{itemize}

\taskno{6}

\begin{itemize}
%\itB Заметим, что:
%
%\begin{center}\begin{tabular}{rcl}
%$9$ & = & $0 \times 9 + (0+9)$ \\
%$19$ & = & $1 \times 9 + (1+9)$ \\
%$29$ & = & $2 \times 9 + (2+9)$ \\
%\multicolumn{3}{c}{...} \\
%$99$ & = & $9 \times 9 + (9+9)$ \\
%$109$ & = & $10 \times 9 + (10+9)$ \\
%\multicolumn{3}{c}{...} \\
%$1239$ & = & $123 \times 9 + (123+9)$
%\end{tabular}\end{center}
%
%Обобщите это наблюдение и докажите результат.
%
%\itr
%Рассмотрим выражение $10k+9$.
%Подставляя 0, 1, 2, ..., 123 вместо $k$, мы получим левые части равенств.
%Теперь, заметив, что $10=1+9$, мы получим выражения целиком:
%$10k+9=9k+(k+9)$.

\itC Многозначное число называется старшим, если при перестановке любой 
группы цифр, стоящих в его начале, в конец, оно уменьшается. Найдите 
несколько старших пятизначных чисел. Сколько имеется старших пятизначных 
чисел, в записи которых встречаются лишь цифры 1, 2 и 3?

\itr 
Заметим, что любое число, старшая цифра 
которого больше всех остальных, является старшим --- например, 21111, 54321. 

Теперь подсчитаем количество пятизначных старших чисел из цифр 1, 2 и 3.

Всего возможно $3^5 = 243$ различных пятизначных числа из таких цифр.
Не являются старшими числа, которые при сдвиге совпадают 
сами с собой. Поскольку длина числа является простой, то, если число совпадает 
с собой при каком-то не кратном длине сдвиге, то оно должно совпадать с собой 
и при единичном сдвиге. А таких чисел только три ($11111$, $22222$ и $33333$). 

У остальных $240$ чисел все 5 сдвигов различны, и эти числа разбиваются
на 48 групп по 5 чисел. 
Максимумы этих групп и являются всеми возможными старшими числами. Ответ: 48 чисел.
\end{itemize}

\taskno{8}

\begin{itemize}
\itC Квадрат состоит из $(n+1) \times (n+1)$ белых клеток. Вовочка закрашивает некоторые клетки квадрата в чёрный цвет, а Дима вычёркивает какие-либо $n-1$ столбцов и $n-1$ строк этого квадрата. Докажите, что Вовочка может закрасить $3n$ клеток так, что при любом димином вычёркивании в квадрате оставалась бы по крайней мере одна белая клеточка.

\itr Приведем вариант раскраски, из которой Дима не сможет вычеркнуть все белые клетки. 

Первые $n$ клеток Вовочка должен отметить в первой колонке квадрата со строки 1 по $n$, 
вторые $n$ клеток~--- в $n+1$ строке со второй колонки до $n+1$, и остающиеся $n$ клеток~---
на диагонали $(2,1) \dots (n+1,n)$, всего закрашено $3n$ клеток.
          
Пример раскраски для $n=8$ приведён ниже:

\begin{center}\tikz[scale=0.5]{
	\fill[pattern=north west lines] (0,1) rectangle (1,8);
	\fill[pattern=north west lines] (1,0) rectangle (8,1);
	\foreach \x in {1,2,3,4,5,6,7} {
		\fill[pattern=north west lines] (\x,8-\x) rectangle (\x+1,9-\x);
	}
	\draw (0,0) grid (8,8);
}\end{center}

Поясним, почему при любом димином вычёркивании останется хотя бы одна белая клетка. 
Заметим, что в данной раскраске любые две 
колонки имеют одновременно закрашенные клетки ровно в одной строке: со второй колонки по $n+1$~--- 
в нижней строке, колонка 1 имеет закрашенную одновременно с колонкой $t$ клетку в строке $t-1$.
Значит, какие мы две колонки и строки не выберем, даже если одна из строк в пересечении с колонками
и даст две одновременно закрашенные клетки, вторая строка всё равно даст в пересечении хотя бы одну белую 
клетку.

\end{itemize}
