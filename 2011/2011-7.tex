\secklas{7}

\taskno{1}

\begin{itemize}

\itB Отрезок AB разбит на n равных отрезков, на каждом из которых как на диаметре построена окружность. 
Какие значения может принимать сумма длин всех этих окружностей, если $|AB| = \SI{2011}{\text{см}}$?

\itr 
\begin{center}\tikz{
    \draw[thick] (0,0) node[left] {$A$} -- (3,0);
    \draw[thick,dashed] (3,0) -- (4,0);
    \draw[thick] (4,0) -- (7,0) node[right] {$B$};
    \foreach \x/\r in {0.5/0.5,1.5/0.5,2.5/0.5,4.5/0.5,5.5/0.5,6.5/0.5} {
        \draw[thick] (\x,0) circle[radius=\r];
    }
}\end{center}

Как известно, длина окружности диаметра $d$ равна $\text{π}d$. 
Поэтому, если отрезок разбит на $n$ отрезков и длина каждого диаметра равна $\frac{|AB|}{n}$~см,
то сумма длин этих окружностей будет
$$
n \cdot (\text{π} d)=n \cdot (\text{π} \frac{|AB|}{n})=\text{π} |AB|=2011\text{π}
$$

\end{itemize}

\taskno{6}
\begin{itemize}

\itB На отрезке $AB$ длиной 38 см между $A$ и $B$ отмечены точки $C_1$, $C_2$, $C_3$, ..., $C_n$ и построены равносторонние треугольники с основаниями $AC_1$, $C_1C_2$, $C_2C_3$, ... , $C_nB$. Зависит ли сумма длин сторон треугольников, лежащих вне отрезка $AB$, от количества отмеченных точек и их размещения на $AB$?

\itr Для треугольника с основанием $C_{k}C_{k+1}$ обозначим $B_{k+1}$ его третью вершину. Для треугольников с основаниями $AC_1$ и $C_nB$ третьи вершины обозначим $B_1$ и $B_{n+1}$ соответственно. 

\begin{center}\tikz{
   \draw[thick] (0,0) node[left] {$A$} -- (5,0);
   \draw[thick,dashed] (5,0) -- (5.9,0);
   \draw[thick] (5.9,0) -- (7,0) node [right] {$B$};

   \foreach \x/\sz [count=\n] in {0/0.5,0.5/2,2.5/1,3.5/0.5,4/1} {
       \draw[thick] (\x,0) -- ++(60:\sz) node[above] {$B_\n$} -- ++(-60:\sz) node[below] {$C_\n$};
   }
   \draw[thick] (6,0) node[below] {$C_n$} -- ++(60:1) node[above] {$B_{n+1}$} -- ++(-60:1);
}\end{center}

Во введенных обозначениях на сумму всех указанных сторон выполняются следующие соотношения:
$$
\begin{array}{l}
(AB_1+B_1C_1)+(C_1B_2+B_2C_2)+...+(C_nB_{n+1}+B_{n+1}B)=\\
=2(AC_1+C_1C_2+C_2C_3+...+C_{n-1}C_n+C_nB)=\\
=2AB,
\end{array}$$
откуда видно, что указанная сумма не зависит от выбранных точек $C_1,...,C_n$.


\itC Каждая грань кубика разделена на 4 квадрата, и каждый квадратик окрашен в один из трёх цветов: синий, жёлтый или красный так, что квадратики, имеющие общую сторону, окрашены в разные цвета. Сколько при этом может быть красных, жёлтых и синих квадратиков?

\itr Рассмотрим какую-нибудь вершину данного кубика. 
К ней прилегает три квадратика, причем все они попарно касаются друг друга по стороне. 
Поэтому рассматриваемые квадратики представляют все три цвета. Значит, при каждой вершине 
нашего кубика есть ровно один квадратик данного цвета. Поскольку вершин всего 8, каждый 
цвет представлен 8 раз.

\end{itemize}
