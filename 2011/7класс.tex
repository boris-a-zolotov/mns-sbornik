\documentclass[12pt]{amsart}
%
% amsart -- стандартный стиль для оформления математических статей, весьма 
% удобен и во многих случаях лучше пользоваться именно им.
%

\usepackage[T2A]{fontenc}
\usepackage[cp1251]{inputenc}
\usepackage[russian]{babel}
%
% Это нужно для стандартной русификации. 
%

\usepackage{amsmath,amssymb}
\usepackage{graphicx}
% Это для вставки графики при компиляции с помощью pdfLaTeX

%\usepackage[dvips]{graphicx}
% Это для вставки графики при компиляции в PostScript

\theoremstyle{definition}
\newtheorem{defin}{Определение}
% Окружение типа "определение"

\newtheorem{claim}{Предложение}

\theoremstyle{remark}
\newtheorem{remark}{Замечание}
% Окружение типа "замечание"

\theoremstyle{plain}
\newtheorem*{thm}{Результат}
% Окружение типа "теорема"
\newtheorem*{mainthm}{Основная теорема}
% * означает, что мы определения не нумеруем автоматически
\newtheorem{lemma}{Лемма}
\newtheorem{corollary}{Следствие}


\begin{document}

1А. В одном кубическом метре 1000 литров воды. Значит, в 1000 м$^3$ миллион литров воды. Поскольку в одном гектаре 10000 м$^2$, для того, чтобы в бассейне было миллион литров воды, необходимо, чтобы глубина воды была равна 0.1м, то есть 10см. Таким образом, вышка должна быть выше 10см.

1В. Как известно, длина окружности диаметра $d$ равна $\pi d$. Поэтому если $d_1$, $d_2$, ..., $d_n$ --- диаметры соответствующих окружностей, то сумма длин этих окружностей будет
$$
\pi d_1+\pi d_2+...+\pi d_n=\pi (d_1+d_2+...+d_n)=\pi AB=2011\pi.
$$

1С. Извините, задача сформулирована некорректно.

2А. Возможны оба варианта. Например, если караднаш стоит дороже тетради, то восемь карандашей стоят дороже семи тетрадей, то есть условие задачи выполнено, а так же девять карандашей стоят дороже восьми тетрадей. Обратно, пусть $k$ --- стоимость карандаша, а $t$ --- стоимость тетради. Положим $t=1\,\text{р}$, а $k=\frac{1}{2}(\frac{8}{9}+\frac{7}{8})\,\text{р}$. Нетрудно проверяется, что при таком выборе цен восемь карандашей дороже семи тетрадей, но девять карандашей дешевле восьми тетрадей.

2В. Пусть $l$ --- длина отрезка пути, во время которого пассажир смотрел в окно, $S$ --- длина всего пути. Тогда
$$
\frac{S}{2}+l+\frac{l}{2}=S,
$$
то есть
$$
l=\frac{S}{3}.
$$
Ответ: треть всего пути.

2С. В Витином примере частное, делимое, делитель и остаток --- нечетные числа. Это невозможно (см. задачу 4А 8го класса).

3А. Оранжевому попрыгунчику следует совершить два больших прыжка в направлении $OA$ и пять маленьких прыжков в обратном направлении.

3В. Пусть записано $n$ чисел $a_1$, $a_2$, ..., $a_n$ (нумерация по кругу). Тогда с одной стороны
$$
3(a_1+a_2+...+a_n)=3\cdot 37,
$$
с другой стороны
$$\begin{array}{l}
3(a_1+a_2+...+a_n)=\\
=(a_1+a_2+a_3)+(a_2+a_3+a_4)+(a_3+a_4+a_5)+...+(a_{n-2}+a_{n-1}+a_{n})+\\
+(a_{n-1}+a_{n}+a_1)+(a_{n}+a_{1}+a_2)=\\
=n\cdot S,
\end{array}$$
где $a_1+a_2+a_3=S$. Таким образом
$$
3\cdot 37=n\cdot S.
$$
Поскольку все числа натуральные и $n>3$ отсюда имеем, что $n=37$ и $S=3$, а значит $a_1=a_2=...=a_n=1$.

3С. Полицейский автомобиль проезжает один километр за одну минуту, поэтому невозможно проезжать один километр на одну или две минуты быстрее. Пусть олигарх Романов проезжает километр на $t\,\text{мин}$ быстрее полицейского. Тогда он проезжает километр за $1-t\,\text{мин}$, откуда видно, что $t<1$. Например, Романов может проезжать километр на пол минуты быстрее.

4А. См. задачу 4А 8го класса.

4В. Пусть написано $n$ чисел $a_1$, $a_2$, ..., $a_n$. Заметим, что
$$
(*)\qquad (n-1)(a_1+a_2+...+a_n)=2(a_1+a_2+...+a_n).
$$
Для того, чтобы проверить это, достаточно в правой части равнества заменить $a_1$ на $\frac{1}{2}(a_2+a_3+...+a_n)$, $a_2$ на $\frac{1}{2}(a_1+a_3+a_4+...+a_n)$ и так далее. Можно ли разделить обе части равенства на $a_1+a_2+...+a_n$? Пусть
$$
a_1+a_2+...+a_n=0.
$$
Тогда
$$
\begin{cases}
a_1=-(a_2+...+a_n)\\
a_1=\frac{a_2+...+a_n}{2},
\end{cases}$$
откуда видим, что $a_1=0$, чего не может быть по условию. Следовательно, $a_1+a_2+...+a_n\neq 0$ и на него можно разделить обе части $*$, откуда получаем, что $n=3$.

4С. Нет, не сможет. Пусть все же существует такая схема. Сопоставим каждому агенту точку на плоскости. Условимся называть точку так же, как соответствующего агента, например, точка 005 соответствует агенту 005. Две точки соединим стрелкой с направлением от первой ко второй, если агент, соответствующий первой, следит за агентом, соответствующим второй. Проанализируем теперь наши построения:

1) Из каждой точки выходит ровно одна стрелка. Это следует из того, что каждый агент следит ровно за одним другим.

2) В каждую точку направлена ровно одна стрелка. Действительно, пусть мы нашли точку, являющуюся концом более одной стрелки. Тогда за соответствующим агентом следят более одного агента. Пусть эта точка 005, тогда за всеми агентами, следящими за 005, следит агент 004. Но 004 следит ровно за одним агентом. В остальных случаях рассуждения аналогичны.

Из утверждений 1) и 2) следует, что мы можем расположить наши точки в вершинах правильного восьмиугольника так, чтобы стрелки лежали на сторонах этого восьмиугольника. Действительно, поместим точку 001 в какую-нибудь вершину. В следующую (в направлении часовой стрелки) вершину поместим точку того агента, за которым следит 001. Соответствующую сторону восьмиугольника отметим стрелкой в направлении от 001. Будем продолжать этот процесс до тих пор, пока не дойдем до агента, точка которого была уже помещена в вершину.

3) Пусть в ходе процесса, описаннного выше, последней в вершину восьмиугольника была помещена точка А. Тогда агент А следит за 001. И вправду, от А должна идти стрелка к одной из уже помещенных в вершины точек. Но единственная точка, к которой мы еще не провели стрелку, есть точка А.

4) Все восемь точек были помещены в вершины. Действительно, 001 была помещена в вершину, по условию задачи через одну вершину от нее была помещена 002, через одну вершину от 002 была помещена 003 и так далее.

Теперь уже хорошо видно противоречие: например, покрасим вершины восьмиугольника в черный и белый цвет так, чтобы соседние вершины были разного цвета. Пусть 001 белого цвета, тогда 002 тоже белого (потому что следующая за 001 вершина черная, а следующая далее --- белая), 003 тоже белая по тем же соображениям, так же белая 004 и 005 и 006 и 007 и 008. Но ведь половина вершин должна быть черного цвета!

5A. Пусть $A_k$ и $A_{k+m}$ --- искомые точки, то есть длина $A_kA_{k+m}$ равна 564. По формуле суммы арифметической прогрессии длина $A_kA_{k+m}$ будет
$$
k+(k+1)+(k+2)+...+(k+m-1)=mk+\frac{m(m-1)}{2},
$$
то есть
\begin{equation}
\label{e} mk+\frac{m(m-1)}{2}=564.
\end{equation}
Заметим, что $\frac{m(m-1)}{2}$ целое число, так как либо $m$, либо $m-1$ четно.

Пусть $m$ нечетно, тогда левая часть (\ref{e}) делится на $m$. Значит 564 должно делиться на $m$. Из того, что $564=4\cdot 3\cdot 47$ заключаем, что возможны четыре случая:

1) $m=141$. Подставляем в (\ref{e}):
$$
k+70=4.
$$
Но это невозможно при натуральных $k$.

2) $m=47$. В этом случае (\ref{e}) вырождается в
$$
k+23=12,
$$
но такое равенство невозможно при натуральных $k$.

3) $m=3$. Подставив в (\ref{e}), имеем:
$$
k+1=188,
$$
откуда $k=187$.

4) $m=1$. Опять же из (\ref{e}) следует, что $k=564$.

Пусть теперь $m$ четное. Запишем $m=2l$ и подставим в (\ref{e}):
$$
2lk+l(2l-1)=564.
$$
Левая часть равенства делится на $l$, поэтому и 564 должно делиться на $l$. Поделим обе части на $l$
$$
2(k+l)-1=\frac{564}{l}.
$$
Левая часть нечетна, потому и правая часть должна быть нечетной. Из того, что $564=4\cdot3\cdot47$, следует, что для этого необходимо, чтобы $l$ делилось на 4. Запишем теперь $l=4n$ и подставим это в полученное равенство:
\begin{equation}
\label{f} 2(k+4n)-1=\frac{141}{n}.
\end{equation}
Снова возможно несколько случаев:

1) $n=141$. Подставляя в (\ref{f}) имеем
$$
2(k+4\cdot 141)-1=1,
$$
что невозможно при натуральных $k$. Так же проверяется, что случай $n=47$ невозможен.

2) $n=3$. Тогда (\ref{f}) вырождается в
$$
2k+23=47,
$$
откуда находим $k=12$.

3) $n=1$. Снова подставляем в (\ref{f}):
$$
2k+7=141,
$$
откуда $k=67$.

Итак, мы получили четыре отрезка:
$$
\begin{array}{l}
[A_{564},A_{565}],\\
\left [A_{187},A_{190}\right ],\\
\left [A_{12},A_{36}\right ],\\
\left [A_{67},B_{75}\right ].
\end{array}
$$

5В. Можно. Проведем первую прямую $l_1$, выберем на ней точку $O$. Построим луч $a_1$ с началом в точке $O$ под углом $27^{\circ}$ к $l_1$ (все углы будем мерить против часовой стрелки). Далее построим луч $a_2$, выходящий из точки $O$ под углом $27^{\circ}$ к $a_1$. Продолжим построения до тих пор, пока не построим луч $a_{10}$. По построению $a_{10}$ составляет угол $270^{\circ}$ с $l_1$, то есть перпендикулярен ей (потому что $360^{\circ}-270^{\circ}=90^{\circ}$). Прямая, содержащая луч $a_{10}$, перпендикулярна прямой $l_1$.

5С. См. задачу 5В 9го класса.

6А. Например, можно продлить его стороны, тогда вертикальный ему угол  можно будет измерить.

6В. Для треугольника с основанием $C_{k}C_{k+1}$ обозначим $B_{k+1}$ его третью вершину. Для треугольников с основаниями $AC_1$ и $C_nB$ третьи вершины обозначим $B_1$ и $B_{n+1}$ соответственно. В введенных обозначениях на сумму всех указанных сторон выполняются следующие соотношения:
$$
\begin{array}{l}
(AB_1+B_1C_1)+(C_1B_2+B_2C_2)+(C_2B_3+B_3C_3)+...+(C_{n-1}B_n+B_nC_n)+\\
+(C_nB_{n+1}+B_{n+1}B)=2(AC_1+C_1C_2+C_2C_3+...+C_{n-1}C_n+C_nB)=\\
=2AB,
\end{array}$$
откуда видно, что указанная сумма не зависит от выбранных точек $C_1,...,C_n$.

6С. Рассмотрим какой-нибудь угол данного кубика. К нему прилегает три квадратика, причем все они попарно касаются друг друга по стороне. Поэтому рассматриваемые квадратики представляют все три цвета. Значит, при каждом угле нашего кубика есть ровно один квадратик данного цвета. Поскольку углов всего 8, каждый цвет представлен 8 раз.

\end{document}
