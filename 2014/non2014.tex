\documentclass[10pt]{scrbook} \usepackage{modules/nonstahp_book}
\usepackage{mathspec}

\setmainfont[
	Path = f/,
	BoldFont=pb.ttf,
	ItalicFont=pi.ttf,
	BoldItalicFont=pbi.ttf
		]{p.ttf}
\setsansfont[
	Path = f/,
	BoldFont=pb.ttf,
	ItalicFont=pi.ttf,
	BoldItalicFont=pbi.ttf
		]{p.ttf}
		
\setmathfont(Digits)[Path = f/]{p.ttf}
\setmathfont(Latin)[Path = f/]{pi.ttf}
\setmathfont(Greek)[Path = f/, Uppercase]{p.ttf}
\setmathfont(Greek)[Path = f/, Lowercase]{pi.ttf}

\setmonofont[Path = f/]{pmono.ttf}

%\setCJKmainfont[
%	Path=f/,
%	BoldFont=notoserifb.ttf,
%	ItalicFont=notoserifi.ttf,
%	BoldItalicFont=notoserifbi.ttf
%		]{notoserif.ttf}

 \begin{document}




\medbreak
\noindent
{\bf Перед Вами 10 задач, на выполнение которых отводится 3 часа. Пункт А каждой задачи оценивается в 3 балла, пункт В --- в 6 баллов, пункт С --- в  9 баллов. Из каждой задачи в зачёт идёт только один пункт, в котором Вы набрали наибольшее количество баллов. Экспертная комиссия вправе снизить баллы за неполное обоснование задачи и добавить 1--2 балла за оригинальность решения. Если в работе представлен только ответ, он не рассматривается и не оценивается. Желаем успеха.}
\medbreak
\noindent

\vfil\eject
\centerline{\bf НЦ "ЛАБОРАТОРИЯ НЕПРЕРЫВНОГО МАТЕМАТИЧЕСКОГО ОБРАЗОВАНИЯ"}
\centerline{\bf ОЛИМПИАДА "МАТЕМАТИКА НОН-СТОП" — 2014}
\bigbreak 
\centerline{\bf (7 класс)}
\medbreak
\noindent
{\bf Перед Вами  10  задач, на выполнение которых отводится  3 часа. Пункт А каждой задачи оценивается в 3 балла, пункт В - в 6 баллов, пункт С - в  9 баллов. В зачет идет из каждой задачи только один пункт, в котором Вы набрали наибольшее количество баллов. Экспертная комиссия вправе снизить баллы за неполное обоснование задачи и добавить 1-2 балла за оригинальность решения. Одни ответы не рассматриваются и не оцениваются. Желаем успеха.}
\medbreak
\noindent
ЗАДАЧА 1.



\vfill\eject
\centerline{\bf НЦ "ЛАБОРАТОРИЯ НЕПРЕРЫВНОГО МАТЕМАТИЧЕСКОГО ОБРАЗОВАНИЯ"}
\centerline{\bf ОЛИМПИАДА "МАТЕМАТИКА НОН-СТОП" — 2014}
\bigbreak
\centerline{\bf  (6 класс)}
\medbreak
\noindent
{\bf Перед Вами  8  задач, на выполнение которых отводится  2 часа 30 минут. Пункт А каждой задачи оценивается в 3 балла, пункт В - в 6 баллов, пункт С - в  9 баллов. В зачет идет из каждой задачи только один пункт, в котором Вы набрали наибольшее количество баллов. Экспертная комиссия вправе снизить баллы за неполное обоснование задачи и добавить 1-2 балла за оригинальность решения. Одни ответы не рассматриваются и не оцениваются. Желаем успеха.}
\medbreak
\noindent
ЗАДАЧА 1. 

А:\qquad Боря в лесу каждые 10 метров находил гриб. Сколько метров он прошел от первого найденного гриба до последнего, если он собрал 30 грибов?

B:\qquad Существует ли такое двузначное число, что если переставить в нем цифры, то оно станет в 3 раза больше?

C:\qquad Найдите наименьшее число, сумма цифр которого делится на 3, на 5 и на 7, если в записи этого числа могут быть использованы только цифры 3, 5 и 7 (не обязательно каждая).
\medbreak
\noindent
ЗАДАЧА 2.

A:\qquad На плоскости расположены 200 точек. Сколькими способами можно построить отрезки с концами в этих точках?

B:\qquad На доске $4\times 4$ произвольно расставлены 6 фишек. Верно ли, что всегда существуют такие две строки и такие два столбца, что все фишки обязательно в них находятся?

C:\qquad На плоскости дано $n$ точек. Можно ли соединить эти точки несамопересекающейся ломаной с вершинами в этих точках?
\medbreak
\noindent
ЗАДАЧА 3.

A:\qquad На какое наименьшее число частей можно разрезать прямоугольник $4\times 9,$ чтобы из них можно было сложить квадрат $6\times 6$?

B:\qquad У мальчика Бори имеется прямоугольник $10\times 12$ и квадратик $1\times 1.$ Может ли Боря разрезать этот прямоугольник на 2 части, не являющиеся прямоугольниками, а затем из этих двух частей и данного квадратика сложить квадрат $11\times 11$?

C:\qquad Мальчик Боря хочет замостить без пропусков и перекрытий квадрат $7\times 7$ плиточками размера $1\times 5$ и $2\times 3.$ Сколько плиток ему понадобится? Приведите пример такого замощения.  Можно ли обойтись другим количеством плиток?
\medbreak
\noindent
ЗАДАЧА 4. 

A:\qquad Космический корабль потерпел аварию в 80 км от базы. На корабле есть 8 аккумуляторов, каждый из которых может обеспечить жизнь космонавта в течение суток. Космонавт может нести с собой только 3 аккумулятора и может проходить 20 км в сутки. Может ли космонавт добраться до базы?

B:\qquad Во время Олимпиады в Сочи бутик А не работает по понедельникам, а бутик В --- по вторникам; бутик С работает ежедневно, а бутик Д --- только по понедельникам, вторникам и средам. Однажды в будний день светские дамы  Ксюша, Настя, Лолита и Алла отправились в бутики за покупками, причем каждая --- в нужный ей бутик. По пути они встретились и обменялись следующими репликами:

Алла: ``А я могла бы пойти и вчера, и завтра''.\par
Настя: ``Я не хотела идти сегодня, но завтра я не смогу купить то, что мне нужно''.\par
Лолита: ``Я могла бы пойти и вчера, и позавчера''.

\noindent
В какой день происходил разговор? В какие бутики шли эти светские дамы?
 
C:\qquad 20 блюдечек расставлены по кругу, на каждом лежит по конфете. Родители разрешили мальчику Вовочке брать конфеты с блюдечек при условии, что он будет послушно придерживаться следующего правила: он может взять конфету с любого (по его выбору) блюдца, но при этом, если хотя бы одно из соседних блюдцев пусто, Вова должен вернуть взятую им конфету, положив ее на одно из этих пустых блюдечек. Если же в обоих соседних блюдцах есть конфеты, то взятую конфету Вовочка забирает себе. Какое наибольшее число конфет может забрать Вовочка?
\medbreak
\noindent
ЗАДАЧА 5. 

A:\qquad Борина комната обладает одним поразительным свойством. Если "поставить ее на любой  бок", то площадь комнаты не уменьшится. Высота потолка в комнате Бори равна 3 м. Какова наибольшая площадь такой комнаты?

B:\qquad  Мальчику Диме родители подарили инновационную линейку и инновационные кубики (кубиков очень много и они все одинаковы). Кроме того, у Димы есть инновационный стол --- он может ставить на него кубики, но только на грань (не на ребро и не на вершину). Также кубики можно ставить друг на друга. Какое наименьшее число кубиков надо взять Диме, чтобы линейкой измерить большую диагональ кубика? Дима не знает теорему Пифагора, калькулятора у него нет.

C:\qquad Дан куб. Какие-то его две вершины окрашены в черный цвет, остальные — в белый. Разрешается менять цвет вершин следующим образом: выбрать какую-нибудь грань и все черные вершины перекрасть в белые, а все белые — в черные.  Затем можно выбрать грань и поступить также и т.д. Можно ли этими операциями перекрасить вершины куба в один цвет?
\medbreak
\noindent
ЗАДАЧА 6.

A:\qquad  Кого больше: котов, кроме тех котов, которые не Васьки, или Васек, кроме тех Васек, которые не являются котами?

B:\qquad На острове Буяне житель либо всегда лжет, либо всегда говорит правду. Антон, Борис и Вовочка — жители острова Буяна. Один из них сказал: "Антон и Борис — оба лжецы", а другой из них — "Борис и Вова — оба лжецы". Кто из них сказал какое утверждение, неизвестно. Сколько же лжецов среди них?

C:\qquad Четыре оппозиционные партии ``Единый франт'', ``Все по барабану'', ``Красная пилюля'' и ``Мысли в тумбочке'' после мининга на Кардане решили объединиться. На объединительной конференции присутствовало поровну делегатов от каждой партии. Разногласия возникли при выборе названия новой партии. Для голосования были отобраны два названия --- ``Единый в тумбочке'' и ``Мысли по барабану''.  Известно, что все делегаты из ``Красной пилюли'' хотят голосовать за одно и то же название. А мнения делегатов от других партий разделились. Среди делегатов из ``Единого франта'' столько же хотят голосовать за первое название, сколько делегатов от ``Все по барабану'' --- за второе. Среди приверженцев второго названия одну треть составляют делегаты ``Мысли в тумбочке''. Как будет называться новая партия?
\medbreak
\noindent
ЗАДАЧА 7.

A:\qquad Три синих попугая капитана Флинта съедают 3 кг корма за 3 дня, 5 зеленых попугаев --- 5 кг корма за 5 дней, а 7 оранжевых --- 7 кг корма за 7 дней. Какие попугаи самые прожорливые?

B:\qquad Маугли попросил обезьян принести ему орехов. Каждая обезьяна собрала одно и то же число орехов и понесла их Маугли.
Но обезьяны имеют обыкновение ссориться по дороге. Во время ссоры
каждая обезьяна бросает в каждую другую обезьяну по ореху. Сколько раз ссорились обезьяны по дороге, если в конце концов Маугли досталось 33 ореха? По скольку орехов собрала каждая обезьяна? Сколько было обезьян?

C:\qquad Каких чисел больше среди всех чисел от 100 до 999: тех, у которых средняя цифра больше обеих крайних, или тех, у которых средняя цифра меньше обеих крайних?
\medbreak
\noindent
ЗАДАЧА 8.

A:\qquad Квадрат некоторого числа состоит из цифр 1,\ 2,\ 5,\ 5,\ 6. Найдите это число.

B:\qquad В клетках таблицы $8\time8$ записаны числа $-1,\ 0$ и 1. Может ли быть так, что суммы чисел по строкам, столбцам и большим диагоналям были все различны?

C:\qquad На доске выписаны семь чисел: 123,\ 234,\ 345,\ 456,\ 567,\ 678 и 789. За один ход разрешается выбрать по цифре из каких-либо чисел и поменять их местами. После хода вычисляется сумма всех полученных чисел. Какое наименьшее значение суммы может получиться при таких ходах? Какое наимеьшее число ходов нужно сделать, чтобы получилась наименьшая сумма?
\vfill\eject
\centerline{\bf НЦ "ЛАБОРАТОРИЯ НЕПРЕРЫВНОГО МАТЕМАТИЧЕСКОГО ОБРАЗОВАНИЯ"}
\centerline{\bf ОЛИМПИАДА "МАТЕМАТИКА НОН-СТОП" — 2014}
\bigbreak
\centerline{\bf  (5 класс)}
\medbreak
\noindent
{\bf Перед Вами  6  задач, на выполнение которых отводится  2 часа. Пункт А каждой задачи оценивается в 3 балла, пункт В - в 6 баллов, пункт С - в  9 баллов. В зачет идет из каждой задачи только один пункт, в котором Вы набрали наибольшее количество баллов. Экспертная комиссия вправе снизить баллы за неполное обоснование задачи и добавить 1-2 балла за оригинальность решения. Одни ответы не рассматриваются и не оцениваются. Желаем успеха.}
\medbreak
\noindent
ЗАДАЧА 1.

A:\qquad Бабушка печет блины и выкладывает их на тарелку. К приходу внука из школы она успевает испечь 20 блинов. Внук Боря, придя из школы, начинает сразу их есть. Пока Боря съедает 4 блина, бабушка успевает испечь, и выкладывает на тарелку  3 новых блина. Боря убежал играть в компьютерные игры, съев 24 блина. Сколько блинов осталось на тарелке?

B:\qquad На столе стоит несколько тарелок (больше одной). На каждой тарелке лежат сливы. Если добавить на каждую тарелку одно и то же для всех тарелок число слив, то общее число слив на столе увеличится на 11. Сколько тарелок стоит на столе?

C:\qquad Вова и Дима независимо друг от друга задумали по числу. Затем каждый из мальчиков умножил задуманное число на 11 и зачеркнул в произведении цифру десятков, после чего каждый из них умножил результат на 7 и опять зачеркнул в полученном произведении цифру десятков. В результате у каждого получилось число 23. Можно ли утверждать, что мальчики задумали одно и то же число?
\medbreak
\noindent
ЗАДАЧА 2.

A:\qquad Жил был Поп Толоконный лоб. Поп нанял для работы 8 работников. Каждый из них ежедневно съедал по буханке хлеба. На следующий год Поп нанял работника Ивана, который работает за пятерых, а ест за троих, и работника Балду, который работает за троих, а ест за двоих. Сколько буханок хлеба будет экономить Поп каждый день?

B:\qquad На далекой Третьей планете созвездия Медузы время  считается не так, как на Земле. Сутки там длятся 15 часов (1 час — 60 минут), в неделе — 5 суток, в месяце — 3 недели, в году — 13 месяцев. Алисе 10 лет, а капитану  Бурану, всю жизнь прожившему  на Третьей планете, — 30 лет. Кто из них старше?

C:\qquad Мальчик Боря утверждает, что он может в клетки квадрата $5\times 5$ расставить числа 0 или 1 так, что в каждом квадрате $2\times 2$ будет стоять ровно три одинаковых числа. Прав ли Борис? Какое наибольшее значение может принимать сумма чисел в этом квадрате?
\medbreak
\noindent
ЗАДАЧА 3.

A:\qquad Семь Заек-Зазнаек съедают 7 морковок за 7 дней. За сколько дней девять  Заек-Зазнаек съедят 9 морковок?

B:\qquad Вот стихотворение:

Мышка ночью пошла гулять.

Кошка ночью видит --- мышка!

Мышку кошка пошла поймать.

А вот перевод (построчный) этого стишка на язык племени Ам-Ям:

Ам ту му ям,

Ту ля бу ам,

Гу ля ту ям.

Составьте фрагмент русско--ам-ямского словаря по этому переводу.

C:\qquad Сумасшедший клоун написал на доске два не последовательных натуральных числа и стал ежесекундно прибавлять к ним по единице. Какие числа могли быть написаны, если известно, что через некоторое время из них получились числа, имеющие общий делитель, больший 1?
\medbreak
\noindent
ЗАДАЧА 4. 

A:\qquad  Какие два целых числа, в десятичной записи не содержащие нулей, дадут в произведении миллион?

B:\qquad Квадрат некоторого числа состоит из цифр 1,\ 2,\ 5,\ 5,\ 6. Найдите это число.

C:\qquad Существует ли такое двузначное число, что если переставить в нем цифры, то оно станет в 3 раза больше?
\medbreak
\noindent
ЗАДАЧА 5.

A:\qquad В игре "Кто хочет стать миллионером" участвуют двое. Первый называет любое целое число от 1 до 9 включительно. Затем второй прибавляет к названному числу любое целое число от 1 до 9 включительно, затем это снова делает первый и т.д. Выигрывает тот, кто первый назовет число $1\ 000\ 000.$ В этой игре начинающий  всегда проигрывает, если только противник откроет один секрет, который обеспечивает ему победу. В чем этот секрет?

B:\qquad На какое наименьшее число частей можно разрезать прямоугольник $4\times 9,$ чтобы из них можно было сложить квадрат $6\times 6$? 

C:\qquad Из клетчатой бумаги, клетки которой --квадраты $1\times 1,$ вырезали 150 прямоугольников, площадь каждого из которых не более 21. Все разрезы проводились только по сторонам клеток. Существует гипотеза, что среди прямоугольников не менее пяти одинаковых. Проверьте эту гипотезу.
\medbreak
\noindent
ЗАДАЧА 6.

A:\qquad Винни-Пух и его друзья поселились в одном доме: Винни-Пух на первом этаже, Пятачок --- на третьем, Кролик --- на шестом. До квартиры Винни-Пуха ступенек нет. Поднимаясь к себе домой, Пятачок прошел 30 ступенек, а потом решил зайти в гости к братцу Кролику. Сколько еще ступенек он пройдет?

B:\qquad В трехзначном числе цифру сотен увеличили на 3, цифру десятков — на 2 и цифру единиц — на 1. В результате получилось новое трехзначное число, в 4 раза больее исходного. Найти исходное число. 

C:\qquad 175 камешков расположены в неравных (по количеству камешков) кучек. Оказалось, что камешки из двух кучек можно поровну разделить по остальным кучкам, причем в каждую кучку добавится 10 камешков. Сколько всего имеется кучек?


\end{document}