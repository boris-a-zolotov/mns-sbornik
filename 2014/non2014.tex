\documentclass[10pt]{scrbook} \usepackage{modules/nonstahp_book}
\usepackage{mathspec}

\setmainfont[
	Path = f/,
	BoldFont=pb.ttf,
	ItalicFont=pi.ttf,
	BoldItalicFont=pbi.ttf
		]{p.ttf}
\setsansfont[
	Path = f/,
	BoldFont=pb.ttf,
	ItalicFont=pi.ttf,
	BoldItalicFont=pbi.ttf
		]{p.ttf}
		
\setmathfont(Digits)[Path = f/]{p.ttf}
\setmathfont(Latin)[Path = f/]{pi.ttf}
\setmathfont(Greek)[Path = f/, Uppercase]{p.ttf}
\setmathfont(Greek)[Path = f/, Lowercase]{pi.ttf}

\setmonofont[Path = f/]{pmono.ttf}

%\setCJKmainfont[
%	Path=f/,
%	BoldFont=notoserifb.ttf,
%	ItalicFont=notoserifi.ttf,
%	BoldItalicFont=notoserifbi.ttf
%		]{notoserif.ttf}

 \begin{document}




\medbreak
\noindent
{\bf Перед Вами 10 задач, на выполнение которых отводится 3 часа. Пункт А каждой задачи оценивается в 3 балла, пункт В --- в 6 баллов, пункт С --- в  9 баллов. Из каждой задачи в зачёт идёт только один пункт, в котором Вы набрали наибольшее количество баллов. Экспертная комиссия вправе снизить баллы за неполное обоснование задачи и добавить 1--2 балла за оригинальность решения. Если в работе представлен только ответ, он не рассматривается и не оценивается. Желаем успеха.}
\medbreak
\noindent
ЗАДАЧА 1.

 A:\qquad Космический корабль потерпел аварию в 80 км от базы. На корабле есть 8 аккумуляторов, каждый из которых может обеспечить жизнь космонавта в течение суток. Космонавт может нести с собой только 3 аккумулятора и может проходить 20 км в сутки. Может ли космонавт добраться до базы?

 B:\qquad Мальчик Боря утверждает, что он может в клетки квадрата $5\times 5$ расставить числа 0 или 1 так, что в каждом квадрате $2\times 2$ будет стоять ровно три одинаковых числа. Прав ли Борис? Какое наибольшее значение может принимать сумма чисел в этом квадрате?

 C:\qquad Числа 18 и 48 обладают свойством, что их квадраты --- числа 324 и 2116 --- имеют ту же самую сумму цифр, что и сами числа. Понятно, что если начать приписывать к этим числам нули, то указанное свойство сохранится. Тем самым, таких чисел бесконечно много. А будет ли таких чисел бесконечно много, если запретить им оканчиваться нулями?

\medbreak
\noindent
ЗАДАЧА 2.

A:\qquad  Кого больше: котов, кроме тех котов, которые не Васьки, или Васек, кроме тех Васек, которые не являются котами?


B:\quad Во время Олимпиады в Сочи бутик А не работает по понедельникам, а бутик В --- по вторникам; бутик С работает ежедневно, а бутик Д --- только по понедельникам, вторникам и средам. Однажды в будний день светские дамы  Ксюша, Настя, Лолита и Алла отправились в бутики за покупками, причем каждая --- в нужный ей бутик. По пути они встретились и обменялись следующими репликами:

Алла: ``А я могла бы пойти и вчера, и завтра''.\par
Настя: ``Я не хотела идти сегодня, но завтра я не смогу купить то, что мне нужно''.\par
Лолита: ``Я могла бы пойти и вчера, и позавчера''.

\noindent
В какой день происходил разговор? В какие бутики шли эти светские дамы?

C:\qquad В очереди в банке в кассу за кредитами стоит 28 человек, среди них тётушки Поля, Тоня и Света. Между Полей и кассой 15 человек, между Тоней и Полей --- 12 человек, между Светой и Полей стоит на одного человека больше, чем между Светой и Тоней. Сколько человек в очереди находятся между Светой и кассой?
\medbreak
\noindent
ЗАДАЧА 3.

A:\qquad Вот стихотворение:

Мышка ночью пошла гулять.

Кошка ночью видит --- мышка!

Мышку кошка пошла поймать.

А вот перевод (построчный) этого стишка на язык племени Ам-Ям:

Ам ту му ям,

Ту ля бу ам,

Гу ля ту ям.

Составьте фрагмент русско--ам-ямского словаря по этому переводу.

B:\qquad Четыре оппозиционные партии ``Единый франт'', ``Все по барабану'', ``Красная пилюля'' и ``Мысли в тумбочке'' после мининга на Кардане решили объединиться. На объединительной конференции присутствовало поровну делегатов от каждой партии. Разногласия возникли при выборе названия новой партии. Для голосования были отобраны два названия --- ``Единый в тумбочке'' и ``Мысли по барабану''.  Известно, что все делегаты из ``Красной пилюли'' хотят голосовать за одно и то же название. А мнения делегатов от других партий разделились. Среди делегатов из ``Единого франта'' столько же хотят голосовать за первое название, сколько делегатов от ``Все по барабану'' --- за второе. Среди приверженцев второго названия одну треть составляют делегаты ``Мысли в тумбочке''. Как будет называться новая партия?

С:\qquad Пограничники отметили, что число вещей, перевозимых эмигрантом  Витей в Арабские эмираты, совпадает с числом $N_1$ --- количеством натуральных чисел, меньших миллиона, в десятичной записи которых единиц больше, чем нулей, а число вещей, отправленных им в Швейцарию, совпадает с числом $N_2$ --- количеством натуральных чисел, меньших миллиона, в десятичной записи которых нулей больше, чем единиц. В какую страну  Витя отправил вещей больше?
\medbreak
\noindent
ЗАДАЧА 4.

A:\qquad Какие два целых числа, в десятичной записи не содержащие нулей, дадут в произведении миллион?

B:\qquad  Маугли попросил принести обезьян ему орехов. Каждая обезьяна собрала одно и то же число орехов и понесла их Маугли.
Но обезьяны имеют обыкновение ссорится по дороге. Во время ссоры
каждая обезьяна бросает в каждую другую обезьяну по ореху. Сколько раз ссорились обезьяны по дороге, если в конце концов Маугли досталось 33 ореха? По скольку орехов собрала каждая обезьяна? Сколько было обезьян?
\medbreak
\noindent

C:\qquad Пусть $a$ --- любое нечетное число, большее 3. Докажите, что предпоследняя цифра числа $a^2$ является четной.
\medbreak
\noindent
ЗАДАЧА 5.

A:\qquad Периметр треугольника составляет 137 м. Две его стороны в сумме составляют 100 м, а их разность --- 12 м. Найдите длины сторон этого треугольника.

B:\qquad Мальчику Диме родители подарили инновационную линейку и инновационные кубики (кубиков очень много и они все одинаковы). Кроме того, у Димы есть инновационный стол --- он может ставить на него кубики, но только на грань (не на ребро и не на вершину). Также кубики можно ставить друг на друга. Какое наименьшее число кубиков надо взять Диме, чтобы линейкой измерить большую диагональ кубика? Дима не знает теорему Пифагора, калькулятора у него нет.

C:\qquad Куб $1\times 1\times 1$ полностью оклеили шестью квадратами, общей площадью 6. Обязательно ли эти квадраты равны?
\medbreak
\noindent
ЗАДАЧА 6.

A:\qquad Золушка обшивает маленькую квадратную салфетку тесьмой по краю за 1 час. Сколько часов ей понадобится, чтобы обшить большую салфетку, площадь которой в 4 раза больше?

B:\qquad На стороне $BC$ прямоугольника $ABCD$ отмечены точки $K$ и $L,$ а на стороне $AD$ --- точка $M.$ Оказалось, что $KM$ --- биссектриса угла $AKC,$ а $LM$ --- биссектриса угла $BLD.$ Найдите сумму длин отрезков $BK$ и $LC,$ если $AK=6,\ KL=7$ и $LD=8.$

C:\qquad Найдите отношение длин диагоналей ромба, если известно, что одна из его диагоналей делится вписанной в ромб окружностью на три равные части.
\medbreak
\noindent
ЗАДАЧА 7.

A:\qquad Может ли прямоугольник площади 1 иметь диагональ длины 1000000?

B:\qquad  В треугольнике $ABC$ проведены биссектрисы $AK$ и $BM.$ Оказалось, что $AK=BM=AB.$ Найдите углы этого треугольника.

C:\qquad Отличница Пятеркина решила задачу по геометрии на построение треугольника. Сообразительный Вовочка завысил вдвое величины двух углов треугольника из этой задачи и уменьшил вдвое величину третьего. И тем не менее он смог построить треугольник с новыми углами. Какой наибольший угол мог быть в треугольнике отличницы Пятеркиной?
\medbreak
\noindent
ЗАДАЧА 8.

A:\qquad Три синих попугая капитана Флинта съедают 3 кг корма за 3 дня, 5 зеленых попугаев --- 5 кг корма за 5 дней, а 7 оранжевых --- 7 кг корма за 7 дней. Какие попугаи самые прожорливые?

B:\qquad Проверьте, что
$$
\frac{19^3-70^3}{19^3+89^3}=\frac{19-70}{19+89}.
$$
Обобщите задачу. Докажите обобщение.

C:\qquad Натуральные числа $m$ и $n$ удовлетворяют равенству
$$
(m-n)^2=\frac{4mn}{m+n-1}.
$$
Докажите, что их сумма есть квадрат натурального числа.
\medbreak
\noindent
 ЗАДАЧА 9.

 A:\qquad 5. Винни-Пух и его друзья поселились в одном доме: Винни-Пух на первом этаже, Пятачок --- на третьем, Кролик --- на шестом. До квартиры Винни-Пуха ступенек нет. Поднимаясь к себе домой, Пятачок прошел 30 ступенек, а потом решил зайти в гости к братцу Кролику. Сколько еще ступенек он пройдет?

 B:\qquad Сумасшедший клоун написал на доске два не последовательных натуральных числа и стал ежесекундно прибавлять к ним по единице. Какие числа могли быть написаны, если известно, что через некоторое время из них получились числа, имеющие общий делитель, больший 1?

С:\qquad Известно, что сумма каких-то трех из данных четырех двузначных чисел АА, ВВ, АВ и ВА равна 147 (А и В — цифры, отличные от нуля).Найдите все такие двузначные числа.

\medbreak
\noindent
ЗАДАЧА 10.

A:\qquad На плоскости расположены 4 точки. Сколькими способами можно построить отрезки, концами которых являются эти точки?

B:\qquad 175 камешков расположены в неравных (по количеству камешков) кучек. Оказалось, что камешки из двух кучек можно поровну разделить по остальным кучкам, причем в каждую кучку добавится 10 камешков. Сколько всего имеется кучек?

C:\qquad Кот Матроскин утверждает, что он может разбить прямую
на 5 равных попарно непересекающихся множеств, а Шарик утверждает, что он может разбить эту прямую на 7 таких множеств. Кто из них прав?
\vfil\eject
\centerline{\bf НЦ "ЛАБОРАТОРИЯ НЕПРЕРЫВНОГО МАТЕМАТИЧЕСКОГО ОБРАЗОВАНИЯ"}
\centerline{\bf ОЛИМПИАДА "МАТЕМАТИКА НОН-СТОП" — 2014}
\bigbreak 
\centerline{\bf (7 класс)}
\medbreak
\noindent
{\bf Перед Вами  10  задач, на выполнение которых отводится  3 часа. Пункт А каждой задачи оценивается в 3 балла, пункт В - в 6 баллов, пункт С - в  9 баллов. В зачет идет из каждой задачи только один пункт, в котором Вы набрали наибольшее количество баллов. Экспертная комиссия вправе снизить баллы за неполное обоснование задачи и добавить 1-2 балла за оригинальность решения. Одни ответы не рассматриваются и не оцениваются. Желаем успеха.}
\medbreak
\noindent
ЗАДАЧА 1.

А:\qquad Винни-Пух и его друзья поселились в одном доме. Винни-Пух поселился на первом этаже, Ослик Иа-Иа на втором, а Сова — на девятом. До квартиры Винни-Пуха ступенек нет. Однажды у Совы заболели крылья, и ей пришлось подниматься домой пешком. Во сколько раз больший путь нужно проделать Сове по сравнению с Осликом, когда он также идет к себе домой?

B:\qquad На столе стоят несколько (больше одной) тарелок. На каждой тарелке лежат конфеты, причем на разных тарелках — разные количества конфет, и ни одна тарелка не пустует. Если бы на каждую тарелку добавили бы некоторое, одно и то же для всех тарелок, число конфет, то общее число конфет на столе удвоилось бы. А если бы удвоили число конфет на каждой тарелке, то общее число конфет увеличилось бы на 21 конфету. Сколько тарелок стоит на столе?

С:\qquad Кот Матроскин считает, что он может нарисовать на плоскости 9 прямых так, что число пар пересекающихся прямых было равно числу пар параллельных прямых. Прав ли кот Матроскин? Изменится ли ответ, если число прямых будет 10?
\medbreak
\noindent
ЗАДАЧА 2.

A:\qquad Какое наименьшее число жильцов нужно вселить в 30-ти квартирный дом, чтобы в любых наугад выбранных трех квартир проживало не менее 7 человек?

 B:\qquad Вова и Дима независимо друг от друга задумали по числу. Затем каждый из мальчиков умножил задуманное число на 11 и зачеркнул в произведении цифру десятков, после чего каждый из них умножил результат на 7 и опять зачеркнул в полученном произведении цифру десятков. В результате у каждого получилось число 23. Можно ли утверждать, что мальчики задумали одно и то же число?

 C:\qquad Из числа $a$ вычли число $b,$ из числа $b$ вычли полученную разность, из первой разности вычли вторую разность и т.д. Для каких двузначных чисел $a$ и $b$ построенная последовательность будет содержать 0? Какое наибольшее число операций надо проделать, чтобы получить 0?
\medbreak
\noindent
ЗАДАЧА 3.

A:\qquad Квадрат некоторого числа состоит из цифр 1,\ 2,\ 5,\ 5,\ 6. Найдите это число.

B:\qquad  Может ли число $a^2-b^2+c^2$ делиться на 5, если ни одно из целых чисел $a,b$ и $c$ не делится на 5?

C:\qquad Найдите наименьшее натуральное число, которое не менее чем четыремя различными способами можно представить в виде $14m+41n,$ где $m$ и $n$ — натуральные числа.
\medbreak
\noindent
ЗАДАЧА 4.

A:\qquad Семь Заек-Зазнаек съедают 7 морковок за 7 дней. Сколько дней девять  Заек-Зазнаек съедят 9 морковок?

B:\qquad В равенствах
$$
a:b=c,\quad c+d=e,\quad e-f=g,\ g\cdot h=10i+j
$$
расставьте вместо букв 10 цифр $0,\ 1,\ 2,\ \ldots,\ 9$ так, чтобы получились верные равенства.

C:\qquad Какое наименьшее число различных чисел от 1 до 9 нужно выбрать, чтобы любое число от 1 до 100 можно представить в виде суммы выбранных чисел, при этом  не разрешается использовать в каждой сумме  никакую из них более 4 раз?
\medbreak
\noindent
ЗАДАЧА 5.

A:\qquad В клетках таблицы $8\time8$ записаны числа $-1,\ 0$ и 1. Может ли быть так, что суммы чисел по строкам, столбцам и большим диагоналям были все различны?

B:\qquad  Можно ли вписать в клетки квадрата $3\times 3$ числа от 1 до 9 (каждое из чисел — ровно 1 раз, и в каждую клетку — ровно одно число) так, чтобы суммы чисел во всех строчках и столбцах были 1)\ нечетные? 2)\ четные?

C:\qquad Дан квадрат, состоящий из $n\times n$ клеток. Для каких $n$ в этом квадрате можно расскрасить клетки в три цвета так, чтобы у каждой клетки среди ее соседних (т.е. с общей стороной) были клетки двух других цветов?
\medbreak
\noindent
ЗАДАЧА 6.

A:\qquad В игре "Кто хочет стать миллионером" участвуют двое. Первый называет любое целое число от 1 до 9 включительно. Затем второй прибавляет к названному числу любое целое число от 1 до 9 включительно, замем это снова делает первый и т.д. Выигрывает тот, кто первый назовет число  $1\ 000\ 000.$ В этой игре начинающий  всегда проигрывает, если только противник откроет один секрет, который обеспечивает ему победу. В чем этот секрет?

B:\qquad В ряд посажены $n>2$ деревьев. На каждом дереве висит табличка, которая указывает, сколько дубов имеется в следующей группе деревьев (группа это само дерево и предыдущее и последующее к нему дерево, если такие существуют). При каком наименьшем $n$  по этим табличкам не всегда можно установить, какие из этих деревьев являются дубами?

C:\qquad Мальчики Вова и Дима по очереди составляют $2p$-значное число, используя только цифры 6,\ 7,\ 8 и 9. Первую цифру числа пишет Вова, вторую — Дима, третью — Вова и т.д. При каких $p$ Дима может добиться того, что полученное число будет делиться на 9? 
\medbreak
\noindent
ЗАДАЧА 7.

A:\qquad Жил был Поп Толоконный лоб. Поп нанял для работы 8 работников. Каждый из них ежедневно съедал по буханке хлеба. На следующий год Поп нанял работника Ивана, который работает за пятерых, а ест за троих, и работника Балду, который работает за троих, а ест за двоих. Сколько буханок хлеба будет экономить Поп каждый день?

B:\qquad При каких $n>3$ на плоскости можно расположить $n$ точек и соединить их отрезками, так, чтобы из каждой точки выходило по 3 отрезка, и никакие из отрезков не пересекались по внутренним точкам?

C:\qquad Имеется $10\times 10$ клеток. Линиями, параллельными ее краям, она разбита на 100 единичных квадратных клеток. В некоторых клетках таблицы расставлено по фишке так, что в любой строке и в любом столце находится ровно одна фишка. Рассматриваются два квадрата: левый нижний квадрат размера $7\times 7$ и правый верхний квадрат размера $3\times 3.$ В каком из них фишек больше? На сколько?
\medbreak
\noindent
ЗАДАЧА 8.

A:\qquad Борина комната обладает одним поразительным свойством. Если "поставить ее на любой  бок", то площадь комнаты не уменьшится. Высота потолка в комнате Бори равна 3 м. Какова наибольшая площадь такой комнаты?

B:\qquad Докажите, что существует 2014 различных пар дробей $\frac{p}{q}$ и $\frac{m}{n}$ таких, что разность дробей в каждой паре равна их произведению.

C:\qquad Докажите, что если число $N$ не делится ни на 2, ни на 5, то найдется число, делящееся на $N,$ десятичная запись которого состоит из одних единиц.
\medbreak
\noindent
ЗАДАЧА 9.

A:\qquad Бабушка печет блины и выкладывает их на тарелку. К приходу внука из школы она успевает испечь 20 блинов. Внук Боря, придя из школы, начинает сразу их есть. Пока Боря съедает 4 блина, бабушка успевает испечь, и выкладывает на тарелку  3 новых блина. Боря убежал играть в компьютерные игры, съев 24 блина. Сколько блинов осталось на тарелке?

B:\qquad Каких чисел больше среди всех чисел от 100 до 999: тех, у которых средняя цифра больше обеих крайних, или тех, у которых средняя цифра меньше обеих крайних?

C:\qquad Числа от 1 до 100 записаны по одному в клетку таблицы $10\times 10.$  Назовем клетку {\tt почти наибольшей,} если среди ее соседей (по горизонтали, по вертикали или диагонали) не более чем одно число больше числа, записанного в клетку. Какое наибольшее число {\tt почти наибольших} клеток может быть в этой таблице?
\medbreak
\noindent
ЗАДАЧА 10.

A:\qquad На далекой Третьей планете созвездия Медузы время  считается не так как на Земле. Сутки там длятся 15 часов (1 час — 60 минут), в неделе — 5 суток, в месяце — 3 недели, а год — 13 месяцев. Алисе 10 лет, а капитану  Бурану, всю жизнь прожившему  на Третьей планете, — 30 лет. Кто из них старше? 

B:\qquad На острове Буяне живут представители 100 национальностей. Национальным меньшинством считается любая национальность  А, для которой найдутся не менее 50 национальностей, каждая из которых имеет численность вчетверо или больше превосходящую численность национальности А. Какое наибольшее (в процентном отношении) количество жителей страны могут считать себя представителями национальных меньщинств?

C:\qquad Про четырехзначное натуральное число $X$ известно, что 1)\ первые две цифры равны между собой;\ 2)\ последние две цифры равны между собой; 3)\ число является квадратом натурального числа. Найдите число $X.$
\vfill\eject
\centerline{\bf НЦ "ЛАБОРАТОРИЯ НЕПРЕРЫВНОГО МАТЕМАТИЧЕСКОГО ОБРАЗОВАНИЯ"}
\centerline{\bf ОЛИМПИАДА "МАТЕМАТИКА НОН-СТОП" — 2014}
\bigbreak
\centerline{\bf  (6 класс)}
\medbreak
\noindent
{\bf Перед Вами  8  задач, на выполнение которых отводится  2 часа 30 минут. Пункт А каждой задачи оценивается в 3 балла, пункт В - в 6 баллов, пункт С - в  9 баллов. В зачет идет из каждой задачи только один пункт, в котором Вы набрали наибольшее количество баллов. Экспертная комиссия вправе снизить баллы за неполное обоснование задачи и добавить 1-2 балла за оригинальность решения. Одни ответы не рассматриваются и не оцениваются. Желаем успеха.}
\medbreak
\noindent
ЗАДАЧА 1. 

А:\qquad Боря в лесу каждые 10 метров находил гриб. Сколько метров он прошел от первого найденного гриба до последнего, если он собрал 30 грибов?

B:\qquad Существует ли такое двузначное число, что если переставить в нем цифры, то оно станет в 3 раза больше?

C:\qquad Найдите наименьшее число, сумма цифр которого делится на 3, на 5 и на 7, если в записи этого числа могут быть использованы только цифры 3, 5 и 7 (не обязательно каждая).
\medbreak
\noindent
ЗАДАЧА 2.

A:\qquad На плоскости расположены 200 точек. Сколькими способами можно построить отрезки с концами в этих точках?

B:\qquad На доске $4\times 4$ произвольно расставлены 6 фишек. Верно ли, что всегда существуют такие две строки и такие два столбца, что все фишки обязательно в них находятся?

C:\qquad На плоскости дано $n$ точек. Можно ли соединить эти точки несамопересекающейся ломаной с вершинами в этих точках?
\medbreak
\noindent
ЗАДАЧА 3.

A:\qquad На какое наименьшее число частей можно разрезать прямоугольник $4\times 9,$ чтобы из них можно было сложить квадрат $6\times 6$?

B:\qquad У мальчика Бори имеется прямоугольник $10\times 12$ и квадратик $1\times 1.$ Может ли Боря разрезать этот прямоугольник на 2 части, не являющиеся прямоугольниками, а затем из этих двух частей и данного квадратика сложить квадрат $11\times 11$?

C:\qquad Мальчик Боря хочет замостить без пропусков и перекрытий квадрат $7\times 7$ плиточками размера $1\times 5$ и $2\times 3.$ Сколько плиток ему понадобится? Приведите пример такого замощения.  Можно ли обойтись другим количеством плиток?
\medbreak
\noindent
ЗАДАЧА 4. 

A:\qquad Космический корабль потерпел аварию в 80 км от базы. На корабле есть 8 аккумуляторов, каждый из которых может обеспечить жизнь космонавта в течение суток. Космонавт может нести с собой только 3 аккумулятора и может проходить 20 км в сутки. Может ли космонавт добраться до базы?

B:\qquad Во время Олимпиады в Сочи бутик А не работает по понедельникам, а бутик В --- по вторникам; бутик С работает ежедневно, а бутик Д --- только по понедельникам, вторникам и средам. Однажды в будний день светские дамы  Ксюша, Настя, Лолита и Алла отправились в бутики за покупками, причем каждая --- в нужный ей бутик. По пути они встретились и обменялись следующими репликами:

Алла: ``А я могла бы пойти и вчера, и завтра''.\par
Настя: ``Я не хотела идти сегодня, но завтра я не смогу купить то, что мне нужно''.\par
Лолита: ``Я могла бы пойти и вчера, и позавчера''.

\noindent
В какой день происходил разговор? В какие бутики шли эти светские дамы?
 
C:\qquad 20 блюдечек расставлены по кругу, на каждом лежит по конфете. Родители разрешили мальчику Вовочке брать конфеты с блюдечек при условии, что он будет послушно придерживаться следующего правила: он может взять конфету с любого (по его выбору) блюдца, но при этом, если хотя бы одно из соседних блюдцев пусто, Вова должен вернуть взятую им конфету, положив ее на одно из этих пустых блюдечек. Если же в обоих соседних блюдцах есть конфеты, то взятую конфету Вовочка забирает себе. Какое наибольшее число конфет может забрать Вовочка?
\medbreak
\noindent
ЗАДАЧА 5. 

A:\qquad Борина комната обладает одним поразительным свойством. Если "поставить ее на любой  бок", то площадь комнаты не уменьшится. Высота потолка в комнате Бори равна 3 м. Какова наибольшая площадь такой комнаты?

B:\qquad  Мальчику Диме родители подарили инновационную линейку и инновационные кубики (кубиков очень много и они все одинаковы). Кроме того, у Димы есть инновационный стол --- он может ставить на него кубики, но только на грань (не на ребро и не на вершину). Также кубики можно ставить друг на друга. Какое наименьшее число кубиков надо взять Диме, чтобы линейкой измерить большую диагональ кубика? Дима не знает теорему Пифагора, калькулятора у него нет.

C:\qquad Дан куб. Какие-то его две вершины окрашены в черный цвет, остальные — в белый. Разрешается менять цвет вершин следующим образом: выбрать какую-нибудь грань и все черные вершины перекрасть в белые, а все белые — в черные.  Затем можно выбрать грань и поступить также и т.д. Можно ли этими операциями перекрасить вершины куба в один цвет?
\medbreak
\noindent
ЗАДАЧА 6.

A:\qquad  Кого больше: котов, кроме тех котов, которые не Васьки, или Васек, кроме тех Васек, которые не являются котами?

B:\qquad На острове Буяне житель либо всегда лжет, либо всегда говорит правду. Антон, Борис и Вовочка — жители острова Буяна. Один из них сказал: "Антон и Борис — оба лжецы", а другой из них — "Борис и Вова — оба лжецы". Кто из них сказал какое утверждение, неизвестно. Сколько же лжецов среди них?

C:\qquad Четыре оппозиционные партии ``Единый франт'', ``Все по барабану'', ``Красная пилюля'' и ``Мысли в тумбочке'' после мининга на Кардане решили объединиться. На объединительной конференции присутствовало поровну делегатов от каждой партии. Разногласия возникли при выборе названия новой партии. Для голосования были отобраны два названия --- ``Единый в тумбочке'' и ``Мысли по барабану''.  Известно, что все делегаты из ``Красной пилюли'' хотят голосовать за одно и то же название. А мнения делегатов от других партий разделились. Среди делегатов из ``Единого франта'' столько же хотят голосовать за первое название, сколько делегатов от ``Все по барабану'' --- за второе. Среди приверженцев второго названия одну треть составляют делегаты ``Мысли в тумбочке''. Как будет называться новая партия?
\medbreak
\noindent
ЗАДАЧА 7.

A:\qquad Три синих попугая капитана Флинта съедают 3 кг корма за 3 дня, 5 зеленых попугаев --- 5 кг корма за 5 дней, а 7 оранжевых --- 7 кг корма за 7 дней. Какие попугаи самые прожорливые?

B:\qquad Маугли попросил обезьян принести ему орехов. Каждая обезьяна собрала одно и то же число орехов и понесла их Маугли.
Но обезьяны имеют обыкновение ссориться по дороге. Во время ссоры
каждая обезьяна бросает в каждую другую обезьяну по ореху. Сколько раз ссорились обезьяны по дороге, если в конце концов Маугли досталось 33 ореха? По скольку орехов собрала каждая обезьяна? Сколько было обезьян?

C:\qquad Каких чисел больше среди всех чисел от 100 до 999: тех, у которых средняя цифра больше обеих крайних, или тех, у которых средняя цифра меньше обеих крайних?
\medbreak
\noindent
ЗАДАЧА 8.

A:\qquad Квадрат некоторого числа состоит из цифр 1,\ 2,\ 5,\ 5,\ 6. Найдите это число.

B:\qquad В клетках таблицы $8\time8$ записаны числа $-1,\ 0$ и 1. Может ли быть так, что суммы чисел по строкам, столбцам и большим диагоналям были все различны?

C:\qquad На доске выписаны семь чисел: 123,\ 234,\ 345,\ 456,\ 567,\ 678 и 789. За один ход разрешается выбрать по цифре из каких-либо чисел и поменять их местами. После хода вычисляется сумма всех полученных чисел. Какое наименьшее значение суммы может получиться при таких ходах? Какое наимеьшее число ходов нужно сделать, чтобы получилась наименьшая сумма?
\vfill\eject
\centerline{\bf НЦ "ЛАБОРАТОРИЯ НЕПРЕРЫВНОГО МАТЕМАТИЧЕСКОГО ОБРАЗОВАНИЯ"}
\centerline{\bf ОЛИМПИАДА "МАТЕМАТИКА НОН-СТОП" — 2014}
\bigbreak
\centerline{\bf  (5 класс)}
\medbreak
\noindent
{\bf Перед Вами  6  задач, на выполнение которых отводится  2 часа. Пункт А каждой задачи оценивается в 3 балла, пункт В - в 6 баллов, пункт С - в  9 баллов. В зачет идет из каждой задачи только один пункт, в котором Вы набрали наибольшее количество баллов. Экспертная комиссия вправе снизить баллы за неполное обоснование задачи и добавить 1-2 балла за оригинальность решения. Одни ответы не рассматриваются и не оцениваются. Желаем успеха.}
\medbreak
\noindent
ЗАДАЧА 1.

A:\qquad Бабушка печет блины и выкладывает их на тарелку. К приходу внука из школы она успевает испечь 20 блинов. Внук Боря, придя из школы, начинает сразу их есть. Пока Боря съедает 4 блина, бабушка успевает испечь, и выкладывает на тарелку  3 новых блина. Боря убежал играть в компьютерные игры, съев 24 блина. Сколько блинов осталось на тарелке?

B:\qquad На столе стоит несколько тарелок (больше одной). На каждой тарелке лежат сливы. Если добавить на каждую тарелку одно и то же для всех тарелок число слив, то общее число слив на столе увеличится на 11. Сколько тарелок стоит на столе?

C:\qquad Вова и Дима независимо друг от друга задумали по числу. Затем каждый из мальчиков умножил задуманное число на 11 и зачеркнул в произведении цифру десятков, после чего каждый из них умножил результат на 7 и опять зачеркнул в полученном произведении цифру десятков. В результате у каждого получилось число 23. Можно ли утверждать, что мальчики задумали одно и то же число?
\medbreak
\noindent
ЗАДАЧА 2.

A:\qquad Жил был Поп Толоконный лоб. Поп нанял для работы 8 работников. Каждый из них ежедневно съедал по буханке хлеба. На следующий год Поп нанял работника Ивана, который работает за пятерых, а ест за троих, и работника Балду, который работает за троих, а ест за двоих. Сколько буханок хлеба будет экономить Поп каждый день?

B:\qquad На далекой Третьей планете созвездия Медузы время  считается не так, как на Земле. Сутки там длятся 15 часов (1 час — 60 минут), в неделе — 5 суток, в месяце — 3 недели, в году — 13 месяцев. Алисе 10 лет, а капитану  Бурану, всю жизнь прожившему  на Третьей планете, — 30 лет. Кто из них старше?

C:\qquad Мальчик Боря утверждает, что он может в клетки квадрата $5\times 5$ расставить числа 0 или 1 так, что в каждом квадрате $2\times 2$ будет стоять ровно три одинаковых числа. Прав ли Борис? Какое наибольшее значение может принимать сумма чисел в этом квадрате?
\medbreak
\noindent
ЗАДАЧА 3.

A:\qquad Семь Заек-Зазнаек съедают 7 морковок за 7 дней. За сколько дней девять  Заек-Зазнаек съедят 9 морковок?

B:\qquad Вот стихотворение:

Мышка ночью пошла гулять.

Кошка ночью видит --- мышка!

Мышку кошка пошла поймать.

А вот перевод (построчный) этого стишка на язык племени Ам-Ям:

Ам ту му ям,

Ту ля бу ам,

Гу ля ту ям.

Составьте фрагмент русско--ам-ямского словаря по этому переводу.

C:\qquad Сумасшедший клоун написал на доске два не последовательных натуральных числа и стал ежесекундно прибавлять к ним по единице. Какие числа могли быть написаны, если известно, что через некоторое время из них получились числа, имеющие общий делитель, больший 1?
\medbreak
\noindent
ЗАДАЧА 4. 

A:\qquad  Какие два целых числа, в десятичной записи не содержащие нулей, дадут в произведении миллион?

B:\qquad Квадрат некоторого числа состоит из цифр 1,\ 2,\ 5,\ 5,\ 6. Найдите это число.

C:\qquad Существует ли такое двузначное число, что если переставить в нем цифры, то оно станет в 3 раза больше?
\medbreak
\noindent
ЗАДАЧА 5.

A:\qquad В игре "Кто хочет стать миллионером" участвуют двое. Первый называет любое целое число от 1 до 9 включительно. Затем второй прибавляет к названному числу любое целое число от 1 до 9 включительно, затем это снова делает первый и т.д. Выигрывает тот, кто первый назовет число $1\ 000\ 000.$ В этой игре начинающий  всегда проигрывает, если только противник откроет один секрет, который обеспечивает ему победу. В чем этот секрет?

B:\qquad На какое наименьшее число частей можно разрезать прямоугольник $4\times 9,$ чтобы из них можно было сложить квадрат $6\times 6$? 

C:\qquad Из клетчатой бумаги, клетки которой --квадраты $1\times 1,$ вырезали 150 прямоугольников, площадь каждого из которых не более 21. Все разрезы проводились только по сторонам клеток. Существует гипотеза, что среди прямоугольников не менее пяти одинаковых. Проверьте эту гипотезу.
\medbreak
\noindent
ЗАДАЧА 6.

A:\qquad Винни-Пух и его друзья поселились в одном доме: Винни-Пух на первом этаже, Пятачок --- на третьем, Кролик --- на шестом. До квартиры Винни-Пуха ступенек нет. Поднимаясь к себе домой, Пятачок прошел 30 ступенек, а потом решил зайти в гости к братцу Кролику. Сколько еще ступенек он пройдет?

B:\qquad В трехзначном числе цифру сотен увеличили на 3, цифру десятков — на 2 и цифру единиц — на 1. В результате получилось новое трехзначное число, в 4 раза больее исходного. Найти исходное число. 

C:\qquad 175 камешков расположены в неравных (по количеству камешков) кучек. Оказалось, что камешки из двух кучек можно поровну разделить по остальным кучкам, причем в каждую кучку добавится 10 камешков. Сколько всего имеется кучек?


\end{document}