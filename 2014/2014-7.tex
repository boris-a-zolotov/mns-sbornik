\secklas{7}

\taskno{1}

\begin{itemize}

	\itA Винни-Пух и его друзья поселились в одном доме. Винни-Пух поселился на первом этаже, Ослик Иа-Иа на втором, а Сова — на девятом. До квартиры Винни-Пуха ступенек нет. Однажды у Совы заболели крылья, и ей пришлось подниматься домой пешком. Во сколько раз больший путь нужно проделать Сове по сравнению с Осликом, когда он также идет к себе домой?
	
	\itr Ослику нужно c первого этажа подняться на один этаж, чтобы попасть к себе домой, а Сове — на восемь. Поэтому ей нужно проделывать в восемь раз больший путь.

	\itB На столе стоят несколько (больше одной) тарелок. На каждой тарелке лежат конфеты, причем на разных тарелках — разные количества конфет, и ни одна тарелка не пустует. Если бы на каждую тарелку добавили бы некоторое, одно и то же для всех тарелок, число конфет, то общее число конфет на столе удвоилось бы. А если бы удвоили число конфет на каждой тарелке, то общее число конфет увеличилось бы на 21 конфету. Сколько тарелок стоит на столе?
	
	\itr Удваивая количество конфет, мы увеличиваем это количество на 21. Значит, всего конфет 21. Удвоения количества конфет (то есть, опять же, увеличения на 21) можно также добиться, если добавить на каждую тарелку одинаковое количество конфет. Значит, количество тарелок — 3, 7 или 21.
	
	Разложить 21 конфету по 21 тарелке так, чтобы ни одна не пустовала, можно единственным способом: на каждую тарелку по конфете. Но тогда на всех тарелках будет одинаковое количество конфет — это не то, что требуется в условии. Чтобы разложить конфеты на 7 тарелок, их должно быть хотя бы $1 + \ldots + 7 = 28$. Поэтому единственный оставшийся ответ — 3 тарелки: например, на них могло лежать 5, 7 и 9 конфет соответственно.

	\itC Кот Матроскин считает, что он может нарисовать на плоскости 9 прямых так, что число пар пересекающихся прямых было равно\linebreak числу пар параллельных прямых. Прав ли кот Матроскин? Изменится ли ответ, если число прямых будет 10?

	\itr Из 9 прямых можно сформировать $9 \cdot 8\,/\,2 = 36$ пар. Значит, Матроскин хочет добиться, чтобы было 18 пар параллельных прямых и 18 пересекающихся. Этого легко добиться, расположив прямые следующим образом:
	
	\begin{center} \tikz{
		\foreach \x in {0,...,5} {\draw[thick] (0.7 * \x cm, -1) -- (0.7 * \x cm, 2.4); }
		\foreach \y in {0,1,2} {\draw[thick] (-1, 0.7 * \y cm) -- (4.5, 0.7 * \y cm); }
	} \end{center}
	
	Если же прямых было бы 10, то они составляли бы $10 \cdot 9\ :\ 2 = 45$ пар — а нечетное количество пар прямых не может разделиться поровну.

\end{itemize}

\taskno{2}

\begin{itemize}

	\itA Какое наименьшее число жильцов нужно вселить в 30-ти квартирный дом, чтобы в любых наугад выбранных трех квартирах проживало не менее 7 человек?
	
	\itr Заметим, что количество жильцов может не превосходить 2 не более чем в двух квартирах, так как иначе возьмем три таких «маленьких» квартиры, в них не наберется 7 жильцов.
	
	Рассмотрим две самых малонаселенных квартиры. В них может\linebreak жить 2, 2, или 2, 1, или 1, 1, или больше жильцов. В первом случае в оставшихся квартирах как минимум по 3 жильца, во втором случае в каждой из оставшихся квартир минимум по 4 жильца, и в третьем случае в каждой из оставшихся квартир минимум по 5 жильцов. 
	
	Посчитав минимальное число жильцов в каждом из случаев выше, получаем, что нижня оценка на количество жильцов в доме —
	$$28 \cdot 7 + 4 = 88\quad\text{жильцов.}$$
	
	88 жильцов действительно можно поселить: в две квартиры по 2 человека, в остальные по 3.

	\itB Вова и Дима независимо друг от друга задумали по числу. Затем каждый из мальчиков умножил задуманное число на 11 и зачеркнул в произведении цифру десятков, после чего каждый из них умножил результат на 7 и опять зачеркнул в полученном произведении цифру десятков. В результате у каждого получилось число 23. Можно ли утверждать, что мальчики задумали одно и то же число?
	
	\itr Нет, утверждать однозначность нельзя: рассмотрим два различных числа, 19 и 29, и проделаем с ними операции из условия.
\begin{align*}
	& 19 \rightarrow 209 \rightarrow  29 \rightarrow 203 \rightarrow 23; \\
	& 29 \rightarrow 319 \rightarrow 39 \rightarrow 273 \rightarrow 23.
\end{align*}

	Получился один и тот же результат — значит, мальчики вполне могли задумать различные числа.

\end{itemize}

\taskno{5}

\begin{itemize}

	\itB Можно ли вписать в клетки квадрата $3\times 3$ числа от 1 до 9 (каждое из чисел — ровно 1 раз, и в каждую клетку — ровно одно число) так, чтобы суммы чисел во всех строчках и столбцах были (а) нечетные? (б) четные?
	
	\itr Сделать все суммы нечетными просто —
	\begin{center} \begin{tabular}{c|c|c}
		2 & 1 & 4 \\ \hline
		3 & 5 & 7 \\ \hline
		6 & 9 & 8 
	\end{tabular} \end{center}
	
	Сделать все суммы четными невозможно: рассмотрим три стро-\linebreak ки~— сумма всех чисел в таблице равна сумме сумм в строках. Если все суммы в строках были бы четны, то и сумма всех чисел в таблице обязана была бы оказаться четной. А она равна $1 + \ldots + 9 = 45$.

\end{itemize}

\taskno{6}

\begin{itemize}

	\itC Мальчики Вова и Дима по очереди составляют $2m$--значное число, используя только цифры 6, 7, 8 и 9. Первую цифру числа пишет Вова, вторую — Дима, третью — Вова и т.д. При каких $m$ Дима может добиться того, что полученное число будет делиться на 9?
	
	\itr При $m$, делящемся на 3, у Димы выигрышная стратегия, разумеется, есть: после каждого кода Вовы ему нужно писать цифру, дающую в сумме с написанной Вовой число 15. Тогда после $2m$ ходов сумма цифр, написанных ребятами, будет делиться на 45 — а значит полученное число окажется делящимся на 9.
	
	Осталось привести «выигрышную стратегию» для Вовы в остальных случаях — то есть, такой план действий, который лишит Диму возможности добиться желаемого.
	
	\begin{enumerate}[label=\arabic*.]
	
	\item Пусть $m$ дает остаток 1 по модулю 3.
	
	Первым своим ходом Вова может написать цифру 7.
	
	Далее перед каждым ходом Вовы следует ход Димы. Вова должен «дополнять» написанную Димой цифру до 15.
	
	Тогда перед последним ходом Димы сумма написанных цифр будет иметь остаток 7 при делении на 9. Прибавление любого из чисел 6–9 к семи не сделает сумму цифр (а значит и записанное число) делящейся на 9.
	
	\item Если же $m$ дает остаток 2 по модулю 3, то рассмотрим отдельно первый ход Вовы и последние три хода: Дима–Вова–Дима.
	
	Первым ходом Вова должен написать цифру 8, потом, вплоть до последних трех ходов, — «дополнять» цифру, написанную Димой на предыдущем ходе, до 15. Наконец, на предпоследнем ходе игры Вова должен написать цифру 9.
	
	Тогда сумма цифр, которую получат мальчики в конце игры, будет равна
	$$8 + 15 \cdot 3k + 9 + x_1 + x_2.$$
	
	Здесь $x_1$ и $x_2$ — цифры, которые напишет Дима в результате двух своих последних ходов, пока нами не обсуждавшихся. Эта сумма дает остаток 8 при делении на 9, и несложно убедиться перебором, что какие две цифры из данных нам к восьми ни прибавляй, сумма, кратная 9, получиться не может. 
	
	\end{enumerate}

\end{itemize}

\taskno{7}

\begin{itemize}

	\itB При каких $n>3$ на плоскости можно расположить $n$ точек и соединить их отрезками, так, чтобы из каждой точки выходило по 3 отрезка, и никакие из отрезков не пересекались по внутренним точкам?
	
	\itr При нечетных $n$ этого сделать нельзя: у отрезков, которые хочется провести, тогда будет $3n$ «концов» — а их должно быть четное число, потому что у каждого отрезка два конца.
	
	При $n=4$ точки можно расположить так:
	
	\begin{center} \tikz{
		\foreach \r in {0,1,2,3} {\filldraw [thick,rotate = 90*\r] (0,0) -- (1,0) circle[radius=0.8mm] -- (0,1); }
		\filldraw (0,0) circle[radius=0.8mm]; 
	} \end{center}
	
	При $n=6$ — следующим образом:
	
	\begin{center} \tikz{
		\filldraw (1.2,1) circle[radius=0.8mm] node(a){ }; 
		\filldraw (1.2,-1) circle[radius=0.8mm] node(b){ }; 
		\filldraw (-1.2,-1) circle[radius=0.8mm] node(c){ }; 
		\filldraw (-1.2,1) circle[radius=0.8mm] node(d){ }; 
		\draw[thick] (a) -- (b) -- (c) -- (d) -- (a); 
		\filldraw (-0.4,0) circle[radius=0.8mm] node(p){ }; 
		\filldraw (0.4,0) circle[radius=0.8mm] node(q){ }; 
		\draw[thick] (d) -- (p) -- (c); 
		\draw[thick] (a) -- (q) -- (b); 
		\draw[thick] (p) -- (q); 
	} \end{center}
	
	Наконец, любое четное число, большее 3, можно представить в виде суммы четверок и шестерок, поэтому, взяв несколько копий картинок, нарисованных выше, мы добьемся нужного нам числа вершин.

\end{itemize}

\taskno{8}

\begin{itemize}

	\itA Борина комната обладает одним поразительным свойством. Если «поставить ее на любой бок», то площадь комнаты не уменьшится. Высота потолка в комнате Бори равна 3 м. Какова наибольшая площадь такой комнаты?
	
	\itr Пусть длина комнаты равна $x$ метров, а ширина — $y$ метров. Тогда из условия задачи
	$$3\cdot x \ge x \cdot y,\qquad 3 \cdot y \ge x \cdot y.$$
	
	Отсюда $x$ и $y$ не превосходят трех, и максимальная площадь комнаты с такими сторонами равна 9 квадратным метрам.

\end{itemize}

\taskno{10}

\begin{itemize}

	\itB На острове Буяне живут представители 100 национальностей. Национальным меньшинством считается любая национальность $A$,\linebreak для которой найдутся не менее 50 национальностей, каждая из которых имеет численность вчетверо или больше превосходящую\linebreak численность национальности $A$. Какое наибольшее (в процентном отношении) количество жителей страны могут считать себя представителями национальных меньшинств?
	
	\itr Наша цель — сделать как можно больше как можно б\'oльших национальностей национальными меньшинствами, а как можно меньше как можно меньших национальностей — не национальными меньшинствами.
	
	Заметим, что 50 самых больших по численности национальностей меньшинствами являться не могут, поэтому логично хотеть, чтобы они все были одинакового, минимально возможного размера.
	
	Остальные же 50 национальностей можно сделать национальными меньшинствами, сделав их численность одинаковой и максимально доступной — то есть, равной $1/4$ от численности «больших» национальностей.
	
	Таким образом, меньшинства могут составлять максимум $25\%$ от населения государства.

	\itC Про четырехзначное натуральное число $X$ известно, что
	
	\begin{enumerate}[label=\arabic*)]
		\item первые две цифры равны между собой;
		\item последние две цифры равны между собой;
		\item число является квадратом натурального числа.
	\end{enumerate}
	
	Найдите число $X$.
	
	\itr То, что первые две цифры и вторые две цифры числа равны между собой, означает, что это число делится на 11. Раз оно квадрат натурального числа, то автоматически обязано делиться на квадрат 11 — 121. Также заметим, что $121 \cdot 8 < 1000$, а $121 \cdot 83 > 10000$. Таким образом, искомое число представляется в виде $121 \cdot n$, где $n$ — квадрат натурального числа и лежит в пределах от 9 до 82.
	
	Теперь, перебрав имеющиеся варианты, получим, что искомое число — $121 \cdot 64 = 7744$.
	
\end{itemize}