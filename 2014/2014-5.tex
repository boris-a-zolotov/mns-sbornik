\secklas{5}

\taskno{3}

\begin{itemize}
	\itB \lookPrev{8}{3A}
	
	\itC Сумасшедший клоун написал на доске два не последовательных натуральных числа и стал ежесекундно прибавлять к ним по единице. Какие числа могли быть написаны, если известно, что через некоторое время из них получились числа, имеющие общий делитель, больший 1?
	
	\itr Это могли быть какие угодно числа, потому как если разность между написанными числами была равна $a$~— то есть, они были равны $N$ и $N+a$ соответственно~— то найдётся число, делящееся на $a$ и большее $N$: $M= m \cdot a > N$.
	
	Прибавим тогда к каждому из чисел по $M - N$ единиц, получим числа
	$$M = m \cdot a\text{\ \ и\ \ }(N+a) + (M-N) = (m+1) \cdot a,$$
	у которых имеется общий делитель $a$, по условию больший единицы.
\end{itemize}

\taskno{5}

\begin{itemize}

	\itB На какое наименьшее число частей можно разрезать прямоугольник $4\times 9$, чтобы из них можно было сложить квадрат $6\times 6$? 

	\itr Разреза на две части будет достаточно:
	
	\begin{center} \tikz{
		\filldraw[fill=white,draw=white] (-0.5,-0.5) rectangle (5,2.5);
		\foreach \x in {1,...,8}
			{\draw [color={rgb:black,4;white,6}] (0.5 * \x cm, 0) -- (0.5 * \x cm, 2);}
		\foreach \x in {1,...,3}
			{\draw [color={rgb:black,4;white,6}] (0, 0.5 * \x cm) -- (4.5, 0.5 * \x cm);}
		\draw [very thick] (0,0) -- (0,2) -- (4.5,2) -- (4.5,0) -- cycle;
		\draw [very thick]
			(1.5,0) -- (1.5,1) --
			(3,1) -- (3,2);
	} \end{center}
	
	Составить из этих двух частей квадрат более чем просто: нужно подвинуть правую часть вниз и влево~— и тем самым расположить её под левой.

\end{itemize}

\taskno{6}

\begin{itemize}

	\itB В трехзначном числе цифру сотен увеличили на 3, цифру десятков~— на 2 и цифру единиц~— на 1. В результате получилось новое трехзначное число, в 4 раза большее исходного. Найти исходное число.
	
	\itr Единственный разряд, не подверженный переносам из более младших разрядов, — самый младший. Раз младшая цифра при умножении на 4 увеличилась на 1, то она была нечётной (потому что иначе она возросла бы на чётное число). Единственная подходящая под эти условия цифра — 7 ($7 \cdot 4 = 28$, а цифры 1, 3, 5 и 9 такими свойствами не обладают, в чём легко убедиться).
	
	Таким образом, последняя цифра в числе равна 7, и при умножении на 4 образовался перенос 2 в средний разряд. Так как при умножении на 4 цифра среднего разряда увеличилась на 2, мы можем вычесть из получившейся цифры перенос, тем самым получив, что средняя цифра числа остаётся неизменной при умножении на 4. Значит, она чётна — и подходит под это условие только 0.
	
	Отсюда число заканчивается на 07, и при умножении на 4 переноса из среднего разряда в старший не возникает. Поэтому старшая цифра при умножении на 4 увеличивается на 3 и не даёт никакого переноса в следующие разряды. Под это условие подходит только единица.
	
	Получаем ответ — 107.

\end{itemize}