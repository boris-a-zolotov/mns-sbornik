\secklas{8}
\taskno{1}

\def\km#1{\SI{#1}{\text{км}}}

\begin{enumerate}

	\itA Космический корабль потерпел аварию в \km{80} от базы. На корабле есть 8 аккумуляторов, каждый из которых может обеспечить жизнь космонавта в течение суток. Космонавт может нести с собой только 3 аккумулятора и может проходить \km{20} в сутки. Может ли космонавт добраться до базы?
	
	\itr Да, космонавт может добраться до базы за 6 дней.
	
	В первый день он может взять 3 аккумулятора и пройти \km{20}. Вечером первого дня в 20 километрах от сломанного корабля будут один использованный аккумулятор, два свежих аккумулятора и космонавт, а на корабле — 5 свежих аккумуляторов.
	
	Во второй день он возьмет один свежий аккумулятор и вернется на корабль, а в третий — возьмет с корабля три аккумулятора и пройдет \km{20} к базе, используя один из них. Таким образом, к вечеру третьего дня ситуация будет следующей:
	
	\begin{center}\begin{tabular}{|l|l|} \hline
		\km{20} от корабля & На корабле \\ \hline
	 	3 свежих аккумулятора & 2 свежих аккумулятора \\
	 	2 использованных аккум. & 1 использованный аккум. \\
	 	Космонавт & \\ \hline
	\end{tabular}\medskip \\\end{center}
	
	Теперь у космонавта есть 3 свежих аккумулятора, и ему осталось 3 дня пути до базы. Это уже легко преодолимо.

	\itB Мальчик Боря утверждает, что он может в клетки квадрата $5\times 5$ расставить числа 0 или 1 так, что в каждом квадрате $2\times 2$ будет стоять ровно три одинаковых числа. Прав ли Борис? Какое наибольшее значение может принимать сумма чисел в этом квадрате?
	
	\itr Рассмотрим 4 угловых квадрата $2 \times 2$ и заметим, что в каждом из них должно быть хотя бы по одному нулю. Значит, максимальная сумма чисел в квадрате ограничена сверху числом $25-4=21$. Приведем пример расстановки 21 единицы и 24 нулей в квадрате, удовлетворяющей условию:
	
	\begin{center} \begin{tabular}{c|c|c|c|c}
		1 & 1 & 1 & 1 & 1 \\ \hline
		1 & 0 & 1 & 0 & 1 \\ \hline
		1 & 1 & 1 & 1 & 1 \\ \hline
		1 & 0 & 1 & 0 & 1 \\ \hline
		1 & 1 & 1 & 1 & 1
	\end{tabular} \end{center}

	\itC Число 18 обладает следующим интересным свойством: его квадрат~— число 324~— имеет ту же самую сумму цифр, что и само число 18. Понятно, что если начать приписывать в конец числа нули, то указанное свойство сохранится. Тем самым, чисел с указанным свойством бесконечно много. А будет ли таких чисел бесконечно много, если запретить им оканчиваться нулями?
	
	\itr Любое число, состоящее только из девяток, обладает исследуемым нами свойством: несложно показать, что
	$${\underbrace{99\ldots 9}_{n}}^2\ \ =\ \ 
		\underbrace{9\ldots 9}_{n-1}8\underbrace{0\ldots 0}_{n-1}1.$$
	Сумма цифр у числа в правой части такая же, как у числа в левой.

\end{enumerate}

\taskno{3}

\begin{itemize}

	\itA Вот стихотворение:

\begin{quote}
Мышка ночью пошла гулять. \\
Кошка ночью видит --- мышка! \\
Мышку кошка пошла поймать.
\end{quote}

А вот перевод (построчный) этого стишка на язык племени Ам-Ям:

\begin{quote}
Ам ту му ям, \\
Ту ля бу ам, \\
Гу ля ту ям.
\end{quote}

Составьте фрагмент русско--ам-ямского словаря по этому переводу.

	\itr Единственное слово, которое встречается в третьей строке, но не встречается в первых двух, — «поймать». Поэтому оно соответствует слову «гу» языка племени. По аналогичным причинам «уникальное» слово второй строки — «видит» — соответствует слову «бу» племени.
	
	Единственное слово, которое встречается во всех строках, — «ту», значит, «мышка». В двух последних строках встречается «ля», которое соответствует кошке.
	
	Остальные слова распределить просто: «ам» — «ночью», «ям»~— «пошла», «му» — «гулять».

	\itC Пограничники отметили, что число вещей, перевозимых эмигрантом  Витей в Арабские эмираты, совпадает с числом $N_1$ --- количеством натуральных чисел, меньших миллиона, в десятичной записи которых единиц больше, чем нулей, а число вещей, отправленных им в Швейцарию, совпадает с числом $N_2$ --- количеством натуральных чисел, меньших миллиона, в десятичной записи которых нулей больше, чем единиц. В какую страну Витя отправил вещей больше?
	
	\itr Десятичная запись натурального числа не может начинаться с нуля. Рассмотрим те натуральные числа, запись которых, кроме того, не начинается с единицы.
	
	Покажем, что среди таких чисел поровну тех, в чьей записи больше единиц, и тех, в чьей записи больше нулей: возьмем число, в записи которого больше единиц, заменим все единицы на нули и все нули на единицы. Мы получим число, в записи которого больше нулей (так как ни один ноль не окажется ведущим). При этом разным числам сопоставляются разные числа, и любое число, в десятичной записи которого нулей больше, чем единиц, может быть получено таким образом.
	
	Теперь рассмотрим числа, запись которых начинается с единицы. Если взять такое число, в записи которого больше нулей, и заменить $0\rightarrow 1$ и $1\rightarrow 0$, мы несомненно получим число, в записи которого больше единиц, чем нулей. Однако, не все числа, в записи которых больше единиц, могут быть получены таким образом: например, число 110 может быть получено только из 101, в записи которого по-прежнему больше единиц.
	
	Значит, в целом количество чисел, в которых больше единиц, больше — отсюда больше вещей уехало в Арабские эмираты.

\end{itemize}