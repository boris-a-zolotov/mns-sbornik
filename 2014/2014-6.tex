\secklas{6}

\taskno{1}

\begin{itemize}

	\itA Боря в лесу каждые 10 метров находил гриб. Сколько метров он прошел от первого найденного гриба до последнего, если он собрал 30 грибов?
	
	\itr Давайте считать так — до каждого найденного Борей\linebreak гриба, кроме самого первого, ему надо было пройти 10 метров. Найденных грибов, кроме самого первого, всего 29 штук. Значит, пройдено было $29 \cdot 10 = 290$ метров.
	
	\itB Существует ли такое двузначное число, что если переставить в нем цифры, то оно станет в 3 раза больше?
	
	\itr Напомним читателю признаки делимости на 3 и на 9: число делится на 3 тогда и только тогда, когда его сумма цифр делится на 3. Аналогично, число делится на 9 только вместе со своей суммой цифр.
	
	Значит, если число, требуемое в условии, есть, то сумма его цифр делится на 3, так как из его цифр может быть составлено число, полученное умножением на 3. Значит, само число делится на 3.
	
	После перестановки его цифр, так как число увеличилось в три раза, результат стал делящимся на 9. Значит, сумма составляющих его цифр делилась на  9 — и исходное число делилось на 9.
	
	Есть всего 10 двузначных чисел, делящихся на 9 — одно из них равно 99 (очевидно не подходит под условие задачи), а цифры остальных имеют разную чётность (ведь их сумма равна 9), поэтому при их перестановке число поменяет чётность.
	
	При умножении же на 3 чётность числа не меняется, поэтому чисел, подходящих под условие задачи, нет.
	
	\itC Найдите наименьшее число, сумма цифр которого делится на 3, на 5 и на 7, если в записи этого числа могут быть использованы только цифры 3, 5 и 7 (не обязательно каждая).
	
	\itr Если сумма цифр числа делится на 3, 5 и 7, то она обязана делиться на их наименьшее общее кратное, равное 105. Мы построим число с суммой цифр, равной 105, так, что будет понятно: чисел с большей суммой цифр (210, 315, ...), меньших его, не бывает.
	
	Минимальное количество разрядов в подходящем нам числе может быть равно $105\,/\,7 = 15$, так как 7 — наибольшая цифра, которую мы можем использовать. Тогда рассмотрим число, состоящее из 15 семёрок:
	$$777\,777\,777\,777\,777.$$
	
	Оно имеет наименьшее возможное количество разрядов\scolon также в числе из 15 разрядов не могут быть использованы тройки и пятёрки, так как тогда сумма цифр обязана будет быть меньше 105. Значит, оно наименьшее — и является ответом на данную задачу.

\end{itemize}

\taskno{2}

\begin{itemize}

	\itB На доске $4\times 4$ произвольно расставлены 6 фишек. Верно ли, что всегда существуют такие две строки и такие два столбца, что все фишки обязательно в них находятся?
	
	\itr Если поставить 4 из имеющихся 6 фишек в клетки главной диагонали таблицы, то, очевиндно, двух строк и столбцов не найдётся.
	
\end{itemize}

\taskno{3}

\begin{itemize}
	\itB У мальчика Бори имеется прямоугольник $10\times 12$ и квадратик $1\times 1.$ Может ли Боря разрезать этот прямоугольник на 2 части, не являющиеся прямоугольниками, а затем из этих двух частей и данного квадратика сложить квадрат $11\times 11$?
	
	\itr Разрежем квадрат $11 \times 11$ на три части, одна из которых~— квадрат со стороной 1, а из двух других можно сложить прямоугольник $10 \times 12$:
	
	\def\dotn#1{\filldraw #1 circle [radius=0.24mm];}
	
	\begin{center} \tikz{
		\draw[thick] (1.5,0) -- (0,0) -- (0,1.5);
		\draw[thick, rotate around={180:(1.5,1.5)}] (1.5,0) -- (0,0) -- (0,1.5);
		
		\draw (3,2.5) -- (2.5,2.5);
		\foreach \x in {0,0.5,1.5,2} {
			\begin{scope}[xshift=\x cm,yshift = 0.5 cm + \x cm]
				\draw (0,0) -- (0.5,0) -- (0.5,0.5);
			\end{scope};
		}
		\draw[thick] (0,2.5) -- (0,3) -- (0.5,3);
		\draw[thick, rotate around={180:(1.5,1.5)}] (0,2.5) -- (0,3) -- (0.5,3);
		
		\dotn{(1.25,1.75)} \dotn{(1.1,1.6)} \dotn{(1.4,1.9)}
		
		\begin{scope}
			\dotn{(1,3)} \dotn{(0.7,3)} \dotn{(1.3,3)}
			\dotn{(3,1)} \dotn{(3,0.7)} \dotn{(3,1.3)}
		\end{scope}
		
		\begin{scope}[rotate around={180:(1.5,1.5)}]
			\dotn{(1,3)} \dotn{(0.7,3)} \dotn{(1.3,3)}
			\dotn{(3,1)} \dotn{(3,0.7)} \dotn{(3,1.3)}
		\end{scope};
		
		\draw[<->,thick] (0,3.35) -- (2.5,3.35);
		\draw (1.25,3.35) node[above]{10};
	} \end{center}

\end{itemize}

\taskno{4}

\begin{itemize}

	\itC 20 блюдечек расставлены по кругу, на каждом лежит по конфете. Родители разрешили мальчику Вовочке брать конфеты с блюдечек при условии, что он будет послушно придерживаться следующего правила: он может взять конфету с любого (по его выбору) блюдца, но при этом, если хотя бы одно из соседних блюдец пусто, Вова должен вернуть взятую им конфету, положив ее на одно из этих пустых блюдечек. Если же в обоих соседних блюдцах есть конфеты, то взятую конфету Вовочка забирает себе. Какое наибольшее число конфет может забрать Вовочка?
	
	\itr Больше 18 конфет Вовочка забрать не может — если среди конфет останутся две, то ни у одной из них не будет двух занятых соседних блюдец. Придумаем стратегию, как Вовочке взять ровно 18 конфет.
	
	Возьмём «каждую вторую» конфету из 20, лежащих в круг. Остальные конфеты можно «сдвинуть» так, чтобы они стали лежать подряд, забирая их с блюдец и помещая на одно из пустующих соседних.
	
	Теперь мы имеем 10 подряд лежащих конфет. Заберём себе вторую из них, возьмём с блюдца первую и переложим её на место второй. Осталось 9 подряд лежащих конфет. Повторяя описанную процедуру, можно получить 3 подряд лежацих конфеты — возьмём среднюю из них, и останется ровно две.

\end{itemize}