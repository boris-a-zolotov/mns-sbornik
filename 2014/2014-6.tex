\documentclass[10pt]{scrbook} \usepackage{modules/nonstahp_book}
\usepackage{mathspec}

\setmainfont[
	Path = f/,
	BoldFont=pb.ttf,
	ItalicFont=pi.ttf,
	BoldItalicFont=pbi.ttf
		]{p.ttf}
\setsansfont[
	Path = f/,
	BoldFont=pb.ttf,
	ItalicFont=pi.ttf,
	BoldItalicFont=pbi.ttf
		]{p.ttf}
		
\setmathfont(Digits)[Path = f/]{p.ttf}
\setmathfont(Latin)[Path = f/]{pi.ttf}
\setmathfont(Greek)[Path = f/, Uppercase]{p.ttf}
\setmathfont(Greek)[Path = f/, Lowercase]{pi.ttf}

\setmonofont[Path = f/]{pmono.ttf}

%\setCJKmainfont[
%	Path=f/,
%	BoldFont=notoserifb.ttf,
%	ItalicFont=notoserifi.ttf,
%	BoldItalicFont=notoserifbi.ttf
%		]{notoserif.ttf}

 \begin{document}
\renewcommand{\theyear}{2014}


\secklas{6}

\taskno{1}

\begin{itemize}

	\itA Боря в лесу каждые 10 метров находил гриб. Сколько метров он прошел от первого найденного гриба до последнего, если он собрал 30 грибов?
	
	\itr Давайте считать так — до каждого найденного Борей гриба, кроме самого первого, ему надо было пройти 10 метров. Найденных грибов, кроме самого первого, всего 29 штук. Значит, пройдено было $29 \cdot 10 = 290$ метров.

\end{itemize}










\end{document}




ЗАДАЧА 1. 

А:\qquad Боря в лесу каждые 10 метров находил гриб. Сколько метров он прошел от первого найденного гриба до последнего, если он собрал 30 грибов?

B:\qquad Существует ли такое двузначное число, что если переставить в нем цифры, то оно станет в 3 раза больше?

C:\qquad Найдите наименьшее число, сумма цифр которого делится на 3, на 5 и на 7, если в записи этого числа могут быть использованы только цифры 3, 5 и 7 (не обязательно каждая).
\medbreak
\noindent
ЗАДАЧА 2.

A:\qquad На плоскости расположены 200 точек. Сколькими способами можно построить отрезки с концами в этих точках?

B:\qquad На доске $4\times 4$ произвольно расставлены 6 фишек. Верно ли, что всегда существуют такие две строки и такие два столбца, что все фишки обязательно в них находятся?

C:\qquad На плоскости дано $n$ точек. Можно ли соединить эти точки несамопересекающейся ломаной с вершинами в этих точках?
\medbreak
\noindent
ЗАДАЧА 3.

A:\qquad На какое наименьшее число частей можно разрезать прямоугольник $4\times 9,$ чтобы из них можно было сложить квадрат $6\times 6$?

B:\qquad У мальчика Бори имеется прямоугольник $10\times 12$ и квадратик $1\times 1.$ Может ли Боря разрезать этот прямоугольник на 2 части, не являющиеся прямоугольниками, а затем из этих двух частей и данного квадратика сложить квадрат $11\times 11$?

C:\qquad Мальчик Боря хочет замостить без пропусков и перекрытий квадрат $7\times 7$ плиточками размера $1\times 5$ и $2\times 3.$ Сколько плиток ему понадобится? Приведите пример такого замощения.  Можно ли обойтись другим количеством плиток?
\medbreak
\noindent
ЗАДАЧА 4. 

A:\qquad Космический корабль потерпел аварию в 80 км от базы. На корабле есть 8 аккумуляторов, каждый из которых может обеспечить жизнь космонавта в течение суток. Космонавт может нести с собой только 3 аккумулятора и может проходить 20 км в сутки. Может ли космонавт добраться до базы?

B:\qquad Во время Олимпиады в Сочи бутик А не работает по понедельникам, а бутик В --- по вторникам; бутик С работает ежедневно, а бутик Д --- только по понедельникам, вторникам и средам. Однажды в будний день светские дамы  Ксюша, Настя, Лолита и Алла отправились в бутики за покупками, причем каждая --- в нужный ей бутик. По пути они встретились и обменялись следующими репликами:

Алла: ``А я могла бы пойти и вчера, и завтра''.\par
Настя: ``Я не хотела идти сегодня, но завтра я не смогу купить то, что мне нужно''.\par
Лолита: ``Я могла бы пойти и вчера, и позавчера''.

\noindent
В какой день происходил разговор? В какие бутики шли эти светские дамы?
 
C:\qquad 20 блюдечек расставлены по кругу, на каждом лежит по конфете. Родители разрешили мальчику Вовочке брать конфеты с блюдечек при условии, что он будет послушно придерживаться следующего правила: он может взять конфету с любого (по его выбору) блюдца, но при этом, если хотя бы одно из соседних блюдцев пусто, Вова должен вернуть взятую им конфету, положив ее на одно из этих пустых блюдечек. Если же в обоих соседних блюдцах есть конфеты, то взятую конфету Вовочка забирает себе. Какое наибольшее число конфет может забрать Вовочка?