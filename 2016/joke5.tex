\task{Шутка}
\begin{itemize}

\itA Поскольку это шутка, засчитывался любой минимально обоснованный ответ. 
Например, можно предположить, что на третий раз не сломается ни одной ножки, 
так как в ходе предыдущих двух падений все шаткие ножки уже сломались. 

Если же решать задачу всерьез, то легко увидеть, что указанных исходных данных
недостаточно для ответа на вопрос. 

\itB Обозначим стоимость книги за $b$. Тогда осталось заплатить $b - 200$ рублей.
Половина заплаченного --- $100$ рублей; половина заплаченного плюс то, что
осталось заплатить ---
	$$b - 200 + 100 = b - 100.$$
Осталось бы заплатить, если заплачена половина 
заплаченного и еще столько, сколько
осталось заплатить ---
	$$b - (b - 100) = 100.$$
И мы знаем, что $b - 200 = 3 \cdot 100$. Отсюда стоимость книги --- $500$ рублей.

\itC Путешественник отправился вокруг света на восток, его дорога 
заняла $80$ дней ($80 \cdot 24$ часов, чтобы быть точным). 
Каждый день путешественник ночует на $\frac{1}{80}$ длины 
экватора восточнее, то есть Cолнце в зените оказывается на $\frac{1}{80}$ часть
суток раньше, чем в месте предыдущей ночевки.

Таким образом, время между 
ночевками составляет у путешественника только $23.7$ часа (да, возможно он не будет 
высыпаться, но это уже другая история). 

В момент возврата в точку отправления окажется, что за $80$ дней для путешественника
солнце было в зените $80 \cdot 24 \div 23.7 \approx 81$ раз, и календари у путешественника
и у ждущих его людей разойдутся. Для избежания таких расхождений при движении с запада на восток
при пересечении линии перемены дат принято перелистывать календарь на один день назад.

Представим теперь, что путешественник обходит землю не рядом с экватором, 
а вокруг полюса по кругу маленького радиуса. В процессе обхода путешественник 
будет быстро сменять часовые пояса: если он, к примеру, начнет движение в полночь 
31 декабря от линии перемены дат, то через несколько сот метров он придет в место, 
где местное время уже 5 часов утра, потом, еще чуть дальше, 
будет 18 часов вечера... Путешественник за какой-нибудь час <<проживет>> день 31 декабря, 
подойдет к линии перемены дат в 23:59 по местному времени последнего перед 
линией часового пояса, подождет чуток, чтобы на часах появилось
0:00, поздравит всех с Новым годом, а затем, сделав еще два шага вперед 
через линию, вернется в ночь 31 декабря. 

Во всех этих случаях не совершается какого-то необычного путешествия во 
времени --- совершается путешествие по часовым поясам. 
А добавление или исчезновение 
дня возникает за счет изменения продолжительности суток. Космонавты на МКС,
совершающей оборот вокруг Земли каждые два часа, во избежание путаницы не
обращают внимание на местное время и живут по гринвичскому времени (по времени
нулевого меридиана).
\end{itemize}
