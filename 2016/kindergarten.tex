\task{Детский сад}
\begin{itemize}
\itA 10 детей рисуют 10 рисунков за 20 минут --- то есть каждый из детей рисует один рисунок в течении 20 минут.
50 детей будут, соответственно, рисовать 50 рисунков за те же 20 минут.
$d$ детей будут рисовать $r$ рисунков за $20 \cdot \lceil r \div d\rceil$ минут: распределим рисунки
более или менее поровну между детьми — и далее мы знаем, что один ребёнок рисует один рисунок за 20 минут.

\itB Площадь растёт быстрее у Вовы: например, потому, что б\'ольшую длину имеет внешняя часть обруча,
а ширина полосы прилепляемого пластилина та же; значит, расход материала с внешней части будет выше.

Тот же результат можно поулчить с помощью формул для площади круговой полосы, сравнив
	$$S_1 = \text{π}\ll\ll r_1+1\rr^2 - r_1^2\rr \text{\ \ и\ \ } S_2 = \text{π}\ll r_2^2 - \ll r_2-1\rr^2\rr.$$

\itC Всего возможно $4! = 24$ варианта расположения детей по весу (мы должны переставить 
детей в правильном порядке, и возможны все перестановки 4 элементов), 
поэтому будет требоваться не менее пяти взвешиваний (4 взвешивания 
дадут только $2^4 = 16$ вариантов ответа, что недостаточно для выбора перестановки).

Покажем, что пяти взвешиваний достаточно. 

Без уменьшения общности можем говорить не о детях, а о значениях их веса.
Пусть задан список весов $(w_1,w_2,w_3,w_4)$. 
Будем стремиться к тому, чтобы значения в нём были записаны по возрастанию веса, чтобы
если выполнено $i < j$, то выполнено $w_i < w_j$

Рассмотрим операцию \emph{сравнения и обмена} $(p,q)$: сравним $w_p$ и $w_q$, и если они 
стоят в списке в неправильном порядке (то есть $p < q$, но $w_p > w_q$) --- 
поменяем местами. Тогда требуемая перестановка будет получена, если будут выполнены
следующие операции сравнения и обмена: $(1,2)$, $(3,4)$, $(1,3)$, $(2,4)$, $(2,3)$.

Для наглядности изобразим эти сравнения и обмены в виде \emph{сортировочной сети}:
каждая горизонтальная линия соответствует элементу списка, вертикальные соединения ---
операции сравнения и обмена.

\begin{center}\tikz{
    \foreach \y [count = \n] in {1.5,1,0.5,0} {
        \draw (0,\y) node[left] {$w_\n$} -- (6,\y);
    }
    \foreach \a/\b [count=\x] in {1/2, 3/4, 1/3, 2/4, 2/3} {
        \draw (\x,2-\a / 2) -- (\x,2-\b / 2);
        \fill (\x,2-\a / 2) circle [radius=0.07];
        \fill (\x,2-\b / 2) circle [radius=0.07];
    }
}\end{center}

Продемонстрируем работу данных сравнений на примере: пусть у детей был вес в килограммах
$(20,25,15,10)$. Тогда:
\begin{enumerate}
\item сравнение и обмен $(1,2)$ не изменит список весов;
\item сравнив $(3,4)$, мы получим $(20,25,10,15)$;
\item сравнение $(1,3)$ даст $(10,25,20,15)$;
\item сравнение $(2,4)$ даст $(10,15,20,25)$;
\item сравнение $(2,3)$ оставит список без изменений.
\end{enumerate}

Почему это работает? После первых двух сравнений $w_1 < w_2$
и $w_3 < w_4$. Затем мы на третьем шагу выбираем общий минимум 
(сравнивая минимумы пар), на четвёртом --- общий максимум. 
И на пятом шагу правильно расставляем два оставшихся средних элемента.

\end{itemize}
