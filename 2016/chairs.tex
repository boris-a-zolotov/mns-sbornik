\task{Падающие стулья}
\begin{itemize}
\itA \lookPrev{5}{1A}

\itB 
%Слово \emph{гарантированно} означает, что 
%как бы мы ни выбирали подпиленные ножки, 
%как минимум $m$ стульев должны быть опасными.

%Предложим самую неудачную для Васи систему подпилов.
%Пусть Вася подпилит у каждого стула по одной ножке, у выбранных заранее
%$m-1$ стульев --- все три оставшиеся; и ещё у одного выбранного стула --- одну ножку.
%Это потребует 

Потребуется $n + m \cdot 3 - 2$ подпила минимум.

Предположим, что данного количества может не хватить, то есть существует
схема подпила, при которой опасным будет только $m-1$ стул. 
Стул безопасен, если у него не более 1 подпила, 
то есть остальные $n-m+1$ стульев имеют $n-m+1$ подпилов в сумме максимум.
Остаётся $n + m \cdot 3 - 2 - n + m -1 = m \cdot 4 - 3$ подпилов на $m-1$ стул,
то есть какой-то стул будет иметь 5 подпилов, что невозможно.

Однако, если неудачно подпилить $n + m \cdot 3 - 3$ ножек (у всех стульев по
одной и у выбранного $m-1$ ещё три оставшихся), то такого количества
подпилов может уже не хватить.

\itC Количество игр --- $4 \cdot ...$

\begin{itemize}
\item Если $a_1 = a_2 = 0$, то в игре будет ничья:
каждый мальчик, оказавшись перед перспективой проигрыша, будет воздерживаться от подпила.

\item Если $a_1 > 0$ и $a_2 = 0$, то выигрывает Вася, если же наоборот, $a_1 = 0$ и $a_2 > 0$,
то выигрывает Петя: мальчик, не имеющий возможности пропустить ход, будет вынужден в какой-то
момент подпилить у какого-то стула вторую ножку.

\item Если $n+1$ делится на $b_2$, то выигрышная тактика есть у Васи:
на любой ход Пети отвечать 


Поскольку перепиливания не выгодны никому из игроков, первый стул упадёт при
$n+1$ перепиливании --- у каждого стула по ножке, и какая-то ножка у следующего

\item 
\end{itemize}

В кафе $n$ четырёхногих стульев. Стул падает, если у него меньше 
трёх целых ножек. У мальчиков Васи и Пети есть две пилы, и они изобретают 
себе игру. Мальчики уже сошлись на том, что первым ходит Петя, а проигрывает 
тот, после чьего хода упадёт первый стул. Осталось выбрать возможное число 
перепиливаний за ход для каждого из них. Пусть за каждый ход Петя 
перепиливает не менее чем $a_1$ и не более чем $b_1$ ножек, Вася — от $a_2$ до 
$b_2$ ножек. Числа $a_1$ и $a_2$ могут быть равны нулю или единице, а 
числа $b_1$ и $b_2$ — $m$ или $m-1$, $m<n$, но при этом обязательно $b_1 \ne b_2$. 
Сколько игр удовлетворяет этим условиям, и кто из мальчиков выиграет в каждой из них?
\end{itemize}
