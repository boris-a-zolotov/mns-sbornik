\task{Падающие стулья}
\begin{itemize}
\itA \lookPrev{5}{1A}

\itB 
%Слово \emph{гарантированно} означает, что 
%как бы мы ни выбирали подпиленные ножки, 
%как минимум $m$ стульев должны быть опасными.

%Предложим самую неудачную для Васи систему подпилов.
%Пусть Вася подпилит у каждого стула по одной ножке, у выбранных заранее
%$m-1$ стульев --- все три оставшиеся; и ещё у одного выбранного стула --- одну ножку.
%Это потребует 

Потребуется $n + m \cdot 3 - 2$ подпила минимум.

Предположим, что данного количества может не хватить, то есть существует
схема подпила, при которой опасным будет только $m-1$ стул. 
Стул безопасен, если у него не более 1 подпила, 
то есть остальные $n-m+1$ стульев имеют $n-m+1$ подпилов в сумме максимум.
Остаётся $n + m \cdot 3 - 2 - n + m -1 = m \cdot 4 - 3$ подпилов на $m-1$ стул,
то есть какой-то стул будет иметь 5 подпилов, что невозможно.

Однако, если неудачно подпилить $n + m \cdot 3 - 3$ ножек (у всех стульев по
одной и у выбранного $m-1$ ещё три оставшихся), то такого количества
подпилов может уже не хватить.

\iffalse {
	В кафе $n$ четырёхногих стульев. Стул падает, если у него меньше 
	трёх целых ножек. У мальчиков Васи и Пети есть две пилы, и они изобретают 
	себе игру. Мальчики уже сошлись на том, что первым ходит Петя, а проигрывает 
	тот, после чьего хода упадёт первый стул. Осталось выбрать возможное число 
	перепиливаний за ход для каждого из них. Пусть за каждый ход Петя 
	перепиливает не менее чем $a_1$ и не более чем $b_1$ ножек, Вася — от $a_2$ до 
	$b_2$ ножек. Числа $a_1$ и $a_2$ могут быть равны нулю или единице, а 
	числа $b_1$ и $b_2$ — $m$ или $m-1$, $m<n$, но при этом обязательно $b_1 \ne b_2$. 
	Сколько игр удовлетворяет этим условиям, и кто из мальчиков выиграет в каждой из них?
} \fi

\itC Игра заканчивается тогда, когда у одного из стульев оказываются перепиленными
хотя бы две ножки. Иными словами, все «интересные» ходы происходят, когда у каждого
из стульев перепилено не более одной ножки, и поэтому количество «интересных» ходов
не превосходит $n$.

Все игры делятся на три группы:

\begin{center} \bfseries
	1. Игры, где $a_1 = a_2 = 0$.
\end{center}

Любая такая игра будет бесконечной, потому что у каждого из игроков есть возможность
«пропустить ход», не перепилив ни одной ножки. Осталось посчитать количество
таких игр. Число $m$ может меняться в промежутке от 1 до $n-1$, поэтому есть $n-1$ вариант
его выбрать. После этого можно дать либо Пете, либо Васе возможность перепиливать
ровно $m$ ножек (против $m-1$ у его противника). Всего игр получилось
	$$2 \cdot (n-1).$$

\begin{center} \bfseries
	2. Игры, где ровно одно из чисел $a_1$, $a_2$ равно нулю.
\end{center}

В любой такой игре побеждает игрок, у которого есть возможность «пропустить ход»,
ничего не перепиливая — это может быть либо Петя, либо Вася, в зависимости
от установленных ими правил.

Посчитаем количество таких игр. При $m=1$ возможны две игры: Петя ничего
не перепиливает, Вася перепиливает ровно одну ножку, либо наоборот.
Для каждого $m$ от 2 до $n-1$ возможно 4 игры:

\begin{center} \begin{tabular}{c|c}
	\begin{minipage}[left]{3cm}
		П.: от 0 до $m$ \\
		В.: от 1 до $m-1$
	\end{minipage} &
	\begin{minipage}[left]{3cm}
		П.: от 0 до $m-1$ \\
		В.: от 1 до $m$
	\end{minipage}
	$\vphantom{\int\limits_{0_0}^{1^1}}$ \\ \hline
	\begin{minipage}[left]{3cm}
		П.: от 1 до $m$ \\
		В.: от 0 до $m-1$
	\end{minipage} &
	\begin{minipage}[left]{3cm}
		П.: от 1 до $m-1$ \\
		В.: от 0 до $m$
	\end{minipage}
	$\vphantom{\int\limits_{0_0}^{1^1}}$
\end{tabular} \end{center}

Всего игр получилось
	$$4 \cdot (n-2) + 2.$$

\begin{center} \bfseries
	2. Игры, где $a_1 = a_2 = 1$.
\end{center}

Этот случай интересен тем, что он наименее тривиальный (на каждом ходу
всегда что-то происходит) и при этом присутствует явная асимметрия
в возможностях игроков: одному из них доступна одна лишняя возможность —
перепилить $m$ ножек.

Докажем, что именно этот игрок всегда будет выходить победителем.

\begin{enumerate}[label={\bfseries\arabic*)}]

	\item Петя перепиливает от 1 до $m$ ножек, Вася — от 1 до $m-1$ ножек.
	
	Если $n$ не делится на $m$, побеждает Петя: первым ходом он перепиливает
	по одной ножке у $n \bmod m$ стульев, затем, если Вася перепиливает
	$k$ ножек, отвечает ему перепиливанием $m-k$ ножек. Таким образом,
	все $n$ стульев исчерпаются после хода Пети.
	
	Если $n$ не делится на $m+1$, побеждает, опять же, Петя, с точно
	такой же стратегией (отвечать Васе перепиливанием $m+1-k$ ножек).
	
	Если $n$ делится и на $m+1$, и на $m$, то первым ходом Петя должен перепилить
	одну ножку у одного стула. Тогда после хода Васи останется от $n - 2$ до $n - m$
	стульев с неподпиленными ножками — это число, очевидно, положительно и
	не делится на $m+1$, поэтому, начиная с этого момента, Петя может действовать
	согласно стратегии, описанной в предыдущем абзаце.
	
	\item Петя перепиливает от 1 до $m-1$ ножек, Вася — от 1 до $m$ ножек.
	
	Если $n$ делится на $m$ или на $m+1$, выигрышная стратегия для Васи очевидна:
	оставлять после каждой пары ходов «Петя, Вася» число, по-прежнему
	кратное соответственно $m$ или $m+1$.
	
	Если $n$ не делится ни на $m$, ни на $m+1$, то после хода Пети Вася может сделать
	число стульев, оставшихся без неподпиленных ножек, кратным одному из этих чисел
	(ему доступны все возможные величины остатков натуральных чисел по модулям $m$
	и $m+1$) — а после этого продолжать согласно стратегии, описанной в предыдущем абзаце.

\end{enumerate}

Посчитаем количество игр такого вида: для каждого $m$ в пределах от 2 до $n-1$ существует
две игры (различающиеся тем, кто из игроков «одарен» ходом в $m$ ножек). Отсюда ответ —
	$$2 \cdot (n-2).$$

Если сложить ответы, полученные нами в разных случаях, получится $8n-12$ игр.

\end{itemize}