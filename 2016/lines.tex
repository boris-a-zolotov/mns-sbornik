\task{Линии и сетки}
\begin{itemize}
\itA В стол вбиты 26 гвоздиков так, как показано на рисунке 5. Расстояние между соседними~— 1 сантиметр. Помогите Любе пропустить по столу нитку длиной 25 сантиметров от гвоздика $A$ к гвоздику $B$ так, чтобы она касалась каждого гвоздика.

(картинка)

\itB Легко заметить, что любая прямая, проведённая через прямоугольник, пересечёт не более $a-1$
вертикальных внутренних линий сетки и не более $b-1$ горизонтальных линий. 

Представим, что мы идём вдоль прямой слева направо. Сперва мы попадаем в какую-то из клеток прямоугольника через его внешнюю границу, и каждое следующее горизонтальное или вертикальное пересечение даёт нам одну новую клетку на нашем пути --- мы переходим в неё либо через правую сторону, либо через верхнюю/нижнюю. Если мы не пересекали углы клеток, то всего получится $1 + (a-1) + (b-1)$ клеток на нашем пути: одна начальная, $a-1$ клеток при пересечении правой стороны и $b-1$ при пересечении верхней/нижней стороны.Если же мы пересекаем угол, количество клеток на пути будет ещё меньше --- мы одновременно проходим и через горизонтальную и через вертикальную линию, без промежуточной клетки.

Таким образом, ответ --- нет. Впрочем, если же мы ослабим условие и будем считать, что линия проходит по клетке если она даже только задевает границу клетки --- то да, достаточно хотя бы раз задеть пересечение линий сетки (мы тогда сможем заявить, что побывали во всех четырёх клетках, соседних с пересечением). 

\itC Среди треугольников с данным основанием $l$ и данным периметром $P$ найти треугольник с максимальной площадью.

Площадь треугольника --- это полупроизведение основания на высоту. Основание фиксировано, а вот
высота зависит от двух других сторон $a$ и $b$. Начнём от ситуации, когда $a = b = \frac{P-l}{2}$ --- то есть треугольник равнобедренный. Если мы будем увеличивать, например, $a$, то $b$ неизбежно придётся уменьшить,
чтобы сохранить периметр. Соответственно, треугольник накренится и сожмётся по высоте, площадь его уменьшится.
Значит, наибольшая площадь будет при случае равнобедренного треугольника.

Теперь выразим это же более формально. 
Посчитаем площадь треугольника по формуле Герона. Для подсчёта нам требуется знать длину всех 
сторон, поэтому введём длину второй стороны $a$ как параметр (третью сторону мы получим как $P-(P-l-a) = l+a$):

$$S = \sqrt{\frac{P\cdot(P-a)\cdot(P-l)\cdot(l+a))}{4}}$$

Чтобы максимизировать $S$, мы можем менять только $a$. Формула монотонно зависит от всех компонентов
произведения, а при изменении $a$ меняются только $P-a$ и $l+a$, поэтому вместо поиска точки
максимума всей функции мы можем искать точку максимум только у $(P-a)\cdot(l+a)$.

Раскроем скобки: $g(x) = -a^2 + (P-l)a + lP$. Это квадратичная парабола, рога параболы направлены вниз,
и потому максимум её, как известно, достигается в точке $\frac{-(P-l)}{2\cdot(-1)} = \frac{P-l}{2}$.

Осталось посчитать ответ (шаги по преобразованию формулы опустим): $S = \sqrt{\frac{P\cdot(P-l)\cdot(P+l)^2}{16}}$.

\end{itemize}

