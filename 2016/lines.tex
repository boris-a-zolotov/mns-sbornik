\task{Линии и сетки}
\begin{itemize}
\itA Нитку можно натянуть, например, так:

\begin{center}\tikz[scale=0.5]{
    \pgfmathsetmacro{\e}{0.1}
    \foreach \x in {1,2,...,5} {
        \foreach \y in {1,2,3,4} {
            \fill (\x,\y) circle[radius=\e];
        }
    }
    \fill (3,0) circle[radius=\e];
    \fill (4,0) circle[radius=\e];
    \fill (0,2) circle[radius=\e];
    \fill (6,3) circle[radius=\e];
    \fill (2,5) circle[radius=\e];
    \fill (3,5) circle[radius=\e];
    \draw [very thick, rounded corners=2] (0-\e,2+\e) node[left] {$A$} -- 
         (1+\e,2+\e) -- (1-\e,1-\e) -- (2,1+\e) -- (3+\e,1+\e) -- 
         (3-\e,-\e) -- (4+\e,-\e) -- (4-\e,1+\e) -- (5+\e,1-\e) --
         (5+\e,2+\e) -- (4,2-\e) -- (3,2+\e) -- (2-\e,2-\e) --
         (2+\e,3+\e) -- (1-\e,3-\e) -- (1-\e,4+\e) -- (2+\e,4-\e) --
         (2-\e,5+\e) -- (3+\e,5+\e) -- (3-\e,4) -- (3-\e,3-\e) --
         (4+\e,3-\e) -- (4-\e,4+\e) -- (5+\e,4+\e) -- (5-\e,3-\e) -- 
         (6+\e,3+\e) node[right] {$B$};
}\end{center}

\itB Легко заметить, что любая прямая, проведённая через прямоугольник, пересечёт не более $a-1$
вертикальных внутренних линий сетки и не более $b-1$ горизонтальных линий. 

Представим, что мы идём вдоль прямой слева направо. Сперва мы попадаем в какую-то из клеток прямоугольника через его внешнюю границу, и каждое следующее горизонтальное или вертикальное пересечение даёт нам одну новую клетку на нашем пути --- мы переходим в неё либо через правую сторону, либо через верхнюю/нижнюю. Если мы не пересекали углы клеток, то всего получится $1 + (a-1) + (b-1)$ клеток на нашем пути: одна начальная, $a-1$ клеток при пересечении правой стороны и $b-1$ при пересечении верхней/нижней стороны.Если же мы пересекаем угол, количество клеток на пути будет ещё меньше --- мы одновременно проходим и через горизонтальную и через вертикальную линию, без промежуточной клетки.

Таким образом, ответ --- нет. Впрочем, если же мы ослабим условие и будем считать, что линия проходит по клетке если она даже только задевает границу клетки --- то да, достаточно хотя бы раз задеть пересечение линий сетки (мы тогда сможем заявить, что побывали во всех четырёх клетках, соседних с пересечением). 

\itC Площадь треугольника --- это полупроизведение основания на высоту. Основание фиксировано, а вот
высота зависит от двух других сторон $a$ и $b$. Начнём от ситуации, когда $a = b = \frac{P-l}{2}$ --- то есть треугольник равнобедренный. Если мы будем увеличивать, например, $a$, то $b$ неизбежно придётся уменьшить,
чтобы сохранить периметр. Соответственно, треугольник накренится и сожмётся по высоте, площадь его уменьшится.
Значит, наибольшая площадь будет в случае равнобедренного треугольника.

\begin{center}\tikz{
    \draw[thick] (-1,0) -- node[midway,left] {$a$} (0,3)
                        -- node[midway,right] {$b$} (1,0) 
                        -- node[midway,below] {$l$} (-1,0);
    \draw[dotted] (-4,0) -- (-1,0) ++ (1,0) -- (4,0);
    %\draw (-1,0) circle[radius=3.1622];
    \draw[dotted] (3.1622,0) arc (0:180:3.1622 and 3);
    \draw[thick,densely dashed] (-1,0) -- (2.5,1.837) -- (1,0);
}\end{center}


Теперь выразим это же более формально. 
Посчитаем площадь треугольника по формуле Герона. 
Для этого введём обозначение для полупериметра треугольника $p = \frac{P}{2}$.
Для подсчёта нам требуется знать длину всех 
сторон, поэтому введём длину второй стороны $a$ как параметр 
(третью сторону мы получим как $P-l-a$):

$$S(a) = \sqrt{p\cdot(p-l)\cdot(p-a)\cdot(l+a-p)}$$

Формула монотонно зависит от всех сомножителей подкоренного произведения, 
а при изменении $a$ меняются только $p-a$ и $l+a-p$, поэтому вместо поиска точки
максимума всей функции мы можем искать точку максимума только у $(p-a)\cdot(l+a-p)$,
обозначим это выражение за $g(a)$.

Раскроем скобки: $g(a) = -a^2 + (2p - l)a + lp - p^2$. Это квадратичная парабола, 
рога параболы направлены вниз, и потому максимум её, как известно, достигается в 
точке $$a = -\frac{(2p-l)}{2\cdot(-1)} = p-\frac{l}{2}$$

Осталось посчитать ответ (шаги по преобразованию формулы опустим): 
$$S = \sqrt{\frac{P\cdot(P-2l)\cdot l^2}{16}}$$

\end{itemize}

