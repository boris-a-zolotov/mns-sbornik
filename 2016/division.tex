\task{Деление и остатки}
\begin{itemize}
\itA Если $d$ делится на 564, то существует такой $k$, что $k \cdot 564 = d$. Попробуем подобрать:
$2 \cdot 564 = 1128$, $3 \cdot 564 = 1692$, $4 \cdot 564 = 2256$. Ни на одно из получившихся чисел
2016 не делится, и дальше перебирать бессмысленно: числа не делятся на числа, бо'льшие себя.
Вывод: Настя ошиблась.

\itB Вторая подруга ошибается. Например, пусть $a = 7, b = 6, c = 5$. Тогда остаток от деления
$a$ на $b$ равен $1$, а остаток от деления $1$ на $с$ снова равен $1$. С другой стороны, остаток
от деления $a$ на $c$ равен 2.

\itC Даны два числа $b$ и $c$. Доказать, что для любого $a$ выполнено 
$(a \bmod b) \bmod c = a \bmod c$ тогда и только тогда, когда $b$ делится на $c$.

Введём обозначения.
Выпишем определение для деления с остатком $a$ на $b$: $a = k \cdot b + p$, причём $0 \le p < b$.
То же для деления $p$ на $c$: $p = s \cdot c + q$ при $0 \le q < c$. 
Произведём подстановку: $a = k \cdot b + s \cdot c + q$. 

Если $b$ делится на $c$, то существует
такое $t$, что $b = t\cdot c$, то есть $a = (k \cdot t + s) \cdot c + q$, причём $0 \le q < c$.
Прямое утверждение доказано.

Обратное утверждение: равенство верно для любого $a$ только тогда, когда $b$ делится на $c$. 
Переформулируем: если $b$ не делится на $c$, то равенство выполнено не для любого $a$.
Давайте найдём такой $a$ по данным $b$ и $c$.
Возьмём $a = b$, тогда левая часть $(a \bmod b) \bmod c = 0$,
а правая --- $a \bmod c = b \mod c \ne 0$.

Отдельно заметим, что даже если $b$ не делится на $c$, то при некоторых $a$ (например, $a = 0$)
равенство из условия всё равно будет выполнено.

\end{itemize}
