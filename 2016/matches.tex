\task{Спички и пионеры}
\begin{itemize}
\itA Пионер Петя выкладывает цифры из спичек так, как это делается на экране стандартного калькулятора. Какое наибольшее число он может сложить из 15 спичек?

Ясно, что величина числа в десятичной записи в первую очередь зависит от 
количества его разрядов. Поэтому построим самое длинное число из самых
<<дешёвых>> цифр: 1111111. У нас остаётся одна спичка: удлиннить число мы
уже не можем, остаётся только увеличивать цифры. Самая большая цифра из трёх
спичек --- 7, поэтому ответ на задачу --- 7111111.

\itB Да, легко: вполне возможно, это были 
Аустри, Бримир, Вестри, Гандальв, Двалин, 
Ёрд, Ингви, Кили, Лит, Мотсогнир, 
Нии, Ори, Регин, Судри, Торин,
Фили, Хефти, Эйкинскьяльди, Яри
и Пётр.
                                                        
\itC Несколько неформальная задача, требовалось любое достаточно разумное 
рассуждение, например такое.
Самая сложная цифра --- 8. Любая другая цифра может быть выложена
упрощением её структуры (изъятием и частичным перекладыванием спичек).
Чтобы выложить восьмёрку, нужно получить две замкнутые области.
Самая простая фигура из прямых с замкнутой областью внутри --- треугольник.
Поэтому меньше трёх спичек никак не получится. Но из трёх спичек 
можно составить только один треугольник, поэтому четыре спички.

(картинка)
\end{itemize}
