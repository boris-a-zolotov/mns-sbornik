\task{Спички и пионеры}
\begin{itemize}
\itA Ясно, что величина числа в десятичной записи в первую очередь зависит от 
количества его разрядов. Поэтому построим самое длинное число из самых
<<дешевых>> цифр: 1111111. У нас остается одна спичка: удлиннить число мы
уже не можем, остается только увеличивать цифры. Самая большая цифра из трех
спичек --- 7, поэтому ответ на задачу --- 7111111.

\itB Да, вполне возможно, это были 
Аустри, Бримир, Вестри, Гандальв, Двалин, 
ерд, Ингви, Кили, Лит, Мотсогнир, 
Нии, Ори, Регин, Судри, Торин,
Фили, Хефти, Эйкинскьяльди, Яри
и Петр.
                                                        
\itC Несколько неформальная задача, требовалось любое достаточно\linebreak
разумное рассуждение, например такое.

Самая сложная цифра --- 8. Любая другая цифра может быть выложена
упрощением ее структуры (изъятием и частичным перекладыванием спичек).
Чтобы выложить восьмерку, нужно получить\linebreak две замкнутые области.
Самая простая фигура из прямых с замкнутой областью внутри --- треугольник.
Поэтому меньше трех спичек никак не получится.

Однако, из трех спичек можно составить только один треугольник, поэтому
на самом деле требуется минимум 4. Имея 4 спички, выложить все
остальные цифры становится просто: \medskip

\begin{center} \tikz{
\begin{scope}[scale=0.75]
	\mov{-5}{\draw[very thick] (0.4,-0.4) -- (0.4,0.4) -- (-0.4,0.4) -- (-0.4,-0.4) -- cycle; }
	\mov{-4}{\draw[very thick] (0,-0.4) -- (0,0.4); }
	\mov{-3}{\draw[very thick] (0.4,-0.4) -- (-0.4,-0.4) -- (0.359,-0.147)
		-- (0.4,0.652) -- (-0.4,0.652); }
	\mov{-2}{\draw[very thick] (-0.4,0.4) -- (0.4,0.4) -- (0.4,0) -- (-0.4,0)
		-- (0.4,0) -- (0.4,-0.4) -- (-0.4,-0.4); }
	\mov{-0.6}{\draw[very thick] (0,-0.4) -- (0,0.4) -- (-0.692,0) -- (0.107,0); }
	\mov{0}{\draw (0,0) node {$\ldots$}; }
	\mov{1}{\foreach \x in {-45,45} \draw[very thick,rotate=\x] (-0.4,0) -- (0.4,0); 
		\foreach \y in {-0.2828,0.2828} \draw[very thick] (-0.4,\y) -- (0.4,\y); }
	\mov{2.2}{\draw[very thick] (-0.4,-0.4) -- (0.4,-0.4) -- (0.4,0.4); 
		\draw[very thick] (0.4,0) -- (-0.293,0.4) -- (0.507,0.4); }
\end{scope}
}\end{center}

\end{itemize}
