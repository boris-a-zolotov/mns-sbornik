\task{Факториалы}
\begin{itemize}
\itA Егор посчитал факториал числа 33 и записал его на бумажку. 
Его сестра решила пошалить, и стёрла одну из цифр факториала. Получилась запись: \smallskip \\
\centerline{$33!=$ 8'683'317'618'811'886'49$\square$'518'194'401'280'000'000} \smallskip
Помогите Егору восстановить стёртую цифру.

$33!$, очевидно, делится на 9. Значит, сумма всех цифр числа должна делиться на 9.
Если мы просуммируем все цифры числа, получим $139+\square$. При этом, должны быть выполнены
два условия:
\begin{center} 
$(139+\square) \bmod 9 = 0$ и $\square \le 9$
\end{center}

Перебрав все 10 вариантов
для $\square$, можем убедиться, что единственный подходящий из них --- 8.

\itB Воспользуемся двумя признаками делимости: на 9 и на 11 (на оба эти числа делится $n!$, если $n \ge 12$).
В большинстве случаев признака делимости на 9 хватит и мы можем восстановить стёртую цифру аналогично
пункту A данной задачи. Однако, если сумма известных цифр числа делится на 9, то
то возможны два варианта для стёртой цифры: 0 и 9.

В этом случае воспользуемся признаком делимости на 11: просуммируем значения, стоящие на чётных местах,
и значения, стоящие на нечётных местах. Если разница между суммами и так делится на 11, была стёрта цифра 0
(его добавление на место не поменяет делимости). В противном случае была стёрта цифра 9.

\itC Перегруппируем исходное выражение:
$$1! \cdot 2! \cdot 3! \cdot \ldots \cdot n! = (1! \cdot 1! \cdot 2) \cdot (3! \cdot 3! \cdot 4) \cdot \ldots \cdot ((n-1)! \cdot (n-1)! \cdot n)$$

И ещё раз:
$$\underbrace{1!^2 \cdot 3!^2 \cdot \ldots \cdot (n-1)!^2}_\textrm{квадрат целого} \cdot \underbrace{2 \cdot 4 \cdot \ldots \cdot n}_\textrm{s}$$

Рассмотрим внимательнее подвыражение $s$:

$$s = 2 \cdot 4 \cdot \ldots \cdot (n-2) \cdot n = 2^{\frac{n}{2}} \cdot \left(\frac{n}{2}\right)!$$

Поскольку $n$ кратно четырём (существует $t$, что $n = 4\cdot t$), то 
$2^{\frac{n}{2}} = (2^t)^2$.

Значит, нужно вычеркнуть факториал $\frac{n}{2}$, это единственный сомножитель, не являющийся полным квадратом.

\end{itemize}
