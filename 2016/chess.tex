\task{Фигуры в шахматах}
\begin{itemize}
\itA Какое минимальное количество клеток нужно пройти ферзю из точки $(a,b)$ в точку $(c,d)$ на шахматном поле?

Рассмотрим разницу координат $(|a-c|,|b-d|)$, в какой-то координате разница меньше, в какой-то больше.
Пусть для определённости $|a-c| > |b-d|$.
Если ферзь будет сперва идти по диагонали, пока а потом по горизонтали, то он пройдёт 



Количество клеток, которые ферзь пройдёт по диагонали, равно $\min(|a-c|,|b-d|)$. 
Затем по горизонтали или вертикали нужно ещё пройти $\max(|a-c|,|b-d|) - \min(|a-c|,|b-d|)$

\itB Фигура Корблюд умеет делать из данной клетки на шахматном поле четыре хода так, как показано на 
рисунке 3. Доказать, что корблюд может прийти из любой клетки бесконечного шахматного поля в любую 
другую. А верно ли такое утверждение для Корблюда Диагонального, делающего ходы как на рисунке 4?

Нам достаточно показать, что корблюд способен сдвинуться на 1 в любую из четырёх сторон.

Влево: три раза сходить <<на 1 клетку левее и на 2 клетки вверх>>, после чего 

\itC Будем рассматривать фигуры, которые из данной клетки умеют делать четыре хода: 
вперёд-налево, вперёд-направо, назад-налево, назад-направо. При этом направо / налево фигура 
смещается лишь на одну клетку, а вперёд / назад каждый ход — либо на две, либо на три клетки. 
Сколько фигур, с точностью до симметрий, удовлетворяют этим свойствам, и какие из них могут 
из любой клетки бесконечного шахматного поля дойти до любой другой?
\end{itemize}

\begin{center}
  \includegraphics[width=8.5cm]{figures/2016/corbleud.png}
\end{center}
