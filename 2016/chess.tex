\task{Фигуры в шахматах}
\begin{itemize}
\itA Для определённости будем учитывать в общем количестве пройденных клеток начальные и конечные клетки.

Ферзь путешествует с поля $(a,b)$ на поле $(c,d)$. Рассмотрим разницу начальных и конечных координат. 
Пусть для определённости разница в номере колонки не меньше, чем в номере строки ($|a-c| \ge |b-d|$).
Поскольку ферзь не прыгает через клетки, он должен пройти не меньше $|a-c|+1$ клетки (иначе окажется,
что через какую-то вертикаль из находящихся между $a$ и $c$ он перепрыгнет, не побывав на ней).

Однако, если ферзь пройдёт по диагонали $|b-d|+1$ клеток, и оставшиеся $|a-c|-|b-d|$ клеток пройдёт
по горизонтали, то посетит в точности $|a-c|+1$ клетку. 
Поэтому ответ на задачу --- максимум разниц координат плюс один: $\max(|a-c|,|b-d|)+1$.

Ниже на картинке показан пример движения ферзя с поля {\bfseries b3} на поле {\bfseries d7}. 
Ферзь посетит на своём маршруте $\max(|2-4|,|3-7|)+1 = 5$ клеток.

\begin{center}\tikz[x=6mm,y=6mm]{
    \draw[fill=black!20,even odd rule] (0,0) 
    foreach \n in {1,2,3,4} { -- ++ (8,0) -- ++ (0,1) -- ++ (-8,0) -- ++ (0,1) } -- (8,8)
    foreach \n in {1,2,3,4} { -- ++ (0,-8) -- ++ (-1,0) -- ++ (0,8) -- ++ (-1,0) } -- cycle
    foreach[count=\n] \m in {a,b,...,h} { (-0.4,\n-0.5) node {\n} (\n-0.5,-0.5) node {\phantom{g}\m\phantom{f}} };

    \draw[very thick,->,>=stealth] (1.5,2.5)  -- (3.5,4.5) -- (3.5,6.5);
}\end{center}

\itB Фигура Корблюд умеет делать из данной клетки на шахматном поле четыре хода так, как показано на 
рисунке 3. Доказать, что корблюд может прийти из любой клетки бесконечного шахматного поля в любую 
другую. А верно ли такое утверждение для Корблюда Диагонального, делающего ходы как на рисунке 4?

Нам достаточно показать, что корблюд способен сдвинуться на 1 в любую из четырёх сторон.

Влево: три раза сходить <<на 1 клетку левее и на 2 клетки вверх>>, после чего 

\itC Будем рассматривать фигуры, которые из данной клетки умеют делать четыре хода: 
вперёд-налево, вперёд-направо, назад-налево, назад-направо. При этом направо / налево фигура 
смещается лишь на одну клетку, а вперёд / назад каждый ход — либо на две, либо на три клетки. 
Сколько фигур, с точностью до симметрий, удовлетворяют этим свойствам, и какие из них могут 
из любой клетки бесконечного шахматного поля дойти до любой другой?
\end{itemize}

\begin{center}
  \includegraphics[width=8.5cm]{figures/2016/corbleud.png}
\end{center}
