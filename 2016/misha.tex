\task{Хитрый Миша}
\begin{itemize}
\itA Посетители тира ведут огонь из вершины $X$ равнобедренного треугольника $XAB$ по его основанию
$AB$. Заметим, что отрезок $XP$, соединяющий вершину с точкой попадания пули на основании 
треугольника $P$,
делит треугольник на два других, $XAP$ и $XBP$. Площадь же этих треугольников равна полупроизведению 
оснований на расстояния от пулевой дырки до соответствующих прямых $h_A$ и $h_B$: 
$$S(XAP) = \frac{|XA|\cdot h_A}{2}\quad\quad
S(XBP) = \frac{|XB|\cdot h_B}{2}$$ 
Однако, поскольку $|XA|=|XB|$, имеем
$$S(XAB) = S(XAP) + S(XBP) = \frac{|XA|\cdot h_A + |XB|\cdot h_B}{2} = |XA|\cdot (h_A + h_B)$$

Площадь треугольника $XAB$ и длина боковых сторон не зависит от выбора $P$, значит, и $h_A + h_B$
тоже не зависит от выбора $P$. Поэтому, если следовать предложению Миши, награждать придётся
всех посетителей тира.

\itB Будем заклеивать куб со стороной $\sqrt 5$ крестиками, сопоставляя одному из шести крестиков 
одну из шести граней кубика. Будем приклеивать крестик к квадратной грани «под наклоном»:

\begin{center}
	\tikz{
		\draw (0,1)
			-- (0,2)
			-- (1,2)
			-- (1,3)
			-- (2,3)
			-- (2,2)
			-- (3,2)
			-- (3,1)
			-- (2,1)
			-- (2,0)
			-- (1,0)
			-- (1,1) -- cycle;
		\draw[thick,dashed] (0,1) -- (1,3) -- (3,2) -- (2,0) -- cycle;
	}
\end{center}

Тогда, загибая уголки у крестиков, получим заклеивание кубика:

\begin{center}
	\tikz{
		\draw (0,0) -- (0,3) -- (3,3) -- (3,0) -- cycle;
		\draw (0,3) -- (1,4) -- (4,4) -- (3,3) -- (3,0) -- (4,1) -- (4,4);
		\draw[very thick] (0,0) -- (6/5,3/5) -- (1.5,0);
		\draw[very thick] (3.5,0.5) -- (3+2/5,1)
			-- (3,0) -- (12/5,6/5) -- (3,1.5) -- (3.2,2) -- (3,3)
			-- (3-6/5,3-3/5) -- (1.5,3) -- (7/5,3+1/5)
			-- (0,3) -- (3-12/5,3-6/5) -- (0,1.5);
		\draw[very thick] (3,3) -- (14/5,3+2/5) -- (3.5,3.5) -- (3.6,3) -- (4,4)
			-- (3-2/5,3+4/5) -- (2.5,4);
		\draw[very thick] (1,4) -- (0.5+3.5/5,4-2/5) -- (0.5,3.5);
		\draw[very thick] (4,1) -- (3+4/5,2) -- (4,2.5)
	}
\end{center}

\itC Построим чертёж. Искомое расстояние $t = |XT|$. Так как $|AT| = |AT'|$, то треугольник $TAT'$ --- 
равнобедренный. Если посередине между $T$ и $T'$ отметить ещё одну точку,
$B$, то треугольник $XAB$ будет прямоугольным (т.к. высота и медиана к основанию в равнобедренном треугольнике 
совпадают). Значит, $\angle XAB = 30^\circ$, отсюда $|XB| = \frac{|XA|}{2} = \SI{10}{\text{см}}$.
Значит, $t = \SI{8}{\textrm{см}}$.

\begin{center}
\tikz{
   \coordinate [label=left:$X$] (X) at (0,0);
   \coordinate [label=above:$P$] (P) at (60:6);
   \coordinate [label=below:$Q$] (Q) at (6,0);
   \coordinate [label=right:$A$] (A) at (60:5);
   \coordinate [label=below:$T$] (T) at (1.5,0);
   \coordinate [label=below:$B$] (B) at (2.5,0);
   \coordinate [label=below:$T'$] (T1) at (3.5,0);
   \draw (X) -- (P);
   \draw (X) -- (Q);
   \draw (A) -- (B);
   \draw (T) -- node[midway,sloped] {\tikz \draw (0,0) -- +(0,-0.1) -- +(0,0.1)} (A) -- 
                node[midway,sloped] {\tikz \draw (0,0) -- +(0,-0.1) -- +(0,0.1)} (T1);
   \draw (B) -- ++(-0.2,0) -- ++(0,0.2) -- ++(0.2,0);
   \draw[thick] (0.3,0) arc (0:60:0.3);
   \draw (0.7,0.25) node {$60^\circ$};
   \draw[rotate=60] (2.5,0.25) node[rotate=60] {$\SI{20}{\text{см}}$};
}
\end{center}
\end{itemize}

