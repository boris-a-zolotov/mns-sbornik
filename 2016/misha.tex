%\documentclass[10pt]{scrbook} \usepackage{modules/nonstahp_book}
%\usepackage{mathspec}

\setmainfont[
	Path = f/,
	BoldFont=pb.ttf,
	ItalicFont=pi.ttf,
	BoldItalicFont=pbi.ttf
		]{p.ttf}
\setsansfont[
	Path = f/,
	BoldFont=pb.ttf,
	ItalicFont=pi.ttf,
	BoldItalicFont=pbi.ttf
		]{p.ttf}
		
\setmathfont(Digits)[Path = f/]{p.ttf}
\setmathfont(Latin)[Path = f/]{pi.ttf}
\setmathfont(Greek)[Path = f/, Uppercase]{p.ttf}
\setmathfont(Greek)[Path = f/, Lowercase]{pi.ttf}

\setmonofont[Path = f/]{pmono.ttf}

%\setCJKmainfont[
%	Path=f/,
%	BoldFont=notoserifb.ttf,
%	ItalicFont=notoserifi.ttf,
%	BoldItalicFont=notoserifbi.ttf
%		]{notoserif.ttf}

 \begin{document}

\task{Хитрый Миша}
\begin{itemize}
\itA Миша подошёл к заведующему городским тиром с предложением открыть экспериментальный тир. Его зона имела бы форму равнобедренного треугольника со стрельбищем в вершине, а в конце дня награждались бы все, достигшие минимальной за этот день суммы расстояний от пулевой дырки до прямых, продолжающих боковые рёбра зоны. Докажите, что Миша тем самым разорит тир.

\itB Маша попросила Мишу вырезать из бумаги шесть развёрток для куба. Вместо этого Миша вырезал шесть крестиков, состоящих из пяти одинаковых тетрадных клеток каждый. Может ли Маша оклеить без наложений хоть какой-нибудь куб крестиками, вырезанными Мишей?

Будем заклеивать куб со стороной $\sqrt 5$ крестиками, сопоставляя одному из шести крестиков одну из шести граней кубика. Будем приклеивать крестик к квадратной грани «под наклоном»:

\begin{center}
	\tikz{
		\draw (0,1)
			-- (0,2)
			-- (1,2)
			-- (1,3)
			-- (2,3)
			-- (2,2)
			-- (3,2)
			-- (3,1)
			-- (2,1)
			-- (2,0)
			-- (1,0)
			-- (1,1) -- cycle;
		\draw[thick,dashed] (0,1) -- (1,3) -- (3,2) -- (2,0) -- cycle;
	}
\end{center}

Тогда, загибая уголки у крестиков, получим заклеивание кубика:

\begin{center}
	\tikz{
		\draw (0,0) -- (0,3) -- (3,3) -- (3,0) -- cycle;
		\draw (0,3) -- (1,4) -- (4,4) -- (3,3) -- (3,0) -- (4,1) -- (4,4);
		\draw[very thick] (0,0) -- (6/5,3/5) -- (1.5,0);
		\draw[very thick] (3,0) -- (12/5,6/5) -- (3,1.5) -- (3.2,2) -- (3,3)
			-- (3-6/5,3-3/5) -- (1.5,3) -- (7/5,3+1/5)
			-- (0,3) -- (3-12/5,3-6/5) -- (0,1.5);
	}
\end{center}

\itC Автор учебника по геометрии попросил Мишу набрать текст учебника. А Миша, как и следовало ожидать, допустил опечатку. В задаче, гласящей: «Отмерьте на одной стороне угла в 60 градусов 20 см, а на другой стороне — $t$ см, и посчитайте расстояние между отмеченными точками», Миша намеренно увеличил $t$ на 4 см, но при этом ответ на задачу остался верным. Чему равно $t$?
\end{itemize}


% \end{document}