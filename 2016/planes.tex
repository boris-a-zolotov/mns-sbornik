\task{Первым делом — самолёты}
\begin{itemize}
\itA Комната освещена маленькой лампой, установленной на стене. Петя пускает бумажные самолётики так, что они пролетают перед лампочкой на её уровне. Вася же берётся определить, был пролетающий самолётик наклонён к лампочке или от лампочки, лишь по длине теней от его крыльев на противоположной стене. Как ему это сделать?

\itB Четыре маленьких вертолёта стоят на земле в вершинах квадрата $1 \times 1$ метр. На какую высоту нужно взлететь двум из них, чтобы попарные расстояния между вертолётами стали одинаковыми?

\itC По прямым, содержащим стороны правильного семиугольника со стороной в два километра, перемещаются камеры. Оператору Blue Origin было дано задание посадить ракету в этот семиугольник так, чтобы суммарное расстояние от неё до прямых, по которым ездят камеры, было минимальным. Куда оператору сажать ракету?
\end{itemize}
