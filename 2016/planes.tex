\task{Первым делом — самолёты}
\begin{itemize}
\itA Поскольку на стене установлен точечный источник света (лампа), лучи от него
расходятся в стороны под некоторым углом. Поэтому тень от ближе расположенного
к лампе объекта будет больше, чем от дальше расположенного (рассмотрим предельный
случай: небольшую близко расположенную лампу можно закрыть ладонью целиком; 
от далеко расположенной ладонью можно разве что заслонить глаза). 

Если самолётик немного наклонить, то тень от более близкой к лампе части крыла
будет больше, чем от более далёкой. Поэтому если тень от нижней часть самолётика 
больше, то самолётик наклонён к лампе, а если тень от верхней части больше ---
самолётик наклонён от лампы.

\itB Пока вертолёты стоят на земле, расстояние между противоположными вертолётами
равно $\sqrt{2}$. Нам надо добиться, чтобы расстояние и между соседними составило
$\sqrt{2}$.

\begin{center}\tikz{
    \draw (0,0) -- node[midway,sloped,above]{1 м} (2,-0.6) -- (4,0) -- (2,0.6) -- cycle;
    \draw[dashed] (0,0) -- node[midway,sloped,above]{1 м} (0,2.2);
    \draw[dashed] (4,0) -- node[midway,sloped,above]{1 м} (4,2.2);
    \draw[dotted] (0,2.2) -- (2,0.6) -- (4,2.2) -- (2,-0.6) -- cycle;
    \draw[dotted] (2,0.6) -- (2,-0.6);
    \draw[dotted] (0,2.2) -- (4,2.2);
    \draw (-0.3,2.4) node {$A$};
    \draw (2,0.9) node {$B$};
    \draw (4.3,2.4) node {$C$};
    \draw (2,-0.9) node {$D$};
}\end{center}

Поднимем в воздух вертолёты $A$ и $C$ на высоту 1 метр.
Теперь все вертолёты имеют попарные расстояния $\sqrt{2}$.

\itC Пусть $S$ --- точка посадки ракеты. Тогда площадь семиугольника равна сумме площадей
треугольников, образованных точкой $S$ и соседними вершинами семиугольника $A_i$ и $A_{i+1}$.

\begin{center}\tikz{
    \foreach \angle in {0,1,2,3,4,5,6} {
        \draw (0.1,0.3) -- (\angle*360/7:1.5) -- (\angle*360/7+360/7:1.5);
    }
    \foreach \angle in {1,2,3,4,5,6,7} {
        \draw (-\angle*360/7:1.8) node {$A_\angle$};
    }
    \draw (0.5,0) node {$S$};
}\end{center}

Площадь треугольника $SA_iA_{i+1}$ равна 
$\frac{h_i\cdot \SI{2}{\text{км}}}{2} = \SI{h_i}{\textrm{км}^2}$,
где $h_i$ --- расстояние от $S$ до прямой, продолжающей сторону семиугольника $A_iA_{i+1}$.

Отсюда можем сделать вывод, что сумма расстояний до прямых $h_1 + h_2 + \ldots + h_7$ численно 
равна площади семиугольника и потому постоянна.
%Значит, оператор Blue Origin может спокойно выполнять свои обязанности по 
%посадке ракеты, не отвлекаясь на просьбу режиссёра съёмки:
%все точки внутри семиугольника с этой точки зрения равноценны.

\end{itemize}
