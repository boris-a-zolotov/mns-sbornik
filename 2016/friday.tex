\task{Пятница}
\begin{itemize}
\itA Перепишем условие формально:
Если $x$ --- количество молока в миллилитрах в обычной кружке, то за первый раз
мама разлила $4 \cdot x + 1.2 \cdot x$ миллилитров --- и это 30\% от двух литров.
Давайте тогда составим уравнение:

$$4 \cdot x + 1.2 \cdot x = 0.3 \cdot 2000$$

Легко видеть, что $x = \frac{600}{5.2} = 115.38...$, 
и искомая 20\% разница составила чуть больше 23 миллилитров.

\itB Представим, что Исинбай бежит с обычной человеческой скоростью $v \SI{м}{с}$, 
за ним бежит тигр в $q$ раз быстрее, со скоростью $qv$. 
За $t$ секунд Исинбай пробежит расстояние $vt$, тигр --- 
$qvt$, поэтому Исинбай никогда не должен подходить к тигру
ближе, чем на $(q-1)vt$ метров, иначе тигр его успеет догнать за один приём,
не отдыхая. Обычный человек бегает короткую дистанцию со
скоростью примерно \SI{30}{\text{км}/\text{ч}}, или \SI{8.(3)}{\text{м}/\text{с}},
что даёт оценку $x = 8.(3) \cdot (q-1)t$ метров для безопасного расстояния.

Отрицательный $x$ означает, что тигр бегает медленнее человека,
и потому к нему можно безопасно подходить практически вплотную.

\itC Для задания каждой расстановки мы должны выбрать пять пятниц для Дани 
и три для Кости.

У Дани всего есть $C^5_7 = 21$ вариант иметь пять пятниц на неделе.
В каждом из вариантов для Кости нужно выбрать из пяти Даниных пятниц две ($C^2_5 = 10$), и из
двух не-пятниц одну ($C^1_2 = 2$). Итого, всего есть $21 \cdot 10 \cdot 2 = 420$
расстановок пятниц.
\end{itemize}
