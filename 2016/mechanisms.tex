\task{Эти необычные механизмы}
\begin{itemize}
\itA Легко заметить, что шестерёнки, находящиеся в зацеплении, всегда вращаются в 
разных направлениях.

(картинка)

\itB 
%В японских шахматах сёги есть фигура Золото, которая бьёт шесть клеток вокруг себя
% (заштрихованы на рисунке). 
%
%\begin{center}\tikz[scale=0.7]{
%    \draw (-1,5) grid (4,10);
%    \draw[pattern = north east lines] (0,9) -- (0,7) -- (1,7) -- (1,6) -- (2,6) -- (2,7) -- (3,7) -- (3,9) -- cycle;
%    \fill (1.5,7.5) circle[radius=0.1];
%}\end{center}
%
%Можно ли расставить на поле $9 \times 9$ несколько таких 
%фигур, чтобы каждая клетка поля билась ровно одним Золотом?
%\itr 
Это невозможно. Рассмотрим левый верхний угол доски (поле $a1$) и подумаем, какое
Золото может его бить. Всего возможно четыре позиции: две в первой вертикали, и две во второй.

Если Золото стоит на второй вертикали, на поле $b2$ (или $a2$), тогда поле $c1$ ($b1$, 
соответственно) окажется не под боем: легко увидеть, что Золото, бьющее это поле, обязательно 
заденет и поле $c2$ ($b2$, соответственно).

\begin{center}\tikz[scale=0.7]{
    \draw (0,0) grid (9,9);
    \draw[thick] (0,0) rectangle (9,9);
    \draw[pattern = north east lines] (0,9) -- (0,7) -- (1,7) -- (1,6) -- (2,6) -- (2,7) -- (3,7) -- (3,9) -- cycle;
    \foreach \y [count=\n] in {a,b,c,d,e,f,g,h,i} {
        \draw (9,9.5-\n) node [right] {$\y$};
    }
    \foreach \y [count=\n] in {1,2,...,9} {
        \draw (\n-0.5,9) node [above] {$\y$};
    }
    \fill (1.5,7.5) circle[radius=0.1];
    \draw (0.5,6.5) circle[radius=0.1];
    %\fill (1.5,7.5) circle[radius=0.1];
}\end{center}


%\shodiag{1}{{\shopiece{3}{3}{\Rps}\shopiece{1}{5}{\Ps}\shopiece{-1}{9}{\Ss}
%            \shopiece{1}{2}{\Ng}\shopiece{1}{3}{\Kg}\shopiece{3}{6}{\Bg}
%            \shopiece{2}{3}{\Pg}}{\shopiece{3}{4}{\Bps}\shopiece{3}{5}{\Bs}
%            \shopiece{-1}{9}{\Gs}\shopiece{1}{4}{\Pg}\shopiece{1}{5}{\Kg}\shopiece{2}{7}{\Rpg}}}

Если же Золото стоит на первой вертикали, на поле $b1$ (или $a1$), то фигуры не <<уложатся>> вдоль доски: 
поскольку Золото имеет ширину 3, то оно должно далее стоять на $b4$ ($a4$) --- чтобы бить поле $a3$, 
затем на $b7$ ($a7$) --- чтобы бить поле $a6$, и поле $a9$ окажется не под боем.

\begin{center}\tikz[scale=0.7]{
    \draw (0,0) grid (9,9);
    \draw[thick] (0,0) rectangle (9,9);
    \draw[pattern = north east lines] (0,9) -- (0,6) -- (1,6) -- (1,7) -- (2,7) -- (2,9) -- cycle;
    \draw[pattern = horizontal lines] (2,9) -- (2,8) -- (3,8) -- (3,7) -- (4,7) -- (4,8) -- (5,8) -- (5,9) -- cycle;
    \draw[pattern = north west lines] (5,9) -- (5,7) -- (6,7) -- (6,6) -- (7,6) -- (7,7) -- (8,7) -- (8,9) -- cycle;
    \foreach \y [count=\n] in {a,b,c,d,e,f,g,h,i} {
        \draw (9,9.5-\n) node [right] {$\y$};
    }
    \foreach \y [count=\n] in {1,2,...,9} {
        \draw (\n-0.5,9) node [above] {$\y$};
    }
    \fill (0.5,7.5) circle[radius=0.1];
    \draw (8.5,8.5) circle[radius=0.1];
    %\fill (1.5,7.5) circle[radius=0.1];
}\end{center}


\itC Давайте вычислим: время на проезд $n$ участков без остановки --- 
это время на проезд первого участка 
со скоростью 54 км/ч, второго --- со скоростью 48 км/ч, третьего --- со скоростью 42 км/ч, и т.д.
плюс время на ремонт колёс. Или, формулой (время выражено в минутах):
 $$T(n) = \left(\frac{12}{54} + \frac{12}{48} + \dots + \frac{12}{60 - 6 \cdot n}\right) \cdot 60 + 10 + 3 \cdot n$$
Средняя скорость (в километрах в минуту), соответственно, будет равна расстоянию, поделённому на время, т.е.
$\frac{12 \cdot n}{T(n)}$.

Теперь с конкретными цифрами:

\begin{tabular}{lll}
Участки & Время на проезд & Средняя скорость \\
\hline
1	& $\frac{12}{54} \cdot 60 + 13 = 26\frac{1}{3}$ & $\frac{36}{79} \approx 0.456$ \\
2	& $\left(\frac{12}{54} + \frac{12}{48}\right) \cdot 60 + 16 = 44\frac{1}{3}$ & $\frac{72}{133} \approx 0.541$ \\
3	& $\left(\frac{12}{54} + \frac{12}{48} + \frac{12}{42}\right) \cdot 60 + 19 = 64\frac{10}{21}$ & $\frac{378}{677} \approx 0.558$ \\
4	& $\left(\frac{12}{54} + \frac{12}{48} + \frac{12}{42} + \frac{12}{36}\right) \cdot 60 + 22 = 87\frac{10}{21}$ & $\frac{1008}{1837} \approx 0.549$
\end{tabular}

Итого, наиболее выгодная тактика --- проезжать три участка, после чего заменять все потерянные колёса.
\end{itemize}