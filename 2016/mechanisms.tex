\section{Эти необычные механизмы}
\begin{itemize}
\itA Легко заметить, что шестерёнки, находящиеся в зацеплении, всегда вращаются в 
разных направлениях.

(картинка)

\itB В японских шахматах сёги есть фигура Золото, которая бьёт шесть клеток вокруг себя
 (закрашены серым на рисунке 2). Можно ли расставить на поле $9 \times 9$ несколько таких 
фигур, чтобы каждая клетка поля билась ровно одним Золотом?

Это невозможно. Рассмотрим левый верхний угол доски. Если золото стоит на h1 или h2, 
тогда поля i2 (i3, соответственно) окажется не под боем: легко увидеть, что Золото,
бьющее это поле, обязательно заденет и поле h2 (h3, соответственно).

(картинка)

Если же золото стоит на i1 или i2, то фигуры не <<уложатся>> вдоль доски: Золото должно
далее стоять на f1 (f2), затем на c1(c2), и поле a1 окажется не под боем.

(картинка)

\itC Давайте вычислим: время на проезд $n$ участков без остановки --- 
это время на проезд первого участка 
со скоростью 54 км/ч, второго --- со скоростью 48 км/ч, третьего --- со скоростью 42 км/ч, и т.д.
плюс время на ремонт колёс. Или, формулой (время выражено в минутах):
 $$T(n) = \left(\frac{12}{54} + \frac{12}{48} + \dots + \frac{12}{60 - 6 \cdot n}\right) \cdot 60 + 10 + 3 \cdot n$$
Средняя скорость (в километрах в минуту), соответственно, будет равна расстоянию, поделённому на время, т.е.
$\frac{12 \cdot n}{T(n)}$.

Теперь с конкретными цифрами:

\begin{tabular}{lll}
Участки & Время на проезд & Средняя скорость \\
\hline
1	& $\frac{12}{54} \cdot 60 + 13 = 26\frac{1}{3}$ & $\frac{36}{79} \approx 0.456$ \\
2	& $\left(\frac{12}{54} + \frac{12}{48}\right) \cdot 60 + 16 = 44\frac{1}{3}$ & $\frac{72}{133} \approx 0.541$ \\
3	& $\left(\frac{12}{54} + \frac{12}{48} + \frac{12}{42}\right) \cdot 60 + 19 = 64\frac{10}{21}$ & $\frac{378}{677} \approx 0.558$ \\
4	& $\left(\frac{12}{54} + \frac{12}{48} + \frac{12}{42} + \frac{12}{36}\right) \cdot 60 + 22 = 87\frac{10}{21}$ & $\frac{1008}{1837} \approx 0.549$
\end{tabular}

Итого, наиболее выгодная тактика --- проезжать три участка, после чего заменять все потерянные колёса.