\task{Числа и суммы}
\begin{itemize}
\itA Число $0$ меньше количества цифр в нём. Число $0.1$ меньше собственной суммы цифр.

\itB Если в числе во всех разрядах, кроме старшего --- девятки, то само это число 
(любое меньшее число неизбежно будет иметь меньшие цифры хоть в одном 
разряде, и не будет иметь бо'льших цифр).

В противном случае, нужно вычесть $1$ из цифры в старшем разряде (а если там и так
$1$ --- отбросить его), а числа во всех остальных разрядах заменить на $9$. 
В самом деле, чтобы ещё увеличить сумму цифр такого числа, надо увеличивать (или добавлять)
старший разряд, а сделав это, мы получим число, большее исходного.

\itC Петя сложил все числа от 1 до $m \cdot n$, а Вася сложил все числа от 1 до $m$, 
от 1 до $n$ и посчитал произведение этих двух сумм. У кого в итоге получилось большее число?

%Воспользуемся способом Гаусса: как известно, сумма чисел от 1 до $k$ равна $\frac{k\cdot(k+1)}{2}$.

%Тогда у Пети $\frac{(m \cdot n) \cdot (m \cdot n + 1)}{2}
 

%m*n(m+1)*(n+1)/4=mn(mn+1) * mn(m+n)   --- (mn(mn+1))/2

\end{itemize}