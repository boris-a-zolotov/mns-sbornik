\task{Числа и суммы}
\begin{itemize}
\itA Число $0$ меньше количества цифр в нём. Число $0.1$ меньше собст-\linebreak венной суммы цифр.
Несложно, тем не менее, убедиться, что \underline{нату-} \underline{ральных} чисел с такими свойствами не найдётся.

\itB Если в числе во всех разрядах, кроме старшего --- девятки, то ответом будет само это число 
(любое меньшее число неизбежно будет иметь меньшие цифры хоть в одном 
разряде, и не будет иметь б\'ольших цифр).

В противном случае, нужно вычесть $1$ из цифры в старшем разряде (а если там и так
$1$ --- отбросить его), а числа во всех остальных разрядах заменить на $9$. 
В самом деле, чтобы ещё увеличить сумму цифр такого числа, надо увеличивать (или добавлять)
старший разряд, а, сделав это, мы получим число, большее исходного.

\itC Будем пользоваться способом Гаусса, который заключается в том,
что сумма чисел от 1 до $k$ равна
$$\frac{k\cdot(k+1)}{2}.$$

Можно сосчитать, что у Пети получилось $$\frac{mn \cdot (mn-1)}{2}.$$
А Васи получилось
$$\frac{m \cdot (m-1)}{2} + \frac{n \cdot (n-1)}{2}.$$

Заметим, что если $m,n > 1$, то $mn \ge m+n$.
В самом деле, в таких ограничениях $$(m-1)(n-1) \ge 1$$ 
А, значит, $$mn = (m-1)(n-1) + m + n - 1 \ge 1 + m + n - 1 = m + n.$$
Несмотря на сложные манипуляции с цифрами, идея данного доказательства --- геометрическая,
ведь в прямоугольнике $m \times n$, если $m$ и $n$ как минимум 2, мы всегда 
можем отметить $n$ клеток вдоль одной стороны,
$m-1$ клетку вдоль другой стороны, и ещё одну клетку где-то в середине.
\begin{center}%\begin{tabular}{ll}
%\tikz{
%   \fill[pattern=dots] (0,0) rectangle (0.5,1);
%   \fill[pattern=north west lines] (0.5,0) rectangle (1,1);
%   \draw[step=0.5] (0,0) grid (1,1);
%   \draw (0,0.5) node[rotate=90,above] {$n=2$};
%   \draw (0.5,1) node[above] {$m=2$}
%}&
\tikz{
   \fill[pattern=dots] (0,0) rectangle (0.5,1.5);
   \fill[pattern=north west lines] (0.5,1) rectangle (3,1.5);
   \fill[pattern=north west lines] (2.5,0.5) rectangle (3,1);
   \draw[step=0.5] (0,0) grid (3,1.5);
   \draw (0,0.7) node[rotate=90,above] {$n=3$};
   \draw (1.5,1.5) node[above] {$m=6$}
}
%\end{tabular}
\end{center}


Рассмотрим следующую цепочку неравенств: 
\begin{align*}
	& (mn)^2 - mn \ge (m+n)^2 - mn = m^2 + mn + n^2 > \\
	>\ & m^2 + n^2 > (m^2 - m) + (n^2 - n).
\end{align*}
Нетрудно убедиться, что левая формула в цепочке --- это удвоенная сумма Пети, а правая --- удвоенная сумма Васи.

Рассмотрим теперь оставшиеся случаи, когда $m=1$ или $n=1$. Поскольку в этом случае
$mn = \max(m,n)$, число у Васи окажется больше.

Ответ: если $m>1$ и $n>1$, то б\'ольшее число у Пети, если же $m=1$ или $n=1$, то 
б\'ольшее число у Васи.

\end{itemize}
