
\task{Дело-то житейское}
\begin{itemize}
\itA См. задачу {\bfseries 1B} из варианта 5 класса.

\itB По принципу Дирихле нельзя рассадить $n$ кроликов в $n-1$ клетку так, 
чтобы в каждой клетке оказалось ровно по кролику.
Мы будем раскладывать по $n-1$ ящикам $n$ сайтов, и в ящик номер $k$ мы будем класть
сайт, у которого $k$ ссылок на другие сайты. Принцип Дирихле докажет требуемое.

\itC Заметим, что максимально в строке имеется 4 промежутка между закрашенными цифрами
(промежуток --- число --- промежуток --- число --- промежуток --- число --- промежуток).
С другой стороны, минимально в строке $9 + 19\cdot 2 = 47$ символов и максимально может 
быть закрашено $3 \cdot 2 = 6$ цифр.
То есть, минимальная длина промежутка $41 \div 4 > 10$ символов, значит, одноразрядное
число может быть закрашено только если это число 1.

Пусть же число 1 не закрашено. Тогда заметим, что длина
промежутка до первой закрашенной цифры --- нечётна ($9 + x \cdot 2$), но длина промежутка
между первым и вторым закрашенным числом --- чётна.

Значит, закраска возможна только если число 1 закрашено.
Для завершённости приведём корректный пример такой закраски:

\begin{center}
{\underline{1}}23456789101112{\underline{13}}14151617181920{\underline{21}}22232425262728{\underline{29}}
\end{center}

\end{itemize}