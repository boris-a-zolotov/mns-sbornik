
\task{Дело-то житейское}
\begin{itemize}
\itA См. ранее...

\itB По принципу Дирихле нельзя рассадить $n$ кроликов в $n-1$ клетку так, 
чтобы в каждой клетке оказалось ровно по кролику.
Мы будем раскладывать по $n-1$ ящикам $n$ сайтов, и в ящик номер $k$ мы будем класть
сайт, у которого $k$ ссылок на другие сайты. Принцип Дирихле докажет требуемое.

\itC Заметим, что максимально в строке имеется 4 промежутка между закрашенными цифрами
(промежуток --- число --- промежуток --- число --- промежуток --- число --- промежуток).
С другой стороны, минимально в строке $9 + 19\cdot 2 = 47$ символов и максимально может 
быть закрашено $3 \cdot 2 = 6$ цифр.
То есть, минимальная длина промежутка $41 \div 4 > 10$ символов, значит, одноразрядная
цифра может быть закрашена только если это цифра 1.

Пусть же число 1 (первая цифра слева в строке) не закрашена. Тогда заметим, что длина
промежутка до первой закрашенной цифры --- нечётна ($9 + x \cdot 2$), но длина промежутка
между первым и вторым закрашенным числом --- чётна.

Значит, закраска возможна только если цифра 1 закрашена.
Для завершённости задачи стоит привести корректный пример закраски:

\begin{center}\ttfamily
{\bfseries 1}23456789101112{\bfseries 13}14151617181920{\bfseries 21}22232425262728{\bfseries 29}
\end{center}


Всего в строке возможно $9 + 19\cdot 2 = 47$, 49, 51 или 53 символа,
что даёт минимум $47 \div 4 = 9$ символов на промежутки. Значит,
одноразрядных цифр может быть закрашено не более одной.

Значит, закрашено 5 или 6 цифр, и на промежутки остаётся от $47-6=41$ до $53-5=48$ символов.

Если промежутков 3, равенство промежутков оставляет кратные трём варианты 42, 45 или 48
символов (14, 15 или 16 символов).


Рассмотрим варианты, сколько могло быть незакрашенных промежутков: 
если закрашены крайние даты --- тогда 3, если одна из крайних дат -- 4,
если крайних дат нет -- 5. 

\end{itemize}