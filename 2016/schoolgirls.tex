\task{Ох уж эти школьницы!}
\begin{itemize}
\itA Разложим 22887 на сомножители (попробуем понять, где Арина поставила знаки умножения). 
Число делится на 9 (т.к. сумма цифр 27), и
частное равно $2543$ --- числу, выглядящему, как четыре оценки, записанные подряд.

Более тщательная проверка покажет, что это действительно простое число (нам надо проверять его
делимость на простые числа, не превосходящие 53, состоящие только из цифр 1,2,3,4 и 5, таких немного),
то есть у Арины получилась формула $2543 \cdot 3 \cdot 3$ (с точностью до перестановки сомножителей).

Теперь осталось посчитать средний балл: $3\frac{1}{3}$.

\itB Пусть Ольга придумала числа $p$ и $q$, и пусть для определенности $p > q$. Введем новую
переменную $t = p - q$. Тогда $p^2 - q^2 = (q + t)^2 - q^2 = t^2 + 2qt = t(t + 2q)$.

Из условия мы знаем, что $p^2 - q^2 = 3476$, разложим на множители: 
$3476 = 2\cdot 2\cdot 11 \cdot 79$. 
Нужно теперь эти множители распределить между $t$ и $t+2q$. 

Заметим, что $t$ не может быть максимальным сомножителем. Кроме того, поскольку $t$ обязательно 
должен быть четным (иначе все произведение нечетно), то и выражение $t+2q$ тоже должно быть
четным. Эти требования дают нам единственное решение: $t = 22$ и $t + 2q = 2 \cdot 79$,
отсюда $q = 68$ и $p = 90$.

\itC 
%Девочка Лиана записывает пятизначные числа, переставляет их первую цифру в конец и 
%записывает полученные числа в пару к исходным. Проходящая мимо мама сказала: «Лиана, а зачем 
%ты пишешь пары чисел, сравнимых по модулю 41?» Права ли мама в своем вопросе --- действительно 
%ли числа в парах всегда сравнимы по этому модулю?

Вычислим разность двух чисел пары:
$$\overline{abcde} - \overline{bcdea} = 10000\cdot a + \overline{bcde} - 10\cdot\overline{bcde} - a = 9999\cdot a - 9 \cdot\overline{bcde}$$

Поскольку $9999 \bmod 41 = 36$, то мы легко подберем числа, при которых правило нарушается.
Например, возьмем $a = b = 1, c = d = e = 0$. Тогда $11000 \bmod 41 = 12$ и $10001 \bmod 41 = 38$.

\end{itemize}
