\task{Эксперименты с клавиатурой}
\begin{itemize}
\itA Для набора некоторого одиночного символа А. должен нажать несколько раз клавишу с символом, 
и, после этого, несколько раз \verb!Backspace! после этого. В итоге у нас должен остаться один
символ на экране: $1 = x \cdot 5 - y \cdot 8$. Из данного уравнения видно, что
$y = \frac{x \cdot 5 - 1}{8}$.

$y$ монотонно возрастает с ростом $x$, поэтому нам достаточно найти такой минимальный $x$, что 
$y$ будет целым и неотрицательным. Если $x=1$, то $y = 0.5$, не подходит (мы не можем нажать
клавишу наполовину). $x=2$ даёт $y = 1.125$, $x=3$ даёт $y = 1.75$, $x=4$ даёт $y = 2.375$,
и $x=5$ даёт $y=3$. То есть, для набора А. одного символа нам нужно 8 нажатий на клавиатуре:
пять раз нажать символ и три раза --- \verb!Backspace!.

Аналогично, для Б. $y = \frac{x \cdot 7 - 1}{4}$, и минимальный подходящий $x$ равен 3, при этом
Б. для набора символа требуется 7 нажатий на клавиатуре.

Поэтому обычно Б. будет печатать несколько быстрее, чем А., хотя в особых случаях (скажем,
в случае текста, в котором каждая буква повторяется по пппппяяяяятттттььььь раз подряд)
преимущество будет у А.

\itB {\bfseries АБВГДЕЖЗ} --- эту строчку А. печатает за 2 секунды (8 букв и два раза \verb!Caps Lock!), 
Б. же печатает её за 4 секунды.

Обратный вариант невозможен: если А. будет даже дважды нажимать \verb!Caps Lock! для ввода 
каждой заглавной буквы, его скорость ввода будет две буквы за 1.2 секунды (6 нажатий на 2 буквы). 
Б., в свою очередь, печатает две буквы за 1 секунду. То есть разность в скорости печати --- 1.2 раза
в худшем случае, что значительно меньше требуемых 2 раз.

\itC В данных условиях всё, что мы можем --- нажать четыре клавиши в каком-то порядке за секунду,
после этих нажатий на экране появится буква. 
Всего возможно $2\cdot 2 \cdot 2 \cdot 2$ комбинаций, что даёт 
возможность выбрать одну из 16 букв за одну секунду. 
Однако, в русском 33 буквы, значит, какие-то из букв мы ввести не сможем. Если же какие-то 17
букв мы не будем различать (объединим в одну) --- скажем, обозначим все буквы, начиная с {\bfseries О},
как {\bfseries Щ}, то мы как раз получим 16-буквенный алфавит: {\bfseries АБВГДЕЁЖЗИЙКЛМНЩ}. 
Текст, кщнещно, после такого изменения алщавища бщдещ выглядеть нещбщщнщ.
\end{itemize}
