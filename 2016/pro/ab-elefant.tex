\task{$(a,b)$--слоны}

\begin{enumerate}

\item \begin{itemize}
\item[(а)] Выпишем изменения координат, которые происходят при ходе. 
Cлон в первой части хода двигается на $a$ клеток 
в одном из четырёх направлений: $(-a,a), (a,-a), (-a,-a), (a,a)$. Во втором 
движении он поворачивает на $90^\circ$ и двигается ещё на $b$ клеток в одном
из двух направлений: $(-a-b,a-b), (-a+b,a+b), \ldots$ и т.п.

Так как $a\ne b$, то каждая из сумм $a+b$, $a-b$, $-a-b$ и $-a+b$ будет отличаться
от других. Итого, все 8 вариантов хода различны.

\item[(б)] Да, $(a,b)$-слон тождественен $(b,a)$-слону. Рассмотрим всё множество точек,
в которые мы можем прийти: это все комбинации плюсов и минусов вида
$(\pm a \pm b, \pm a \pm b)$, за исключением
вариантов вида $(\pm (a+b), \pm (a+b))$ и $(\pm (a-b), \pm (a-b))$
--- иными словами, кроме тех вариантов, где во втором движении слон идёт в направлении
первого движения или в обратном направлении. Легко заметить, что от обмена
$a$ и $b$ это множество не поменяется.

\item[(в)] Обычный слон в некоторых положениях бьёт 13 клеток, поэтому в точном смысле он
не может являться никаким $(a,b)$-слоном, у которого только 8 вариантов хода. Однако,
если рассмотреть $(1,0)$-слона, то множество полей, которые он способен в принципе достичь,
совпадает с множеством для обычного слона.
\end{itemize}

\item Чёрные клетки --- клетки, у которых сумма координат чётна. Тогда давайте
назовём \emph{горизонтальными} чёрными клетками те клетки, обе координаты которых чётны. 
Остальные чёрные клетки назовём \emph{вертикальными}. 

Если взять $(1,0)$-слона, то если $x$ и $y$ --- чётны, то $x \pm 1$ и $y \pm 1$ нечётны,
и наоборот.

\begin{center}\tikz[scale=0.7]{
    \draw[thick] (0,0) rectangle (4,4);
    \foreach \x/\y in {0/0,0/2,2/0,2/2} {
        \fill[pattern = horizontal lines] (\x,\y) rectangle (\x+1,\y+1);
    }
    \foreach \x/\y in {1/1,1/3,3/1,3/3} {
        \fill[pattern = vertical lines] (\x,\y) rectangle (\x+1,\y+1);
    }
    \draw (0,0) grid (4,4);
    \foreach \y [count=\n] in {0,1,2,3} {
        \draw (0,\n-0.5) node [left] {$\y$};
    }
    \foreach \y [count=\n] in {0,1,2,3} {
        \draw (\n-0.5,0) node [below] {$\y$};
    }
    %\fill (1.5,7.5) circle[radius=0.1];
}\end{center}

Для того, чтобы $(a,b)$-слон ходил только по горизонтальным (вертикальным)
чёрным клеткам, нужно, чтобы он сохранял чётность каждой из координат.
То есть, $(a+b) \divsby 2$ и $(a-b) \divsby 2$. Заметим, что оба условия
выполняются или не выполняются одновременно, то есть в качестве требуемого
свойства мы можем выбрать, например, $(a+b) \divsby 2$.

Если же $(a+b)$ не кратно 2, то такой слон будет менять чётность координат
при каждом ходе.



\end{enumerate}