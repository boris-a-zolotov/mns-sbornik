\task{Рекуррентные функции}

\begin{enumerate}
\item Рассмотрим $f(a,b)$. Если $a \ge b$, тогда либо $b = 0$ (и результат функции определён однозначно),
либо $f(a,b) = f(a-b,a)$, при $b = 0$. Далее мы можем продолжить процесс вычисления, и поскольку
сумма $a+b$ при каждой итерации уменьшается хотя бы на 1, то неизбежно один из аргументов
станет равным $0$ и процесс завершится. 

Если же $a < b$, то $f(a,b) = f(b,a)$ и вычисление сводится к предыдущему случаю.

Таким образом, функция задана для любых пар $(a,b)$, причём процесс вычисления единственен.

\item Если $f(a,0) = a$, то $f(a,b) = \gcd(a,b)$.
В самом деле, $\gcd(a,b) = \gcd(a-b,b)$, то есть на каждом шаге вычисление НОД сохраняется.
На последнем же шаге $\gcd(a,0) = a$, но это же значение и является результатом вычисления.

Заметим, что данное вычисление --- не что иное, как алгоритм Евклида.

\item В общем случае, если $f(a,0) = g(a)$, то $f(a,b) = g(\gcd(a,b))$, поскольку данное 
изменение повлияет только на последний шаг вычисления.

\item Данное определение функции не отличается по структуре вычисления от первого пукта ---
различается только результат. Поэтому функция по-прежнему будет определена во всех точках.

\item Пусть $T(x) = x + k$ и пусть $g(a) = f(a,0) = f(0,a)$. Тогда перепишем условие:

$$f(a,b) = \left\{\begin{array}{ll}
   g(a), & b = 0\\
   k+f(a-b,b), & a \ge b > 0\\
   f(b,a), & a > b\end{array}\right.$$

или иначе: $f(a,b) = g(\gcd(a,b)) + ks$, где $s$ --- количество вычитаний, требуемых,
чтобы получить $\gcd(a,b)$ (т.е. количество шагов алгоритма Евклида). Точной формулы
для вычисления количества шагов алгоритма Евклида на сегодняшний день не известно.

\item (h(a,b)>0)

$h(p_1\cdot p_2\cdot \ldots \cdot p_n, q_1 \cdot q_2 \cdot \ldots \cdot q_n) =
\Pi_{1 \le i,j \le n} h(p_i, p_j)$

$h(a,b) = T^k(0,\gcd(a,b))$



%h(17,3) = h(2,3) = w
%h(19,7) = h(5,7) = T(h(7,5)) = T(h(2,5))

%h(ac mod b,b) = h(a,b)*h(c,b) = h(a mod b,b)*h(c mod b,b)
%h(a,b) = T^s_1(p_1,q_1)*T^s_2(p_2,q_2)*...*T^s_n(p_n,q_n)

%(25,5) = (5,5)*(5,5) = 0
%(25,5) = (5,5)

%h(x^2-1,x) = (x-1,x)
% = ((x-1)*(x+1),x) = (x-1,x)*(1,x)

%h(24,5) = (4,5) = (6,5)*(4,5) => (6,5)=1
%h(x,x) = 1

%h(1,x) = h(x,1) = 1

\end{enumerate}
