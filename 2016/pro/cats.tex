\task{Кошки на координатной плоскости}

\begin{enumerate}
\item Нарисуем происходящее:

\begin{center}\tikz{ 
    \draw[thick] (-2,-1) rectangle (2,1);
    \draw[->] (0,-1) -- (0,1);
    \draw (2,0) circle[radius=0.1]; 
    \draw[->,dashed] (2,0) -- (-2,0);
    \draw[->,dashed] (2,0) -- (-2,0.9);
}\end{center}

Построим математическую модель движения кошки и П. --- параметризуем
их движение временем (параметром $t$), изменяющимся от 0 до 1.
Для этого вспомним формулу, позволяющую вычислить положение точки на отрезке
$[x_0,x_1]$ в момент времени $t$ при равномерном прямолинейном движении (данная
формула выводится в пункте 7 данной задачи):
$$x_t = x_1 \cdot t + x_0 \cdot (1-t)$$

Итак, в момент времени 0 оба участника находятся в начальных точках.
В момент времени $t$ кошка находится в точке $(0,2t-1)$,
а П. --- в точке $$(-2,y)\cdot t + (2,0)\cdot (1-t) = (2-4t,yt)$$
Легко заметить, что П. пересечёт путь кошки в момент $t=\frac{1}{2}$.
Поскольку кошка движется в сторону увеличения координаты $Y$, имеем:
$$y\cdot\frac{1}{2} = \frac{y}{2} > 2\cdot\frac{1}{2} -1 = 0$$
То есть, П. должен стремиться к финишным точкам вида $(-2,y)$, где $y > 0$.

\item Выведем формулу для координаты кошки в зависимости от времени.
Известно, что равномерное прямолинейное движение описывается уравнением
$$p(t) = v\cdot t + p_0$$
В двумерном случае формула получается такой:
$$(x(t),y(t)) = (v_x\cdot t + x_0,v_y\cdot t + y_0)$$
Также мы знаем, что в момент времени $t=1$ кошка окажется в точке $(x_1,y_1)$.

Значит, $x(1) = x_1 = v_x \cdot 1 + x_0$, то есть $v_x = x_1 - x_0$.
То есть, $$x(t) = (x_1 - x_0) \cdot t + x_0 = x_1 \cdot t + x_0 \cdot (1-t)$$.
Аналогично проведя вычисления с координатой $Y$, получим:
$$(x(t),y(t)) = (x_1 \cdot t + x_0 \cdot (1-t), y_1 \cdot t + y_0 \cdot (1-t))$$

\end{enumerate}
