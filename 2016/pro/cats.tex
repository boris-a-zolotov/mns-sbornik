\task{Кошки на координатной плоскости}

\begin{enumerate}
\item Нарисуем происходящее:

\begin{center}\tikz{ 
    \draw[thick] (-2,-1) rectangle (2,1);
    \node[below] at (-2,-1) {$-2$};
    \node[below] at (2,-1) {$2$};
    \node[left] at (-2,-1) {$-1$};
    \node[left] at (-2,1) {$1$};
    \node[below] at (0,-1) {\itshape кошка};
    \draw[->] (0,-1) -- (0,1);
    \draw (2,0) circle[radius=0.1] node[right] {\itshape П.}; 
    \draw[->,dashed] (2,0) -- (-2,0);
    \draw[->,dashed] (2,0) -- (-2,0.9);
}\end{center}

Построим математическую модель движения кошки и П. --- параметризуем
их движение временем (параметром $t$), изменяющимся от 0 до 1.
Для этого вспомним формулу, позволяющую вычислить положение точки на отрезке
$[x_0,x_1]$ в момент времени $t$ при равномерном прямолинейном движении (данная
формула выводится в пункте 7 данной задачи):
$$x_t = x_1 \cdot t + x_0 \cdot (1-t)$$

Итак, в момент времени 0 оба участника находятся в начальных точках.
В момент времени $t$ кошка находится в точке $(0,2t-1)$,
а П. --- в точке $$(-2,y)\cdot t + (2,0)\cdot (1-t) = (2-4t,yt)$$
Легко заметить, что П. пересечёт путь кошки в момент $t=\frac{1}{2}$.
Поскольку кошка движется в сторону увеличения координаты $Y$, имеем:
$$y\cdot\frac{1}{2} = \frac{y}{2} > 2\cdot\frac{1}{2} -1 = 0$$
То есть, П. должен стремиться к финишным точкам вида $(-2,y)$, где $y > 0$.

\item 

Заметим, что если финишная точка П. лежит правее прямой $x=3$,
то тогда П. не пересечёт след кошки несчастья и выполнит условие
задачи. 

Теперь посмотрим, какие возможны финишные точки за прямой $x=3$.
Чем дальше (левее) точка, тем быстрее до прямой $x=3$ дойдёт П.,
значит, тем меньше шансы, что П. не пересечёт след кошки.

Пусть П. идёт в точку $(x,y)$. Составим уравнение встречи П.
и кошки несчастья. 
$$(3,0)(1-t) + (3,3)t = (9,1.5)(1-t) + (x,y)t$$
Отсюда можно выразить координаты $(x,y)$ через параметр $t$
(момент встречи):
$$\left\{\begin{array}{l}x(t) = \frac{3(3t-2)}{t}\\y(t)=\frac{3(3t-1)}{2t}\end{array}\right.$$
И далее можно выразить $y$ через $x$:
$$\left\{\begin{array}{l}t = \frac{6}{9-x}\\y=\frac{x+9}{4}\end{array}\right.$$

Из изложенного ясно, что П. устроят все финишные точки, которые
лежат слева от прямой $y = \frac{x+9}{4}$ внутри прямоугольника $[0,9]\times[0,3]$,
а также все точки внутри прямоугольника $(3,9]\times[0,3]$.

\begin{center}\tikz{ 
    \draw[thick] (0,0) rectangle (9,3);
    \fill[pattern=north west lines] (0,2.25) -- (0,3) -- (3,3);
    \fill[pattern=north west lines] (3,0) -- (3,3) -- (9,3) -- (9,0);
    \node[below] at (0,0) {$0$};
    \node[below] at (9,0) {$9$};
    \node[left] at (0,0) {$0$};
    \node[left] at (0,3) {$3$};
    \draw (3,0) circle[radius=0.05] node[below] {\itshape {кошка несчастья\vphantom{д}}};
    \draw (6,3) circle[radius=0.05] node[above] {\itshape {кошка удачи}};
    \draw[->] (6,3) -- (6,2.5);
    \draw[->] (3,0) -- (3,0.5);
    %\draw (3,0.5) -- (3,3) -- (0,2.25);
    \draw (9,1.5) circle[radius=0.1] node[right] {\itshape П.}; 
    %\draw[dashed,->] (9,1.5) -- (0,2.25);
    %\draw[->,dashed] (2,0) -- (-2,0);
    %\draw[->,dashed] (2,0) -- (-2,0.9);
}\end{center}

По аналогии с кошкой несчастья, желание зарядиться от первой кошки удачей потребует
от П., чтобы его финишные точки находились не правее линии $x=6$, но при этом
не левее линии, получаемой из уравнения
$$(6,3)(1-t) + (6,0)t = (9,1.5)(1-t) + (x,y)t$$
%9-9t+xt = 6t + 6-6t   => 3 = (9-x)t  => t = 3/9-x
%y = -3(t-1)/(2t) => y = 3 - x/2
или, после преобразований
$$\left\{\begin{array}{l}t = \frac{3}{9-x}\\y=3-\frac{x}{2}\end{array}\right.$$

Построив пересечение этих двух областей, получим решение:

\begin{center}\tikz{ 
    \draw[thick] (0,0) rectangle (9,3);
    \fill[pattern=north west lines] (0,2.25) -- (0,3) -- (3,3);
    \fill[pattern=north west lines] (3,0) -- (3,3) -- (9,3) -- (9,0);
    \fill[pattern=north east lines] (0,0) -- (0,3) -- (6,0);
    %\draw (6,0) -- (3,1.5);
    %\fill[color=white] (3,1.5) circle [radius=0.05];
    %\draw (3,1.5) circle [radius=0.05];
    %\draw (0,3) -- (1,2.5);
    %\fill[color=white] (1,2.5) circle [radius=0.05];
    \draw (1,2.5) circle [radius=0.05];
    \node[below] at (0,0) {$0$};
    \node[below] at (9,0) {$9$};
    \node[left] at (0,0) {$0$};
    \node[left] at (0,3) {$3$};
    \draw (3,0) circle[radius=0.05] node[below] {\itshape {кошка несчастья\vphantom{д}}};
    \draw (6,3) circle[radius=0.05] node[above] {\itshape {кошка удачи}};
    \draw[->] (6,3) -- (6,2.5);
    \draw[->] (3,0) -- (3,0.5);
    %\draw (3,0.5) -- (3,3) -- (0,2.25);
    \draw (9,1.5) circle[radius=0.1] node[right] {\itshape П.}; 
    %\draw[dashed,->] (9,1.5) -- (0,2.25);
    %\draw[->,dashed] (2,0) -- (-2,0);
    %\draw[->,dashed] (2,0) -- (-2,0.9);
}\end{center}

Или, в виде уравнения:

$$\left\{\begin{array}{l}y \le 3-\frac{x}{2}\\y > \frac{x+9}{4} \vee x > 3\\x \in [0,6]\\y \in [0,3]\end{array}\right.$$

\item Чтобы кошка, направляющаяся в точку $(3,0)$, успела бы перебежать П.
дорогу, требуется, чтобы в момент $t=\tfrac{2}{3}$ она оказалась бы не выше прямой
$y=1.5$ (условие 1), начав движение не ниже прямой $y=1.5$ (условие 2).
Условие 2 необходимо для того, чтобы факт пересечения следа кошки имел бы место.
Если кошка начала бы движение даже из точки $(3,3)$, то в момент $t$ она 
уже окажется в точке $(3,1)$, успев перебежать П. дорогу. Чем ниже мы поставим 
кошку, тем раньше она пройдёт точку $(3,1.5)$. Поэтому условие 1
выполнено всегда, и нам достаточно удовлетворить условие 2.

Похожие рассуждения справедливы и для второй кошки: если встреча с кошкой, 
направляющейся в $(6,3)$, нежелательна,
то кошка должна начать движение либо с прямой выше $y=1.5$ (чтобы исключить
сам факт пересечения следов), либо две трети пути кошки должны быть больше 
$1.5$ (чтобы на трети пути кошка ещё не достигла прямой $y=1.5$).
                                                              
Вместе это даёт следующие требования на координаты кошек:
$$\left\{\begin{array}{l}x_1 = 3; y_1\ge 1.5\\x_2 = 6; y_2 < 0.75 \vee y_2>1.5\end{array}\right.$$

\item Итак, встреча с двумя кошками желательна, а ещё с двумя --- нежелательна.
Пусть желательные кошки начинают свой путь с прямой $y=1.5$
($\textit{Ж}_1$, $\textit{Ж}_2$), а нежелательные --- заканчивают там
($\textit{Н}_1$, $\textit{Н}_2$):

\begin{center}\tikz{ 
    \draw[thick] (0,0) rectangle (9,3);
    \node[below] at (0,0) {$0$};
    \node[below] at (9,0) {$9$};
    \node[left] at (0,0) {$0$};
    \node[left] at (0,3) {$3$};
    %\draw (3,0.5) -- (3,3) -- (0,2.25);
    \draw (9,1.5) circle[radius=0.1] node[right] {\itshape П.}; 
    \draw[dashed,->] (9,1.5) -- (0,1.5);
    \draw[->] (3,1.5) node[above] {$\textit{Ж}_1$} -- (3,1); \draw[dotted] (3,1) -- (3,0);
    \draw[->] (3,0) node[below] {$\textit{Н}_1$} -- (3+0.6,0.3); \draw[dotted] (3+0.6,0.3) -- (6,1.5);
    \draw[->] (6,1.5) node[above] {$\textit{Ж}_2$} -- (6,1); \draw[dotted] (6,1) -- (6,0);
    \draw[->] (6,0) node[below] {$\textit{Н}_2$} -- (6-0.6,0.3); \draw[dotted] (6-0.6,0.3) -- (3,1.5);
    \draw (3,1.5) circle[radius=0.05];
    \draw (6,1.5) circle[radius=0.05];
    \draw (3,0) circle[radius=0.05];
    \draw (6,0) circle[radius=0.05];
    %\draw[->,dashed] (2,0) -- (-2,0);
    %\draw[->,dashed] (2,0) -- (-2,0.9);
}\end{center}

Очевидно, что желательные кошки перебегут дорогу П. в момент $t=0$, а нежелательные дойдут до 
пути П. только тогда, когда он уже придёт в точку назначения.

\item Такое невозможно. В самом деле,
по условию, у каждой кошки её финишная позиция является стартовой для другой кошки, и
ни у каких двух кошек нет одинаковых начальных и конечных позиций.
Значит, эти позиции образуют перестановку. Перестановку можно разложить на циклы,
наибольший интерес представляет случай одного цикла длины 4. Здесь 
первая кошка идёт от $(x_1,y_1)$ до $(x_2,y_2)$, 
вторая --- от $(x_2,y_2)$ до $(x_3,y_3)$, третья --- от $(x_3,y_3)$ до $(x_4,y_4)$ и 
четвёртая --- от $(x_4,y_4)$ до $(x_1,y_1)$. 
При этом, чтобы три кошки были кошками удачи, 
они должны иметь начальные позиции выше конечных: то есть 
$y_1 > y_2 > y_3 > y_4$ (мы всегда можем сдвинуть номера кошек по циклу так, 
чтобы стартовая позиция кошки несчастья шла бы последней). Но, значит, максимум две из этих
кошек удачи могут повстречаться П. (в случае, если $y_2 = 1.5$ или $y_3 = 1.5$). 

Если же перемещения кошек раскладываются на два или более циклов, то неизбежно в 
каждом цикле должна быть как минимум одна кошка, не являющаяся кошкой удачи.

\item Кошка не сдвигается по горизонтали, поэтому, чтобы решить задачу, давайте добьёмся 
синхронного движения по вертикали кошки и П. Тогда П., 
оказавшись на вертикали $x$, встретится с кошкой (поскольку обе 
координаты совпадут). При этом П. обязательно окажется на вертикали $x$, поскольку
он проходит всё поле слева направо.

Возьмём $a=9$ и $b=3$. Тогда $$y_{\textit{кошки}} = 9(1-t) + 3t = y_{\textit{П.}}$$
и поэтому горизонталь кошки и П. всегда будет одной и той же.

\item Выведем формулу для координаты кошки в зависимости от времени.
Известно, что равномерное прямолинейное движение описывается уравнением
$$p(t) = v\cdot t + p_0$$
В двумерном случае формула получается такой:
$$(x(t),y(t)) = (v_x\cdot t + x_0,v_y\cdot t + y_0)$$
Также мы знаем, что в момент времени $t=1$ кошка окажется в точке $(x_1,y_1)$.

Значит, $x(1) = x_1 = v_x \cdot 1 + x_0$, то есть $v_x = x_1 - x_0$.
То есть, $$x(t) = (x_1 - x_0) \cdot t + x_0 = x_1 \cdot t + x_0 \cdot (1-t)$$.
Аналогично проведя вычисления с координатой $Y$, получим:
$$(x(t),y(t)) = (x_1 \cdot t + x_0 \cdot (1-t), y_1 \cdot t + y_0 \cdot (1-t))$$

\end{enumerate}
