\task{Кошки на координатной плоскости}

\begin{enumerate}
\item Нарисуем происходящее:

\begin{center}\tikz{ 
    \draw[thick] (-2,-1) rectangle (2,1);
    \node[below] at (-2,-1) {$-2$};
    \node[below] at (2,-1) {$2$};
    \node[left] at (-2,-1) {$-1$};
    \node[left] at (-2,1) {$1$};
    \node[below] at (0,-1) {\itshape кошка};
    \draw[->] (0,-1) -- (0,1);
    \draw (2,0) circle[radius=0.1] node[right] {\itshape П.}; 
    \draw[->,dashed] (2,0) -- (-2,0);
    \draw[->,dashed] (2,0) -- (-2,0.9);
}\end{center}

Построим математическую модель движения кошки и П. --- параметризуем
их движение временем (параметром $t$), изменяющимся от 0 до 1.
Для этого вспомним формулу, позволяющую вычислить положение точки на отрезке
$[x_0,x_1]$ в момент времени $t$ при равномерном прямолинейном движении (данная
формула выводится в пункте 7 данной задачи):
$$x_t = x_1 \cdot t + x_0 \cdot (1-t)$$

Итак, в момент времени 0 оба участника находятся в начальных точках.
В момент времени $t$ кошка находится в точке $(0,2t-1)$,
а П. --- в точке $$(-2,y)\cdot t + (2,0)\cdot (1-t) = (2-4t,yt)$$
Легко заметить, что П. пересечёт путь кошки в момент $t=\frac{1}{2}$.
Поскольку кошка движется в сторону увеличения координаты $Y$, имеем:
$$y\cdot\frac{1}{2} = \frac{y}{2} > 2\cdot\frac{1}{2} -1 = 0$$
То есть, П. должен стремиться к финишным точкам вида $(-2,y)$, где $y > 0$.

\item 

Заметим, что если финишная точка П. лежит правее прямой $x=3$,
то тогда П. не пересечёт след кошки несчастья и выполнит условие
задачи. 

Теперь посмотрим, какие возможны финишные точки за прямой $x=3$.
Чем дальше (левее) точка, тем быстрее до прямой $x=3$ дойдёт П.,
значит, тем меньше шансы, что П. не пересечёт след кошки.

Пусть П. идёт в точку $(x,y)$. Составим уравнение встречи П.
и кошки несчастья. 
$$(3,0)(1-t) + (3,3)t = (9,1.5)(1-t) + (x,y)t$$
Отсюда можно выразить координаты $(x,y)$ через параметр $t$
(момент встречи):
$$\left\{\begin{array}{l}x(t) = \frac{3(3t-2)}{t}\\y(t)=\frac{3(3t-1)}{2t}\end{array}\right.$$
И далее можно выразить $y$ через $x$:
$$\left\{\begin{array}{l}t = \frac{6}{9-x}\\y=\frac{x+9}{4}\end{array}\right.$$

Из изложенного ясно, что П. устроят все финишные точки, которые
лежат слева от прямой $y = \frac{x+9}{4}$ внутри прямоугольника $[0,9]\times[0,3]$,
а также все точки внутри прямоугольника $(3,9]\times[0,3]$.


\begin{center}\tikz{ 
    \draw[thick] (0,0) rectangle (9,3);
    \fill[pattern=north west lines] (0,2.25) -- (0,3) -- (3,3);
    \fill[pattern=north west lines] (3,0) -- (3,3) -- (9,3) -- (9,0);
    \node[below] at (0,0) {$0$};
    \node[below] at (9,0) {$9$};
    \node[left] at (0,0) {$0$};
    \node[left] at (0,3) {$3$};
    \node[below] at (3,0) {\itshape {кошка несчастья\vphantom{д}}};
    \node[below] at (6,0) {\itshape {кошка удачи}};
    \draw[->] (6,3) -- (6,0);
    \draw[->] (3,0) -- (3,3);
    \draw (9,1.5) circle[radius=0.1] node[right] {\itshape П.}; 
    %\draw[->,dashed] (2,0) -- (-2,0);
    %\draw[->,dashed] (2,0) -- (-2,0.9);
}\end{center}

\item Выведем формулу для координаты кошки в зависимости от времени.
Известно, что равномерное прямолинейное движение описывается уравнением
$$p(t) = v\cdot t + p_0$$
В двумерном случае формула получается такой:
$$(x(t),y(t)) = (v_x\cdot t + x_0,v_y\cdot t + y_0)$$
Также мы знаем, что в момент времени $t=1$ кошка окажется в точке $(x_1,y_1)$.

Значит, $x(1) = x_1 = v_x \cdot 1 + x_0$, то есть $v_x = x_1 - x_0$.
То есть, $$x(t) = (x_1 - x_0) \cdot t + x_0 = x_1 \cdot t + x_0 \cdot (1-t)$$.
Аналогично проведя вычисления с координатой $Y$, получим:
$$(x(t),y(t)) = (x_1 \cdot t + x_0 \cdot (1-t), y_1 \cdot t + y_0 \cdot (1-t))$$

\end{enumerate}
