\task{Числа, выписанные на доску}
\begin{itemize}
\itA Коля в свой День рождения выписал на доску наименьшее число, дважды содержащее 
все цифры от 0 до 9 и делящееся на 72. Выпишите и вы это число!


Число делится на 72 тогда и только тогда, когда делится на 8 и на 9.
Поскольку $1+2+3+4+5+6+7+8+9 = 45$, любое число, составленное из этих цифр, делится на 9.
Чтобы число делилось на 8, требуется, чтобы последние его три цифры как число делились
на 8. 

Чтобы число было минимальным, в его начале должны быть минимально возможные цифры.
К сожалению, мы не можем ставить в первый разряд 0 (запись чисел с ведущими нулями не 
очень грамотная, а Коля --- мальчик очень аккуратный), поэтому в первый разряд поставим
следующую возможную цифру --- 1. Остальные же цифры отсортируем по убыванию.
Мы получим 10012233445566778899. 
Это число нечётное, поэтому не делится на 8.
Надо найти ближайшее к нему большее его, которое бы делилось.

По признаку
делимости число $\overline{abc}$ делится на 8, если $4a + 2b +c$ делится на 8.
Нетрудно видеть, что двух цифр 8 и нечётных 7 и 9 недостаточно для делимости:
поскольку неизбежно $c = 8$, то и $4а + 2b$ должно делиться на 8, значит, 
и $a$ и $b$ --- чётные. Поэтому нам необходимо ещё иметь и 6 в этом числе --- ближайшую
чётную цифру.

Минимальное чётное число с 6 в правых трёх разрядах --- это 10012233445567788996
(перенесли 6 вправо, остальные цифры разместили в порядке возрастания).
Но это число не делится на 8. Следующее за ним чётное число будет 10012233445567789698 (перенос в четвёртый
справа разряд, остальные цифры по возрастанию),
а затем 10012233445567789896 --- и оно нас устраивает.

\itB А Оля записала на доску числа от 1 до 121 и теперь занимается следующим: 
стирает с доски числа $a$ и $b$, записывая вместо них разность вида $a-2b$ либо $b-2a$. 
Могло ли в конце на доске остаться единственное число — ноль?


(решение)

\itC Пусть заданы числа $a_1, \dots, a_n$. Рассмотрим остатки от деления суммы начальных 
отрезков этих чисел на $n$:
$a_1 \bmod n$, $(a_1+a_2) \bmod n$, $(a_1+a_2+a_3) \bmod n$ и т.д. 
Всего есть $n$ таких отрезков, поэтому, возможно, все остатки будут различны.
Если это так, тогда среди них обязательно будет остаток 0. 

В противном случае обязательно какой-то из остатков повторится два раза:
$(a_1 + \dots + a_k) \bmod n = (a_1 + \dots + a_l) \bmod n$, $k < l$. Тогда
неизбежно $(a_{k+1} + \dots + \a_l) \bmod n = 0$.

\end{itemize}