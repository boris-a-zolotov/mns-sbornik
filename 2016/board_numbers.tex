\task{Числа, выписанные на доску}
\begin{itemize}
\itA Число делится на 72 тогда и только тогда, когда делится на 8 и на 9.
Поскольку $1+2+3+4+5+6+7+8+9 = 45$, любое число, составленное из этих цифр, делится на 9.
Чтобы число делилось на 8, требуется, чтобы последние его три цифры как число делились
на 8. 

Чтобы число было минимальным, в его начале должны быть минимально возможные цифры.
К сожалению, мы не можем ставить в первый разряд 0 (запись чисел с ведущими нулями не 
очень грамотная, а Коля --- мальчик очень аккуратный), поэтому в первый разряд поставим
следующую возможную цифру --- 1. Остальные же цифры отсортируем по возрастанию.
Мы получим $$10012233445566778899$$
Это число нечетное, поэтому не делится на 8.
Надо найти ближайшее к нему большее его, которое бы делилось.

По признаку
делимости число $\overline{abc}$ делится на 8, если $4a + 2b +c$ делится на 8.
Нетрудно видеть, что двух цифр 8 и нечетных 7 и 9 недостаточно для делимости:
поскольку неизбежно $c = 8$, то и $4a + 2b \divsby 8$, значит, $2a+b \divsby 4$, 
то есть $b \divsby 2$ и $(a + b \div 2) \divsby 2$. Поэтому, если $b = c = 8$,
то $a \divsby 2$. Поэтому нам надо добавить еще одну четную цифру к рассмотрению,
6.

Какие числа, делящиеся на 8, можно собрать из одной шестерки и цифр 8,8,9,9? 
Рассуждая аналогично предыдущему абзацу, построим таблицу:

\begin{center}\begin{tabular}{lll}
Позиция 6 & Последние 3 цифры & Все число\\
\hline
$a=6$ & $688$ & $10012233445567799688$\\
$b=6$ & $968$ & $10012233445567789968$\\
$c=6$ & $896$ & $10012233445567789896$
\end{tabular}\end{center}

Из перечисленных вариантов выберем минимальный, это и будет ответ на задачу:
$$10012233445567789896$$

\itB Вместо чисел $a$ и $b$, сумма которых равна $a+b$, на доске может оказаться
одно из чисел $a-2b$ или $b-2a$. Заметим, что в первом случае сумма чисел
на доске уменьшится на $3b$, а во втором случае — на $3a$. Это значит, что остаток
от деления на 3 всей суммы чисел на доске остается неизменным.

Можно заключить, что если на доске получился ноль, то изначальная сумма чисел делилась на 3. Однако
	$$1+2+3+\ldots+121 = \frac{121 \cdot 122}{2}\text{\quad —}$$
на три не делится. Поэтому ноль не мог получиться.

\itC Пусть заданы числа $a_1, \dots, a_n$. Рассмотрим остатки от деления суммы начальных 
отрезков этих чисел на $n$:
$a_1 \bmod n$, $(a_1+a_2) \bmod n$, $(a_1+a_2+a_3) \bmod n$ и т.д. 
Всего есть $n$ таких отрезков, поэтому, возможно, все остатки будут различны.
Если это так, тогда среди них обязательно будет остаток 0. 

В противном случае обязательно какой-то из остатков повторится два раза:
$(a_1 + \dots + a_k) \bmod n = (a_1 + \dots + a_l) \bmod n$, $k < l$. Тогда
неизбежно $(a_{k+1} + \dots + a_l) \bmod n = 0$.

\end{itemize}