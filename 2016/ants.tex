\task{Очень умные муравьи}
\begin{itemize}
\itA На плоскости живут четыре муравья. Могут ли они нарисовать себе на плоскости четыре связные 
области так, чтобы любые два муравья могли бы общаться друг с другом~— их области имели бы участок 
общей границы?

(картинка)

\itB Ответ на задачу зависит от того, какую точку мы считаем началом восхождения.
Давайте считать, что муравьи совершают восхождение из внутреннего объёма коробки, из самого её центра.
Тогда расстояние от центра до середины стенки --- $0.5$ метра, расстояние от середины стенки до угла
коробки по теореме Пифагора --- $\sqrt{0.5^2 + 0.5^2}$, и расстояние от центра коробки до угла по той же теореме ---
$\sqrt{(0.5^2 + 0.5^2) + 0.5^2} \approx 0.866}$.

Поскольку муравьи в 1000 раз короче людей, нам нужно увеличить все размеры в 1000 раз --- и мы получим высоту холма чуть больше 866 метров.

\itC Счётное сообщество муравьёв хочет организовать на плоскости треугольную сетку такую, чтобы 
к каждой вершине прилегало ровно пять треугольников, жители которых могли бы попить чай в этой 
вершине. По силам ли муравьям это предприятие?

Рассмотрим треугольник. Из его вершин во внешнее пространство выходит по 3 отрезка. С каждой стороны 2 отрезка
обязаны идти к общей вершине. Это 3 треугольника 2 поколения.

Затем --- к каждой стороне треугольников 2 поколения прилегает по треугольнику 3 поколения, итого их 6.

И из вершин идут треугольники 4 поколения. 

Рассмотрев этот процесс, мы видим, что все построения треугольников были вынуждены, и этот

(картинка)

\end{itemize}