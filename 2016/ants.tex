\task{Очень умные муравьи}
\begin{itemize}
\itA Муравьи могли бы разделить плоскость так:

\begin{center}\tikz{
	\draw (0,0) circle[radius=0.5] node{1}; 
	\draw (0.5,0) -- (1.5,0); 
	\draw[rotate=60] (1,0) node{2}; 
	\draw[rotate=120] (0.5,0) -- (1.5,0); 
	\draw[rotate=180] (1,0) node{4}; 
	\draw[rotate=240] (0.5,0) -- (1.5,0); 
	\draw[rotate=300] (1,0) node{3}; 
}\end{center}

\itB Ответ на задачу зависит от того, какую точку мы считаем началом восхождения.
Давайте считать, что муравьи совершают восхождение из внутреннего объема коробки, из самого ее центра.
Тогда расстояние от центра до середины стенки равно $0.5$ метра, расстояние от середины стенки до угла
коробки по теореме Пифагора равно $\sqrt{0.5^2 + 0.5^2}$, и расстояние от центра коробки до угла по той же теореме
$$\sqrt{(0.5^2 + 0.5^2) + 0.5^2} \approx 0.866$$

Поскольку муравьи в 1000 раз короче людей, нам нужно увеличить все размеры в 1000 раз --- и мы 
получим высоту холма чуть больше 866 метров.

\itC Рассмотрим какую-нибудь вершину сетки. Из нее выходит пять отрезков, причем эти отрезки --- стороны пяти 
треугольников, касающихся данной вершины. 

\begin{center}\tikz{
    \foreach \a in {0,72,144,144+72,144+144} {
        \draw[rotate=\a,very thick] (0,0) -- (1,0) -- (72:1) -- cycle; 
        \draw[rotate=\a] (10:1.5) -- (1,0) -- (-10:1.5); 
    }
}\end{center}

Отрезки, выходящие из вершин, соединенных стороной треугольника, обязаны соединяться ---
иначе к треугольнику будет прилегать не другой треугольник, а более сложная фигура.

\begin{center}\tikz{
    \foreach \a in {0,72,144,144+72,144+144} {
        \draw[rotate=\a,very thick] (0,0) -- (1,0) -- (72:1) -- cycle; 
        \draw[rotate=\a,very thick] (1,0) -- (36:2) -- (72:1); 
        \draw[rotate=\a] (45:2.5) -- (36:2) -- (27:2.5); 
	\draw[rotate=\a] (36:2) -- (36:2.5); 
    }
}\end{center}

В свою очередь, отрезки, выходящие из вершин, соединенных общей двузвенной ломаной,
должны совпадать --- иначе в сетке появятся многоугольники с числом вершин, большим трех.

\begin{center}\tikz{
    \foreach \a in {0,72,144,144+72,144+144} {
        \draw[rotate=\a,very thick] (0,0) -- (1,0) -- (72:1) -- cycle; 
        \draw[rotate=\a,very thick] (1,0) -- (36:2) -- (72:1); 
        \draw[rotate=\a,very thick] (-36:2) -- (36:2); 
        \draw[rotate=\a] (36:2) -- (36:2.5); 
    }
}\end{center}

И снова, все отрезки, выходящие из вершин, соединенных стороной, должны соединяться.
Заметим, что точка соединения --- общая для 1 и 2 отрезка, 2 и 3 отрезка, и т.п., поэтому 
она общая для всех отрезков. Из нее будет выходить 5 отрезков, значит, мы не можем
больше добавить ни одной вершины к нашей сетке. 

Все шаги по построению сетки были вынужденными, и у нас получилась конечная сетка.
А поскольку муравьев счетное количество, им нужно счетное количество треугольников.
Поэтому ответ на вопрос задачи отрицательный.

Дополнительно заметим, что структура из треугольников, которая получилась --- это
структура икосаэдра, одного из пяти правильных многогранников.


\begin{center}\tikz[scale=5]{
\draw (0.814410,1.546317) -- (0.905925,1.129469); 
\draw (0.814410,1.546317) -- (1.010636,1.121206); 
\draw[dotted] (0.814410,1.546317) -- (1.013094,1.603959); 
\draw (0.814410,1.546317) -- (1.190298,1.592072); 
\draw (0.814410,1.546317) -- (1.191974,1.833762); 
\draw (0.905925,1.129469) -- (1.010636,1.121206); 
\draw[dotted] (0.905925,1.129469) -- (1.013094,1.603959); 
\draw (0.905925,1.129469) -- (1.324886,0.964915); 
\draw[dotted] (0.905925,1.129469) -- (1.325667,1.205978); 
\draw (1.010636,1.121206) -- (1.190298,1.592072); 
\draw (1.010636,1.121206) -- (1.324886,0.964915); 
\draw (1.010636,1.121206) -- (1.502871,1.194090); 
\draw[dotted] (1.013094,1.603959) -- (1.191974,1.833762); 
\draw[dotted] (1.013094,1.603959) -- (1.325667,1.205978); 
\draw[dotted] (1.013094,1.603959) -- (1.505328,1.676844); 
\draw (1.190298,1.592072) -- (1.191974,1.833762); 
\draw (1.190298,1.592072) -- (1.502871,1.194090); 
\draw (1.190298,1.592072) -- (1.601985,1.662932); 
\draw[dotted] (1.191974,1.833762) -- (1.505328,1.676844); 
\draw (1.191974,1.833762) -- (1.601985,1.662932); 
\draw[dotted] (1.324886,0.964915) -- (1.325667,1.205978); 
\draw (1.324886,0.964915) -- (1.502871,1.194090); 
\draw (1.324886,0.964915) -- (1.693500,1.246083); 
\draw[dotted] (1.325667,1.205978) -- (1.505328,1.676844); 
\draw[dotted] (1.325667,1.205978) -- (1.693500,1.246083); 
\draw (1.502871,1.194090) -- (1.601985,1.662932); 
\draw (1.502871,1.194090) -- (1.693500,1.246083); 
\draw[dotted] (1.505328,1.676844) -- (1.601985,1.662932); 
\draw[dotted] (1.505328,1.676844) -- (1.693500,1.246083); 
\draw (1.601985,1.662932) -- (1.693500,1.246083); 
}\end{center}

\end{itemize}
