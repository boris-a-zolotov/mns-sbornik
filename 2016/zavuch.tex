\task{Проблемы завуча}
\begin{itemize}
\itA Средний рост в параллели: $\frac{161 + 162 + \ldots + 220}{60} = \frac{361\cdot 30}{60} = 190.5$,
обозначим его за $A$.

Обозначим сумму ростов школьников в классе $i$ как $H_i$. Покажем, что максимальный 
минимальный средний рост школьников достигается, когда $H_1 = H_2 = H_3$: в этом случае,
очевидно, $\frac{H_i}{20} = A$. Покажем это в два приёма: во-первых, продемонстрируем,
что максимальный минимальный средний рост не превышает $A$, а потом покажем, что
$A$ достижим.

Пусть есть два класса, в которых средний рост различается, и пусть $H_1 < H_2 \le H_3$. 
Тогда минимальный средний рост ниже $A$: 

$$A = \frac{H_1 + H_2 + H_3}{60} > \frac{3H_1}{60} = \frac{H_1}{20}$$

Теперь покажем достижимость $A$: построим пример такого разбиения на классы,
что $H_1 = H_2 = H_3$.
Заметим, что если мы симметрично исключим из вычисления среднего роста параллели двух 
школьников (скажем, двух школьников с ростом $161+k$ и $220-k$ для некоторого $k$), 
то средний рост не изменится. Основываясь на этой идее, легко построить требуемый
пример:

\begin{center}\begin{tabular}{lll}
Класс & Состав класса: рост учеников \\
\hline
1 & $181, 182 \ldots 200$\\
2 & $171, 172 \ldots 179, 201, 202 \ldots 210$ \\
3 & $161, 162 \ldots 169, 211, 212 \ldots 220$
\end{tabular}\end{center}

\itB Монета пролезет, если найдутся такая проекция монеты и такая деформация 
листа с отверстием, что монета может быть размещена целиком внутри деформированного
отверстия. Так как монета круглая, то любая проекция содержит отрезок длины $2r$. 
С другой стороны, проекция ребра ничего кроме этого отрезка и не содержит. 
Отсюда вывод: монета пролезет тогда и только тогда, когда в отверстие помещается 
отрезок длины $2r$.

Давайте найдём наиболее выгодную деформацию.
Поскольку лист очень гибкий, мы можем изменить форму отверстия, сохранив периметр.
Давайте отверстие вытягивать, в пределе получив прямую узкую щель длины $\text{π}R$; это
самая длинная фигура с требуемым периметром, однако, она нас по-прежнему устраивает.

Чтобы монета пролезла, нужно, чтобы длина щели равнялась диаметру монеты. Отсюда, $R = \frac{2r}{\text{π}}$.

\itC Теперь у завуча есть лист совершенно не гибкого стекла, также с вырезанным в нём круглым отверстием 
радиуса $r$. Какую длину должно иметь ребро очень шершавого куба, чтобы он не пролезал через это отверстие?

Если мы имеем возможность вращать куб, то нам нужно найти его проекцию с самым маленьким
радиусом описанной окружности. 

\end{itemize}
