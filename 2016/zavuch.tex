\task{Проблемы завуча}
\begin{itemize}
\itA Средний рост в параллели: $\frac{161 + 162 + \ldots + 220}{60} = \frac{361\cdot 30}{60} = 190.5$,
обозначим его за $A$.

Обозначим сумму ростов школьников в классе $i$ как $H_i$. Покажем, что максимальный 
минимальный средний рост школьников достигается, когда $H_1 = H_2 = H_3$: в этом случае,
очевидно, $\frac{H_i}{20} = A$. Покажем это в два приёма: во-первых, продемонстрируем,
что максимальный минимальный средний рост не превышает $A$, а потом покажем, что
$A$ достижим.

Пусть есть два класса, в которых средний рост различается, и пусть $H_1 < H_2 \le H_3$. 
Тогда минимальный средний рост ниже $A$: 

$$A = \frac{H_1 + H_2 + H_3}{60} > \frac{3H_1}{60} = \frac{H_1}{20}$$

Теперь покажем достижимость $A$: построим пример такого разбиения на классы,
что $H_1 = H_2 = H_3$.

\begin{center}\begin{tabular}{lll}
Класс & Состав класса: рост учеников \\
\hline
1 & $181, 182 \ldots 200$\\
2 & $171, 172 \ldots 179, 201, 202 \ldots 210$ \\
3 & $161, 162 \ldots 169, 211, 212 \ldots 220$
\end{tabular}\end{center}

\itB Монета пролезет, если найдутся такая проекция монеты и такая деформация 
листа с отверстием, что монета может быть размещена целиком внутри деформированного
отверстия. Так как монета круглая, то любая проекция содержит отрезок длины $2r$. 
С другой стороны, проекция ребра ничего кроме этого отрезка и не содержит. 
Отсюда вывод: монета пролезет тогда и только тогда, когда в отверстие помещается 
отрезок длины $2r$.

Давайте найдём наиболее выгодную деформацию.
Поскольку лист очень гибкий, мы можем изменить форму отверстия, сохранив периметр.
Давайте отверстие вытягивать, в пределе получив прямую узкую щель длины $\text{π}R$; это
самая длинная фигура с требуемым периметром, однако, она нас по-прежнему устраивает.

Чтобы монета пролезла, нужно, чтобы длина щели равнялась диаметру монеты. Отсюда, $R = \frac{2r}{\text{π}}$.

\itC Если мы имеем возможность вращать куб, то нам нужно найти его проекцию с самым маленьким
радиусом описанной окружности. Такая проекция получится, сли поставить куб на одну из
его вершин, так чтобы противоположная вершина находилась строго над ней.

Обозначим вершину, на которую мы ставим куб, через $v_0$, а противоположную ей вершину — за $v_1$. Тогда окружность, описанная вокруг проекции куба, будет проходить через проекции сразу всех 6 его оставшихся вершин. Заметим, что вершина $v_0$ и три вершины, смежные с ней, образуют симметричную треугольную пирамиду, основание которой — треугольник со стороной $\sqrt{2}$.

В свою очередь, радиус описанной окружности такого треугольника равен
	$$\frac{\sqrt{2}}{2} \cdot \frac{2}{\sqrt{3}} = \sqrt{\frac{2}{3}}$$

\end{itemize}
