\task{Проблемы завуча}
\begin{itemize}
\itA На параллели 60 школьников ростом 161, 162, ..., 219, 220 сантиметров. Завуч хочет распределить их на три класса так, чтобы минимальный средний рост школьников в классе был максимален. Как ей это сделать?

\itB Завуч, одновременно являясь преподавателем физики, демонстрирует детям абсолютно гибкий, но нерастяжимый и неповреждаемый металлический лист с вырезанным в нём круглым отверстием радиуса $r$. Монета какого максимального диаметра пролезет через это отверстие?

\itC Теперь у завуча есть лист совершенно не гибкого стекла, также с вырезанным в нём круглым отверстием радиуса $r$. Какую длину должно иметь ребро очень шершавого куба, чтобы он не пролезал через это отверстие?
\end{itemize}
