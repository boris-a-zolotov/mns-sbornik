\task{Лыжная секция}
\begin{itemize}
\itA Зададим себе уточняющий вопрос: в какой день в секции станет ровно 12 участников?
На второй день рано: мы можем получить максимально $1+5+5=11$ участников, если каждый день приходит по пятеро.
На третий уже поздно: минимально получится $1+4+4+4=13$ участников. 
Поэтому ровно 12 участников в секции не могло быть ни в какой день.

\itB Рассмотрим треугольник, образованный отрезками, соединяющими текущее положение лыжников.
При соблюдении правил перемещения его площадь всегда постоянна: если принять
отрезок, соединяющий двух стоящих лыжников, за основание треугольника, 
то перемещения третьего лыжника не меняют его высоту.
Посчитаем площадь итогового треугольника (нетрудно видеть, что он прямоугольный): $$\frac{90\cdot 120}{2}$$
Однако, площадь исходного треугольника явно меньше:

$$\frac{\left(\frac{\sqrt{3}}{2}\cdot 100\right) \cdot 100}{2} < \frac{(0.9 \cdot 100) \cdot 100}{2} < \frac{90 \cdot 120}{2}$$

Значит, правила перемещения в какой-то момент были нарушены.

\itC Возможны два сценария, в результате которых лыжник возвращается на своё место: <<локальный>> 
(если лыжник №1 обогнал лыжника №2, а потом лыжник №2 обогнал лыжника №1) и <<глобальный>>
(лыжник №1 обогнал всех и вернулся на своё место, имея круг в запасе).

При <<локальном>> сценарии количество обгонов всегда чётное (всех, кого вы обогнали, вы должны пропустить вперёд).
При <<глобальном>> сценарии, когда лыжник обгоняет своих остальных 563 товарищей, возможно нечётное
количество обгонов.
Если же в гонке участвует 563 лыжника, то оба сценария предполагают чётное количество обгонов.

При необходимости 
более формально данный результат можно доказать, переведя его на язык перестановок. 
Будем записывать положение лыжников, как $n$-элементную перестановку начального 
положения. 
При этом, поскольку гонка кольцевая и положение лыжников
можно начать отсчитывать с любого места, то мы отождествляем между собой циклические
сдвиги перестановок: 
\begin{center}если $\sigma(x) = (\tau(x)+k) \bmod n + 1$, то $\sigma \equiv \tau$
\end{center}

Если $n = 564$, то циклический сдвиг позволяет сделать из чётной перестановки
нечётную. Однако, если $n=563$, то сдвиг всегда сохраняет чётность.
А раз к тому же исходная и целевая перестановки --- чётные, то и количество транспозиций неизбежно будет чётным. 

\end{itemize}
