\task{Попытки осмысления биссектрис}
\begin{itemize}
\itA %У Робинзона Крузо есть стандартный советский чертёжный треугольник с углами в 30, 60 и 90 градусов. 
%Как ему с помощью этого треугольника начертить на песке угол в 15 градусов?
%
Обведём треугольник ({\itshape контур 1}), потом перевернём его и снова обведём ({\itshape контур 2}).
Луч, исходящий из прямого угла, проходящий через пересечение контуров 1 и 2 --- биссектриса прямого угла,
делящая его на два угла по $45^\circ$.
Если приложить к этой биссектрисе треугольник ({\itshape контур 3}), то мы получим искомый $15^\circ$ угол.

\begin{center}\tikz{
    \draw[thick,dotted] (0,0) -- (-3,0) -- ++(60:6) node[right] {\itshape контур 1} -- cycle;
    \draw[thick,dashed] (0,0) -- (0,3) -- ++(210:6) node[below] {\itshape контур 2} -- cycle;
    \draw[thick] (0,0) -- ++(135:6) -- ++(225:3) ++(0,0.1) node[right] {~~~\itshape контур 3} ++(0,-0.1) -- (0,0);
    \draw[thick] (-0.4,0) arc (180:163:0.4);
    \draw (-0.6,0) node[below] {$15^\circ$};
    \draw[thick, double] (0,0.4) arc (90:135:0.4);
    \draw (-0.3,0.5) node[above] {$45^\circ$};
}\end{center}

\itB %Мальчик по имени Текдра решил определять биссектрису трёхгранного угла следующим образом: проведём 
%биссектрису одной из его граней, затем плоскость, проходящую через эту биссектрису и противоположное 
%использованной грани ребро. Теперь проведём биссектрису угла в этой плоскости, образованного ребром и 
%старой биссектрисой. Докажите, что определение Текдры некорректно — в одном трёхгранном угле можно построить 
%несколько различных биссектрис.
%
Данное определение биссектрисы зависит от выбора начальной грани. Рассмотрим пример,
где различие очевидно. Дан трёхгранный угол, образованный гранями $AOB$, $AOC$ и $BOC$,
причём угол $BOC$ значительно меньше углов $AOC$ и $BOA$.
Если мы разобьём сперва плоский угол $BOC$ биссектрисой $OP$, а потом 
построим биссектрису $OQ$, разбив угол $AOQ$, то она будет лежать почти в середине угла (левый рисунок).
Противоположный порядок разбиения угла даст нам биссектрису $OS$, сильно смещённую к грани $BOC$ (правый рисунок).


\begin{tabular}{cc}
\tikz{
    \draw (0,0) node [left] {$O$} -- (80+240:4) node [right] {$A$};
    \draw (0,0) -- (10+240:4) node [left] {$B$};
    \draw (0,0) -- (20+240:2);
    \draw (20+240:2) -- (20+240:4) node [below] {$C$};
    %\draw[dotted] (80+240:3.5) -- (10+240:3.5) -- (20+240:3.5) -- (80+240:3.5);
    \fill[pattern=north east lines, pattern color = gray] (0,0) -- (15+240:3.5) -- (80+240:3.4) -- cycle;
    %\fill[pattern=north east lines, pattern color = gray] 
    %     (20+240:2) -- (80+240:2) -- (80+240:3) arc (80+240:20+240:3) -- cycle;
    \fill[pattern=horizontal lines, pattern color = gray] 
         (0,0) -- (10+240:3) -- (15+240:3) -- cycle;
    %\fill[pattern=north west lines, pattern color = gray] (0,0) -- (20+240:2) -- (80+240:2) -- cycle;
    \draw[dashed] (0,0) -- (15+240:4) node [below] {$P$};
    \draw[dashed] (0,0) -- (95/2+240:1.7);
    \draw[dashed] (95/2+240:1.7) -- (95/2+240:3.5) node [below] {$Q$};
    \draw[double] (10+240:1.5) arc (10+240:15+240:1.5);
    \draw[double] (15+240:1.6) arc (15+240:20+240:1.6);
    \draw (15+240:0.5) arc (15+240:95/2+240:0.5);
    \draw (95/2+240:0.6) arc (95/2+240:80+240:0.6);
}
&
\tikz[rotate=240]{
    \draw (0,0) node [left] {$O$} -- (80:4) node [right] {$A$};
    \draw (0,0) -- (10:4) node [left] {$B$};
    \draw (0,0) -- (20:4) node [below] {$C$};
    %\draw[dotted] (80:3.5) -- (10:3.5) -- (20:3.5) -- cycle;
    %\draw[dotted,gray] (20:3.5) -- (45:2.865);
    \draw[double] (10:0.5) arc (10:20:0.5);
    \draw[double,gray] (20:0.5) arc (20:45:0.5);
    \draw[double] (45:0.6) arc (45:80:0.6);
    \fill[pattern=north east lines, pattern color = gray] (0,0) -- (45:2.865) -- (20:3.5) -- cycle;
    \fill[pattern=horizontal lines, pattern color = gray] (0,0) -- (45:2.865) -- (80:3.5) -- cycle;
    \fill[pattern=horizontal lines, pattern color = gray] (0,0) -- (10:3.1) -- (20:3.1) -- cycle;
    %\fill[pattern=horizontal lines, pattern color = gray!20] (0,0) -- (45:2.865) -- (20:3.1) -- cycle;
    \draw[dashed] (0,0) -- (45:3.4) node [right] {$R$};
    \draw[dashed] (0,0) --  (32.5:4) node [right] {$S$};
    \draw (45:1) arc (45:32.5:1);
    \draw (32.5:1.1) arc (32.5:20:1.1);
}
\end{tabular}

Чуть более формальное пояснение. Возьмём трёхгранный угол и будем уменьшать угол $BOC$ до нуля. 
При очень маленьких значениях этого угла окажется, что биссектриса $QO$ почти совпадает с 
биссектрисой угла $AOB$. Однако, угол $SOC$ --- только половина угла $ROC$, то есть примерно
четверть угла $AOB$.

%Формализовать это рассуждение можно, рассмотрев треугольник, образованный пересечением 
%трёхгранного угла и плоскости, проходящей через точки, отстоящие от него на фиксированном расстоянии.
%Пусть $\angle BOC = 10^\circ$, а $\angle AOB = \angle AOC = 60^\circ$.
%Здесь $\angle ROA = 30^\circ$, 
%При этом можно показать, что $\angle QOA < \frac{\angle AOC}{2} = 30^\circ$ (проведём 
%плоскость поперёк трёхгранного угла, 

\itC В треугольнике $MNK$ проведены биссектрисы углов $N$ и $K$. Из оставшейся вершины на эти биссектрисы 
опустили перпендикуляры и провели прямую через их основания. Доказать, что она будет параллельна стороне $NK$.

%\tikz{
%    \draw (0,0) -- (2,0) -- ++(32:2.3);
%}

\end{itemize}
