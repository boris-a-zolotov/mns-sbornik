\task{Попытки осмысления биссектрис}
\begin{itemize}
\itA У Робинзона Крузо есть стандартный советский чертёжный треугольник с углами в 30, 60 и 90 градусов. 
Как ему с помощью этого треугольника начертить на песке угол в 15 градусов?

Обведём треугольник, потом перевернём его, и снова обведём.
Угол между горизонтальной линией и направлением из прямого угла к пересечению контуров будет $45^\circ$.
Если приложить к этой линии треугольник углом $30^\circ$ к обведённому прямому углу, то мы получим 
искомый $15^\circ$ угол.

\begin{center}\tikz{                              
    \draw (0,0) -- (-3,0) -- ++(60:6) -- cycle;
    \draw (0,0) -- (0,3) -- ++(210:6) -- cycle;
    \draw (0,0) -- ++(135:6) -- ++(225:3) -- cycle;
    \draw (-0.4,0) arc (180:163:0.4);
    \draw (-0.6,0) node[below] {$15^\circ$};
}\end{center}

\itB Мальчик по имени Текдра решил определять биссектрису трёхгранного угла следующим образом: проведём 
биссектрису одной из его граней, затем плоскость, проходящую через эту биссектрису и противоположное 
использованной грани ребро. Теперь проведём биссектрису угла в этой плоскости, образованного ребром и 
старой биссектрисой. Докажите, что определение Текдры некорректно — в одном трёхгранном угле можно построить 
несколько различных биссектрис.

\itC В треугольнике $MNK$ проведены биссектрисы углов $N$ и $K$. Из оставшейся вершины на эти биссектрисы 
опустили перпендикуляры и провели прямую через их основания. Доказать, что она будет параллельна стороне $NK$.

\tikz{
    \draw (0,0) -- (2,0) -- ++(32:2.3);
    
}

\end{itemize}
