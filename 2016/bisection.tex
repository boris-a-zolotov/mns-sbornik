\task{Попытки осмысления биссектрис}
\begin{itemize}
\itA У Робинзона Крузо есть стандартный советский чертёжный треугольник с углами в 30, 60 и 90 градусов. Как ему с помощью этого треугольника начертить на песке угол в 15 градусов?

\itB Мальчик по имени Текдра решил определять биссектрису трёхгранного угла следующим образом: проведём биссектрису одной из его граней, затем плоскость, проходящую через эту биссектрису и противоположное использованной грани ребро. Теперь проведём биссектрису угла в этой плоскости, образованного ребром и старой биссектрисой. Докажите, что определение Текдры некорректно — в одном трёхгранном угле можно построить несколько различных биссектрис.

\itC В треугольнике $MNK$ проведены биссектрисы углов $N$ и $K$. Из оставшейся вершины на эти биссектрисы опустили перпендикуляры и провели прямую через их основания. Доказать, что она будет параллельна стороне $NK$.
\end{itemize}
