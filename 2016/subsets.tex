\task{Разделение на подмножества}
\begin{itemize}
\itA Могут, если в компаниях разное количество мальчиков: допустим, в первой компании десять мальчиков с 1 рублём, а во второй --- пять мальчиков с 1 рублём.

\itB 

Будем обозначать чёрный отрезок цифрой 1, а белый --- 0, и пусть $a_n$ --- цвет отрезка прямой, содержащей
точку $n$.

Покажем от противного, что Петя не сможет покрасить отрезки желаемым образом.
Пусть ему это удалось.
Заметим тогда, что при любом $n$ выполнено 
$a_n+a_{n+1}+a_{n+2}+a_{n+3} = 2$ и $a_{n+1}+a_{n+2}+a_{n+3}+a_{n+4} = 2$. 
Значит, $a_n = a_{n+4}$. 

Возьмём такое $n$, что $a_{n-1} = 0$ и $a_n = 1$ (такое $n$ существует, поскольку иначе
начиная с некоторого места все отрезки покрашены одинаково, что невозможно). 
По этим данным однозначно вытекает, что $a_{n+3} = 0$. А вот $a_{n+1}$ может быть
разным.
Рассмотрим два случая: 

\begin{enumerate}
\item $a_{n+1} = 1$. Тогда $a_{n+2} = 0$ и
$a_{n+1} + a_{n+2} + \dots + a_{n+11} = 1 + 0 + 0 + 1 + 1 + 0 + 0 + 1 + 1 + 0 + 0 = 5$,
противоречие.

\item $a_{n+1} = 0$. Тогда $a_{n+2} = 1$ и
$a_{n+1} + a_{n+2} + \dots + a_{n+11} = 0 + 1 + 0 + 1 + 0 + 1 + 0 + 1 + 0 + 1 + 0 = 5$,
снова противоречие.
\end{enumerate}

\itC В каждой клетке могут водиться муравьи данного вида, а могут и не водиться. 
Итого, $2 \cdot 2 \cdot 2 \cdot 2 = 16$ вариантов. 

(картинка)
\end{itemize}
