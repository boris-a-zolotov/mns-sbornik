\task{Разделение на подмножества}
\begin{itemize}
\itA Могут, если в компаниях разное количество мальчиков: допустим, в первой компании десять мальчиков с 1 рублём, а во второй --- пять мальчиков с 1 рублём.

\itB 

Будем обозначать чёрный отрезок цифрой 1, а белый --- 0, и пусть $a_n$ --- цвет отрезка прямой, содержащей
точку $n$.

Покажем от противного, что Петя не сможет покрасить отрезки желаемым образом.
Пусть ему это удалось.
Заметим тогда, что при любом $n$ выполнено 
$a_n+a_{n+1}+a_{n+2}+a_{n+3} = 2$ и $a_{n+1}+a_{n+2}+a_{n+3}+a_{n+4} = 2$. 
Значит, $a_n = a_{n+4}$. 

Возьмём такое $n$, что $a_{n-1} = 0$ и $a_n = 1$ (такое $n$ существует, поскольку иначе
начиная с некоторого места все отрезки покрашены одинаково, что невозможно). 
По этим данным однозначно вытекает, что $a_{n+3} = 0$. А вот $a_{n+1}$ может быть
разным.
Рассмотрим два случая: 

\begin{enumerate}
\item $a_{n+1} = 1$. Тогда $a_{n+2} = 0$ и
$a_{n+1} + a_{n+2} + \dots + a_{n+11} = 1 + 0 + 0 + 1 + 1 + 0 + 0 + 1 + 1 + 0 + 0 = 5$,
противоречие.

\item $a_{n+1} = 0$. Тогда $a_{n+2} = 1$ и
$a_{n+1} + a_{n+2} + \dots + a_{n+11} = 0 + 1 + 0 + 1 + 0 + 1 + 0 + 1 + 0 + 1 + 0 = 5$,
снова противоречие.
\end{enumerate}

\itC Рассмотрим некоторую клетку, и рассмотрим вопрос <<водятся ли в клетке муравьи типа $T$>>.
Всего возможно четыре вопроса и ответы на вопрос не зависят друг от друга:
мы можем ответить <<да, да, нет, нет>>, можем <<нет, нет, да, нет>>, и вообще, 
любая комбинация возможна. 
Итого, получается $2 \cdot 2 \cdot 2 \cdot 2 = 16$ вариантов. 

Пример расселения муравьёв, в котором реализуются все комбинации, приведён ниже.
Каждый тип линии соответствует какому-то типу муравьёв; если линия проходит
через клетку, то данная клетка принадлежит ареалу.

\begin{center}\tikz{
  \draw[step=0.5,dotted] (0,0.5) grid (8,2.5);

  \draw[very thick] (0.1,1.75) -- (15/2+0.1,1.75);
  \foreach \i in {1,3,5,7,9,11,13,15} {
     \draw[very thick] (\i/2+0.1,1.75) -- (\i/2+0.1,1.4) -- (\i/2+0.3,1.4);
  }

  \draw[very thick,double] (0.1+4,1.2) -- (0.4+7.5,1.2);

  \draw[very thick,dashed] (0.1,0.75) -- (14/2+0.25,0.75);
  \foreach \i in {2,6,10,14} {
     \draw[very thick,dashed] (\i/2+0.25,0.75) -- (\i/2+0.25,1.1) -- (\i/2+0.75,1.1);
  }

  \draw[very thick,dotted] (0.1,2.25) -- (12/2+0.3,2.25);
  \foreach \i in {4,12} {
     \draw[very thick,dotted] (\i/2+0.3,2.25) -- (\i/2+0.3,1.3) -- (\i/2+1.9,1.3);
  }
}\end{center}
\end{itemize}
