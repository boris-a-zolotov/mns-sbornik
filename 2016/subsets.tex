\task{Разделение на подмножества}
\begin{itemize}
\itA Могут, если в компаниях разное количество мальчиков: допустим, в первой компании десять мальчиков с 1 рублём, а во второй --- пять мальчиков с 1 рублём.

\itB 

Будем обозначать чёрный отрезок цифрой 1, а белый --- 0, и пусть $a_n$ --- цвет отрезка прямой, содержащей
точку $n$. Рассмотрим тогда два фрагмента прямой: $(a_1,a_2,a_3,a_4)$ --- мы знаем, что $a_1+a_2+a_3+a_4 = 2$ 
и $(a_2,a_3,a_4,a_5)$, где тоже $a_2+a_3+a_4+a_5 = 2$. Отсюда понятно, что вообще выполнено $a_n = a_{n+4}$.



Утверждение: если на прямой среди любых соседних чёрных отрезков есть ровно два чёрных, то чёрные и белые отрезки
должны чередоваться. В самом деле, пусть не так, и где-то на прямой есть два соседних чёрных отрезка. 
Тогда 

Рассмотрим фрагмент прямой из 11 единичных отрезков. В нём шесть чёрных единичных отрезков. Возможны две ситуации: 
\begin{enumerate}
\item Белые и чёрные отрезки чередуются --- значит, фрагмент начинается и заканчивается чёрным отрезком.

\item Где-то на фрагменте есть два соседних чёрных отрезка. 

Петя хочет покрасить единичные отрезки в составе прямой в два цвета так, чтобы среди любых четырёх подряд идущих отрезков было ровно два чёрных, а среди любых одиннадцати подряд идущих — ровно шесть чёрных. Может ли он это сделать?

\itC В каждой клетке могут водиться муравьи данного вида, а могут и не водиться. Итого, $2 \cdot 2 \cdot 2 \cdot 2 = 16$ вариантов. 

(картинка)
\end{itemize}
