\task{Разделение на подмножества}
\begin{itemize}
\itA Могут, если в компаниях разное количество мальчиков: допустим, в первой компании десять мальчиков с 1 рублём, а во второй --- пять мальчиков с 1 рублём.

\itB 

Будем обозначать чёрный отрезок цифрой 1, а белый --- 0, и пусть $a_n$ --- цвет отрезка прямой, содержащей
точку $n$.

Заметим тогда, что $a_n+a_{n+1}+a_{n+2}+a_{n+3} = 2$ и $a_{n+1}+a_{n+2}+a_{n+3}+a_{n+4} = 2$. 
Значит, $a_n = a_{n+4}$. 

Теперь из второго условия следует, что $a_0+a_1+ \dots +a_{10} = 6$. 



Утверждение: если на прямой среди любых соседних чёрных отрезков есть ровно два чёрных, то чёрные и белые отрезки
должны чередоваться. В самом деле, пусть не так, и где-то на прямой есть два соседних чёрных отрезка. 
Тогда 


Петя хочет покрасить единичные отрезки в составе прямой в два цвета так, чтобы среди любых четырёх подряд идущих отрезков было ровно два чёрных, а среди любых одиннадцати подряд идущих — ровно шесть чёрных. Может ли он это сделать?

\itC В каждой клетке могут водиться муравьи данного вида, а могут и не водиться. Итого, $2 \cdot 2 \cdot 2 \cdot 2 = 16$ вариантов. 

(картинка)
\end{itemize}
