\task{День, когда Стёпа всё испортил}
\begin{itemize}
\itA Лиза обещала подарить Стёпе диск со всеми сериями «Adventure time», если он 
сможет досчитать на пальцах до 1023 так, что все числа будут показаны разными 
комбинациями пальцев. Как Стёпе заполучить желанный диск?

Стёпа, например, мог бы попробовать посчитать в двоичной системе:
согнутый палец означает 0, распрямлённый означает 1. Всего десять пальцев,
поэтому самое большое число, которое можно показать так, равно $2^10-1 = 1023$.

Хотя, конечно, данный совет --- это только самое начало дела. Показ некоторых
комбинаций (скажем, числа $0101001010_2$ --- распрямлены безымянный и указательный
пальцы на обеих руках) может потребовать тренировки.


\itB (косяк в условии) ... Стёпа покрутил какой-то рычажок, после чего при взвешивании
они всегда показывают вес, отличающийся от настоящего на $w_0$. При этом, Стёпа не знает
это значение $w_0$ (потому он не знает, как надо подкрутить рычажок для исправления ситуации), 
Что хуже, если ничего не взвешивать, то весы показывают 0.  ...

Весы Всезнамуса после вмешательства Стёпы показывают всегда на $w_0$ больше, надо определить $w_0$.
Взвесив бутылку воды получим $r_1 = w_1 + w_0$, взвесив кусок циркония получим $r_2 = w_2 + w_0$,
взвесив оба предмета получим $r_{12} = w_1 + w_2 + w_0$. Давайте теперь вычтем из двух 
первых результатов третий:
$r_1 + r_2 - r_{12} = w_1 + w_0 + w_2 + w_0 - w_1 - w_2 - w_0 = w_0$.

\itC Придя домой, Стёпа уселся за свою любимую компьютерную игру — Portal. Один портал стоит на большой 
высоте на вертикальной стене. Стёпа вылетает из него, перед самым приземлением ставит под собой второй 
портал и попадает в него с тем, чтобы вновь вывалиться из первого портала на стене, причём параллельно 
земле. Как быстро будет нижний портал отдаляться от стены? При прохождении через портал скорость Стёпы 
не изменяется.

к условию: (скорость не изменяется по абсолютному значению, но направлена параллельно земле)

Стёпа, падая с высоты $h$, падает равноускоренно. За время падения $t$ Стёпа пролетит 
$\frac{at^2}{2}$ метров, где $a$ --- ускорение свободного падения в мире игры Portal, 
причём $\frac{at^2}{2} = h$. Отсюда $t = \sqrt{\frac{2h}{a}}$ и вертикальная скорость
у земли $v_\downarrow = \sqrt{2ah}$.

После прохождения через портал он летит горизонтально с данной скоростью, но вертикальная
скорость снова равна 0, поэтому Стёпа снова наберёт $v_\downarrow = \sqrt{2ah}$, и
итоговая скорость будет $\sqrt{v_\downarrow^2 + v_\downarrow^2}$.
И вообще, после $n$ прыжков итоговая скорость будет $\sqrt{n \cdot v_\downarrow^2}$.

Соответственно, нижний портал в ходе $n+1$ прыжка окажется на расстоянии $h \cdot \sqrt{n \cdot v_\downarrow^2}$.

\end{itemize}
