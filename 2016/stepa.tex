\task{День, когда Стёпа всё испортил}
\begin{itemize}
\itA Стёпа, например, мог бы попробовать посчитать в двоичной системе:
согнутый палец означает 0, распрямлённый означает 1. Всего десять пальцев,
поэтому самое большое число, которое можно показать так, равно $2^{10}-1 = 1023$.

Хотя, конечно, данный совет --- это только самое начало дела. Показ некоторых
комбинаций (скажем, числа $0101001010_2$ --- распрямлены безымянный и указательный
пальцы на обеих руках) может потребовать тренировки.


\itB (косяк в условии) ... весы Всезнамуса обладают той особенностью, что без груза
всегда показывают 0 ...

Весы Всезнамуса после вмешательства Стёпы показывают всегда на $w_0$ больше, надо определить $w_0$.
Взвесив бутылку воды получим $r_1 = w_1 + w_0$, взвесив кусок циркония получим $r_2 = w_2 + w_0$,
взвесив оба предмета получим $r_{12} = w_1 + w_2 + w_0$. Давайте теперь вычтем из двух 
первых результатов третий:
$r_1 + r_2 - r_{12} = w_1 + w_0 + w_2 + w_0 - w_1 - w_2 - w_0 = w_0$.

\itC 
к условию: (скорость не изменяется по абсолютному значению, но направлена параллельно земле)

Стёпа, падая с высоты $h$, падает равноускоренно. За время падения $t$ Стёпа пролетит 
$\frac{at^2}{2}$ метров, где $a$ --- ускорение свободного падения в мире игры Portal, 
причём $\frac{at^2}{2} = h$. Отсюда $t = \sqrt{\frac{2h}{a}}$ и вертикальная скорость
у земли $v_\downarrow = \sqrt{2ah}$.

После прохождения через портал он летит горизонтально с данной скоростью, но вертикальная
скорость снова равна 0, поэтому Стёпа снова наберёт $v_\downarrow$ вертикальной скорости, и
итоговая скорость будет $\sqrt{v_\downarrow^2 + v_\downarrow^2}$.
И вообще, после $n$ прыжков итоговая скорость будет $\sqrt{n \cdot v_\downarrow^2}$.

Соответственно, нижний портал в ходе $n+1$ прыжка окажется на расстоянии 
$h \cdot \sqrt{n \cdot v_\downarrow^2}$ от стены,
то есть будет удаляться всё с меньшим и меньшим шагом, пропорционально $\sqrt{n}$.

\end{itemize}
