\task{Несправедливый турнир}
\begin{itemize}
\itA Нет, поскольку самый слабый участник всегда проигрывает свою встречу.
Значит, он всегда будет оказываться в худшей группе, с кем бы ни играл.
В итоге, он окажется внизу таблицы.

\itB Введём определение: если N. состязается сильнее некоторого другого участника,
то мы назовём того участника <<слабым>>, если слабее --- назовём его <<сильным>>.
Исходно имеется не более $2^{t-1}-2$ сильных участников.

Чтобы оказаться в финале с аутсайдером, N. должен проиграть все встречи,
кроме последней. Значит, в первой встрече N. должен встречаться с сильным.
Вместе с N. в группе проигравших окажется не более $\frac{2^{t-1}-2}{2}-1 = 2^{t-2}-2$ 
сильных участников: сильный может проиграть только сильному, всего 
$\frac{2^{t-1}-2}{2}$ пар, но одну из пар сильных мы обязаны разорвать, чтобы 
N. проиграл.

Повторив рассуждение можно вывести общую формулу: после раунда $k$ 
в группе проигравших всего может остаться не более $2^{t-k-1}-2$ сильных. 
Значит, после раунда $t-2$ сильных не останется и N. избежит встречи с аутсайдером, 
выиграв в раунде $t-1$.

\itC Введём определения аналогично предыдущему пункту: если N. состязается сильнее 
некоторого другого участника, то мы назовём того участника <<слабым>>, если 
слабее --- назовём его <<сильным>>. Исходно есть не менее $2^{t-1}$ сильных. 

Пусть в первом туре N. соревнуется с сильным и все остальные сильные соревнуются 
в паре с другими сильными, 
тогда после раунда в группе проигравших останется не менее $2^{t-2}-1$ сильных.
Во втором раунде мы снова составим пару N. с сильным, добавим не меньше $\frac{2^{t-2}-2}{2} = 2^{t-3}-1$ 
пар сильных, и после второго раунда в группе проигравших останется N и
не меньше $2^{t-3}-1$ сильных. Повторив это построение пар $k$ раз
мы получим, что после $k$ раунда в группе проигравших останется N. и не меньше $2^{t-k-1}-1$ сильных. 

Итак, N. после $t-1$ раунда окажется в проигравшей все раунды 
группе. А вторым участником этой группы будет самый слабый участник соревнования:
ведь все остальные участники, кроме этих двоих, смогли выиграть хоть у кого-то. 

Для точности заметим, что условие неявно предполагает, что аутсайдер и N. --- 
это разные участники. Если же такого требования нет, то мы сможем утверждать только
об участии N. в состязании за последнее и предпоследнее места.
\end{itemize}
