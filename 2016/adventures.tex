\task{Числа, цифры и приключения}
\begin{itemize}
\itA Стоимость спички в листочках при использовании в процессе
оплаты некоторой цифры --- это значение данной цифры, 
поделенное на количество спичек в ней. Давайте посчитаем эти значения:

$$\frac{0}{6}, \frac{1}{2}, \frac{2}{5}, \frac{3}{5}, \frac{4}{4}, \frac{5}{5}, \frac{6}{6}, \frac{7}{3}, \frac{8}{7}, \frac{9}{6}$$

Очевидно, самая выгодная цифра --- 0 (можно получить спички бесплатно), однако, если
отбросить данный вариант как жульничес-\linebreak
кий, то стоит остановиться на цифре 2, которой соответствует
самое маленькое ненулевое число из ряда выше.

\itB $953 = 32 \cdot 29 + 25$. То есть, за один проход по экватору 
гусеница сделает 32 оборота и еще сдвинется на 25
секций в ходе 33 оборота. Исходя из этого, номер секции на линии старта после $k$ оборотов можно
выразить как $(k \cdot 25) \bmod 29$. Поскольку 29 и 25 взаимно просты, то за 29 оборотов каждая из 
секций по разу побывает
на линии старта.

То же касается и свежеустановленной секции: она также будет оказываться 
на линии старта раз в 29 оборотов. 

\itC Этот способ получения простых не работает уже с 6 
простым числом: $2\cdot 3 \cdot 5 \cdot 7 \cdot 11 \cdot 13 + 1 = 30031 = 59 \cdot 509$.

\end{itemize}

