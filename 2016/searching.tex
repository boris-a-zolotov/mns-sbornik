\task{В поисках чисел}
\begin{itemize}
\itA Первый ребус. Справа число РОМАШКА --- в нём 7 знаков, оно не меньше 1000000.
Слева, в произведении, 6 различных цифр, одна цифра повторяется два раза, значит, максимальное
значение этого выражения $9\cdot 9 \cdot 8 \cdot 7 \cdot 6 \cdot 5 \cdot 4 = 544320$.
Данный ребус не имеет решений.

Второй ребус. Заметим, что $\text{УРАУРА} = 1001 \cdot \text{УРА}$, однако $1001 = 11 \cdot 91$.
Число 11 простое, но в левой части равенства все числа --- одноразрядные, поэтому левая часть
на 11 не делится. Значит, и этот ребус не имеет решений.

\itB Поскольку групп всего 4, а чисел --- 6, то либо найдётся группа с тремя числами, либо две
группы будут содержать по два числа. В первом случае заметим, что произведение трёх минимальных
чисел из диапазона от 3 до 8 равно $60$. Во втором случае выберем ту группу, в которой нет 
числа 3: произведение её элементов как минимум $4 \cdot 5 = 20$.

\itC (не сводящийся к перебору значений)

Для данного числа $n$ предъявите алгоритм нахождения наименьшего составного числа $N$ 
такого, что $n!$ не делится на $N$, и докажите, что полученное составное число действительно
будет наименьшим.

Рассмотрим минимальное простое число, б\'ольшее $n$, пусть это $p$.
Понятно, что $n!$ не делится на $p$ и тем более не делится на $2p$. 
Пусть существует составное число $t$, такое, что $n! \divsby t$.

Согласно постулату Бертрана, между $n$ и $2n$ есть хотя бы одно простое число,
поэтому $2p < 4n$. Отсюда, $t < 4n$. 

Поскольку $p$ --- минимальное простое, большее $n$, то $t$ может быть целиком
представлено в виде произведения сомножителей $n!$: 
$$t = 2^{a_2} \cdot \ldots n^{a_n}$$

Заметим, что наиболее выгодный вариант --- иметь в качестве $t$ степень некоторого
простого числа: $t = q^{a_q}$. Ведь если $n!$ не делится на $t$, то значит, что один
из простых сомножителей $t$ имеет слишком высокую степень. Давайте оставим только его,
это только сделает $t$ меньше, а делимости не добавит.

В этот момент мы можем остановиться и предложить следующий алгоритм поиска 
числа: рассмотрим простые числа от $1$ до $n$, и найдём минимальную степень 
$s_q$ для каждого, такую, что $k^{s_q}$ не делит $n!$. Также найдём минимальное
простое число $p$, такое, что $p > n$. После чего найдём среди $2p$ и полученных
чисел $q^{s_q}$ минимальное --- это и будет ответ.

Однако, мы можем продолжить, поскольку мы можем попробовать оценить требуемые
степени $a_q$. Разобравшись, мы можем существенно алгоритм упростить.

Оценим, какая минимальная степень $a_q$ должна быть у простого числа $q$, чтобы 
$n!$ не делилось на $t=q^{a_q}$.

%Заметим, что $q < \sqrt{n}$ при $n\ge 16$, поскольку минимальная степень, на 
%которую не делится $n!$ равна $\sqrt{n}$, а даже $q^3 \ge \sqrt{n}\cdot n \ge 4n > 2p$, 
%потому такое $q$ не подходит.


%Во-первых, в множители $1\cdot 2 \cdot 3 \cdot \ldots \cdot n$ число $k$ входит
%$\lfloor\frac{n}{k}\rfloor$ раз. Во-вторых, поскольку $k < 2\sqrt{n}$, то 


%При этом, если $k^{\frac{n}{k}+1} \ge 4n$, то тогда $2p < t$ и данное простое
%число $k$ нет смысла рассматривать.

Рассмотрим классическое неравенство $k^n > n^k$ (выполнено при $k,n>3$).


%Заметим же, что $2^{\frac{n}{2}+1} > 4n > p$, если $n \ge 12$.

%Но вообще, в $n!$ входит $\frac{n}{k}$ множителей $k$, то есть, если $a_k > \frac{n}{k}$, 
%то $k^{\frac{n}{k}+1} > 4n$                    n^2 > 4n
%k^n > (4n)^k   // k <= n
%k^(k+n) > 4n^k                 n^(n+n) > (4n)^n
%                               (n+n) * log n > n * log (4n) = n * log n + n * log 4
%
%e^n > n^e
%n * log e > e * log n
%n * log k > (e (log k/log e)) * log n > 
%
%0 <= k <= n
%n log k <= k log n
%
%k^n > n^k
%n^n > n^n
%
%
%k^n > n^k:
%(n-1)^n      n^(n-1)
%
%(k+n) log k > k (log n + log 4)
%
%(k+n) log k > k log n
%
%n log k > k log n
%k log k >= k log 4 // k >= 4
%
%(k+n) log k > k (log n + log 4)
%k^(n+k) > 4n^k



%n^n > (4n)^n ???



%$$k^n > (4n)^k= n^k \cdot 4^k$$
%
%$$k^10 > 10^k$$
%k<n


\end{itemize}
