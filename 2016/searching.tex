\task{В поисках чисел}
\begin{itemize}
\itA Первый ребус. Справа число РОМАШКА --- в нём 7 знаков, оно не меньше 1000000.
Слева, в произведении, 6 различных цифр, одна цифра повторяется два раза, значит, максимальное
значение этого выражения $9\cdot 9 \cdot 8 \cdot 7 \cdot 6 \cdot 5 \cdot 4 = 544320$.
Данный ребус не имеет решений.

Второй ребус. Заметим, что $\text{УРАУРА} = 1001 \cdot \text{УРА}$, однако $1001 = 11 \cdot 91$.
Число 11 простое, но в левой части равенства все числа --- одноразрядные, поэтому левая часть
на 11 не делится. Значит, и этот ребус не имеет решений.

\itB Поскольку групп всего 4, а чисел --- 6, то либо найдётся группа с тремя числами, либо две
группы будут содержать по два числа. В первом случае заметим, что произведение трёх минимальных
чисел из диапазона от 3 до 8 равно $60$. Во втором случае выберем ту группу, в которой нет 
числа 3: произведение её элементов как минимум $4 \cdot 5 = 20$.

\itC Для данного числа $n$ предъявите алгоритм нахождения наименьшего составного числа $N$ 
такого, что $n!$ не делится на $N$, и докажите, что полученное составное число действительно
будет наименьшим.


\end{itemize}
