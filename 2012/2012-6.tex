\secklas{6}

\taskno{2}

\begin{itemize}

\itB Гастрабайтер Гаджи хочет замостить квадратную комнату $7 \times 7$ метров панелями $1 \times 5$ и $2 \times 3$ метров. Сколько панелей ему понадобится? Приведите пример такого замощения. Может ли он обойтись другим количеством панелей?

\itr Разберёмся сначала с количеством панелей. Пусть есть замощение с $A$ панелями $1 \times 5$ и $B$ панелями $2 \times 3$. Тогда число $49 - 5A$ должно делиться на 6 — в частности быть чётным. Среди всех чётных чисел, получающихся как $49 - 5A$ (4, 14, 24, 34, 44) на 6 делится только 24. Таким образом, замощение может состоять исключительно из 5 панелей $1 \times 5$ и 4 панелей $2 \times 3$.

Зная это, придумать замощение несложно:

\def\hsp{\ \hspace{0.7cm}}
\def\bcel{\multicolumn{1}{|c|}{$\cdot$}}
\def\lcel{\multicolumn{1}{|c}{$\cdot$}}
\def\rcel{\multicolumn{1}{c|}{$\cdot$}}
\def\cel{\multicolumn{1}{c}{$\cdot$}}
\begin{center}
\begin{tabular}{|c|c|c|c|c|c|c|}
\hline
	\bcel & \cel & \cel & \cel & \lcel & \cel & \rcel \\
	\bcel & \cel & \cel & \cel & \lcel & \cel & \rcel \\ \cline{2-7}
	\bcel & \cel & \cel & \lcel & \cel & \lcel & \bcel \\
	\bcel & \cel & \cel & \lcel & \cel & \lcel & \bcel \\
	\bcel & \cel & \cel & \lcel & \cel & \lcel & \bcel \\ \cline{1-5}
	\lcel & \cel & \cel & \cel & \cel & \lcel & \bcel \\ \cline{1-5}
	\lcel & \cel & \cel & \cel & \cel & \lcel & \bcel \\ \hline
\end{tabular}
\end{center}

\itC У Карлсона было 20 банок варенья. Он расставил их на трех полках так, что на каждой полке стояло одинаковое количество литров варенья. На первую полку Карлсон поставил одну большую и четыре средних банки варенья, на вторую — две большие и шесть литровых банок, на третью — одну большую, три средних и три литровых банки. Сколько литров варенья было у Карлсона?

\itr Сравним первую и третью полки Карлсона. На первой четыре средних банки, а на третьей — три средних и ещё три литра. Значит, в одной средней банке три литра варенья. Соответственно, в трёх средних банках — девять литров.

Теперь сравним вторую и третью полки. На третьей — одна большая банка и ещё 12 литров варенья. На второй — две больших и ещё 6 литров. Значит одна большая банка содержит в себе 6 литров варенья.

Таким образом, на второй полке $2 \cdot 6 + 6 = 18$ литров варенья — и столько же на на первой, и столько же на третьей. Значит, у Карлсона 54 литра варенья.
\bigbreak\noindent

\end{itemize}

\taskno{3}

\begin{itemize}

\itC Мальчик Вовочка записал в строчку один за другим 10 первых простых чисел в возрастающем порядке (единица не считается простым числом). Мальчик Дима сначала вычеркнул половину цифр в полученном числе так, что получилось наибольшее число, а затем снова из этого же числа вычеркнул половину цифр так, что получилось наименьшее число. На сколько наибольшее число больше наименьшего?

\itr Выпишем строку, полученную Вовочкой:

\vspace{-0.4cm}
$$2357111317192329.$$

В ней 16 цифр, значит вычеркнуть надо 8. Первая цифра числа, которе мы получим, лежит среди первых 9 цифр строки — чтобы получить наибольшее число, мы должны максимизировать её. Подойдёт 7. Продолжая тем же образом, получим, что наибольшее число, которое мог получить Дима, —

\vspace{-0.4cm}
$$77192329,$$

а наименьшее —

\vspace{-0.4cm}
$$11111232.$$

Их разность равна 66081097.

\end{itemize}


\taskno{7}

\begin{itemize}

\itB Винни-Пух и Пятачок одновременно отправились в гости к друг другу. Но поскольку Винни-Пух всю дорогу сочинял очередную «шумелку», а Пятачок считал пролетавших галок, они не заметили друг друга при встрече. После встречи Пятачок подошел к дому Винни-Пуха через четыре минуты, а Винни-Пух подошел к дому Пятачка через одну минуту. Сколько минут был в пути каждый из них?

\def\vv{v_\text{В}} \def\vp{v_\text{П}}
\itr Пусть $\vv$ — скорость Винни-Пуха, $\vp$ — скорость Пятачка, а $t$ — время от их выхода из дома до встречи. Тогда из условия задачи ясно, что расстояние между домами Винни-Пуха и Пятачка выражается как

\vspace{-0.4cm}
$$S = (t+1)\vv = (t+4)\vp = t(\vv+\vp).$$

Рассмотрим вторую и четвёртую части этого равенства:

\begin{align*}
	& t\vv + 1 \cdot \vv = t\vv + t\vp \\
	& 1 \cdot \vv = t \vp \\
	& \frac{\vv}{\vp} = \frac{t}{1}
\end{align*}

Теперь, зная это, разберёмся со второй и третьей частями:

\begin{align*}
	& (t+1)t\vp = (t+4)\vp,\ \ t>0 \\
	& t(t+1) = t+4 \\
	& t^2 = 4\ \ \Longrightarrow\ \ t=2.
\end{align*}


Таким образом, Винни-Пух был в пути 3 минуты, а Пятачок — 6 минут.
\end{itemize}

\taskno{8}
\begin{itemize}

\itA Может ли какая-нибудь степень двойки содержать в своей записи поровну нулей, единиц, двоек, $\ldots$, девяток?

\itr Пусть каждая из цифр встречается в этой степени двойки по $k$ раз. Тогда сумма цифр степени двойки равна $k(0 + \ldots + 9) = k \cdot 45$. То есть, по признаку делимости на 3, степень двойки делится на 3. Такого не может быть — получили противоречие.

\itB Однажды перед сном мальчик Вовочка просчитал вслух от одного до тысячи. Сколько слов произнес Вовочка? Каждое слово считается столько раз, сколько оно произнесено.

\itr Пусть $D$ — количество слов, которое нужно произнести, чтобы посчитать от 1 до 99. Тогда всего Вовочка произнёс $10D + 900 + 1$ слов: в каждой стоне нужно посчитать от 1 до 99, плюс 900 трёхзначных чисел дают по одному слову для обозначения сотен, плюс слово «тысяча».

Осталось найти $D$. 27 чисел требуют одно слово для произнесения: 1, 2, 3, $\ldots$, 19, 20, 30, 40, \ldots, 80, 90. Все остальные — а их $99 - 27 = 72$ — по два слова. Отсюда

\begin{align*}
	& D = 72 \cdot 2 + 27 = 171\scolon \\
	& 10D + 901 = 2611.
\end{align*}

\end{itemize}