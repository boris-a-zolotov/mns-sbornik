\secklas{8}

\taskno{3}

\begin{itemize}

\itB Натуральные числа $m$, $n$ удовлетворяют равенству

$$(m-n)^2=\frac{4mn}{m+n-1}.$$

Докажите, что $m+n$ -- квадрат натурального числа.

\itr По условию задачи,

$$(m-n)^2 (m+n-1) - 4mn = 0.$$

Продолжим цепочку равенств:

\begin{align*}
	& (m-n)^2 (m+n-1) - 4mn = \\
	& = m^3 + n^3 - m^2n - mn^2 - m^2 - 2mn - n^2 =\\
	& = (m+n)(m^2-mn+n^2) -mn(m+n) -(m+n)^2 = \\
	& = (m+n)(m-n)^2 -(m+n)^2.
\end{align*}

Отсюда можно понять, что число $m+n$ является отношением двух квадратов а, значит, и само квадрат.

\itC Вещественные числа $x$, $y$ удовлетворяют соотношениям

$$x^2+xy+y^2=4\quad\text{и}\quad x^4+x^2y^2+y^4=8.$$

Найти значение выражения $x^6+x^3y^3+y^6$.

\itr

\begin{align*}
& \begin{cases}
	(x+y)^2 - xy = 4 \\
	(x+y)^4 - 4xy(x+y)^2 + 2(xy)^2 = 8 \\
	\label{herestav} \text{\it Найти $(x+y)^6 - 6xy(x+y)^4 + 9(xy)^2(x+y)^2 - (xy)^3$}
\end{cases} \\
& \begin{cases}
	T_1 - T_2 =4 \qquad \text{($T_1=T_2+4$)} \\
	T_1^2 - 4T_1T_2 + 2T_2^2 = 8
\end{cases} \\
	& (T_2+4)^2 - 4(T_2+4)T_2 + 2T_2^2 = 8 \\
	& -T_2^2 - 8 T_2 + 8 = 0 \\
	& T_2 = -4 \pm 2 \sqrt 6\qquad \text{($T_1 = \pm 2\sqrt 6$)}
\end{align*}

Теперь мы знаем сумму и произведение $x$ и $y$. Можно как подставить значения $T_1$ и $T_2$ в \hyperref[herestav]{выражение в первой системе, равное исходному}, так и явно найти $x$, $y$ и посчитать нужное выражение для них.

\end{itemize}

\taskno{4}

\begin{itemize}

\itA Какое из чисел 1, 2, 3, 4, 5, 6, 7 надо выбросить, чтобы сумма квадратов из трёх оставшихся чисел оказалась равной сумме квадратов других трёх оставшихся чисел?

\itr Сейчас у нас есть семь «подозреваемых» чисел, которые можно выкинуть. Давайте сократим их количество. Заметим, что если две суммы чисел равны, то они равны и по модулю 2, и по модулю 3. Выпишем остатки от деления квадратов чисел 1–7 на 2 и на 3 и посмотрим, какой из них нужно выкинуть, чтобы можно было поделить оставшееся на две части.

\begin{center} \begin{tabular}{|c|c|c|c|c|c|c|c|}
	\hline
	x & 1 & \bf 2 & 3 & \bf 4 & 5 & 6 & 7 \\ \hline
	$x^2 \bmod 2$ & 1 & \bf 0 & 1 & \bf 0 & 1 & 0 & 1\\ \hline
	$x^2 \bmod 3$ & 1 & \bf 1 & 0 & \bf 1 & 1 & 0 & 1\\ \hline
\end{tabular} \end{center}

Легко понять, что надо сверху выкинуть ноль (чтобы сумма оставшихся чисел была чётна), а снизу единицу. То есть, либо $2^2$, либо $4^2$.

Сумма $1^2 + \ldots + 7^2$ равна 140. $\tfrac{140-4}{2}=68$, $68-49=19$ — двумя из трёх квадратов, которые мы поместим вместе с 49, нельзя собрать 19 (потому что 19 вообще не получается как сумма двух квадратов), поэтому вариант с выкидыванием двойки не подходит.

Наконец, $\tfrac{140-16}{2} = 62$, $49+9+4=62=36+25+1$. Это и есть ответ.

\itC Найдите все пары $(x, y)$ неотрицательных целых чисел, удовлетворяющих равенству

$$x-y=x^2+xy+y^2.$$

\itr Заметим, что при $x \ge 2$ правая часть строго больше $x$, а левая — не больше его. Значит, $x=0$ или $x=1$. Если $x=0$, то $-y = y^2$, $y=0$. Если $x=1$, то $1-y=y^2+y+1$, $y(y+2)=0$, $y=0$.

Ответ: $(0,0)$, $(1,0)$.
\end{itemize}

\taskno{7}

\begin{itemize}
\itA У бизнесмена Березова было предприятий в 3 раза меньше, чем у бизнесмена Романова. Если бы Березов отсудил еще столько же предприятий у Романова, сколько имел, то у них обоих число предприятий стало бы одинаково. Сколько предприятий было у бизнесмена Романова?

\itr Любое число предприятий, кратное трём.

Действительно, если у Романова сейчас $3x$ предприятий, то у Березова $x$. И если Березов отсудит $x$ предприятий у Романова, то у обоих станет поровну — по $2x$. И этот результат не зависит от конкретного $x$, то есть, рассуждения можно проделать при любом его значении.
\end{itemize}