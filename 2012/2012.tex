\documentclass[11pt,a4paper,book]{ncc} \usepackage{modules/nonstahp_book}
\usepackage{mathspec}

\setmainfont[
	Path = f/,
	BoldFont=pb.ttf,
	ItalicFont=pi.ttf,
	BoldItalicFont=pbi.ttf
		]{p.ttf}
\setsansfont[
	Path = f/,
	BoldFont=pb.ttf,
	ItalicFont=pi.ttf,
	BoldItalicFont=pbi.ttf
		]{p.ttf}
		
\setmathfont(Digits)[Path = f/]{p.ttf}
\setmathfont(Latin)[Path = f/]{pi.ttf}
\setmathfont(Greek)[Path = f/, Uppercase]{p.ttf}
\setmathfont(Greek)[Path = f/, Lowercase]{pi.ttf}

\setmonofont[Path = f/]{pmono.ttf}

%\setCJKmainfont[
%	Path=f/,
%	BoldFont=notoserifb.ttf,
%	ItalicFont=notoserifi.ttf,
%	BoldItalicFont=notoserifbi.ttf
%		]{notoserif.ttf}

 \begin{document}



\medbreak\centerline{\bfseries 6 класс}\bigbreak\noindent

\bigbreak\noindent
Задача 1.\par\noindent
A:\qquad Вовочке удалось, взяв два раза цифры 1,\ 2,\ 3 и 4, написать восьмизначное число, у которого между единицами стоит одна цифра, между двойками -- две, между тройками -- три, а между четверками -- четыре. Найдите такое число.
\medbreak\noindent

\bigbreak\noindent Задача 2.\par\noindent

\medbreak\noindent
B:\qquad Гастрабайтер Гаджи хочет замостить квадратную комнату 7х7 метров панелями 1х5 и 2х3 метров. Сколько панелей ему понадобится? Приведите пример такого замощения. Может ли он обойтись другим количеством панелей?
\medbreak\noindent
C:\qquad  У Карлсона было 20 банок варенья. Он расставил их на трех полках так, что на каждой полке стояло одинаковое количество литров варенья. На первую полку Карлсон поставил одну большую и четыре средних банки варенья, на вторую --  две большие и шесть литровых банок, на третью -- одну большую, три средних и три литровых банки. Сколько литров варенья было у Карлсона?
\bigbreak\noindent
Задача 3.\par\noindent
А (шутка): \qquad Стул упал с лестницы, состоящей из четырех ступенек, и сломал одну ножку. Сколько ножек сломает стул, если он затем упадет с лестницы, состоящей из 16 ступенек?\medbreak\noindent

\medbreak\noindent
C:\qquad Мальчик Вовочка записал в строчку один за другим 10 первых простых чисел в возрастающем порядке (единица не считается простым числом). Мальчик Дима сначала вычеркнул половину цифр в полученном числе так, что получилось наибольшее число, а затем снова из этого же числа вычеркнул половину цифр так, что получилось наименьшее число.  На сколько наибольшее число больше наименьшего?
\bigbreak\noindent
Задача 4.\par\noindent
A:\qquad Старуха Шапокляк очень любит животных. Все ее животные, кроме двух, -- собаки, все, кроме двух, -- кошки, и все кроме двух, -- попугаи, остальные -- крыски. Сколько каких животных у старухи Шапокляк?
\medbreak\noindent

\bigbreak\noindent
Задача 5.\par\noindent
A:\qquad Историк  нашел в архиве проект бюджета, в котором министр финансов, живший в XXI веке, писал:\par
В этом бюджете написано сто следующих утверждений:

-- в этом бюджете одно неверное утверждение;

-- в этом бюджете два неверных утверждения;

-- в этом бюджете три неверных утверждения;

.......................................................................................

-- в этом бюджете сто неверных утверждений.

Есть ли среди них верное утверждение?
\medbreak\noindent

\medbreak\noindent
C:\qquad Буратино не хотел ходить в школу, и черепаха Тортилла решила его проучить. Она не просто отдала ему Золотой ключик, а задала ему непростую задачу. Она вынесла три коробочки: красную, синюю и зеленую. На красной коробочке было написано "Здесь лежит Золотой ключик", на синей -- "Зеленая коробочка пуста", а на зеленой -- "Здесь сидит гадюка". Тортилла прочитала надписи Буратино и сказала: "Действительно, в одной коробочке лежит Золотой ключик, в другой -- сидит гадюка, а третья коробочка пуста, но все эти надписи неверные. Где лежит Золотой ключик?\bigbreak\noindent
Задача 6.\par\noindent
A:\qquad Квадрат разбит на шесть прямоугольников (см. рисунок). Периметры четырех из них равны между собой. Докажите, что центральный прямоугольник является квадратом.\medbreak\noindent

\bigbreak\noindent
Задача 7.\par\noindent

\medbreak\noindent
B:\qquad Винни-Пух и Пятачок одновременно отправились в гости к друг другу. Но поскольку Винни-Пух всю дорогу сочинял очередную "шумелку", а Пятачок считал пролетавших галок, они не заметили друг друга при встрече. После встречи Пятачок подошел к дому Винни-Пуха через четыре минуты, а Винни-Пух подошел к дому Пятачка через одну минуту. Сколько минут был в пути каждый из них?
\medbreak\noindent

\bigbreak\noindent
Задача 8.\par\noindent
A:\qquad Может ли какая-нибудь степень двойки содержать в своей записи половину нулей, единиц, двоек, $\ldots$, девяток?\medbreak\noindent
B:\qquad Однажды перед сном мальчик Вовочка просчитал вслух от одного до тысячи. Сколько слов произнес Вовочка? Каждое слово считается столько раз, сколько оно произнесено.\medbreak\noindent


\vfill\eject

\medbreak\centerline{\bfseries 7 класс}\bigbreak\noindent

\bigbreak\noindent
Задача 1.\par\noindent
A:\qquad Существуют ли два таких натуральных числа, наибольший общий делитель которых равен 110, а наименьшее общее кратное равно 2000?
\medbreak\noindent
B:\qquad Вовочка умножил два подряд стоящих натуральных числа и получил число, состоящее из цифр\ 1,3,4,5,7,8 и 9. Покажите, что Вовочка ошибся.
\medbreak\noindent
C:\qquad Существуют ли 2012 ненулевых числа, никакие два из которых не равны между собой, таких, что их сумма равна их произведению?
\bigbreak\noindent
Задача 2.\par\noindent
A:\qquad Прямоугольник разрезами, параллельными его сторонам, разбит на 4 маленьких прямоугольника. Площади трех из них известны. Это 3, 4, 5. Найдите площадь четвертого.
\medbreak\noindent
B:\qquad Мальчики Вова и Дима купили по одной новогодней открытке и каждый разрезал свою открытку на два прямоугольника равной площади. Один из прямоугольников мальчики выбросили, а другой оставили себе. Оказалось, что периметр Вовиного прямоугольника равен 14 см, периметр Диминого -- 19 см.Найдите периметр и стороны прямоугольной открытки у каждого из мальчиков.
\medbreak
\noindent

\bigbreak\noindent
Задача 3.\par\noindent
A:\qquad Не используя технические средства, сравните дроби
$$
\frac{2012}{2013}\qquad\text{и}\qquad\frac{2012000000002012}{2013000000002013}.
$$
\medbreak\noindent

\medbreak\noindent

\bigbreak\noindent
Задача 4.\par\noindent
A:\qquad Какую четверку цифр надо приписать справа к числу 2012, чтобы полученное восьмизначное число делилось на 2013?
\medbreak\noindent

\medbreak\noindent
C:\qquad Произведение шести последовательных натуральных чисел может быть равно произведению трех последовательных натуральных чисел. Например,
$$1\cdot 2\cdot 3\cdot 4\cdot 5\cdot 6=8\cdot 9\cdot 10=720,$$Есть ли еще такие числа?
\bigbreak\noindent
Задача 5.\par\noindent


\medbreak\noindent
C:\qquad Вдоль прямолинейной аллеи городского парка растет 5 деревьев. Известны 8 из 10 попарных расстояний между ними:\ 1 м, 1м, 2м, 2м, 3м, 3м, 3м, 4м. Найдите два остальных расстояния.
\bigbreak\noindent
Задача 7.\par\noindent
A (шутка):\qquad Стул упал с лесницы, состоящей из 4 ступенек и сломал одну ножку. Затем, стул упал с лесницы, также состоящей из 4 ступеней, и снова сломал еще одну ножку. Затем стул упал с лестницы, состоящей из 8 ступенек. Сколько ножек сломал стул?


Задача 8.

\medbreak\noindent
C:\qquad Какое наименьшее число различных цифр нужно выбрать,
чтобы любое число от 1 до 100 можно представить в виде суммы выбранных цифр, каждую из которых нельзя использовать более четырех раз?

бмо с.69
\vfill\eject

\medbreak\centerline{\bfseries 8 класс}\bigbreak\noindent

\bigbreak\noindent

\bigbreak\noindent
Задача 3.\par\noindent

\medbreak\noindent
B:\qquad Натуральные числа $m$  $n$ удовлетворяют равенству
$$
(m-n)^2=\frac{4mn}{m+n-1}.
$$
Докажите, что $m+n$ -- квадрат натурального числа.
\medbreak\noindent
C:\qquad Вещественные числа $x,\  y$ удовлетворяют соотношениям
$$
x^2+xy+y^2=4\qquad\text{и}\qquad x^4+x^2y^2+y^4=8.
$$
Найти значение выражения $x^6+x^3y^3+y^6.$
\bigbreak\noindent
Задача 4. \par\noindent
A:\qquad Какое из чисел 1,2,3,4,5,6,7 надо выбросить, чтобы сумма квадратов из трех оставшихся чисел оказалась равной сумме квадратов других трех оставшихся чисел?
\medbreak\noindent

\medbreak\noindent
C:\qquad Найдите все пары $(x,\  y)$ неотрицательных целых чисел, удовлетворяющих равенству
$$x-y=x^2+xy+y^2.$$
\bigbreak\noindent

Задача 7. \par\noindent
A:\qquad У бизнесмена Березова было предприятий в 3 раза меньше, чем у бизнесмена Романова. Если бы Березов отсудил еще столько же предприятий у Романова, сколько имел, то у них обоих число предприятий стало бы одинаково. Сколько предприятий было у бизнесмена Романова?
\medbreak\noindent



\end{document}