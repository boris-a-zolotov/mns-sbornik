\secklas{5}

\taskno{4}

\begin{itemize}
\itA Старуха Шапокляк очень любит животных. Все ее животные, кроме двух,~— собаки, все, кроме двух,~— кошки, и все, кроме двух,~— попугаи, остальные~— крыски. Сколько каких животных у старухи Шапокляк?

\itr Пусть у Шапокляк $s$ собак, $c$ кошек, $p$ попугаев и $k$ крысок. Тогда

$$
\begin{cases}
c+p+k = 2 \\
s+p+k = 2 \\
s+c+k = 2 \\
\end{cases}
$$

Вычитая эти равенства друг из друга, можно получить, что $s=c=p$. Более того, эти три числа обязаны не превосходить 1, потому что иначе суммы в левой части равенств будут больше двух.

Первый случай: $s=c=p=0$. Тогда у Шапокляк две крыски, а других животных нет. Второй случай: $s=c=p=1$. Тогда у шапокляяк одна собака, одна кошка и один попугай, а крыс нет.

Оба ответа являются верными~— более того, для получения полного балла за задачу оба должны быть приведены.

\itB Четверо пятиклассников Андрей, Борис, Вася и Гена решили определить свой вес. Однако  все четверо мальчиков на весы не помещались, поэтому они стали взвешиваться по трое или по двое. Оказалось, что Андрей, Боря и Вася вместе весят 90 кг, Боря, Вася и Гена~— 92 кг, а Андрей и Гена~— 58 кг. Сколько весят все мальчики вместе?

\itr Давайте сложим четыре результата, которые получились у мальчиков в условии задачи. Заметим, что тогда получится дважды вес всех мальчиков вместе, и он равен 240 кг. Значит, мальчики вместе весят 120 кг.

\end{itemize}


\taskno{5}

\begin{itemize}
\itB С 1 января цена билета в кинотеатр возросла по сравнению с декабрем на $20 \%$, а выручка от продажи билетов возросла на $14 \%$. Как изменилась посещаемость кинотеатра? Увеличилась или уменьшилась?

\itr Если бы посещаемость кинотеатра не изменилась, то выручка бы увеличилась так же, как и цена билета~— на 20 процентов. Однако общая выручка уменьшилась~— значит, посетителей стало приходить меньше. Можно даже посчитать, насколько меньше: $\tfrac{1.14}{1.20} = 0.95$~— то есть, в кинотеатр теперь ходит $95 \%$ от прежнего количества посетителей.

\itC В мешке у Деда Мороза лежат 10 красно-синих (т.е. одна половина красная, а другая~— синяя), 7 сине-зеленых, 5 зелено-красных шаров. Какое наименьшее число шаров должен вынуть из мешка Дед Мороз, чтобы нашелся такой цвет, который будет присутствовать в окраске не менее чем в пяти из вынутых шаров?

\itr Что означает, что нашёлся цвет, присутствующий в окраске пяти шаров? Это означает, что нашлись два сорта шаров (например, красно-зелёные и сине-зелёные), шаров из которых {\itshape в сумме} хотя бы пять.

Пусть Дед Мороз вынул пять шаров. Среди них должны присутствовать все три имеющихся типа, иначе есть два, которые в сумме дадут вынутые пять. Значит, шары распределены по типам как 2-2-1 или 3-1-1. Во втором случае какой шар ни возьми следующим~— первый и второй или первый и третий типы в сумме дадут 5.

В первом случае можно взять ещё один шар третьего типа, а следующий, седьмой, уже даст нам пять шаров в сумме у двух типов.

Ответ~— 7 шаров.
\bigbreak\noindent

\end{itemize}


\taskno{6}
\begin{itemize}

\itA Дед рассказывал своим внучатам: «В комнате стояло 5 стульев. На них сидели 4 матери, 4 дочки, 3 бабушки, 2 прабабушки и 1 прапрабабушка. При этом каждая из них сидела на отдельном стуле.» «Это невозможно»,~— возразили внучата. «Я сам видел»,~— ответил дед. Правду ли сказал дед внучатам?

\def\upno#1{$ ^{(\text{#1})}$}
\itr Как ни странно, дед не соврал. Представим себе пять женщин: \upno{1}совсем девочка, \upno{2}её мама, \upno{3}её бабушка, \upno{4}её прабабушка, \upno{5}её прапрабабушка.

Дочерьми здесь являются четверо: (1)–(4), и матерями тоже четверо: (2)–(5). Трое, (3)–(5),~— бабушки\scolon двое, (4)–(5),~— прабабушки\scolon одна прапрабабушка.

\itC Буратино не хотел ходить в школу, и черепаха Тортилла решила его проучить. Она не просто отдала ему Золотой ключик, а задала ему непростую задачу. Она вынесла три коробочки: красную, синюю и зеленую. На красной коробочке было написано «Здесь лежит Золотой ключик», на синей~— «Зеленая коробочка пуста», а на зеленой~— «Здесь сидит гадюка». Тортилла прочитала надписи Буратино и сказала: «Действительно, в одной коробочке лежит Золотой ключик, в другой~— сидит гадюка, а третья коробочка пуста, но все эти надписи неверные». Где лежит Золотой ключик?

\itr Надпись на синей коробочке неверна~— значит, зелёная коробочка непуста. И надпись на зелёной коробочке неверна~— значит, в ней не гадюка. Отсюда в зелёной коробке лежит Золотой ключик.

\end{itemize}