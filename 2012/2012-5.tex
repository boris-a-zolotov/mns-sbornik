\documentclass[11pt,a4paper,book]{ncc} \usepackage{modules/nonstahp_book}
\usepackage{mathspec}

\setmainfont[
	Path = f/,
	BoldFont=pb.ttf,
	ItalicFont=pi.ttf,
	BoldItalicFont=pbi.ttf
		]{p.ttf}
\setsansfont[
	Path = f/,
	BoldFont=pb.ttf,
	ItalicFont=pi.ttf,
	BoldItalicFont=pbi.ttf
		]{p.ttf}
		
\setmathfont(Digits)[Path = f/]{p.ttf}
\setmathfont(Latin)[Path = f/]{pi.ttf}
\setmathfont(Greek)[Path = f/, Uppercase]{p.ttf}
\setmathfont(Greek)[Path = f/, Lowercase]{pi.ttf}

\setmonofont[Path = f/]{pmono.ttf}

%\setCJKmainfont[
%	Path=f/,
%	BoldFont=notoserifb.ttf,
%	ItalicFont=notoserifi.ttf,
%	BoldItalicFont=notoserifbi.ttf
%		]{notoserif.ttf}

 \begin{document}

\chapyear{2012}

\secklas{5}

\taskno{4}

\begin{itemize}
\itA Старуха Шапокляк очень любит животных. Все ее животные, кроме двух, — собаки, все, кроме двух, — кошки, и все, кроме двух, — попугаи, остальные — крыски. Сколько каких животных у старухи Шапокляк?

\itr Пусть у Шапокляк $s$ собак, $c$ кошек, $p$ попугаев и $k$ крысок. Тогда

$$
\begin{cases}
c+p+k = 2 \\
s+p+k = 2 \\
s+c+k = 2 \\
\end{cases}
$$

Вычитая эти равенства друг из друга, можно получить, что $s=c=p$. Более того, эти три числа обязаны не превосходить 1, потому что иначе суммы в левой части равенств будут больше двух.

Первый случай: $s=c=p=0$. Тогда у Шапокляк две крыски, а других животных нет. Второй случай: $s=c=p=1$. Тогда у шапокляяк одна собака, одна кошка и один попугай, а крыс нет.

Оба ответа являются верными — более того, для получения полного балла за задачу оба должны быть приведены.

\itB Четверо пятиклассников Андрей, Борис, Вася и Гена решили определить свой вес. Однако  все четверо мальчиков на весы не помещались, поэтому они стали взвешиваться по трое или по двое. Оказалось, что Андрей, Боря и Вася вместе весят 90 кг, Боря, Вася и Гена — 92 кг, а Андрей и Гена — 58 кг. Сколько весят все мальчики вместе?

\itr Давайте сложим четыре результата, которые получились у мальчиков в условии задачи. Заметим, что тогда получится дважды вес всех мальчиков вместе, и он равен 240 кг. Значит, мальчики вместе весят 120 кг.

\end{itemize}

Задача 5.\par\noindent
A:\qquad Переставьте в "равенстве" $20\cdot 203=4114$ цифры так, чтобы получилось верное равенство.
\medbreak\noindent
B:\qquad С 1 января цена билета в кинотеатр возросла по сравнению с декабрем на 20\%, а выручка от продажи билетов возросла на 14\%. Как изменилась посещаемость кинотеатра? Увеличилась или уменьшилась?
\medbreak\noindent
C:\qquad В мешке у Деда Мороза лежат 10 красно-синих (т.е. одна половина красная, а другая — синяя), 7 сине-зеленых, 5 зелено-красных шаров. Какое наименьшее число шаров должен вынуть из мешка Дед Мороз, чтобы нашелся такой цвет, который будет присутствовать в окраске не менее чем в пяти из вынутых шаров?
\bigbreak\noindent
Задача 6.\par\noindent
A:\qquad Дед рассказывал своим внучатам: "В комнате стояло 5 стульев. На них сидели 4 матери, 4 дочки, 3 бабушки, 2 прабабушки и 1 прапрабабушка. При этом каждая из них сидела на отдельном стуле." "Это невозможно", — возразили внучата. "Я сам видел", — ответил дед. Правду ли сказал дед внучатам?
\medbreak\noindent
B:\qquad Историк  нашел в архиве проект бюджета, в котором министр финансов, живший в XXI веке, писал:\par
В этом бюджете написано сто следующих утверждений:

-- в этом бюджете одно неверное утверждение;

-- в этом бюджете два неверных утверждения;

-- в этом бюджете три неверных утверждения;

.......................................................................................

-- в этом бюджете сто неверных утверждений.

Есть ли среди них верное утверждение?
\medbreak\noindent
C:\qquad Буратино не хотел ходить в школу, и черепаха Тортилла решила его проучить. Она не просто отдала ему Золотой ключик, а задала ему непростую задачу. Она вынесла три коробочки: красную, синюю и зеленую. На красной коробочке было написано "Здесь лежит Золотой ключик", на синей — "Зеленая коробочка пуста", а на зеленой — "Здесь сидит гадюка". Тортилла прочитала надписи Буратино и сказала: "Действительно, в одной коробочке лежит Золотой ключик, в другой — сидит гадюка, а третья коробочка пуста, но все эти надписи неверные. Где лежит Золотой ключик?























\end{document}