\secklas{7}

\taskno{1}

\begin{itemize}
\itA Существуют ли два таких натуральных числа, наибольший общий делитель которых равен 110, а наименьшее общее кратное равно 2000?

\itr Таких чисел, конечно же, не существует, ведь наименьшее общее кратное всегда делится на наибольший общий делитель — а 2000 не делится на 110.

\itB Вовочка умножил два подряд стоящих натуральных числа и получил число, состоящее из цифр~1, 3, 4, 5, 7, 8 и 9. Покажите, что Вовочка ошибся.

\itr $1+3+4+5+6+7+8+9=45-2=43$ — имеет остаток 1 при делении на 3. Значит, в соответствии с признаком делимости на 3, число, составленное из этих цифр, также будет иметь остаток 1. Произведение же двух последовательных натуральных чисел может иметь либо остаток 0 ($0 \cdot 1 = 2 \cdot 0 = 0$), либо остаток 2 ($1 \cdot 2 = 2$). Поэтому Вовочка неправ.

\itC Существуют ли 2012 ненулевых числа, никакие два из которых не равны между собой, таких, что их сумма равна их произведению?

\itr Укажем, как построить набор из таких чисел: он будет иметь вид

\vspace{-0.4cm}
$$1, 2, 3, 4, \ldots, 2011, x\scolon$$

\vspace{-0.4cm}
$$1+2+3+\ldots+2011+x=1\cdot 2\cdot 3\cdot\ldots\cdot2011\cdot x.$$

Чтобы выполнить условие задачи, нужно взять $x$, равный

$$\frac{1+2+3+\ldots+2011}{1 \cdot 2 \cdot 3 \cdot \ldots \cdot 2011 - 1}.$$

Это число строго меньше единицы (очевидно), поэтому не будет совпадать ни с одним из взятых до него.
\end{itemize}

\taskno{2}

\begin{itemize}

\itA Прямоугольник разрезами, параллельными его сторонам, разбит на 4 маленьких прямоугольника. Площади трех из них известны. Это 3, 4, 5. Найдите площадь четвертого.

\itr Ответ в этой задаче зависит от того, какой из прямоугольников, площадь которого известна, находится «между» двумя другими:

\def\lins{\cline{1-2} \cline{4-5} \cline{7-8}}
\begin{center} \begin{tabular}{|c|c|c|c|c|c|c|c|}
	\lins
	4 & 3 & \hspace{0.8cm} & 3 & 4 & \hspace{0.8cm} & 3 & 5 \\ \lins
	$5 \cdot \tfrac{4}{3} = \tfrac{20}{3}$ & 5 & &
	$5 \cdot \tfrac{3}{4} = \tfrac{15}{4}$ & 5 & &
	$4 \cdot \tfrac{3}{5} = \vphantom{\int\limits_0^0}
		\tfrac{12}{5}$ & 4 \\ \lins
\end{tabular} \end{center}

\itB Мальчики Вова и Дима купили по одной новогодней открытке и каждый разрезал свою открытку на два прямоугольника равной площади. Один из прямоугольников мальчики выбросили, а другой оставили себе. Оказалось, что периметр Вовиного прямоугольника равен 14 см, периметр Диминого — 19 см. Найдите периметр и стороны прямоугольной открытки у каждого из мальчиков.

\itr Пусть исходная открытка имела размер $a \times b$ сантиметров и, соответственно, периметр $2a+2b$. Единственный способ получить разные прямоугольники — Вове разрезать открытку по вертикали, а Диме — по горизонтали.

Тогда Вовина открытка будет иметь размер $\tfrac{a}{2} \times b$ и периметр $a+2b$, а Димина — размер $a \times \tfrac{b}{2}$ и периметр $2a+b$. Из условия нам известны значения периметров открыток:
\begin{align*}
	a+2b = 14\scolon \\
	2a+b = 19.
\end{align*}

Тогда разность $a-b$ равна пяти,
	$$a=8,\quad b=3.$$
	
В частности, периметр исходной открытки равен \SI{22}{\text{см}}.

\end{itemize}

\taskno{3}

\begin{itemize}

\itA Не используя технические средства, сравните дроби

$$
\frac{2012}{2013}\qquad\text{и}\qquad\frac{2012000000002012}{2013000000002013}.
$$

\itr Эти дроби равны между собой: вторая получается из первой домножением числителя и знаменателя на 1000000000001.

\bigbreak\noindent

\end{itemize}

\taskno{4}

\begin{itemize}

\itA Какую четверку цифр надо приписать справа к числу 2012, чтобы полученное восьмизначное число делилось на 2013?

\itr Очевидно, что числа 20130000 и 2013 делятся на 2013. Вычитая их друг из друга, получим все возможные ответы:

\begin{align*}
	& 20130000 - 2013 = 20127987 \\
	& 20130000 - 2 \cdot 2013 = 20125974 \\
	& 20130000 - 3 \cdot 2013 = 20123961 \\
	& 20130000 - 4 \cdot 2013 = 20121948 \\
\end{align*}

\itC Произведение шести последовательных натуральных чисел может быть равно произведению трех последовательных натуральных чисел. Например,

$$1\cdot 2\cdot 3\cdot 4\cdot 5\cdot 6=8\cdot 9\cdot 10=720.$$

Есть ли еще такие числа?

\itr В задаче требуется найти числа $x$, $y$ такие, что

\vspace{-0.4cm}
$$x(x+1)\ldots(x+5) = y(y+1)(y+2).$$

Частично раскроем скобки в этом выражении — 

\vspace{-0.4cm}
$$(x^2 + 5x)(x^2 + 5x + 4)(x^2 + 5x + 6) = y(y + 1)(y + 2).$$

Отсюда

\vspace{-0.4cm}
$$x^2+5x<y<x^2+5x+4.$$

Действительно: если бы неравенство выше не выполнено, то либо все три множителя в правой части меньше трёх множителей в левой, либо, наоборот, каждый из множителей правой части больше соответствующего множителя левой.

Для переменной $y$ остаётся три варианта:
	$$x^2+5x+1,\quad x^2+5x+2,\quad x^2+5x+3.$$

Подставив каждый из этих вариантов в правую часть, можно убедиться, что решений для $x>1$ у уравнения не будет.

\end{itemize}

\taskno{5}

\begin{itemize}

\def\met#1{\SI{#1}{\text{м}}}
\itC Вдоль прямолинейной аллеи городского парка растет 5 деревьев. 
Известны 8 из 10 попарных расстояний между ними: 
\met 1, \met 1, \met 2, \met 2, \met 3, \met 3, \met 3, \met 4. 
Найдите два остальных расстояния.

\itr Очевидно, что все попарные расстояния между деревьями целые — 
в том числе остальные два: в противном случае не-целых расстояний 
было бы больше, чем два. Значит, мы можем считать, что все деревья 
расположены в целых точках вещественной оси.

Некоторые из самых маленьких расстояний среди перечисленных являются 
расстояниями между соседними деревьями (назовём их атомарными). 
Обозначим эти расстояния за $a$, $b$, $c$ и $d$ в порядке рассмотрения
слева направо. 
Ясно, что единичные расстояния могут быть только атомарными. 
Заметим, что остальные попарные расстояния~--- это суммы
$a+b$, $b+c$, $c+d$, $a+b+c$, $b+c+d$ и $a+b+c+d$.

Давайте посчитаем общую сумму всех попарных расстояний:
$$4a + 6b + 6c + 4d = 1+1+2+2+3+3+3+4 + x+y = 19+x+y$$
где $x$ и $y$ --- два неизвестные расстояния.
Причём, поскольку левая часть чётна, $x+y$ должно быть нечётным.
Разберём варианты (с точностью до направления движения по аллее).

\begin{itemize}
\item $a+b+c+d < 4$ невозможно, так как все расстояния ненулевые. 
\item $a+b+c+d = 4$ невозможно, иначе все атомарные расстояния равны \met 1, 
в наличии же только два таких расстояния, ещё два надо добавить~--- 
то есть $x=y=1$, но $x+y$ должно быть нечётным.
\item $a+b+c+d = 5$ также невозможно, иначе $x=5$ и $y=1$
(поскольку $5=1+1+1+2$, необходимо к попарным расстояниям добавить 
атомарное расстояние \met 1 и расстояние \met 5), и $x+y$ снова чётное. 
\item Пусть $a+b+c+d = 6$. 
Рассмотрев все варианты получения требуемой суммы ($1+1+2+2$, $1+2+2+1$, 
$1+2+1+2$, $1+1+1+3$ и $1+1+3+1$), приходим к выводу, что единственный
вариант, согласующийся с условием, --- это $a=c=1$, $b=d=2$, и тогда $x=5$ и $y=6$.
\item Наконец, $a+b+c+d \ge 7$ невозможно, поскольку тогда 
как минимум одно из атомарных расстояний не меньше \met 3 --- так как
в наборе есть два атомарных по \met 1. Значит,
как минимум четыре попарные расстояния не меньше \met 4 ---
в этом можно убедиться, рассмотрев различные подсуммы
слагаемых в левой части неравенств $a+b+c+(3+k) \ge 7$ и $a+b+(3+k)+d \ge 7$
и случаи $k = 0$ и $k > 0$.
Но среди известных попарных расстояний только одно не меньше \met 4, 
то есть к попарным расстояниям требуется добавить минимум 
три новых, по условию же возможно добавить долько два.
\end{itemize}

Итого, рассмотрев варианты, мы пришли к выводу, что недостающие попарные
расстояния~--- \met 5 и \met 6.

%\begin{comment}
%\def\apart#1{\stackrel{\text{\met{#1}}}{\longleftrightarrow}}
%
%Может быть так, что метровые промежутки между деревьями 
%являются соседними ($a \apart 1 b \apart 1 c$), а может, что и нет.
%
%В первом случае одно из двухметровых расстояний получается как сумма однометровых. У оставшегося двухметрового расстояния не остаётся других вариантов, кроме как быть длиной промежутка между соседними деревьями.
%
%\subitem Если это два дерева, отличные от $a$, $b$, $c$, то получается следующая ситуация:
%
%$$a \apart 1 b \apart 1 c \qquad d \apart 2 e.$$
%
%Тогда $cd = \met 3$, и другие трёхметровые расстояния нам попросту никуда не вписать.
%
%\subitem Если одно из этих двух деревьев — $a$ или $c$, то получаем однозначный вариант:
%
%$$d \apart 2 a \apart 1 b \apart 1 c \apart 3 e,$$
%
%Который, очевидно, не удовлетворяет условию задачи.
%
%Во втором случае двухметровые расстояния также являются расстояниями между соседними деревьями. Это даёт нам однозначный с точностью до симметрии ответ: деревья стоят в точках 0, 1, 3, 4, 6, и оставшиеся два расстояния — 5 и 6.
%\end{comment}

\end{itemize}