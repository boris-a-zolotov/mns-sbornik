\secklas{8}

\taskno{1}

\begin{itemize}

	\itA Найдите значение выражения $13-x$, если значение разности $x-13$ противоположно числу $-13$.
	
	\itr Число $13-x$ является противоположным к числу $x-13$, и в условиях задачи равно $-13$.
	
	\itB В выражении
	$$\frac{x+y}{z} + t$$
	буквы заменили числами 1, 2, 3 и 4 (разные буквы на разные числа). Какое наименьшее значение может принимать это выражение?
	
	\itr Увеличение чисел $x$, $y$ и $t$ при прочих равных увеличивает значение выражения из задачи, а увеличение числа $z$ — уменьшает. Поэтому, чтобы как можно сильнее уменьшить значение выражения из задачи, нужно подставить самое большое число вместо $z$. Поэтому $z=4$.
	
	Получается выражение $\tfrac{1}{4} (x+y) + t$. Самое маленькое число выгодно оставить как есть, а числа побольше — «усмирить» коэффициентом~$\tfrac{1}{4}$. Поэтому ответ на задачу —
	$$\frac{2+3}{4} + 1 = \frac{9}{4}.$$
	
	\itC Решите уравнение
	$$x = \cfrac{1}{
		1 + \cfrac{1}{
			1 + \cfrac{1}{
				1+x \vphantom{\int^0}}}}$$
	
	\itr
	$$\cfrac{1}{
		1 + \cfrac{1}{
			1 + \cfrac{1}{
				1+x \vphantom{\int^0}}}}
	= \cfrac{1}{1+\cfrac{1+x}{2+x}}
	= \frac{2+x}{3+2x \vphantom{\int^0}}\scolon$$
	$$x = \frac{2+x}{3+2x}.$$
	
	Осталось решить квадратное уравнение:
	\begin{align*}
		& 2x^2 + 3x = x+2 \\
		& 2x^2 + 2x - 2 = 0 \\
		& x^2 + x - 1 = 0 \\
		& \qquad \mathcal D = 5 \\
		& x = \frac{-1 \pm \sqrt{5}}{2}.
	\end{align*}
	
	Нужно также проверить, что каждое из двух значений $x$ не обращает в ноль ни один знаменатель при вычислении цепной дроби — в нашем случае, когда в знаменателе всегда будет $\sqrt 5$ с ненулевым коэффициентом, это очевидно.

\end{itemize}

\taskno{2}

\begin{itemize}

	\itA Какое из следующих утерждений неверно?
	
	\begin{enumerate}[label=\arabic*)]
		\item 4101 и 2115 — не взаимно простые числа\scolon
		\item 97 — простое число\scolon
		\item $\text{НОД}\,(21,1001)=1$\scolon
		\item $\text{НОК}\,(15,14)=210$.
	\end{enumerate}
	
	\itr Неверно третье утверждение: $\text{НОД}\,(21,1001)$ на самом \linebreak деле равен 7. Проверку остальных утверждений мы оставляем читателю.
	
	\itB Сколько девяток встретится в последовательности 1, 2, 3, $\ldots$, 2014, 2015?
	
	\itr Среди чисел 2001–2015 ровно одна девятка, поэтому мы исключим их из рассмотрения на протяжении дальнейшего решения.
	
	Среди чисел от 1 до 10 встречается ровно одна девятка. Значит, среди чисел от 1 до 100 в младшем разряде встретится 10 девяток. Ещё 10 девяток на каждую сотню придут из второго разряда чисел 90–99.
	
	Поэтому среди чисел от 1 до 1000 встретится $10 \cdot (10+10) = 200$ девяток, пришедших из двух младших разрядов. Ещё 100 девяток встретится в третьем разряде чисел 900–999. Итого, на каждую тысячу приходится 300 цифр 9 в трёх младших разрядах.
	
	В задаче мы имеем дело числами от 1 до 2000 — в них встретится 600 цифр 9, и ещё одна — после 2000. Ответ: 601 цифра 9.
	
	\itC Последовательность составляется по следующему правилу:
	$$53,\ 503,\ 5003,\ 50003,\ 500003,\ \ldots$$
	Докажите, что в ней есть по крайней мере одно составное число.
	
	\itr Мы докажем, что в данной последовательности найдётся {\bfseries бесконечно много} чисел, делящихся на 7. Для начала заметим, что $k$--ое число в нашей последовательности имеет вид
	$$5\cdot 10^k + 3.$$
	
	Также обратим внимание на то, что при возведении числа 10 в натуральные степени остатки результата при делении на 7 «зацикливаются»:
	$$1,\ 3,\ 2,\ 6,\ 4,\ 5,\ 1,\ 3,\ \ldots$$
	
	В частности, для бесконечно многих $k$ число $10^k$ имеет остаток 5 \linebreak при делении на 7. В таком случае $5 \cdot 10^k + 3$ делится на 7, что и требовалось.

\end{itemize}

\taskno{3}

\begin{itemize}

	\itA Четыре прямые попарно пересекаются. Какое наибольшее число \linebreak точек пересечения может получиться?
	
	\itr Точек пересечения не может быть больше, чем собственно пересечений. В свою очередь, пересечений
	$$\frac{4 \cdot 3}{2} = 6.$$
	Получить 6 точек пересечения совсем просто:
	
	\begin{center} \tikz{
	\begin{scope}[scale=0.75]
		\draw[thick] (0,0) -- (4,2);
		\draw[thick] (4,1.5) -- (0,3.5);
		\draw[thick] (0.5,0) -- (3,2.5);
		\draw[thick] (3,1) -- (0.5,3.5);
	\end{scope}
	} \end{center}

\end{itemize}

\taskno{5}

\begin{itemize}

	\itA У какого трёхзначного числа больше всего делителей?
	
	\itr
	Делители числа — это произведения различных комбинаций его простых множителей. При равном количестве простых множителей больше делителей будет у числа, у которого простые множители более разнообразны.
	
	Рассмотрим наименьшие простые числа:
	\begin{align*}
		& 2 \cdot 3 \cdot 5 \cdot 7 = 210\scolon\\
		& 2 \cdot 3 \cdot 5 \cdot 7 \cdot 11 = 2310.
	\end{align*}
	
	Добавим простых множителей числу 210. $210 \cdot 2 \cdot 3$ — уже четырёхзначное число, поэтому ответом в нашей задаче будет
	$$2 \cdot 2 \cdot 2 \cdot 3 \cdot 5 \cdot 7 = 840.$$
	
	У этого числа 32 делителя. То, что у других трёхзначных чисел не бывает больше делителей, можно установить, например, перебором.
	
	\itB Из цифр 2, 5, 8 составили семизначное число (возможно, некоторые из этих цифр и не участвовали в записи). Может ли оно делиться нацело на 15?
	
	\itr Докажем, что результат проделанной в условии процедуры не мог делиться даже на 3. Действительно, всякое число сравнимо по модулю 3 со своей суммой цифр, а каждая из цифр 2, 5, 8 сравнима с двойкой.
	
	Поэтому составленное из них число будет сравнимо по модулю 3 с числом $2 \cdot 7 = 14$, которое на 3 не делится.
	
	\itC Пусть $a$ — нечётное число, большее 3. Какой цифрой (чётной или нечётной) является предпоследняя цифра числа $a^2$?
	
	\itr Число $a$ можно представить в виде $10 \cdot Y + x$, где $x$ — последняя цифра числа $a$, и потому нечётная. В свою очередь,
	$$\ll 10 \cdot Y + x\rr^2 = 100 \cdot Y^2 + 20 \cdot Yx + x^2.$$
	В этой сумме первое слагаемое не оказывает никакого влияния на предпоследнюю цифру $a^2$, а второе слагаемое не влияет на чётность предпоследней цифры числа $a^2$.
	
	Остаётся перебрать квадраты нечётных цифр — 01, 09, 25, 49, 81 — и выяснить, что их предпоследняя цифра чётна. Поэтому искомая предпоследняя цифра в данной задаче будет чётной.

\end{itemize}

\taskno{7}

\begin{itemize}

	\itA Как изменится частное, если делитель увеличить на $\tfrac{1}{5}$ его величины?
	
	\itr $\frac{x}{y + \frac{1}{5}y} = \frac{5}{6} \cdot \frac{x}{y}$.
	
	\itB В выражении $2 : 3 : 5 : 7 : 11 : 13$, расставляя по-разному скобки, можно получить разные дроби. Можно ли таким образом получить дробь
	$$\frac{2 \cdot 5 \cdot 7}{3 \cdot 11 \cdot 13}?$$
	
	\itr Отметим, что все числа, данные в задаче, являются простыми, поэтому в некотором месте получающейся дроби будет находиться данный множитель, только если это было «предусмотрено» тем, как оказались расставлены знаки деления. Иными словами, произведение или частное двух чисел из задачи не могут делиться на другое число из задачи.
	
	Теперь осталось заметить, что числа 5 и 7, стоящие рядом в исходном выражении, после приведения выражения к виду единой дроби всегда будут оказываться в разных её частях, вне зависимости от того, как между ними и вокруг них расставлены пары скобок. То есть они не могут одновременно попасть в числитель.
	
	Поэтому дробь, данную в условии, получить нельзя.
	
\end{itemize}