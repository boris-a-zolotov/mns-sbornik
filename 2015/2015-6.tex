\secklas{6}

\taskno{1}

\begin{itemize}

	\itA Мальчик Боря согнул из проволоки два квадрата. Когда мальчик Вова положил эти квадраты друг к другу, то получился прямоугольник с б\'oльшей стороной \SI{8}{\text{дм}}. Может ли Боря определить, сколько проволоки он израсходовал на квадраты?
	
	\itr Да, конечно: из того, что б\'oльшая сторона прямоугольника равна \SI{8}{\text{дм}}, легко следует, что его меньшая сторона — \SI{4}{\text{дм}}. Тогда квадраты, сложенные Борей, имели сторону \SI{4}{\text{дм}}, и потрачено им было $16+16$ $=$ \SI{32}{\text{дм}} проволоки.
	
	\def\cm#1{\SI{#1}{\text{см}}}
	\itB Коротышки из Цветочного города решили украсить цветами клумбу в форме прямоугольника. Полезная длина клумбы — \cm{240}, а полезная ширина — \cm{120}. Цветы было решено сажать в узлах квадратной сетки на расстоянии \cm{20} друг от друга. Незнайка подсчитал, что нужно
	$$(240 : 20) \times (120 : 20) = 12 \times 6 = 72$$
	кустика рассады. К сожалению, их не хватило. Сколько же необходимо было привезти кустиков рассады?
	
	\itr Незнайкины расчёты сработали бы, если бы кустики были посажены в центры квадратов $20 \times 20$, на которые можно побить клумбу. Однако коротышки сажали цветы не в центры, а в вершины таких квадратов. Вершин очевидно больше, чем центров: у каждого квадрата есть, например, нижняя левая вершина, и у различных квадратов они различны — однако после выбора всех нижних левых вершин останутся ещё какие-то — например, самая верхняя правая.
	
	Как же посчитать все вершины квадратной сетки? Вершины делятся на строчки, и строчек этих $(120 : 20) +1$ — внизу от каждого ряда квадратов, а также самая верхняя. Вершин в каждой строчке $(240 : 20) +1$ — слева от каждого квадрата, а также самая правая.
	
	Получаем ответ на задачу:
	$$\ll\ll120 : 20\rr +1\rr \times \ll\ll240 : 20\rr +1\rr = 7 \times 13 = 91.$$
	
	\itC У мальчика Димы есть инновационные ножницы и кусок нанолески длиной \cm{192}. Может ли Дима отрезать от этого куска кусок длиной \cm{90}? (Нанолеска такова, что её куски можно сгибать пополам, а также прикладывать друг к другу.)
	
	\itr $192 / 2^5 = 6$ — сложив леску пять раз пополам, мы получим 32 слоя длиной \cm{6}. Взяв 15 таких слоёв подряд, мы получим кусок длиной \cm{90}, его остаётся только отрезать ножницами.

\end{itemize}

\taskno{4}

\begin{itemize}

	\itA У доктора Ватсона в пальто 4 кармана. В каждый карман он кладёт не менее одного патрона и не больше четырёх патронов. Может ли Шерлок Холмс узнать, сколько патронов в карманах у доктора Ватсона, если ему известно, что в карманах у него разное число патронов?
	
	\itr Различных натуральных чиселот 1 до 4 — ровно 4 штуки, столько же, сколько карманов у Ватсона. Поэтому единственный вариант разложить патроны так, как указано в условии, — в первый карман один патрон, $\ldots$, в четвёртый карман четыре патрона.
	
	На этом основании Холмс может наверняка утверждать, что у Ватсона
	$$1+2+3+4=10\text{ патронов.}$$
	
	\itB На прощальной вечеринке танцуют девушка Катя и 7 юношей — Боря, Женя, Илья, Гоша, Андрей, Данил и Максим. У каждого из них есть воздушные шарики. Какое наименьшее число шариков может быть у этих весельчаков, если среди них нет двух с одинаковым числом воздушных шариков?
	
	\itr Попробуем минимизировать количество шариков у танцующих. У танцора с наименьшим количеством шариков всё-таки хотя бы один воздушный шарик. У «второго снизу» по количеству шариков их хотя бы два — и, наконец, у самого богатого на шарики их как минимум 8. Отсюда получаем оценку снизу на общее количество шариков —
	$$1+2+3+4+\ldots+8 = 36.$$
	
	\itC Однажды Винни Пух, Пятачок и ослик Иа-Иа пошли ловить рыбу. Улов оказался не очень большим. Винни Пух поймал половину общего улова без $\tfrac{2}{5}$ того, что поймали Пятачок и Иа-Иа. Пятачок поймал треть общего улова и $\tfrac{1}{5}$ того, что поймали Винни Пух и Иа-Иа. Улов  ослика Иа-Иа отличался от улова Пятачка на \SI{1}{\text{кг}}. Сколько весил весь улов?
	
	\def\vinni{\text{В}}
	\def\piggi{\text{П}}
	\def\iaiai{\text{И}}
	\itr Обозначим улов Винни Пуха через $\vinni$, улов Пятачка через $\piggi$ и улов Иа-Иа через $\iaiai$. Тогда на основании данных задачи можно составить систему уравнений:

	\vspace{-0.6cm}
	\begin{align*}
		&
		\begin{cases}
			\vinni = \frac{1}{2} \cdot \ll \vinni + \piggi + \iaiai \rr
				-\frac{2}{5} \cdot \ll \piggi + \iaiai \rr\scolon \\
			\piggi = \frac{1}{3} \cdot \ll \vinni + \piggi + \iaiai \rr
				+\frac{1}{5} \cdot \ll \vinni+\iaiai \rr\scolon \\
			\iaiai = \piggi \pm 1.
		\end{cases} \\
		&
		\begin{cases}
			-\vinni + 0.2 \piggi + 0.2 \iaiai = 0\scolon \\
			0.8 \vinni - \piggi + 0.8 \iaiai = 0\scolon \\
			\iaiai = \piggi \pm 1.
		\end{cases}
	\end{align*}
	
	Прибавим ко второй строке первую, умноженную на $0.8$:
	
	\vspace{-0.6cm}
	\begin{align*}
		&
		\begin{cases}
			-\vinni + 0.2 \piggi + 0.2 \iaiai = 0\scolon \\
			-0.84 \piggi + 0.96 \iaiai = 0\scolon \\
			\iaiai = \piggi \pm 1.
		\end{cases}
	\end{align*}
	
	Из второй строки получившейся системы видно, что $0.84\piggi = 0.69\iaiai$, то есть $\iaiai<\piggi$, и, значит, $\iaiai=\piggi-1$. Теперь всё просто:
	
	\vspace{-0.6cm}
	\begin{align*}
		& 0.84 \cdot \piggi = 0.96\cdot\ll\piggi-1\rr \\
		& 0.12 \cdot \piggi = 0.96 \\
		& \qquad \bullet\ \piggi = 8 \\
		& \qquad \bullet\ \iaiai = 7 \\
		& \qquad \bullet\ \vinni = 0.2 \cdot 8 + 0.2 \cdot 7 = 3
	\end{align*}
	
	Таким образом, весь улов весил $8+7+3=18$ килограммов.

\end{itemize}

\taskno{5}

\begin{itemize}

	\itA «Сейчас 6 часов вечера,» — сказал мальчик Вовочка. Интересно, какую часть составляет оставшаяся часть суток от прошедшей и какая часть суток осталась?
	
	\itr Осталось 6 часов, а прошло 18. Поэтому осталась четвёртая часть суток, и она составляет третью часть от трёх четвертей, которые уже прошли.
	
	\itC Можно ли в выражении
	$$\frac{A+B}{C} + \frac{D}{E+F}$$
	заменить буквы числами 1, 2, 3, 4, 5, 6 так, чтобы различным буквам соответствовали различные числа, и значение получившегося выражения было бы равно 1?
	
	\itr Приведём дроби в выражении к общему знаменателю:
	$$\frac{(A+B)(E+F) + CD}{C(E+F)}.$$
	
	То, что значение этой дроби равно 1, значит, что её числитель равен её знаменателю:

	\vspace{-0.6cm}
	\begin{align*}
		& (A+B) \underline{(E+F)} + CD = C \underline{(E+F)} \\
		& (A+B-C)(E+F) + CD = 0 \\
		& (C-A-B)(E+F) = CD
	\end{align*}
	
	Чтобы равенство было выполнено, число $C$ должно быть больше, чем сумма $A$ и $B$, и, значит, каждое из них. Вооружившись этим соображением, несложно найти ответ: $A=1$, $B=2$, $C=6$, $D=4$, $E=3$, $F=5$.
	$$\frac{1+2}{6}+\frac{4}{3+5} = 1.$$
	
\end{itemize}

\taskno{6}

\begin{itemize}

	\itA Одним пакетиком чая можно заварить 2 или 3 чашки чая. Маша и Оля разделили пачку чая пополам. Маша заварила 51 чашку чая, а Оля — 73 чашки. Можно ли догадаться, сколько пакетиков чая было в пачке?
	
	\itr $73-51 = 22$ — на столько больше раз Оля использовала пакетик на три чашки, чем Маша. Это значит, что Оля по крайней мере столько раз заваривала пакетиком только две чашки чая.
	
	$22 \cdot 2 = 44$. Значит, Оле осталось заварить 7 чашек, это можно сделать только тремя пакетиками — $7=3+2+2$. Значит, у Оли было 25 пакетиков, а в пачке — 50.
	
	\itB Незнайка составил числовой ребус
	$$\text{ЭНЭ}+\text{БЭНЭ}=\text{РАБА},$$
	в котором разным буквам соответствуют разные цифры. Знайка, посмотрев на ребус, сказал, что цифра, зашифрованная буквой Р, не может быть чётной. Прав ли Знайка?
	
	\itr Знайка неправ, цифра Р вполне может быть чётной: например, равенство
	$$636+7636=8272$$
	является решением ребуса из условия.
	
	\itC В коробочке лежат болтики, шайбочки винтики и гаечки — всего 107 штук. Известно, что болтиков в 3 раза больше, чем шайбочек, а шайбочек в 2 раза больше, чем винтиков. Кроме того, винтиков больше, чем гаечек. Сколько гаечек в коробке?
	
	\itr Заметим, что болтиков, по условию, в 6 раз больше, чем винтиков. То есть, если обозначить количество винтиков за В, а гаечек — за Г, мы получим, что
	$$(6+2+1) \cdot \text{В} + \text{Г} = 107,\qquad \text{В} > \text{Г}.$$
	
	Единственный ответ в таком случае — $\text{В} = 11$, $\text{Г} = 8$: в противном случае число гаечек будет больше, чем множитель у девятки.
	
	Ответ: 8 гаечек.

\end{itemize}