\secklas{7}

\taskno{1}

\begin{itemize}

	\def\metr#1{\SI{#1}{\text{м}}}
	\itA Заяц убегает от Волка, который находится от него на расстоянии \metr{100}. Один прыжок Зайца равен \metr{2}, а Волка — \metr{3}. Пока Волк делает 4 прыжка, Заяц делает 5 прыжков. За сколько своих прыжков Волк догонит Зайца?
	
	\itr $4 \cdot 3 = 12$, $5 \cdot 2 = 10$ — отсюда за четыре волчьих прыжка расстояние между Зайцем и Волком сокращается на \metr{2}. Значит, Волку потребуется
	$$4 \cdot \frac{100}{2} = 200\text{ прыжков.}$$
	
	\itB Вовочке известно, что 24 числа таковы, что среди их попарных произведений ровно 100 отрицательных. Может ли Вовочка определить, сколько среди этих чисел положительны, сколько отрицательных и сколько нулей?
	
	\itr Количество отрицательных попарных произведений в\linebreak наборе равно $P \cdot N$, где $P$ — количество положительных чисел в наборе, $N$ — количество отрицательных чисел в наборе.
	
	Для того, чтобы установить $P$ и $N$, нам нужно представить 100 в виде произведения двух чисел, сумма которых не превосходит 24. Способ сделать это ровно один — $P=10$, $N=10$.
	
	Ответ: 10 положительных чисел, 10 отрицательных и 4 нуля.
	
	\itC Какое наименьшее число различных цифр нужно выбрать, чтобы любое число от 1 до 100 включительно можно было представить в виде суммы выбранных цифр, в которой каждую из них разрешается использовать не более четырёх раз?
	
	\itr Докажем, что четырьмя цифрами обойтись нельзя. Среди выбранных цифр, очевидно, должна быть единица. Единицей можно «набрать» числа от 1 до 4 — поэтому второе число, которое мы берём, не должно превосходить 5.
	
	Добавив к нашему «черновику» набора самые большие цифры, 8 и 9, мы заметим, что
	$$4 \cdot (1+5+8+9) = 92 < 100,$$
	поэтому набор из четырёх цифр нам не подойдет.
	
	А вот набора из пяти цифр — 1, 5, 6, 8, 9 — вполне хватит.

\end{itemize}

\taskno{3}

\begin{itemize}

	\itA Какой длины получится полоса, если 1 кубический километр разрезать на кубические метры и выложить их в длину?
	
	\itr $\SI{1}{\text{км}^3} = \SI{1\,000\,000\,000}{\text{м}^3}$ — поэтому получившаяся полоса будет иметь длину
	$$\SI{1\,000\,000\,000}{\text{м}} = 1\,000\,000\,\text{км}.$$
	
	\itB Возьмём отрезок $[0,1]$. Отрежем от него четверть слева, потом четверть от оставшейся части справа, потом четверть от оставшейся части слева и т.\,д. Какая точка отрезка точно не будет отрезана?
	
	\itr На первом шаге мы отрезали слева кусок длиной $\tfrac{1}{4}$. На втором шаге, справа, — кусок длиной $\tfrac{1}{4} \cdot \tfrac{3}{4}$. На третьем шаге, снова слева, — кусок длины $\tfrac{1}{4} \cdot (\tfrac{3}{4})^2$. Таким образом, на $2k-1$--м шаге мы будем отрезать с левой стороны оставшегося отрезка кусок длиной
	$$\frac{1}{4} \cdot \ll\ll\frac{3}{4}\rr^2\rr^{k-1}.$$
	
	\def\znamen{\ll\frac{9}{16}\rr}
	Так можно посчитать длину всего, что будет отрезано слева:
	$$\frac{1}{4} \cdot \ll 1 + \frac{9}{16} + \znamen^2 + \znamen^3 + \ldots \rr.$$
	
	Чтобы узнать, чему равна эта сумма, вспомним, как считается сумма геометрической прогрессии:
	$$1 + q + q^2 + \ldots + q^n = \frac{1-q^{n+1}}{1-q}.$$
	
	При $q<1$ число $q^{n+1}$ стремится к нулю при больших $n$, поэтому сумма всего ряда $1 + q + q^2 + \ldots$ будет равна
	$$\frac{1}{1-q}.$$
	
	Применив это соображение к сумме, которую мы считаем, получим
	$$\frac{1}{4} \cdot \frac{1}{1 - \frac{9}{16}} = \frac{4}{7}.$$
	
	На $2k$--м шаге мы отрезаем с правой стороны кусок длиной
	$$\frac{3}{4} \cdot \frac{1}{4} \cdot \ll\ll\frac{3}{4}\rr^2\rr^{k-1}.$$
	
	Поэтому всего с правой стороны будет отрезано
	$$\frac{3}{4} \cdot \frac{4}{7} = \frac{3}{7}.$$
	
	Таким образом, нетронутой будет оставаться единственная точка~— $\tfrac{4}{7}.$
	
	\itC На плоскости расположено несколько отрезков. Мальчик Вова отмечает несколько точек пересечения этих отрезков (не обязательно все), считает для каждой отмеченной точки число отрезков, содержащих эту точку, и находит сумму этих чисел. Мальчик Дима считает для каждого из отрезков число отмеченных Вовой точек, лежащих на нём, и находит сумму своих чисел. У кого из мальчиков сумма больше?
	
	\itr Рассмотрим все пары вида ({\itshape отрезок}, {\itshape отмеченная точка на этом отрезке}). Вова посчитал в точности все такие пары, так как для каждой отмеченной точки им были учтены все отрезки, которым принадлежит эта точка. Дима тоже посчитал в точности все такие пары, потому что для каждого отрезка учёл все точки, образующие подобные пары с этим отрезком.
	
	Поэтому числа у мальчиков получились одинаковые.

\end{itemize}

\taskno{5}

\begin{itemize}

	\itA Красная Шапочка имела 2 заряженных револьвера. Убегая от Волка, она дважды в него попала, трижды промазала, и один раз случилась осечка. У Красной Шапочки осталось 7 патронов. Сколькими патронами заряжается револьвер?
	
	\itr Посчитаем, сколько патронов было у Красной шапочки изначально:
	$$7\text{ осталось} + 2\text{ в цель} + 3\text{ мимо} = 12.$$
	
	При осечке патрон не вылетает из пистолета, поэтому технически остаётся у хозяина. Так как у Красной Шапочки два револьвера, то в каждом из них — по 6 патронов.
	
	\itB В каждую клетку таблицы $4 \times 4$ вписано число 0 или 1 так, что в клетках любого квадрата $2 \times 2$ стоит ровно 3 одинаковых числа. Какие значения может принимать сумма чисел в такой таблице?
	
	\itr Таблица $4 \times 4$ естественным образом делится на 4 непересекающихся квадрата $2 \times 2$. Сумма чисел внутри каждого из них нечётна (так как там либо три единицы и ноль, либо три нуля и единица). Значит, сумма чисел, поставленных в таблицу вообще, должна быть чётной.
	
	Также заметим, что сумма чисел в таблице не меньше 4 и не превосходит 12 (опять же, посмотрим на суммы чисел в четырёх квадратах $2 \times 2$). Значит, остались только варианты 4, 6, 8, 10, 12. Приведём примеры, когда достигается каждый из них:
	
\begin{center} \begin{tabular}{rccrc}
	4:\qquad &
	\begin{tabular}{c|c|c|c}
		1 & 0 & 1 & 0 \\ \hline
		0 & 0 & 0 & 0 \\ \hline
		1 & 0 & 1 & 0 \\ \hline
		0 & 0 & 0 & 0
	\end{tabular} & \qquad &
	6:\qquad &
	\begin{tabular}{c|c|c|c}
		1 & 0 & 1 & 1 \\ \hline
		0 & 0 & 0 & 1 \\ \hline
		0 & 1 & 0 & 0 \\ \hline
		0 & 0 & 0 & 1 
	\end{tabular} \\ \\
	8:\qquad &
	\begin{tabular}{c|c|c|c}
		0 & 1 & 1 & 1 \\ \hline
		0 & 0 & 1 & 0 \\ \hline
		1 & 0 & 0 & 0 \\ \hline
		1 & 1 & 0 & 1
	\end{tabular} & \qquad &
	10:\qquad &
	\begin{tabular}{c|c|c|c}
		1 & 0 & 1 & 1 \\ \hline
		0 & 0 & 0 & 1 \\ \hline
		1 & 0 & 1 & 1 \\ \hline
		1 & 1 & 1 & 0 
	\end{tabular} \\ \\
\end{tabular}

\begin{tabular}{rc}
	12:\qquad &
	\begin{tabular}{c|c|c|c}
		 1 & 0 & 1 & 0 \\ \hline
		 1 & 1 & 1 & 1 \\ \hline
		 1 & 0 & 1 & 0 \\ \hline
		 1 & 1 & 1 & 1		 
	\end{tabular}
\end{tabular}

\end{center}

	\itC Квадрат $4 \times 4$ состоит из 16 клеток. Какое наименьшее число сторон клеток в этой таблице надо отметить, чтобы у каждой клетки было не менее двух отмеченных сторон?
	
	\itr Отмеченная сторона принадлежит максимум двум клеткам. У каждой клетки должно быть минимум по 2 отмеченных стороны  — поэтому отметок нам нужно сделать не менее чем
	$$16 \cdot 2\ :\ 2 = 16.$$
	
	Приведём пример того, как можно сделать 16 отметок:
	\begin{center} \tikz{
		\foreach \x in {0,...,4} {
			\draw[thick] (0.7 * \x cm, 0) -- (0.7 * \x cm, 2.8);
			\draw[thick] (0, 0.7 * \x cm) -- (2.8, 0.7 * \x cm);
		};
		\foreach \x in {1,2,3} {
			\draw (0.7 * \x cm, 2.45) node{$\times$};
			\draw (0.7 * \x cm, 0.35) node{$\times$};
			\draw (2.45, 0.7 * \x cm) node{$\times$};
			\draw (0.35, 0.7 * \x cm) node{$\times$};
		};
		\begin{scope}[xshift=1.4cm,yshift=1.4cm]
			\foreach \x in {0,...,4} \draw[rotate=90*\x] (0.35,0) node{$\times$};
		\end{scope}
	} \end{center}
		
\end{itemize}