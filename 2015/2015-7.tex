\documentclass[10pt]{scrbook} \usepackage{modules/nonstahp_book}
\usepackage{mathspec}

\setmainfont[
	Path = f/,
	BoldFont=pb.ttf,
	ItalicFont=pi.ttf,
	BoldItalicFont=pbi.ttf
		]{p.ttf}
\setsansfont[
	Path = f/,
	BoldFont=pb.ttf,
	ItalicFont=pi.ttf,
	BoldItalicFont=pbi.ttf
		]{p.ttf}
		
\setmathfont(Digits)[Path = f/]{p.ttf}
\setmathfont(Latin)[Path = f/]{pi.ttf}
\setmathfont(Greek)[Path = f/, Uppercase]{p.ttf}
\setmathfont(Greek)[Path = f/, Lowercase]{pi.ttf}

\setmonofont[Path = f/]{pmono.ttf}

%\setCJKmainfont[
%	Path=f/,
%	BoldFont=notoserifb.ttf,
%	ItalicFont=notoserifi.ttf,
%	BoldItalicFont=notoserifbi.ttf
%		]{notoserif.ttf}

 \begin{document}
\renewcommand{\theyear}{2015}

\secklas{7}

\taskno{1}

\begin{itemize}

	\def\metr#1{\SI{#1}{\text{м}}}
	\itA Заяц убегает от Волка, который находится от него на расстоянии \metr{100}. Один прыжок Зайца равен \metr{2}, а Волка — \metr{3}. Пока Волк делает 4 прыжка, Заяц делает 5 прыжков. За сколько своих прыжков Волк догонит Зайца?
	
	\itr $4 \cdot 3 = 12$, $5 \cdot 2 = 10$ — отсюда за четыре волчьих прыжка расстояние между Зайцем и Волком сокращается на \metr{2}. Значит, Волку потребуется
	$$4 \cdot \frac{100}{2} = 200\text{ прыжков.}$$
	
	\itB Вовочке известно, что 24 числа таковы, что среди их попаврных произведений ровно 100 отрицательных. Может ли Вовочка определить, сколько среди этих чисел положительны, сколько отрицательных и сколько нулей?
	
	\itr Количество отрицательных попарных произведений равно $P \cdot N$, где $P$ — количество положительных чисел в наборе, $N$ — количество отрицательных чисел в наборе.
	
	Для того, чтобы установить $P$ и $N$, нам нужно представить 100 в виде произведения двух чисел, сумма которых не превосходит 24. Способ сделать это ровно один — $P=10$, $N=10$.
	
	Ответ: 10 положительных чисел, 10 отрицательных и 4 нуля.
	
	\itC Какое наименьшее число различных цифр нужно выбрать, чтобы любое число от 1 до 100 включительно можно было представить в виде суммы выбранных цифр, в которой каждую из них разрешается использовать не более четырёх раз?
	
	\itr Докажем, что четырьмя цифрами обойтись нельзя. Среди выбранных цифр, очевидно, должна быть единица. Единицей можно «набрать» числа от 1 до 4 — поэтому второе число, которое мы берём, не должно превосходить 5.
	
	Добавив к нашему «черновику» набора самые большие цифры, 8 и 9, мы заметим, что
	$$4 \cdot (1+5+8+9) = 92 < 100,$$
	поэтому набор из четырёх цифр нам не подойдет.
	
	А вот набора из пяти цифр — 1, 5, 6, 8, 9 — вполне хватит.

\end{itemize}

\taskno{3}

\begin{itemize}

	\itA Какой длины получится полоса, если 1 кубический километр разрезать на кубические метры и выложить их в длину?
	
	\itr $\SI{1}{\text{км}^3} = \SI{1\,000\,000\,000}{\text{м}^3}$ — поэтому полоса получится длиной
	$$\SI{1\,000\,000\,000}{\text{м}} = 1\,000\,000\,\text{км}.$$
	
	\itB Возьмём отрезок $[0,1]$. Отрежем от него четверть слева, потом четверть от оставшейся части справа, потом четверть от оставшейся части слева и т.\,д. Какая точка отрезка точно не будет отрезана?
	
	\itr На первом шаге мы отрезали слева кусок длиной $\tfrac{1}{4}$. На втором шаге, справа, — кусок длиной $\tfrac{1}{4} \cdot \tfrac{3}{4}$. На третьем шаге, снова слева, — кусок длины $\tfrac{1}{4} \cdot (\tfrac{3}{4})^2$. Таким образом, на $2k-1$--м шаге мы будем отрезать с левой стороны оставшегося отрезка кусок длиной
	$$\frac{1}{4} \cdot \ll\ll\frac{3}{4}\rr^2\rr^{k-1}.$$
	
	\def\znamen{\ll\frac{9}{16}\rr}
	Так можно посчитать длину всего, что будет отрезано слева:
	$$\frac{1}{4} \cdot \ll 1 + \frac{9}{16} + \znamen^2 + \znamen^3 + \ldots \rr =$$
	$$= \frac{1}{4} \cdot \frac{1}{1 - \frac{9}{16}} = \frac{4}{7}.$$
	
	На $2k$--м шаге мы отрезаем с правой стороны кусок длиной
	$$\frac{3}{4} \cdot \frac{1}{4} \cdot \ll\ll\frac{3}{4}\rr^2\rr^{k-1}.$$
	
	Поэтому всего с правой стороны будет отрезано
	$$\frac{3}{4} \cdot \frac{4}{7} = \frac{3}{7}.$$
	
	Таким образом, нетронутой будет оставаться единственная точка~— $\tfrac{4}{7}.$

\end{itemize}



























\end{document}