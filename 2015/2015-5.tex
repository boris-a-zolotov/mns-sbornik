\secklas{5}

\taskno{1}

\begin{itemize}

	\itA Вовочка согнул из куска проволоки квадрат со стороной 9 сантиметров. Затем он разогнул проволоку и согнул из неё равносторонний треугольник. Какова длина стороны этого треугольника?

	\itr $9 \times 4 \div 3 = \SI{12}{\text{см}}$.

	\itB Мальчик Дима в течение 2 часов надувает шары. Каждые три минуты он надувает 8 шаров, а каждый десятый шар у него лопается. Сколько шаров будет у Димы?
	
	\itr В двух часах 120 минут, отсюда Дима надует $\tfrac{120}{3} \cdot 8 = 320$ шариков. В результате того, что некоторые шарики лопаются, останется $320 - 320 \div 10$ $=$ $320 - 32$ $=$ $288$ шариков.
	
	\itC Мальчики Миша, Никита и Олег делят конфеты. Сначала Миша взял себе 20\% всех конфет и ещё 12 конфет. Затем Никита взял 25\% ос- тавшихся конфет и ещё 15 конфет. Наконец, Олег взял 30\% оставшихся конфет и ещё 21 конфету. И конфеты закончились. Кто из мальчиков взял больше конфет?
	
	\itr То, сколько конфет взял каждый из млальчиков, легко установить «обратным ходом».
	
	После первой части хода Олега оставалась 21 конфета, и это составляло 70\% от количества конфет, которое было до хода Олега. Значит, до хода Олега на столе лежало $\tfrac{21}{0.7} = 30$ конфет — и все их взял Олег.
	
	После первой части хода Никиты оставалось $30+15=45$ конфет, которые составляли 75\% от всех имеющихся конфет. Таким образом, перед ходом Никиты на столе было $\tfrac{45}{0.75}=60$ конфет, 30 из которых взял Олег, а, соответственно, 30 — Никита.
	
	Наконец, после первой части хода Миши на столе оставалось 72 \linebreak конфеты, которые составляли 80\% от конфет, имевшихся в наличии до начала дележа. Значит, в начале было $\tfrac{72}{0.8} = 90$ конфет, по 30 из которых взяли Олег и Никита — соответственно, Мише также досталось 30 конфет.
	
	Таким образом, мальчики взяли поровну конфет.

\end{itemize}

\taskno{2}

\begin{itemize}

	\itB Три числа $A$, $B$ и $C$ связаны соотношениями:
	$$A+B = 12.3\scolon\qquad
		B+C=18.9\scolon\qquad
		A+C=6.1.$$

	Не находя эти числа, укажите самое большое среди них. Результат обоснуйте.
	
	\itr $B>C$, так как $A+B$ больше, чем $A+C$. Также $B>A$, так как $B+C>A+C$. Отсюда $B$ — самое большое из чисел.
	
	\itC Тренер расставил спортсменов на прямой дорожке. По сигналу тренера спортсмены бегут к одному из них, на которого указывает тренер, а затем возвращаются на свои места. Какой из спортсменов пробежит наибольшее расстояние после нескольких таких стартов?
	
	\itr Отметим на прямой спортсмена, на которого в какой-то момент указал тренер, и изобразим зависимость расстояния, которое должны пробежать другие спортсмены, от их положения на прямой.
	
	\begin{center} \tikz{
	\begin{scope}[xscale=0.8,yscale=0.4]
		\draw[pattern=vertical lines, pattern color=gray, thick]
			(-2.25,0) -- (-2.25,2.85) -- (0.6,0) -- (2.25,1.65) -- (2.25,0) -- cycle;
		\draw[very thick] (-2.5,0) -- (0.6,0) node[circle,fill=black,inner sep=0.4mm]{}
			-- (2.5,0);
	\end{scope}
	} \end{center}
	
	Например, после выбора тренером трёх спортсменов зависимость расстояния, которое пробегут другие спортсмены за три старта, от их положения на прямой, будет выглядеть так:
	
	\begin{center} \tikz{
	\begin{scope}[xscale=0.8,yscale=0.35]
		\draw[pattern=vertical lines, pattern color=gray, thick]
			(-2.25,0) -- (-2.25, 2.25 + 0.6 + 2.25 - 0.1 + 2.25 - 1.3)
			-- (-1.3, 1.3+0.6+1.3-0.1)
			-- (-0.1, 1.3-0.1+0.6+0.1)
			-- (0.6, 0.6+1.3+0.6+0.1)
			-- (2.25, 2.25-0.6+2.25+0.1+2.25+1.3)
			-- (2.25,0) -- cycle;
		\foreach \x / \y in {-1.3 / 1.3+0.6+1.3-0.1,
			-0.1 / 1.3-0.1+0.6+0.1,
			0.6 / 0.6+1.3+0.6+0.1
		} {
			\draw[thick,dashed] (\x,0) -- (\x,\y);
		};
		\draw[very thick] (-2.5,0)
			-- (-1.3,0) node[circle,fill=black,inner sep=0.4mm]{}
			-- (-0.1,0) node[circle,fill=black,inner sep=0.4mm]{}
			-- (0.6,0) node[circle,fill=black,inner sep=0.4mm]{}
			-- (2.5,0);
	\end{scope}
	} \end{center}
	
	В любом случае, после нескольких стартов зависимость преодолённого расстояния от положения на дорожке будет выглядеть как сумма нескольких функций, изображённых на первом рисунке в этом пункте. Всякая такая функция имеет два локальных максимума на краях дорожки — то есть достаточно близко к каждому из краёв дорожки {\bfseries все} функции, появляющиеся в процессе стартов, будут возрастать при движении к этому краю.
	
	Поэтому наибольшее расстояние будет пройдено одним из крайних спортсменов — чтобы понять, каким именно, в каждом случае нужно смотреть на них индивидуально.

\end{itemize}

\taskno{4}

\begin{itemize}

	\itA На ёлке 2015 шаров. На один синий шар приходится 4 красных. На сколько процентов синих шаров на ёлке меньше, чем красных?
	
	\itr По определению,
	
	\begin{quote}
		Величина $A$ на $k$ процентов меньше величины $B$, если
		$$A = \frac{100-k}{100} \cdot B.$$
	\end{quote}
	
	Красных шаров на ёлке в 4 раза больше, чем синих — иными словами, синих шаров в 4 раза или на 75\% меньше, чем красных.

	\itB Добрыня Никитич раз мечом направо махнёт — 3 врага упадёт, раз мечом налево махнёт — 2 врага упадёт. Рыбится богатырь — раз налево, два раза направо. За сколько взмахов богатырь разобьёт вражье войско, состоящее из 564 человек? А если рубится богатырь — раз направо, два раза налево?
	
	\itr За один «период» при первом способе борьбы Добрыня убивает 8 врагов, а при втором способе — 7.
	
	\subitem $564 : 8 = 70\ (\text{остаток\ } 4)$. В таком случае Добрыне понадобится $70 \cdot 3 + 2$ взмаха: после 70 «периодов» останутся 4 врага, которых можно будет убить за два дополнительных взмаха.
	
	\subitem $564 : 7 = 80\ (\text{остаток\ } 4)$. Добрыне понадобится $80 \cdot 3 + 2$ взма- ха~— после 80 «периодов», опять же, останется 4 врага, которых \linebreak можно убить за два взмаха.
	
	\itC Вовочке на дом задали разделить некоторое число на 2, 3 и 6. Папа, проверяя домашнее задание, услышал от Вовочки следующее: «Я забыл, какое число задали, поэтому делил другое число, которое сам придумал, — и два раза разделилось без остатка, а один раз получился остаток». Папа уверен, что Вовочка допустил ошибку. Как он об этом догадался?
	
	\itr Если у Вовочки получилось разделить придуманное им число нацело на 6, то он точно неправ: если число нацело делится на 6, то оно делится на 2 и на 3, и остатка получиться не могло.
	
	Если же Вовочке удались деления на 2 и на 3, то придуманное им число автоматически делится на 6. То есть, если у Вовочки получились два деления, то и третье тоже должно было получиться.

\end{itemize}

\taskno{5}

\begin{itemize}

	\itA Боря утверждает, что он может нарисовать 6 точек на двух прямых, три на одной и 4 на другой. Может ли такое быть?
	
	\itr Да, конечно может:
	
	\begin{center} \tikz{
	\begin{scope}[rotate=-59]
		\dwt
			(-1.2,0)
			-- (-0.8,0) node[circle,fill=black,inner sep=0.8mm]{}
			-- (0,0) node[circle,fill=black,inner sep=0.8mm]{}
			-- (0.8,0) node[circle,fill=black,inner sep=0.8mm]{}
			-- (1.2,0);
		\dwt
			(0,-2)
			-- (0,-1.6) node[circle,fill=black,inner sep=0.8mm]{}
			-- (0,-0.8) node[circle,fill=black,inner sep=0.8mm]{}
			-- (0,0.8) node[circle,fill=black,inner sep=0.8mm]{}
			-- (0,1.2);
	\end{scope}
	} \end{center}
	
	\def\cm#1{\SI{#1}{\text{см}}}
	\itB Сколькими способами можно разрезать шнурок от ботинка длиной \cm{36} на кусочки длиной \cm{3} и \cm{5}?
	
	\itr Выпишем сначала все способы представить {\itshape число} 36 в виде суммы нескольких чисел 3 и 5. Их не так много:
	
	\begin{tabular}{ll}
	\qquad & $36 = 3 \cdot 12$ \\
		& $36 = 3 \cdot 7 + 5 \cdot 3$ \\
		& $36 = 3 \cdot 2 + 5 \cdot 6$
	\end{tabular}
	
	Способ разрезать шнурок на 12 одинаковых кусков ровно один. Теперь рассмотрим второе представление: нам предстоит разрезать шнурок на 10 кусков, 3 из которых имеют длину 5. В каждом разрезании куски упорядочены от левого конца шнурка к правому, поэтому каждое разрезание определяется тем, под какими номерами из имеющихся десяти идут куски длины 5.
	
	Выбрать три номера из десяти можно $C_{10}^3$ способами — таково определение биномиального коэффициента. Заметим только, что симметричные способы разрезать шнурок посчитаны нами независимо, хотя отличаются поворотом шнурка. Поэтому биномиальный коэффициент нужно разделить пополам.
	
	Со случаем, когда 6 кусков имеют длину 5, поступим аналогично. Получаем ответ:
	$$1 + \frac{1}{2} \cdot C_{10}^3 + \frac{1}{2} \cdot C_{10}^6.$$
	
	\itC У мальчика Лёвы есть волшебная линейка длиной 9 сантиметров. Может ли он нанести на эту линейку три промежуточных деления так, чтобы любой отрезок длиной от 1 до 9 сантиметров можно было измерить с точностью до сантиметра?
	
	\itr Поставим дополнительные деления на расстоянии 2, 5 и 8 сантиметров от левого края линейки. Любой отрезок натуральной длины может быть получен как отрезок между этими делениями и краями линейки — покажем, как это сделать:

		\def\distabove#1{node[above,rectangle,fill=white]{\scriptsize \SI{#1}{\text{см}}}}
	\def\otrmark{node[fill=black,rectangle,xscale=0.225,inner sep=1mm]{}}
	
	\begin{center} \tikz{
	\begin{scope}[scale=0.78]
		\foreach \x in {0,2,5,8,9} {
			\draw[thick,dashed,color=gray] (\x,0.6) -- (\x,-4.6); }
		\dwt
			(0,0.2) \otrmark
				-- (1,0.2) \distabove{2}
			-- (2,0.2) \otrmark
				-- (3.5,0.2) \distabove{3}
			-- (5,0.2) \otrmark
				-- (6.5,0.2) \distabove{3}
			-- (8,0.2) \otrmark
				-- (8.5,0.2) \distabove{1}
			-- (9,0.2) \otrmark; 
		\dwt
			(8,-0.6) \otrmark
				-- (8.5,-0.6) \distabove{1}
			-- (9,-0.6) \otrmark; 
		\dwt
			(0,-0.6) \otrmark
				-- (1,-0.6) \distabove{2}
			-- (2,-0.6) \otrmark; 
		\dwt
			(2,-1.2) \otrmark
				-- (3.5,-1.2) \distabove{3}
			-- (5,-1.2) \otrmark
				-- (7,-1.2) \distabove{4}
			-- (9,-1.2) \otrmark; 
		\dwt
			(0,-1.8) \otrmark
				-- (2.5,-1.8) \distabove{5}
			-- (5,-1.8) \otrmark; 
		\dwt
			(2,-2.4) \otrmark
				-- (5,-2.4) \distabove{6}
			-- (8,-2.4) \otrmark; 
		\dwt
			(2,-3) \otrmark
				-- (5.5,-3) \distabove{7}
			-- (9,-3) \otrmark; 
		\dwt
			(0,-3.6) \otrmark
				-- (4,-3.6) \distabove{8}
			-- (8,-3.6) \otrmark; 
		\dwt
			(0,-4.2) \otrmark
				-- (4.5,-4.2) \distabove{9}
			-- (9,-4.2) \otrmark; 
	\end{scope}
	} \end{center}

\end{itemize}

\taskno{6}

\begin{itemize}

	\itA Пять хамелеонов съедают всех мух с пяти кустов за пять минут. На сколько надо увеличить число хамелеонов, чтобы они съели всех мух с 50 кустов за 50 минут?
	
	\itr Количество хамелеонов не нужно увеличивать, потому что в их изначальном составе за 10 «подходов» по 5 минут хамелеоны объедят все 50 кустов.
	
	\itB В школе 350 учеников и 175 парт. Ровно половина девочек сидит за одной партой с мальчиками. Можно ли пересадить учеников так, чтобы ровно половина мальчиков сидела за одной партой с девочками?
	
	\itr Если ровно половина девочек сидит с мальчиками, то другая половина девочек занимает некоторое количество парт полностью. То есть, половина всех девочек — это чётное число, а отсюда количество девочек делится на 4.
	
	Если бы мы хотели, чтобы ровно половина мальчиков сидела за партой с девочками, то по аналогичным причинам количество мальчиков должно было бы делиться на 4. Однако одновременно делиться на 4 количества девочек и мальчиков не могут, так как тогда оказалось бы, что 350 делится на 4, а это неверно.
	
	\itC В классе учится не менее 12 девочек и не более 16 мальчиков. У каждого из них в классе одинаковое число друзей, среди которых обязательно есть девочка и мальчик. Известно также, что у каждой девочки друзей среди мальчиков больше, чем среди девочек, а у каждого мальчика друзей среди девочек не больше, чем среди мальчиков. Какое наименьшее число друзей может быть у школьников в этом классе?
	
	\def\grl{\text{Д}} \def\mal{\text{М}} \def\frnd{\mathcal F}
	Обозначим наименьшее число друзей у школьников через $D$ и докажем, что $D=6$. Мы знаем, что в классе $\grl$ девочек и $\mal$ мальчиков, при этом
	$$12 \le \grl\scolon\qquad \mal \le 16.$$
	
	\begin{enumerate}[label=\arabic*)]
		\item $D>2$. Действительно, если бы у школьников было по 2 друга, то девочки не могли бы иметь друзей–девочек, что противоречит условию.
		
		\item $D>3$. Если у каждого школьника было бы по 3 друга, то каждая девочка могла бы иметь не менее двух друзей–мальчиков, а каждый мальчик — не более одной подруги–девочки. Обозначив за $\frnd$ количество дружеских связей «мальчик–девочка», получим
		\vspace{-0.3cm}
		$$2\cdot 12 \le 2\cdot\grl \le \frnd \le \mal \le 16
			\text{ — противоречие.}$$
		
		\item $D>4$. Если у каждого школьника по 4 друга, то каждая де-\linebreak вочка имеет не менее трёх друзей–мальчиков, а каждый мальчик~— не более двух друзей–девочек. Получаем, что
		\vspace{-0.3cm}
		$$3 \cdot 12 \le 3\cdot\grl \le \frnd \le 2\cdot\mal \le 2 \cdot 16
			\text{ — противоречие.}$$
		
		\item $D>5$ — аналогично предыдущему пункту.
		
		\item $D=6$. Приведём пример дружеских связей между 16 мальчиками и 12 девочками, удовлетворяющих условию. Разделим мальчиков на 4 группы по 4 человека, а девочек — на 4 группы по 3 человека\scolon разобьём эти группы на пары. Пусть в каждой группе все девочки знакомы друг с другом, все мальчики знакомы друг с другом, а также каждая девочка знает всех мальчиков из группы, парной её группе.
		
		Тогда каждый мальчик знаком с 3 девочками и 3 мальчиками, а каждая девочка — с 2 девочками и 4 мальчиками. Эта ситуация подходит под условие задачи.
	\end{enumerate}

\end{itemize}