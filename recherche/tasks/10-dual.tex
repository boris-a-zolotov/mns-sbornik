\task{Двойственность}

Как известно, линейная функция задаётся уравнением $y = kx + b$, а графиком линейной функции является прямая. Каждая такая прямая определяется парой чисел $(k,b)$. Построим новую координатную плоскость $(k,b)$, {\itshape точки} на которой обозначают прямые на исходной координатной плоскости $(x,y)$. 

\begin{itemize}
\item Нарисуйте на плоскости $(x,y)$ какие-нибудь четыре прямые и отметьте их в виде четырёх точек на плоскости $(k,b)$. 
\item Обратно: выберите три точки на плоскости $(k,b)$ и нарисуйте соответствующие три прямые на плоскости $(x,y)$. Что будет, если выбирать точки на $(k,b)$ лежащими на одной вертикальной прямой? А горизонтальной?
\item Рассмотрим на плоскости $(k,b)$ прямую $b=k$. Каждая точка этой прямой задаёт на плоскости $(x,y)$ какую-то прямую, а вся прямая $b=k$ задаёт на плоскости $(x,y)$ набор прямых. Каким свойством обладает этот набор прямых?
\item На координатной плоскости $(k,b)$ проведено три прямые, проходящие через одну точку. Каждая такая прямая изображает некоторый набор прямых на плоскости $(x,y)$. Как эти три набора прямых связаны между собой?
\item Аналогичный вопрос для трёх параллельных прямых на $(k,b)$.
\item Рассмотрим набор всех прямых плоскости $(x,y)$, которые проходят через точку $(0,0)$. Как этот набор изображается на плоскости $(k,b)$? Тот же самый вопрос при замене точки $(0,0)$ на $(m,n)$.
\item Рассмотрим на плоскости $(k,b)$ прямую $b = uk + v$. Какой набор прямых на плоскости $(x,y)$ изображает эта прямая? 
\item Прямые $b = uk + v$ на плоскости $(k,b)$ задают точки на новой плоскости $(u,v)$. Как новая плоскость связана с плоскостью $(x,y)$?
\end{itemize}