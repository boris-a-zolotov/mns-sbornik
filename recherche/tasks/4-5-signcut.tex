\task{Закопеременные представления}

\begin{itemize}
\item Представьте число $1$ в виде произведения нескольких чисел, сумма которых равна нулю.
\item Решите ту же самую задачу для чисел $2,4,6$.
\item Решите эту задачу для числа $3$. Сможете ли вы найти разложение, в котором все числа являются именно целыми, а не рациональными?
\item Исследуйте вопрос представимости для произвольных натуральных чисел.
\end{itemize}

\task{Как же я люблю разрезать!}

\begin{itemize}
\item Возможно ли разрезать на равнобедренные треугольники: а) квадрат\scolon б) прямоугольник? Если да, то покажите как.
\item Ответьте на тот же вопрос, если нужно разрезать а) параллелограмм\scolon б) равнобокую трапецию. Если можно, то покажите как.
\item Попытайтесь разрезать фигуры на наименьшее возможное число равнобедренных треугольников.
\item Попробуйте изменить форму фигур из списка выше. Как тогда изменится Ваше разбиение на треугольники? Рассмотрите экстремальные ситуации.
\end{itemize}