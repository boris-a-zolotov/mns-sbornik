\task{Задача Иосифа Флавия}

В книге <<Иудейская война>> Иосифа Флавия есть история о том, как он в составе отряда из 41 иудейского воина был загнан римлянами в пещеру. Предпочитая самоубийство плену, воины решили выстроиться в круг и последовательно убивать каждого третьего из живых, до тех пор пока не останется ни одного человека. Однако Иосиф наряду с одним из своих единомышленников счел подобный конец бессмысленным --- он быстро вычислил спасительные места в порочном круге, на которые поставил себя и своего товарища. И лишь поэтому мы знаем его историю.

В нашем варианте мы начнём с того, что выстроим в круг $n$ человек, пронумерованных числами от $1$ до $n$, и будем исключать каждого {\itshape второго} из оставшихся до тех пор, пока не уцелеет только один человек. Например, если $n = 10$, то порядок исключения будет такой: 2, 4, 6, 8, 10, 3, 7, 1, 9, так что остаётся номер 5. Убедитесь в этом!
\begin{itemize}
\item Обозначим через $J(n)$ номер последнего уцелевшего человека. Мы только что выяснили, что $J(10) = 5$. Можно было предположить, что $J(n) = n/2$ при чётном $n$. Верно ли это? Начните с $n = 2,3,4,\ldots$.
\item Выпишите табличку, в которой для малых $n$ указаны порядки исключения чисел. Правда ли, что $J(n)$ всегда нечётно? Почему?
\item Пусть $n$ чётно. Выясните, что происходит в тот момент, когда из круга исключается последнее чётное число. Как связаны числа
$$J(n), \ \ J(n/2)\text{?}$$
\item Найдите аналогичную закономерность для нечётного $n$.
\item Пусть $n = 2^m$. Найдите $J(n)$.
\item Выпишите таблицу значений $J(n)$ для $n$ от 1 до 16 и найдите для $J(n)$ явную формулу.
\end{itemize}