\task{Треугольник Паскаля}

Этот исследовательский проект состоит из нескольких шагов. Если у вас не получается до конца разобраться в каком-то шаге, пропустите его и изучайте следующий.

\begin{itemize}
\item Найдите сумму чисел в каждой строке треугольника Паскаля.
\item Найдите сумму первого, третьего, пятого, $\ldots$ (и так далее) элемента в какой-нибудь строке треугольника Паскаля. Найдите закономерность.
\item Найдите сумму второго, четвёртого, шестого, $\ldots$ (и так далее) элемента в строке треугольника Паскаля. Найдите закономерность.
\item Найдите сумму квадратов чисел в каждой строке треугольника Паскаля.
\item Выясните, где в треугольнике Паскаля спрятались треугольные числа $T_n = 1+2+\ldots + n$. 
\item А где спрятались пирамидальные числа $P_n = T_1 + T_2 + \ldots + T_n$? А где квадратные числа $S_n = n^2$?
\item Выясните, как связаны числа $1, 11, 11^2, 11^3, 11^4, \ldots$ с треугольником Паскаля.
\item Что будет если взять самый левый элемент в строке треугольника Паскаля, вычесть из него следующий (по горизонтали), прибавить следующий за ним, затем вычесть следующий, $\ldots$ (и т.д.) до тех пор, пока не кончатся числа в строке? Какое число получится? Найдите закономерность.
\item Ответьте на предыдущий вопрос, если брать не сами элементы треугольника Паскаля, а их квадраты.
\item Выясните, делятся ли все элементы (кроме крайних двух единиц) в строке на номер этой строки (нумерация строк начинается с нуля)? А когда делятся? 
\item Выберите число 1 у края треугольника Паскаля и идите по диагонали вниз. Начните складывать все встречающиеся числа и в какой-нибудь момент остановитесь. Какое число получилось в сумме? А что получится в общем случае?
\item Что получится, если заштриховать все четные числа в треугольнике? Какой будет узор?
\item А если зашриховать все числа, делящиеся на 3?
\item Где спрятались числа Фибоначчи в треугольнике Паскаля?
\item Чему равна сумма чисел в каком-нибудь параллелограмме треу-\linebreak гольника Паскаля? Найдите закономерность.
\item Откройте свои собственные закономерности в треугольнике и назовите их в свою честь
\end{itemize}