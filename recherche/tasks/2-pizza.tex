\task{Задача о разрезании пиццы}

Три разреза через центр круглой пиццы дают шесть кусочков. Если же сделать третий разрез не через центр, то мы получим семь кусочков разной формы и размера. Немного поэкспериментировав, вы убедитесь в том, что наибольшее число кусков, которое вы можете получить с помощью трех разрезов --- это семь. Но сколько кусочков вы можете получить с помощью большего числа разрезаний?

\begin{center} \tikz{
	\begin{scope}[xshift=-2.2cm]
	\filldraw[thick,fill={rgb:black,1.4; white,8.6}, draw=black] (0,0) circle [radius=1.4]; 
	\foreach \r in {0,60,120} {\draw [rotate=\r] (1.4,0) -- (-1.4,0); }
	\end{scope}
	
	\begin{scope}[xshift=2.2cm]
	\filldraw[thick,fill={rgb:black,1.4; white,8.6}, draw=black] (0,0) circle [radius=1.4]; 
	\foreach \r in {60,120} {\draw [rotate=\r] (1.4,0) -- (-1.4,0); }
	\draw (25:1.4) -- (145:1.4); 
	\end{scope}
} \end{center}

\begin{itemize}
\item Какое наибольшее число кусочков пиццы вы можете получить с помощью четырех разрезов?
\item Опишите принцип максимизации для разрезания пиццы. А именно, ответьте на вопрос: как должен проходить новый разрез, чтобы получилось как можно больше кусочков?
\item Пицца Якоба Штейнера имеет бесконечный радиус, а потому о ней можно мыслить просто как о плоскости. Как связан ответ на общую задачу о пицце с ответом на задачу о разрезании пиццы Якоба Штейнера?
\item Через $P_n$ (от слова {\itshape pieces} --- кусочки) обозначим максимальное число кусков, на которое можно разрезать пиццу с помощью $n$ разрезов. Например, $P_1 = 2$ и $P_2 = 4$. Выразите $P_n$ через $P_{n-1}$ и найдите $P_{137}$.
\item Стандартный кусочек пиццы очень похож на треугольник. А как\linebreak связана задача о разрезании пиццы с треугольными числами?
\item Предположим, что мы получили $P_n$ кусочков пиццы за $n$ разрезаний. Обозначим через $C_n$ (от слова crust --- корка) число тех кусочков пиццы, граница которых не содержит корочки пиццы. Найдите явную формулу для $C_n$.
\item Сложим первые три числа на каждой строчке треугольника Паскаля. Какое число получается?
\item Решите аналогичную {\itshape задачу о разрезании прямой}. Как вы думаете, а как будет связан ответ на новую задачу с треугольником Паскаля?
\item Решите аналогичную {\itshape задачу о разрезании арбуза}. Как вы думаете, а как будет связан ответ на новую задачу с треугольником Паскаля? Уже догадались, какой ответы мы получим, если будем разрезать $n$--мерный шар?

\item Выпишем элементы последовательности $P_n$ в строчку, а под ней запишем ее {\itshape производную последовательность} $\text{Δ} P_n$ (вычитаем из элемента его предыдущий)

\begin{center} \scriptsize \begin{tabular}{llllllllllllll}
	\hline
	{\tiny последовательность} & 1 & & 2 & & 4 & & 7 & & 11 & & 16 & & 22\\
	{\tiny разности} & & 1 & & 2 & & 3 & & 4 & & 5 & & 6 & $\ldots$ \\
	{\tiny разности разностей} & & & 1 & & 1 & & 1 & & 1 & & 1 & $\ldots$ \\
	{\tiny разн. разностей разностей} & & & & 0 & & 0 & & 0 & & 0 & $\ldots$ \\
	\hline
\end{tabular} \end{center} \medskip

Как видите, четвертая строка состоит из нулей. Напишите такую же табличку для последовательности каких--нибудь фигурных чисел. 

\begin{center} \scriptsize \begin{tabular}{llllllllllll}
	\hline
	{\tiny последовательность} & $a_0$ & & $a_1$ & & $a_2$ & & $a_3$
		& & $a_4$ & & $a_5$ \\
	{\tiny разности} & & $b_0$ & & $b_1$ & & $b_2$ & & $b_3$ & & $b_4$ & $\ldots$ \\
	{\tiny разности разностей} & & & $c_0$ & & $c_1$ & & $c_2$ & & $c_3$ & $\ldots$ \\
	{\tiny разн. разностей разностей} & & & & 0 & & 0 & & 0 & $\ldots$ \\
	\hline
\end{tabular} \end{center} \medskip

Докажите, что если есть произвольная последовательность $a_n$ обладает тем свойством, что $\text{Δ} c_n = 0$, где $\text{Δ} a_n = b_n$ и $\text{Δ} b_n = c_n$, то
\begin{align*}
a_n = a_0 \binom{n}{0} + b_0 \binom{n}{1} + c_0 \binom{n}{2}.
\end{align*}
А что можно сказать про последовательности, у которых $c_n = 0$? А $b_n = 0$?
\end{itemize}