\task{Шарики в коробках}

Перед вами бесконечный набор коробок, на каждой из которых написано простое число. 
\begin{align*}
2 \quad 3 \quad 5 \quad 7 \quad 11 \quad 13 \quad 17 \quad 19 \quad 23 \quad 29 \quad 31 \quad 37 \quad 41 \quad \ldots
\end{align*}

Вам дали несколько белых шариков и вы решили положить их все в какие-то коробки (в одну коробку можно положить сразу много шириков). \medskip

После этого пришел эксперт, многозначно посмотрел на вашу расстановку шариков по ящикам и выдал вам число следующим образом: он возвел число $2$ в степень $\alpha_2$, равную количеству шариков в коробке с надписью $2$, потом возвел число $3$ в степень $\alpha_3$, равную количеству шариков в коробке с надписью $3$, потом возвел число $5$ в степень $\alpha_5$, равную количеству шариков в коробке с надписью $5$, и так далее. Затем он умножил все эти числа между собой. Получись число $n = 2^{\alpha_2}\cdot 3^{\alpha_3}\cdot 5^{\alpha_5}\cdot \ldots$.

\begin{itemize}
\item В коробке $2$ лежит $3$ шарика, а в коробках $3$ и $11$ лежат по одному шарику. Какое число назовет эксперт?
\item Сколько шариков вам понадобится и куда их нужно положить, чтобы эксперт назвал вам число $6$? А число $12$? А число $21$?
\item В каком случае эксперт назовет вам простое число? А составное? 
\item Вы расположили шарики в коробках и эксперт назвал число $n$. Что нужно сделать, чтобы он назвал число $np$, где $p$ --- простое число?
\item Известно, что число $n$, которое назвал эксперт, делится на простое число $p$. Что нужно сделать, чтобы эксперт назвал число $n/p$?
\item Вы положили несколько белых шариков в коробки и эксперт назвал Вам число $n$, а потом то же самое происходит с расстановкой черных шариков вашего друга --- он получает число $m$. Что нужно сделать, чтобы эксперт назвал число $nm$? А что нужно сделать, чтобы эксперт назвал НОД\,$(n,m)$? А НОК\,$(n,m)$? А что можно сказать про расположения белых и черных шариков, если числа $n,m$ взаимно просты?
\item Оказалось, что все $m$ шариков положили в одну коробку. Сколько делителей у числа, которое назвал эксперт?
\item Оказалось, что $m$ шариков положили в одну коробку, а $k$ --- в другую. Сколько делителей у числа, которое назвал эксперт?
\item Как определить количество простых делителей числа, которое назовет эксперт?
\item Правда ли, что можно заставить эксперта назвать любое натуральное число, если правильно подобрать шарики?

\end{itemize}