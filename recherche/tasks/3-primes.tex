\task{Знакомство с простыми числами}

Этот исследовательский проект состоит из нескольких шагов. Если у вас не получается до конца разобраться в каком--то шаге, пропустите его и изучайте следующий. 
%\begin{multicols}{2}
\begin{itemize}
\item Начните писать натуральные числа последовательно вдоль извивающейся линии, как показано на рисунке.

\begin{center} \tikz{
	\foreach \x in {1,...,3} {\draw (-0.8 cm + 0.8 * \x cm, -0.5 cm + 0.5 * \x cm)
		-- (0.8 * \x cm, 0.5 * \x cm) node[left]{\scriptsize\x}; }
	\foreach \x in {4,...,6} {\draw (3.2 cm, 0.5 * \x cm) node[left]{\scriptsize\x}; }
	\draw (3.2,2) -- (3.2,3);  \draw (2.4,1.5) -- (2.7,1.5) arc (-90:0:0.5cm); 
	\foreach \x in {7,...,9} {\draw (4 cm, -0.5 cm + 0.5 * \x cm) node[left]{\scriptsize\x}; }
	\draw (4,3.5) -- (4,4);  \draw (3.2,3) -- (3.5,3) arc (-90:0:0.5cm); 
	
	%% Умножить все на 0.8
	
	\draw(4.8,4) node[left]{\scriptsize 10}; 
	\draw(4.8,3.5) node[left]{\scriptsize 11}; 
	\foreach \x in {12,...,15} {\draw (5.6,-2.5 cm + 0.5 * \x cm) node[left]{\scriptsize\x}; }
	\foreach \x in {18,...,21} {\draw (7.2,-4.5 cm + 0.5 * \x cm) node[left]{\scriptsize\x}; }
	\draw(6.4,5) node[left]{\scriptsize 16}; 
	\draw(6.4,4.5) node[left]{\scriptsize 17}; 
	
	\draw (4.8,3.5) arc (-180:0:0.4); 
	\draw (6.4,4.5) arc (-180:0:0.4); 
	\draw (8,5.5) node{$\vdots$}; 
	
	\draw(4,4) arc (180:0:0.4); 
	\draw(5.6,5) arc (180:0:0.4); 
	\draw(7.2,6) arc (180:0:0.4); 
		\draw (4.8,4) -- (4.8,3.5);  \draw (6.4,5) -- (6.4,4.5); 
	\draw (5.6,3.5) -- (5.6,5);  \draw (7.2,4.5) -- (7.2,6); 
} \end{center}

Найдите на полученной спирали закономерность, связанную с простыми числами, и объясните ее.
\item Сколько делителей имеют числа $1, 2, 4, 8, 16, 32, \ldots$? А когда число вида $2^n - 1$ является простым? Составьте таблицу чисел такого вида и найдите закономерность.
\item Выясните, для каких чисел $n$ число $(n-1)!+1$ делится на $n$. Найдите закономерность.
\item Выясните, для каких чисел $n$ число $n^2-1$ делится на 24. Найдите закономерность. 
\item Выясните, для каких чисел $n$ число $n^2+1$ является простым. Найдите закономерность. 
\item Выясните, для каких чисел $n$ число $2^n-2$ делится на $n$. Найдите закономерность.
\item Посмотрим на числа вида $n^2-n$. Получаем $0,2,6,12,20,\ldots$. Все эти числа делятся на два. Посмотрим на числа вида $n^3-n$. Получаем $0, 6, 24, 60, 120, 210, 336, \ldots$. Все эти числа делятся на три! А что дальше? Будут ли числа вида $n^4-n$ делиться на четыре? А числа вида $n^5-n$ на пять? Найдите закономерность.
\end{itemize}