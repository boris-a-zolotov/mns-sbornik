\task{Геометрическая миниатюра}

\begin{itemize}
\item Как вы думаете, какую часть (по площади) составляет треугольник внутри прямоугольника на картинке ниже?

\begin{center} \tikz{
	\draw[very thick] (-2.5,-1.1) -- (-2.5,1.1) -- (2.5,1.1) -- (2.5,-1.1) -- cycle;
	\filldraw[thick,draw=black,fill={rgb:black,1.4;white,8.6}]
		(-2.5,-1.1) -- (-1.4,1.1) -- (2.5,-1.1) -- cycle;
} \end{center}


\item А какую часть составляет такой треугольник?

\begin{center} \tikz{
	\draw[very thick] (-2.5,-1.1) -- (-2.5,1.1) -- (2.5,1.1) -- (2.5,-1.1) -- cycle;
	\filldraw[thick,draw=black,fill={rgb:black,1.4;white,8.6}]
		(2.5,1.1) -- (-2.5,0.5) -- (2.5,-1.1) -- cycle;
} \end{center}

\item А такой?

\begin{center} \tikz{
	\draw[very thick] (-2.5,-1.1) -- (-2.5,1.1) -- (2.5,1.1) -- (2.5,-1.1) -- cycle;
	\filldraw[thick,draw=black,fill={rgb:black,1.4;white,8.6}]
		(-0.6,-1.1) -- (-2.5,1.1) -- (2.5,-1.1) -- cycle;
} \end{center}

\item Как вы думаете, какую часть (по площади) составляет выделенная область от всего правильного шестиугольника?

\begin{center} \tikz{
	\filldraw[thick,draw=black,fill={rgb:black,1.4;white,8.6}]
		(0,0) -- (0: 1.2cm) -- (330: 1.38564cm) -- (300: 1.2cm);
	\draw[very thick] (0: 2.4cm) -- (60: 2.4cm) -- (120: 2.4cm) -- (180: 2.4cm)
		-- (240: 2.4cm) -- (300: 2.4cm) -- cycle;
	\draw[thick] (300: 2.4cm) -- (60: 2.4cm) -- (120: 2.4cm) -- cycle;
	\draw[thick] (0: 2.4cm) -- (180: 2.4cm) -- (240: 2.4cm) -- cycle;
} \end{center}

\end{itemize}


\task{Конфигурации точек}

В 2014 г. на Санкт-Петербургской олимпиаде школьников по математике была предложена следующая задача:
\begin{quote}
На двух параллельных прямых отмечено по 40 точек. Их разбивают на 40 пар так, чтобы отрезки, соединяющие точки в одной паре, не пересекались друг с другом. (В частности, конец одного из отрезков не может лежать на другом отрезке). Докажите, что число способов это сделать не превосходит числа $3^{39}$.
\end{quote}
Последовательность, возникающая в этой задаче, обладает богатыми комбинаторными реализациями, их разнообразие просто изумляет. Опишем общую ситуацию: пусть даны две параллельные прямые, на одной отмечено $k$ точек, на другой $n$ точек. 

\def\lenpalet{
	\filldraw[fill=white,draw=white] (-0.55,-0.3) rectangle (2.4,1.1);
	\filldraw (0,0.8) node(a){ } circle[radius=0.45mm];
	\filldraw (0.65,0.8) node(b){ } circle[radius=0.45mm];
	\filldraw (0,0) node(c){ } circle[radius=0.45mm];
	\filldraw (0.65,0) node(d){ } circle[radius=0.45mm];
	\filldraw (1.3,0) node(e){ } circle[radius=0.45mm];
	\filldraw (1.95,0) node(f){ } circle[radius=0.45mm];
}

\def\dt{\draw[thick]}

\begin{center} \begin{tabular}{c|c}
	\tikz{
		\lenpalet;
		\dt (a)--(b); \dt (c)--(d); \dt (e)--(f);
	} &
	\tikz{
		\lenpalet;
		\dt (a)--(c); \dt (b)--(d); \dt (e)--(f);
	} \\ \hline
	\tikz{
		\lenpalet;
		\dt (a)--(c); \dt (d)--(e); \dt (b)--(f);
	} &
	\tikz{
		\lenpalet;
		\dt (c)--(d); \dt (a)--(e); \dt (b)--(f);
	}
\end{tabular} \end{center}

Отмеченные точки разбивают на пары так, чтобы отрезки, соединяющие точки в одной паре, не пересекались друг с другом. В частности, конец одного из отрезков не может лежать на другом отрезке. 
\begin{center}
 
\begin{tabular}{c c c c c}
       &       &$[0,0]$&       &     \\
       &$[2,0]$&$[1,1]$&$[0,2]$&     \\
$[4,0]$&$[3,1]$&$[2,2]$&$[1,3]$&$[0,4]$ \\
       &       &$\ldots$&      &
\end{tabular}
 
\end{center}
Полученную картинку будем называть {\itshape конфигурацией} (точек и отрезков на двух прямых) или {\itshape разбиением} (точек на пары). Количество разбиений обозначим через $[k,n]$. Например, $[2,4] = 4$, как показывает рисунок выше.
\begin{center}
 
\begin{tabular}{c c c c c c c c c c c c c c}
 &  &  &  & 1&  &  &  & \\ 
 &  &  & 1& 1& 1&  &  & \\ 
 &  & 1& 2&  & 2& 1&  & \\ 
 & 1& 3&  &  &  & 3& 1& \\ 
1& 4&  &  &  &  &  & 4& 1 
\end{tabular}
 
\end{center}
 
Кроме того, если $n+k$ нечётно, то $[k,n] = 0$, а также $[k,n] = 0$, если $n,k < 0$. Если $n$ или $k$ равно нулю, то $[k,n] = 1$ просто по определению.
\begin{itemize}
\item Заполните треугольник выше и найдите в нём закономерности, аналогичные закономерностям в треугольнике Паскаля.
\item Выясните, какую последовательность образуют суммы элементов в строках треугольника.
\item Правда ли, что в каждой строчке числа $[k,n]$ обязательно возрастают при движении от краёв к центру?
\item Выясните, как выразить $[n,n]$ через суммы квадратов чисел на диагоналях в треугольнике Паскаля.
\end{itemize}