\section{Пока не пришёл лифтёр}

\noindent Витя и Петя живут в бесконечном вверх и вниз доме и очень любят кататься на лифте. Как-то раз неведомые хулиганы сломали кнопки во всех лифтах так, что те могли двигаться только на $n$ этажей вверх или вниз и на $m$ этажей вверх или вниз. 
\begin{itemize}

\itA Мальчики не растерялись — сели каждый в свой лифт и одновременно выехали с нулевого этажа, причём Витя с каждый раз едет на $n$ этажей вверх, а Петя - на $m$ этажей вверх. Оказалось, что первый раз они побывали на одном и том же этаже под номером 123. Чему могли быть равны $n$, $m$?

\itB Петя находится этажом выше Вити. Петин лифт умеет ездить на $k$ этажей вверх или вниз, Витин — на $k+1$ этаж вверх или вниз. Мальчики начинают ездить на лифтах, как им заблагорассудится. Может ли Петя управлять своим лифтом так, чтобы никогда не встретиться с Витей на одном этаже? Обязательно ли для этого Пете знать этаж, на котором в данный момент находится Витя?

\itC Теперь Витя решил с помощью двух кнопок — на $n$ этажей вверх или на $m$ вниз — добраться на лифте с нулевого этажа до первого. И у него получилось. Докажите, что НОД\,$(n,m)=1$.
\end{itemize}