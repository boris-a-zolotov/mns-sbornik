\task{Средние арифметические}

\begin{itemize}
\itA Пример наборов, удовлетворяющих условию задачи —

1, 2, 3, 4, 5;\quad 1, 2, 3, 4, 5;\quad 1, 2, 3, 4, 5;\quad\\
10001, 10002, 10003, 10004, 10005.

Максимум средних равен 10003, а среднее арифметическое максимумов — 255.

\itB После разбиения детей на классы у нас будет четыре «самых низких» ребенка, по одному на класс. Расставим их по росту. Одним из них точно будет тот, чей рост — \SI{101}{\text{сантиметр}}. Рост второго будет не больше 131, третьего — не больше 161, четвертого — не больше 191, потому что между этими отметками вмещается ровно по тридцать детей, и если не все они будут в одном классе, то более высокий из детей с самым низким ростом окажется среди них.

Таким образом, у нас есть оценка сверху на величину, которую мы пытаемся максимизировать — $\tfrac{1}{4}(101+131+161+191)$. Попробуем добиться того, чтобы среднее арифметическое четырех ростов было именно таким. Для этого можно разбить детей на классы «подряд» — первые тридцать в первый класс, вторые тридцать — во второй, $\ldots$

Так и сделаем.

\itC Заметим, что какое разбиение детей на классы ни возьми, — сумма средних арифметических ростов детей в классах будет постоянна (и равна 362 см). Значит, минимум наибольший, когда все средние арифметические совпадают. Значит, нужно составлять классы, симметричные относительно 90,5. Например, в первый класс отправить первые десять детей и последние десять, а во второй — вторую и предпоследнюю десятки детей.
\end{itemize}