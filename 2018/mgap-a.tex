\task{Ужасный гадкий аккуратный подсчёт}
\begin{itemize}
\itA Квадратов $1 \times 1$~— $4 \cdot 5 = 20$ штук. Квадратов размером $2 \times 2$ найдётся $3 \cdot 4 = 12$ штук. Квадратов $3 \times 3$ и $4 \times 4$~— 6 и 2 соответственно. Таким образом, всего квадратов

$$20 + 12 + 6 + 2 = 40.$$

Количество прямоугольников можно посчитать более «продвинутым» образом: заметим, что прямоугольников размером $a \times b$ (где $a$~— высота, $b$~— ширина, то есть, мы различаем прямоугольники $2 \times 3$ и $3 \times 2$) можно найти ровно $(4-a+1) \cdot (5-b+1)$ штук. Число $a$ меняется от 1 до 4~— отсюда $4-a+1$ меняется в тех же пределах. То же самое с $5-b+1$~— оно меняется от 1 до 5.

\ms Отсюда можно заключить, что сумма чисел вида $(4-a+1) \cdot (5-b+1)$ при всевозможных $a$ и $b$ будет равна сумме всех чисел вида $a \cdot b$. Как посчитать сумму всех чисел вида $a \cdot b$? Заметим, что при раскрытии скобок в произведении

$$(1+2+3+4+5) \cdot (1+2+3+4)$$

получится сумма из всех слагаемых, которые нам нужны. Отсюда прямоугольников можно найти $15 \cdot 10 = 150$ штук.

\itB \label{mgap-ba} Всего раскрасок $n!$~— в «первом» секторе может стоять $n$ цветов, в следующем~— $n-1$, и так далее. Из одной раскраски вращением круга можно получить ровно $n$ раскрасок (включая её саму)~— поэтому ответ равен $\tfrac{n!}{n} = (n-1)!$.

\itC В верхней полосе может стоять один из шести имеющихся цветов. Во второй полосе — любой из шести цветов, кроме уже стоящего в первой. В нижней полосе — любой из цветов, кроме уже стоящего во второй. Таким образом, ответ — $6 \cdot 5 \cdot 5 = 150$.
\end{itemize}