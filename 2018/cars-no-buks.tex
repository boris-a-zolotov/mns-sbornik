\task{Без пробуксовки}
\begin{itemize}

\itA Машина ездит по окружности вокруг точки, где пересекаются линии, перпендикулярные переднему и заднему колёсам, проходящие через их центр. Эти линии вместе с отрезком между колёсами машины образуют прямоугольный треугольник (см. рисунок), один из углов которого~— 60 градусов, а один из катетов~— \SI{5}{\text{м}}.

Тогда $R = 5 \sqrt{3}$ (отношения сторон прямоугольного треугольника с углом $60^\circ$~— известные величины).

\begin{center}
\tikz{
  \draw (-5,0) -- (0,0) -- (0,3.5);
  \draw (-0.4,0) -- (-0.4,0.4) -- (0,0.4);
  \draw[rotate around={30:(0,2.5)}] (-5.5,2.5) -- (0.5,2.5) -- (0,2.5) -- (0,3.5);
  \filldraw[fill=black] (-0.1,-0.3) rectangle (0.1,0.3);
  \filldraw[fill=black, rotate around={30:(0,2.5)}] (-0.1,2.2) rectangle (0.1,2.8);
  \draw[double,very thick] (0,2) arc (-90:-150:0.5cm);
  \draw[very thick] (0,3) arc (90:120:0.5cm);
  \draw (-0.5,1.7) node {$60^\circ$};
  \draw (-0.6,2.8) node {$30^\circ$};
  \draw (0.4, 1.1) node {\SI{5}{\text{м}}};
  \draw (-2.3,-0.4) node {$R$};
}
\end{center}

\itB Теперь нас интересует высота прямоугольного равнобедренного треугольника с основанием \SI{1.8}{\text{м}}. Она равна \SI{0.9}{\text{м}}. То есть, погрузчик ездит вокруг точки, расположенной на \SI{0.9}{\text{м}} левее, чем середина его левого борта.

\begin{center}
\tikz{
	\draw (0,2.3) -- (0,-2.3);
	\draw[rotate around={45:(0,-1.5)}] (0,-2) -- (0,1.1213);
	\draw[rotate around={-45:(0,1.5)}] (0,2) -- (0,-1.1213);
	\filldraw[fill=black, rotate around={-45:(0,-1.5)}]
		(-0.1,-1.8) rectangle (0.1,-1.2);
	\filldraw[fill=black, rotate around={45:(0,1.5)}]
		(-0.1,1.2) rectangle (0.1,1.8);
	\draw[rotate around={-45:(0,-1.5)}] (0,-1.5) -- (0,-2.5);
	\draw[rotate around={45:(0,1.5)}] (0,1.5) -- (0,2.5);
	\draw(0.5,0) node{\SI{1.8}{\text{м}}};
%%%%%%%%%%%%
	\foreach \y / \r in {-1.5 / -1, -1.5 / 1, 1.5 / -1, 1.5 / 1} {
		\draw[very thick] (0, \y cm + \r * 0.5 cm)
			arc (\r * 90:\r * 135:0.5);
	};
	\draw(-0.3,0.75) node {$45^\circ$};
	\draw (-2.4,1.6) node {\bfseries B.};
%%%%%%%%%%%%%%%%
%%%%%%%%%%%%%%%%
\begin{scope}[xshift=5.5cm,yshift=-1.2cm]
	\draw (0,3.6+0.5+0.3) -- (0,-1.2-0.5-0.3);
	\foreach \y / \r in {-1.2 / -30, 0 / 0, 3.6 / 60} {
		\filldraw[fill=black, rotate around={\r:(0,\y cm)}]
			(-0.1 cm,-0.3 cm+ \y cm) rectangle (0.1 cm,0.3 cm + \y cm);
	}
	\draw (0.5,0)
		-- (-2.0785,0) node[circle,fill,inner sep=0.6mm](){}
		-- (-2.5785,0);
	\draw[rotate around={-30:(0,3.6)}] (-0.8,3.6) -- (0,3.6) -- (0,4.4) -- (0,-1.057);
	\draw[rotate around={60:(0,-1.2)}] (0.8,-1.2) -- (0,-1.2) -- (0,-2) -- (0,1.7);
%%%%%%%%%%%%
	\draw (-0.35,0) -- (-0.35,0.35) -- (0,0.35);
	\draw[very thick] (0,4.1) arc (90:150:0.5);
	\draw[very thick] (0,2.9) arc (-90:-120:0.7);
	\draw[very thick,double] (0,-0.7) arc (90:60:0.5);
%%%%%%%%%%%%
	\draw (-0.35,4.25) node{$60^\circ$};
	\draw (-0.3,2.4) node{$30^\circ$};
	\draw (0.6,-0.8) node{\bfseries ???};
	\draw (0.35,1.8) node{\SI{9}{\text{м}}};
	\draw (-0.35,-0.5) node{\SI{3}{\text{м}}};
	\draw (-2.0785 / 2,0.24) node{$3\sqrt{3}$\,м};
	\draw (-2.4,2.8) node {\bfseries C.};
\end{scope} }
\end{center}

\itC Аналогично первому пункту данной задачи, найдём расстояние от не поворачивающегося колеса до точки, вокруг которой ездит автобус. Оно равно $\tfrac{9}{\sqrt{3}} = 3\sqrt{3}$: опять же, мы, зная один из катетов прямоугольного треугольника с углом $60^\circ$, ищем другой.

Теперь заметим, что среднее и заднее колёса, а также точка, вокруг ездит автобус, образуют прямоугольный треугольник с катетами $3$ и $3\sqrt{3}$ метра. Значит, его углы~— 30 и 60 градусов. Отсюда заднее колесо нужно повернуть на 30 градусов.

\end{itemize}
