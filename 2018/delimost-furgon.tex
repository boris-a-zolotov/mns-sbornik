\task{Фургончик}
\begin{itemize}
\itA Мы знаем, что $(p_1+1)(p_2+1) = p_1p_2 + 15$. Если раскрыть скобки, получается $p_1 + p_2 = 14$. Единственные простые числа, подходящие под это условие,~— 11 и 3. Это и есть ответ.

\itB Для того, чтобы выяснить, какие ноги ещё не были переставлены, нам нужно отыскать все нечётные числа между 2 и 40, не делящиеся на 3. Это 5, 7, 11, 13, 17, 19, 23, 25, 29, 31, 35, 37. Проверить, что мы выписали все нужные числа, несложно~— достаточно посмотреть на их остатки при делении на 6: числа должны иметь вид $6k-1$ или $6k+1$ (остальные остатки от деления на 6 либо чётные, либо 3). Получилось 12 чисел~— это ответ на задачу.

\itC Будем измерять расстояние, которое проехал Саша за день, не в километрах, а в метрах. Понятно, что расстояние между A и G равно сумме со знаками $+$ или~$-$ расстояний между городами, которые указаны в задаче. Осталось только заметить, что все расстояния в метрах ($12000$, $18000$, $10500$, $19500$, $\ldots$) делятся на 3, а их предполагаемая сумма~— $41000$~— почему-то нет. Значит, в атласе дана неверная информация.
\end{itemize}