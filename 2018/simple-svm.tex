\task{Об одной задаче классификации}
\begin{itemize}

\itA Очевидно, что ширина этой полосы не может быть больше, чем расстояние между самыми близкими друг к другу точками кругов. Сделаем ширину полосы равной этому расстоянию.

Для этого соединим центры кругов отрезком и построим касательные к кругам в в точках пересечения отрезка с их границей. Они будут параллельны и образуют полосу максимально допустимой ширины (смотреть рисунок).

\definecolor{lghtgryq}{RGB}{205,205,205}

\begin{center} \tikz{
\begin{scope}[rotate=55]
	\filldraw[draw=white,fill=lghtgryq] (-2.45,-0.9) rectangle (2.45,0.9);
	\foreach \y / \m in {0.9 / 1, -0.9 / -1} {
		\draw[thick] (-2.7,\y) -- (2.7,\y);
		\draw[thick] (0, \y) arc (-90*\m:270*\m:0.75);
		\draw[thick,dashed] (0, \y cm + \m*0.75 cm) -- (0, \y cm);
	}
	\draw[thick,densely dotted] (0,0.9) -- (0,-0.9);
	\draw (0,0.6) -- (0.3,0.6) -- (0.3,0.9);
\end{scope}
} \end{center}

\item[\bfseries B.–C.] Сработает похожий метод: надо найти ближайшие друг к другу точки квадратов и соединить их отрезком — искомая полоса получится, если провести к данному отрезку перпендикуляры в его концах.

С одной стороны, её ширина будет максимально допустимой, потому что она будет равна расстоянию между ближайшими точками квадратов, а большая ширина запрещена.

С другой стороны, ни одна точка из квадратов не попадёт внутрь этой полосы, потому что квадрат — выпуклый многоугольник. В силу этого он либо лежит по одну сторону от прямой, проходящей через точку его границы, либо лежит по обе стороны, и с каждой из сторон от прямой находится часть стороны, на которой лежала точка, через которую мы проводили прямую.

Но тогда на части этой стороны, лежащей внутри полосы, найдётся точка, которая ближе к другому концу отрезка, лежащем на другом краю полосы, что противоречит построению полосы.
\end{itemize}