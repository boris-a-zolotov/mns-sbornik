\task{У магазина}
\begin{itemize}
\itA Понятно, что Федор и Кирилл увеличивают все числа в одинаковое число раз. И „144“, названное Федором, есть квадрат этого числа (так как он назвал то, во сколько раз увеличивает все Кирилл, сам увеличив это число). Тогда оба продавца умножают все на 12.

\ms Соответственно, учебник стоит $43200 \div 144 = \SI{300}{\text{рублей}}$ — так как его цена прошла через уста, опять же, обоих продавцов.

\itB Делимость на 99 значит делимость на 9 и на 11. Восстановить стертую цифру можно почти однозначно, посчитав сумму оставшихся цифр и найдя остаток от деления ее на 9. Проблема может возникнуть, если сумма оставшихся на номере цифр делится на 9 — тогда непонятно, 0 нам ставить на пустое место или 9.

\ms Признак делимости на 11 говорит нам, что знакопеременная сумма цифр числа должна делиться на 11. Заметим, что при постановке цифр 9 и 0 на одно и то же место не может оказаться так, что оба результата будут делиться на 11. Поэтому получится однозначный ответ.

\itC То, как происходит торг между продавцом и покупателем, на самом деле, повторяет работу алгоритма Евклида. Алгоритм Евклида всегда завершается — значит, и торг завершится.

\ms При этом на каждом шаге торга хотя бы одна из названных цен уменьшается хотя бы на 1, поэтому в любой момент времени количество шагов торга оценивается сверху суммой цен, называемых покупателем и продавцом. Поэтому количество шагов всегда будет строго меньше суммы текущих чисел.

\ms Пусть изначально названы цены $a$ и $b = a+t$. Тогда на следующем шаге торга будут названы цены $a$ и $t$. Тогда количество шагов торга строго меньше, чем

$$a+t \underset{\text{\begin{minipage}[l]{1.1cm} \tiny
	один шаг \\ уже сделан
\end{minipage}}}{+\ 1}= b+1 \leq 21.$$

Торг с 20 шагами легко придумать: пусть изначально были названы цены 1 и 20.
\end{itemize}