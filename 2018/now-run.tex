\task{Необходимости и достаточности}
\begin{itemize}
\itA Скорость мышки равна $10 \cdot 35 = \SI{350}{\text{см}/\text{с}}$, а скорость кошки — $55 \cdot 9 = \SI{495}{\text{см}/\text{с}}$. Несомненно, кошка быстрее.

\itB Сколько вылетов нужно сделать винтовому самолёту? Каждого Йожина надо осыпать трижды — получается 300 осыпаний. Каждый вылет даёт два осыпания — поэтому нужно 150.

\ms А реактивному? Аналогичным образом получаем $100 \cdot 8 \div 5 = 160$ вылетов. Таким образом, винтовой самолёт на 10 вылетов эффективнее.

\itC Обозначим через $x_k$ массу еды, которая была в наличии у велосипедистов {\itshape перед} $k$--ым обедом. Мы знаем, что $x_{31}=0$, и ищем $x_1$. Давайте выразим $x_k$ через $x_k+1$. В соответствии с условием задачи,

$$x_k\ \ \ = \underbrace{0.1 \cdot x_k + 2}_{\text{съедят за $k$--ым обедом}} +\ \ \ x_{k+1}.$$

Откуда

$$x_k = \frac{20}{9} + \frac{10}{9} x_{k+1}\scolon$$

\begin{align*}
x_1 =\ & \frac{20}{9} + \frac{10}{9} \cdot \frac{20}{9} + \left(\frac{10}{9}\right)^2 \cdot x_3 = \\
=\ & \frac{20}{9} + \frac{10}{9} \cdot \frac{20}{9} + \ldots + \left(\frac{10}{9}\right)^{29}
	\cdot\frac{20}{9} + \left(\frac{10}{9}\right)^{30} \cdot x_{31} = \\
=\ & \frac{20}{9} + \frac{10}{9} \cdot \frac{20}{9} + \ldots + \left(\frac{10}{9}\right)^{28}
	\cdot\frac{20}{9} + \left(\frac{10}{9}\right)^{29} \cdot \frac{20}{9} = \\
=\ & \frac{20}{9} \cdot \frac{\left(\tfrac{10}{9}\right)^{30}-1}{\tfrac{10}{9}-1}
	\quad \text{— это ответ на задачу.}
\end{align*}

\end{itemize}