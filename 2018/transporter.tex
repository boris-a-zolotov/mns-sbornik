\task{Как провожают транспортеры...}

\begin{itemize}
\itA Если наблюдатель движется со скоростью $\tfrac{1}{3}v$ навстречу транспортеру, собственная скорость которого равна $\tfrac{1}{6}v$, их скорость сближения равна $\tfrac{1}{2}v$ — то есть, для наблюдателя этот транспортер выглядит всего лишь в два раза медленннее, чем исходный.

В такой ситуации взрослый питон проехал бы мимо наблюдателя за 28 секунд. Но питон–детеныш короче, и для его проезда понадобится $28 \cdot \tfrac{3}{4} = 21$~секунда.

\itB Чтобы не обманываться длинами кубиков (как это сделало большинство участников олимпиады), мы на время заменим их на передние их точки относительно движения транспортера. Расстояние между этими точками будет равно 15 сантиметров.

При попадании на более быстрый транспортер расстояние между этими точками увеличится вдвое и составит \SI{30}{\text{см}}. Чтобы получить расстояние между кубиками, из этой величины надо вычесть 5 сантиметров — получится \SI{25}{\text{см}}.

Распространенная ошибка заключалась в том, что участники олимпиады умножали на 2 расстояние между \underline{концом} первого кубика и \underline{$\vphantom{\text{ц}}$началом} второго. Это неправомерно, потому что две названные точки играют разную роль, и умножать расстояние между ними на 2 при решении задачи — это как мерить прыжки в длину одной половины атлетов по дальней точке касания, а другой — по ближней.

\itC Очевидно, что оптимальное деление песка между транспортерами происходит тогда, когда они заканчивают работу одновременно: иначе у опустевшего транспортера остается ресурс, когда он простаивает, а второй транспортер работает вместо двоих.

Поэтому песок нужно поделить в отношении $2\,:\,1$, отдав в два раза больше транспортеру, который работает в два раза быстрее. Получится \SI{400}{\text{кг}} первому и \SI{800}{\text{кг}} второму.

\end{itemize}