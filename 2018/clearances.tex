\task{Клиренсы}
\begin{itemize}

\itA $$740/2-175 = \SI{195}{(\text{мм})}.$$

\itB Угол дома «поднимается» над линией, соединяющей основания ножек стула, на расстояние, равное высоте прямоугольного треугольника с гипотенузой 50 сантиметров. Эта величина максимальна, очевидно, когда треугольник равнобедренный~— тогда она равна \SI{25}{\text{см}}. Поэтому расстояние от сиденья до земли должно быть не меньше \SI{25}{\text{см}}.

\begin{center} \tikz{
	\draw[very thick,rotate=45]
		(-1,-1) node[circle,fill,inner sep=0.6mm](){}
		-- (-1,0.3) -- (1,0.3) -- (1,1.3)
		-- (1,-1) node[circle,fill,inner sep=0.6mm](){};
	\draw[thick,dashed,rotate=45] (-1,-1) -- (1,-1);
	\draw[thick] (0,-2) -- (0,0) -- (2,0);
	\draw (0,-0.28) -- (0.28,-0.28) -- (0.28,0);
	\draw (1.1,-0.9) node{\footnotesize \SI{50}{\text{см}}}
} \end{center}

\itC Треугольник, образованный центром планеты и центрами колёс автобуса,~— равносторонний со стороной \SI{10.5}{\text{см}}: одна из сторон равна колёсной базе, а две других~— сумме радиуса планеты (10 метров) и радиуса колеса (0.5 метра).

\begin{center}
   \tikz{
      \draw[thick,rotate=20] (3,0) arc (0:140:3);
      \draw[thick,rotate=60] (3,0) arc (-180:180:0.32);
      \draw[thick,rotate=120] (3,0) arc (-180:180:0.32);
      \draw[thick] (0,0) -- (60:3.32) -- (120:3.32) -- cycle;
      \draw[thick,dashed] (-4,2.55) -- (4,2.55);
      \draw[->,>=stealth] (0,0) -- (0,2.555);
      \draw (0.2,1.4) node[rotate=90]{\footnotesize $H - 0.5$};
   }
\end{center}

Дорожный просвет автобуса~— расстояние от его пола (который должен касаться верхней точки планеты) до прямой, соединяющей нижние точки колёс. В нашем случае~— это разность $R-(H-0.5)$, где $H$~— высота равностороннего треугольника, а $0.5$~— радиус колеса.

\begin{align*}
	& H = 10.5 \cdot \frac{\sqrt{3}}{2} \\
	& R-(H-0.5) = 11 - 10.5\frac{\sqrt{3}}{2}. \\
	& \text{(это примерно 1.9 метра)}
\end{align*}

Это и есть ответ на задачу.
\end{itemize}