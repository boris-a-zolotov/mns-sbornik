\task{Разрезания}
\begin{itemize}
\itA Поделим каждую из сторон квадрата на семь равных отрезков и рассмотрим 28 треугольников, получающихся, если соединить центр квадрата с краями каждого из этих отрезков. Все эти треугольники имеют одинаковую площадь (так как у них одинаковы основание и высота) и равные длины сторон, лежащие на сторонах квадрата (по построению).

Чтобы получить 7 многоугольников, требуемых в условии, объединим по четыре соседних треугольника — смотреть рисунок:

\def\ret#1{-- #1 -- (0,0)}

\begin{multicols}{2}

\begin{center} \tikz{
\begin{scope}[scale=0.8]
	\draw[very thick] (-2.1,-2.1) -- (2.1,-2.1) -- (2.1,2.1) -- (-2.1,2.1) -- cycle;
	\foreach \x in {0,...,6} \draw[thick, densely dotted]
		(-0.6*\x cm + 2.1cm, -2.1) -- (0.6*\x cm - 2.1cm, 2.1);
	\foreach \x in {0,...,6} \draw[thick, densely dotted]
		(-2.1, 0.6*\x cm - 2.1cm) -- (2.1, -0.6*\x cm + 2.1cm);
	\draw[very thick] (0,0) \ret{(-0.9,2.1)} \ret{(1.5,2.1)}
		\ret{(2.1,0.3)} \ret{(2.1,-2.1)} \ret{(-0.3,-2.1)}
		\ret{(-2.1,-1.5)} \ret{(-2.1,0.9)};
\end{scope}
} \end{center}

\columnbreak

\begin{center} \ \\ [0.15cm]
\tikz{
\begin{scope}[scale=0.8]
	\draw [very thick] (0,0) -- ++(1,0) -- ++(1.5,1.5) -- ++(2,-2)
		-- ++(1,0) -- ++(-1,-1) -- ++(-1,0) -- ++(-0.5,-0.5)
		-- ++(-1,0) -- ++(-2,2);
	\draw[thick] (1.5,0.5) -- ++(1,0) -- ++(0.5,-0.5) -- ++(1,0)
		-- ++(-1,0) -- ++(0.5,-0.5) -- ++(-1.5,-1.5);
	\draw[thick] (2.5,0.5) -- ++(-1.5,-1.5) -- ++(1,0) -- ++(0.5,0.5)
		-- ++(1,0) -- ++(1,-1);
\end{scope}
}\end{center}

\end{multicols}

\itB Смотреть рисунок выше.


\itC Аналогично тому, что было проделано в первом пункте данной задачи, мы умеем резать квадрат на три многоугольника равной площади с равной длиной сторон, лежащих на сторонах квадрата.

Разрежем каждый квадратный «слой» пирамиды на три таких многоугольника одинаковым образом (с точностью до подобия). Тогда в объединении всех слоёв получатся три многогранника одинакового объёма, несущие на себе одинаковое количество краски («выходящие» на стороны пирамиды одинаковой площадью своей границы).

Смотреть рисунок:

\begin{center} \tikz{
\begin{scope}[scale=1.2]
%%%%%%%%
	\draw (2.25,3.5) node[inner sep = 0mm](h){};
	\draw (2.25,0.5) node[inner sep = 0mm](m){};
%%%%%%%%
	\draw[very thick]
		(0,0) node[inner sep = 0mm](a){}
		-- (3,0) node[inner sep = 0mm](b){}
		-- (4.5,1) node[inner sep = 0mm](c){};
	\draw[very thick,dashed]
		(c) -- (1.5,1) node[inner sep = 0mm](d){} -- (a);
%%%%%%%%
	\draw[very thick] (h) -- (a);
	\draw[very thick] (h) -- (b);
	\draw[very thick] (h) -- (c);
	\draw[very thick,dashed] (h) -- (d);
%%%%%%%%
	\foreach \x in {1,2} {
		\draw[dotted] (h) -- (\x cm, 0 cm) -- (m);
		\draw[dotted] (h) -- (3cm + 0.5*\x cm, 1/3 * \x cm) -- (m);
		\draw[loosely dotted] (h) -- (1.5 cm + \x cm, 1 cm) -- (m);
		\draw[loosely dotted] (h) -- (0.5*\x cm, 1/3 * \x cm) -- (m);}
%%%%%%%%
	\draw[pattern=north west lines, pattern color=gray, thick]
		(2.25,3.5) -- (2.25,0.5) -- (3,0) -- cycle;
	\draw[pattern=north east lines, pattern color=gray, thick]
		(2.25,3.5) -- (2.25,0.5) -- (1/2,1/3) -- cycle;
	\draw[pattern=north east lines, pattern color=gray, thick]
		(2.25,3.5) -- (2.25,0.5) -- (3.5,1) -- cycle;
%%%%%%%%
\end{scope}
} \end{center}

\end{itemize}