\task{Летающий цирк}

\begin{itemize}
\itA Все слова в этой задаче состоят из букв А, М, Р, С, Т. Постараемся поставить эти буквы в соответствие с действиями Лэмберта. Для этого составим таблицу: сколько каких букв находится в словах, адресованных Лэмберту.

\begin{center} \begin{tabular}{|l|l|l|l|l|l|}
\hline		& А & М & Р & С & Т \\ \hline
\ttfamily МАТРАС & 2 & 1 & 1 & 1 & 1 \\ \hline
\ttfamily СТАРТ & 1 & 0 & 1 & 1 & 2 \\ \hline
\ttfamily МАРС & 1 & 1 & 1 & 1 & 0 \\ \hline
\end{tabular} \end{center}

Услышав слово {\bfseries«МАТРАС»}, Лэмберт среди прочего поёт два куплета из песни — значит, буква `А' отвечает за куплеты. По аналогичным причинам (посмотрим, каких букв две в слове {\bfseries «СТАРТ»}), `Т' — это ноги в коробке. `М' — это то, чего нет в слове {\bfseries «СТАРТ»}, но есть в {\bfseries «МАТРАС»} — это надевание ведра.

Для `Р' и `С' остаются снятие перчаток и „Караул!“ — но нам неважно, что из действий какой букве соответствует, потому что `Р' и `С' встречаются во всех рассматриваемых словах по одному разу.

Отсюда ответ: Лэмберт закричит „Караул!“, споёт один куплет, наденет на голову ведро и снимет перчатки.

\itB Да, джентльмен сможет купить себе шляпу, так как цена, называемая продавцом, не возрастает (пока финансовые возможности джентльмена остаются ниже её), а количество финансов, имеющееся у джентльмена, на каждом шаге растёт ровно на~1.

Можно также явно проделать процедуру, описанную в задаче, и выяснить, через сколько именно шагов шляпа окажется у джентльмена (получится точно меньше десяти) — но мы не будем делать этого здесь, оставив читателю в качестве упражнения.

\itC Пусть Тревор преувеличивает всё в $a$ раз, а Джереми — преуменьшает в $b$ раз, а кот стоит $s$ рублей. Тогда, из условия задачи,

\begin{align*}
	& s \cdot a = 9600 \\
	& a \div b = 4 \\
	& s \cdot a \div b = 2400 \\
	& s \div b = 150 \\
\end{align*}

\vspace{-0.4cm}
Сравнив первое и третье равенства, получаем, что $b=4$. Подставив найденное $b$ в четвертое равенство, получим $s=600$.

\end{itemize}