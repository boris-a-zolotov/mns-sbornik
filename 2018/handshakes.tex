\task{Рукопожатия}
\begin{itemize}
\itA Давайте «расклеим» восьмёрку, превратив её в обычный круглый хоровод — тогда существо, стоящее в центре восьмёрки, «продублируется». Если оно было крабом, то получится хоровод из 19 крабов и 17 пауков\scolon в противном случае — 18 крабов и 18 пауков. Если в круговом хороводе крабов больше, чем пауков, то какие-то два краба неизбежно будут держаться за лапы, что запрещено.

\ms Отсюда можно заключить, что в центре стоял паук. Придумать хоровод, соответствующий условию, с пауком в центре не представляет ни малейшего труда.

\itB Могло оказаться так, что ровно один человек в компании выиграл машину. Построим соответствующий пример. Возьмём «победителя» — у него есть пять друзей. У каждого из них есть ещё по четыре друга (кроме выигравшего машину), пусть все эти друзья различны. $1+5+4 \cdot 5$ — у нас получилось 26 человек, от каждого из которых не более чем два рукопожатия до выигравшего машину человека.

\ms Однако, для того чтобы довести пример до конца, нам надо установить дружеские связи между людьми, у которых их пока меньше 5 — а именно, между теми, от кого до победителя лотереи два рукопожатия (их 20 человек). Каждому из них нужно «изобрести» ещё по 4 друга.

\ms Поступим просто: поставим эти 20 человек по кругу в произвольном порядке и назначим друзьями каждого двух его правых соседей и двух его левых соседей. Задача решена.

\itC Пусть внутренних рейсов в Авиаландии ровно $M$, а международных из неё — ровно $N$. Каждый внутренний рейс имеет в Авиаландии два «конца», а каждый международный — только один. Всего в города Авиаландии прибывает $5 \cdot 6 = 30$ рейсов. Получаем
	$$2 \cdot M + N = 30.$$

Отсюда $N$ должно быть чётным числом (так как $2 \cdot N$ — чётное).
\end{itemize}