\task{Где-то я это уже видел}
\begin{itemize}
\itA Первое число в дате (оно соответствует дню в месяце) меняется от 1 до 31, а второе (соответствует месяцу) — от 1 до 12. С другой сторооны, как мы знаем, часы пронумерованы от 0 до 23, а минуты — от 0 до 59.

\ms Таким образом, днём в месяце и одновременно часом могут быть числа от 1 до 23, а месяцем и одновременно минутой — от 1 до 12. Кроме того, в каждом месяце точно есть хотя бы 23 дня.

\ms Поэтому ответ — $23 \cdot 12 = 276$.

\itB Давайте всегда использовать «развёрнутую» дату. Тогда любой месяц (от 1 до 12) может стоять на месте часа, а любой день (от 1 до 31) на месте минуты. Ответ — все дни в году.

\itC Есть всего 12 букв русского алфавита, похожих на буквы английского алфавита (\href{http://base.garant.ru/12142212}{ГОСТ Р 50577-93}): \medskip

\centerline{{\underline А}, В, {\underline Е}, К, М, Н, {\underline О}, Р, С, Т, X, {\underline У}.}

\ms Жирным мы отметили гласные — их всего 4\scolon соответственно, согласных 8. Выбрать сочетание «гласная-согласная-согласная» можно $4 \cdot 8 \cdot 8$ способами, а «гласная-гласная-согласная» — $4 \cdot 4 \cdot 8$ способами. Вариантов для числа на номере всегда ровно 1000 — от 000 до 999.

\ms Когда гласная одна, она может стоять на одном из трёх мест, поэтому ответ в таком случае будет равен

$$3\,\cdot\,4\cdot8\cdot8\,\cdot\,1000.$$

\ms Когда гласных две, согласная может стоять на одном из трёх мест. Поэтому ответ —

$$3\,\cdot\,4\cdot4\cdot8\,\cdot\,1000.$$
\end{itemize}