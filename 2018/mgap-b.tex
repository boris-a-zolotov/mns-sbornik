\task{Ужасный гадкий аккуратный подсчёт}
\begin{itemize}
\itA \lookPrev{4}{6B}

\itB \lookPrev{4}{6C}

\itC Эта задача чуть сложнее \hyperref[mgap-ba]{пункта A}: нужно поделить $6!$ на число вращений куба. Сколько же их?

\ms Возьмём «верхнюю» грань куба. При вращении она может оказаться на месте одной из шести граней (включая себя). Теперь посмотрим на одну из граней, соседних с ней. При вращении та может перейти в одну из четырёх граней, соседних с той, на месте которой оказалась верхняя. Заметим, что положение этих двух граней (для которого есть ровно 24 варианта) однозначно определяет положение всех остальных. Поэтому ответ на задачу — $\tfrac{6!}{24} = 30$.
\end{itemize}