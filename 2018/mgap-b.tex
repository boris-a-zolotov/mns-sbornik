\task{Ужасный гадкий аккуратный подсчёт}
\begin{itemize}
\itA \label{mgap-ba} Всего раскрасок $n!$~— в «первом» секторе может стоять $n$ цветов, в следующем уже только $n-1$, и так далее. Из одной раскраски вращением круга можно получить ровно $n$ раскрасок (включая её саму)~— поэтому ответ равен $\tfrac{n!}{n} = (n-1)!$.

\itB В верхней полосе может стоять один из шести имеющихся цветов. Во второй полосе — любой из шести цветов, кроме уже стоящего в первой. В нижней полосе — любой из цветов, кроме уже стоящего во второй. Таким образом, ответ — $6 \cdot 5 \cdot 5 = 150$.

\itC Эта задача чуть сложнее \hyperref[mgap-ba]{пункта A}: нужно поделить $6!$ на число вращений куба. Сколько же их?

\ms Возьмём «верхнюю» грань куба. При вращении она может оказаться на месте одной из шести граней (включая себя). Теперь посмотрим на одну из граней, соседних с ней. При вращении та может перейти в одну из четырёх граней, соседних с той, на месте которой оказалась верхняя. Заметим, что положение этих двух граней (для которого есть ровно 24 варианта) однозначно определяет положение всех остальных. Поэтому ответ на задачу — $\tfrac{6!}{24} = 30$.
\end{itemize}