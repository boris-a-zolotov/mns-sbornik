\task{Сетки на плоскости}
\begin{itemize}

\itA Заметим, что рёбра на пути можно менять местами без изменения начала, конца и длины пути. Заметим также, что по рёбрам каждого из трёх направлений в сетке кратчайший путь ходит максимум в одну сторону, потому что иначе «подвинем» противоположно направленные проходы друг к другу и сократим их, укоротив путь.

Пусть путь использовал все три сорта рёбер в сетке. Тогда мы переставим его рёбра так, что сначала он будет идти в одну сторону по рёбрам первого сорта, затем по рёбрам второго, затем по рёбрам третьего.

Теперь возьмём треугольник и проставим на его сторонах те направления, в которых мы ходим по рёбрам соответствующей ориентации. Получилось три вектора — заметим, что либо один из них равен сумме двух других, либо их сумма равна нулю.

В первом случае можно переставить рёбра в пути так, чтобы проход по двум подряд идущим рёбрам превратить в проход по одному. То есть, путь не был кратчайшим.

Во втором случае можно переставить рёбра так, чтобы, не изменяя начала и конца пути, выкинуть из него три ребра. Тогда он тем более не был кратчайшим.

\itB Окуню достаточно перегрызть два узла, соседних с углом сетки. Если же в сетке нет даже двух рядов узлов, то никакая это не сетка.

\itC Рассмотрим четырёхугольник со сторонами $a$, $b$, $c$, $d$ и углами $\alpha_{ab}$, $\alpha_{bc}$, $\alpha_{cd}$, $\alpha_{da}$ (две буквы в индексе обозначают прилежащие стороны угла). Приложим к стороне $a$ такой же четырёхугольник так, чтобы в каждой из двух общих вершин двух четырёхугольников оказалось по углу, равному $\alpha_{ab}$ и $\alpha_{da}$. Сделаем то же на сторонах $b$, $c$ и $d$, а затем с каждым из четырёх вновь приложенных четырёхугольников.

Заметим, что тогда по построению в каждой из имеющихся вершин сойдутся все четыре угла четырёхугольника — в частности из-за того, что сумма углов треугольника равна $360^\circ$:

\begin{center} \tikz{
	\foreach \x in {10,80,190,310} \draw (0,0) -- (\x:1.7);
	\draw (45:0.7) node{$\alpha_{ab}$};
	\draw (135:0.7) node{$\alpha_{bc}$};
	\draw (250:0.7) node{$\alpha_{cd}$};
	\draw (340:0.7) node{$\alpha_{da}$};
	\draw (3:1.35) node{$a$};
	\draw (73:1.35) node{$b$};
	\draw (183:1.35) node{$c$};
	\draw (303:1.35) node{$d$};
} \end{center}

Продолжая описанную выше процедуру прикоадывания, мы замостим всю плоскость.
\end{itemize}