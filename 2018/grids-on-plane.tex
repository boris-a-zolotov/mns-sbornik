\task{Сетки на плоскости}
\begin{itemize}

\itA Заметим, что рёбра на пути можно менять местами без изменения начала, конца и длины пути. Заметим также, что по рёбрам каждого из трёх направлений в сетке кратчайший путь ходит максимум в одну сторону, потому что иначе «подвинем» противоположно направленные проходы друг к другу и сократим их, укоротив путь.

Пусть путь использовал все три сорта рёбер в сетке. Тогда мы переставим его рёбра так, что сначала он будет идти в одну сторону по рёбрам первого сорта, затем по рёбрам второго, затем по рёбрам третьего.

Теперь возьмём треугольник и проставим на его сторонах те направления, в которых мы ходим по рёбрам соответствующей ориентации. Получилось три вектора — заметим теперь, что один из них равен сумме других!

Если кратчайший путь действительно использовал все три ориентации рёбер, то можно переставить рёбра в этом пути так, чтобы проход по двум подряд идущим рёбрам превратить в проход по одному. То есть, путь не был кратчайшим.

\itB Окуню достаточно перегрызть три узла около угла сетки.

\itC Сумма углов четырёхугольника равна $360^\circ$. Искомое замощение плоскости получится, если складывать четырёхугольники так, чтобы в одной вершине сходились четыре угла соседних четырёхугольников и чтобы два соседних четырёхугольника всегда соприкасались по соответственной стороне соответственными вершинами.
\end{itemize}