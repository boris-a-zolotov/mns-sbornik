\task{Не модельная, а модальная!}
\def\sq{\square}
\def\td{\triangledown}

\begin{itemize}
\itA Фраза $\sq\td X$ означает дословно следующее: для каждого дня, начиная с сегодняшнего, в какой-то момент после него случится событие $X$. То есть, из какого дня вперед ни посмотри — там, в будущем, обязательно хотя бы единожды случится событие $X$. На самом деле эта фраза эквивалентна следующей: «в бесконечное количество дней после сегодняшнего произойдет событие $X$».

Очевидно, что $\sq\td \text{ сегодня суббота}$ — верно: после любого дня ког-\linebreak да-то в будущем обязательно наступит суббота.

\itB \label{logicb} Докажем, что из первой фразы следует вторая. Действительно: первая утверждает, что $\td\sq X$ верно для любого дня, начиная с сегодняшнего — в том числе и для сегодняшнего.

Теперь докажем, что из второй фразы следует первая. Вторая фраза означает: начиная с какого-то дня в будущем (назовем его {\itshape\bfseries D}) каждый день будет происходить событие $X$. Зная это, нам нужно доказать $\sq\td\sq X$: для каждого дня $d$ указать такой день после него, начиная с которого $X$ выполняется каждый день.

Так вот если $d$ раньше {\itshape\bfseries D}, то {\itshape\bfseries D} подойдет в качестве искомого дня. Если же {\itshape\bfseries D} раньше $d$, то после самого $d$ событие $X$ выполняется каждый день — возьмем $d$ в качестве искомого дня.

\itC Легко убедиться, что $\sq X$, $\td X$, $\sq\td X$ и $\td\sq X$ — попарно неэквивалентные фразы. Пусть $X_1$ — «сегодня не 1 января 2000 года», $X_2$ — «сегодня у Пети Иванова последний звонок в школе», $X_3$ — «сегодня День рождения Пети Иванова», $X_4$ — «Пете Иванову уже исполнилось 18 лет»; достаточно проверить, что все $X_i$ делают верными разные наборы утверждений.

Теперь докажем, что любая фраза с другой приставкой из $\sq$ и $\td$ эквивалентна одной из приведенных ранее. Понятно, что $\sq\sq$ и $\td\td$ в любом месте приставки можно заменить на соответственно $\sq$ и $\td$ без изменения смысла фразы. Значит, мы можем рассматривать только фразы, в приставке которых идет не более одного квадратика / треугольничка подряд.

Согласно \hyperref[logicb]{пункту B}, $\sq\td\sq$ можно заменить на $\td\sq$ без изменения\linebreak смысла фразы. Аналогично, $\td\sq\td$ можно заменить на $\sq\td$. Поэтому любую приставку мы можем сократить до содержащей не более двух символов — а все такие мы уже перечислили.

\end{itemize}