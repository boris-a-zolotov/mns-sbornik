\task{Прогрессивное сложение}
\begin{itemize}
\itA $95500 > 50095$.

\itB Если ни одно из трёх чисел $P$, $Q$, $R$ не является префиксом другого, то всё просто: надо отсортировать числа лексикографически и сложить в порядке «от большего к меньшему». Если одно из чисел — префикс другого (например, $P$ — префикс $Q$), то всё не так однозначно: надо сравнить их общую первую цифру и первую цирфу $Q$, следующую за вхождением $P$ в $Q$. Если второе больше, то надо ставить $Q$ перед $P$, иначе — $P$ перед $Q$.

\ms Если $P$ — префикс $Q$, которое, в свю очередь, является префиксом $R$, или $P$ и $Q$ — различные префиксы $R$, действовать следует аналогично.

\itC Нет, так не бывает:

$$P \oplus Q\ =\ P \cdot \underset{n\geq1}{10^n} + Q\ >\ P+Q.$$
\end{itemize}