\task{Прогрессивное сложение}
\begin{itemize}
\itA $95500 > 50095$.

\itB Заметим, что есть всего шесть вариантов расстановки чисел $P$, $Q$, $R$ — поэтому перебор их всех и сравнение результатов уже является не таким плохим вариантом алгоритма. Однако мы попробуем продемонстрировать ещё более рациональную идею.

Если числа $P$, $Q$, $R$ имеют одинаковое число разрядов или по крайней мере ни одно из них не является префиксом другого, то всё просто: надо отсортировать числа лексикографически и сложить в порядке «от большего к меньшему».

Если же ни одно из условий выше не выполнено, попытаемся сделать так, чтобы нам пришлось сравнивать числа с одинаковым числом разрядов. Для этого нам нужно как-то «дополнить» более короткие числа. Давайте для каждого числа, которое короче самого длинного из набора $P$, $Q$, $R$, выберем из оставшихся число, начинающееся с самой большой цифры (или наибольшее лексикографически) и припишем в конец исходного числа несколько первых разрядов выбранного.

Так мы получили несколько строк одинаковой длины, которые можно отсортировать лексикографически — и сложить числа в том порядке, в котором высторились получившиеся из них строчки.

Возьмём, например, числа 59, 598, 5979. Они разной длины, и 59 является префиксом всех остальных. По указанному выше правилу из этих чисел получатся строчки
	$$59\underline{59},\quad 598\underline{5},\quad 5979.$$

После упорядочения их получим 5985, 5979, 5959. Поэтому максимальный результат, который может получится при сложении данных трёх чисел, —
	$$598 \oplus 5979 \oplus 59 = 598597959.$$

\itC Нет, так не бывает:

$$P \oplus Q\ =\ P \cdot \underset{n\geq1}{10^n} + Q\ >\ P+Q.$$
\end{itemize}