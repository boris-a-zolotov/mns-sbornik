\task{Катим круг}
\begin{itemize}
\itA «Расправим» большую окружность — заметим, что дуга, составляющая её половину, равна по длине окружности круга, который мы катим. Это значит, что, проехав эту дугу, круг снова коснётся её отмеченной точкой.

\begin{center}
	\tikz{
		\draw (0,-2) arc (-90:195:2)
			node[circle,fill,inner sep=0.6mm,label=below:{ }](c){}
			arc (195:270:2);
		\draw (0,-2) node[circle,fill,inner sep=0.6mm,label=below:$P$](){};
		\draw (0,2) node[circle,fill,inner sep=0.6mm,label=below:$M$](){};
		\begin{scope}[rotate=75]
			\draw (0,-2) arc (-90:120:1)
				node[circle,fill,inner sep=0.6mm,label=below:{ }](a){}
				arc (120:270:1);
			\draw (0,-2) node[circle,fill,inner sep=0.6mm,label=below:{ }](b){};
		\end{scope}
		\draw[thick] (c) -- ++(-1,0) -- (a) -- (b) -- ++(1,0);
		\draw (1,-1.92) node {$x$};
		\draw (-1,-1.92) node {$x$};
		\draw (1,-1.15) node {$x$};
	}
\end{center}

\itB Длина дуги круга между точкой его касания с окружностью и отмеченной точкой равна длине дуги окружности между точкой её касания с кругом и точкой $P$. Обозначим эту длину через $x$. Отложим дугу длиной $x$ налево от точки $P$ (смотреть рисунок). Отрезок, соединяющий точку касания круга с окружностью и конец новой дуги, будет горизонтальным.

При этом дуга длины $x$ на круге получается из дуги длины $x$ на ок-\linebreak ружности гомотетией с коэффициентом $\tfrac{1}{2}$ и центром в точке касания круга с окружностью: эти дуги отложены из одной точки на окружностях, радиусы которых отличаются в два раза.

Горизонтальный отрезок переходит при гомотетии в горизонтальный отрезок — поэтому концы дуги на круге также лежат на одной горизонтальной прямой.

\itC В силу того же факта, что дуга на квадрате получается из дуги на окружности гомотетией с коэффициентом $\tfrac{1}{2}$, её конец будет находится ровно посередине между концами большой дуги — то есть ровно над точкой $P$, потому как большая дуга изначально строилась симметричной.
\end{itemize}