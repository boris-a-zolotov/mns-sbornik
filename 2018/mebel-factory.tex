\task{Современная мебельная фабрика}
\begin{itemize}

\itA Закроем один из открытых ящиков, открыв тот, что через два ящика «налево» от него. Затем закроем его, открыв следующий, еще через два ящика слева. Повторим то же действие еще раз. На четвертом шаге мы закроем оба открытых ящика, тем самым решив полученную задачу.

\itB При первом сценарии после действий Федора в ведре осталось $\tfrac{6}{10}$ красителя, разведенного там Сергеем, так как 4 литра раствора из 10 были вылиты.

При втором сценарии Федор сначала выливал обычный раствор, а затем — раствор с меньшей концентрацией красителя. То есть, количества красителя в ведре до выливания двух литров и после отличались в $0.8$ раз. В итоге в ведре осталось $\tfrac{8}{10} \cdot \tfrac{8}{10} = 0.64$ от исходного красителя.

Ответ: больше красителя осталось во второй день.

\itC Пусть стул потерял $t$ ножек. Составим уравнение:

$$t = \frac{1}{3} \cdot \biggl(
\underbrace{720 -
	\underbrace{\frac{1}{3} \cdot
		\underbrace{\ll 720-t\rr}_{\begin{minipage}{1.55cm}
			\footnotesize чем у него осталось сейчас
		\end{minipage}}}_{\begin{minipage}{2.5cm}
			\footnotesize в три раза меньше ножек
		\end{minipage}}}_{\begin{minipage}{3.2cm}
			\footnotesize осталось бы, потеряй он
		\end{minipage}}
\biggr)$$

Это линейное уравнение. Его решением является число $t=180$.
\end{itemize}