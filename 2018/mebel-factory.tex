\task{Современная мебельная фабрика}
\begin{itemize}

\itA Закроем один из открытых ящиков, открыв тот, что через два ящика «налево» от него. Затем закроем его, открыв следующий, ещё через два ящика слева. На четвёртом шаге мы закроем два ящика, один из которых был противоположным исходному.

\itB При первом сценарии после действий Фёдора в ведре осталось $\tfrac{6}{10}$ красителя, разведённого там Сергеем, так как 4 литра раствора из 10 были вылиты.

При втором сценарии Фёдор сначала выливал обычный раствор, а затем — раствор с меньшей концентрацией красителя. То есть, количества красителя в ведре до выливания двух литров и после отличались в $0.8$ раз. В итоге в ведре осталось $\tfrac{8}{10} \cdot \tfrac{8}{10} = 0.64$ от исходного красителя.

Ответ: больше красителя осталось во второй день.

\itC Экспериментальный стул с использованием нанотехнологий (одна из инноваций заключается, например, в том, что у такого стула ровно 720 ножек) падает с лестницы (в качестве испытания, разумеется). Выяснилось, что при падении он потерял в три раза меньше ножек, чем у него бы осталось, потеряй он в три раза меньше ножек, чем у него осталось сейчас. Так сколько же ножек осталось у стула?

Пусть стул потерял $t$ ножек. Составим уравнение:

$$t = \frac{1}{3} \cdot \biggl(
\underbrace{720 -
	\underbrace{\frac{1}{3} \cdot
		\underbrace{\ll 720-t\rr}_{\begin{minipage}{1.55cm}
			\footnotesize чем у него осталось сейчас
		\end{minipage}}}_{\begin{minipage}{2.5cm}
			\footnotesize в три раза меньше ножек
		\end{minipage}}}_{\begin{minipage}{4cm}
			\footnotesize осталось бы, потеряй он
		\end{minipage}}
\biggr)$$

Это линейное уравнение. Его решение — $t=180$. Это и есть ответ на данную задачу.
\end{itemize}