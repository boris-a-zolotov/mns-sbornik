\task{Игры}
\begin{itemize}
\itA Выигрышная стратегия есть у первого игрока: превым ходомон должен положить игровую фигуру в центр прямоугольника, разделив его на две равных непересекающихся фигуры — а затем ему достаточно повторять ходы, сделанные вторым игроком в одной из фигур, симметрично в другой фигуре. Понятно, что если второй смог сделать ход, то первый тоже сможет.

\itB Выигрышная стратегия есть у второго игрока: пока он находится сзади, первый может делать только ходы по одному шагу вперёд. Второму надо следовать за ним — и когда тот шагнёт в клетку №301 (второй будет стоять в 229-ой), перепрыгнуть через него и выиграть.

\itC Первому надо вырезать свою букву `Г' по центру верхней стороны квадрата. То, что он выигрывает после этого, доказывается перебором возможных ходов второго игрока: после каждого возможного хода первый игрок может походить так, что буква `L' второго игрока не может быть вписана никуда.
\end{itemize}