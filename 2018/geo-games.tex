\task{Игры}
\begin{itemize}
\itA Выигрышная стратегия есть у первого игрока: первым ходом он должен положить игровую фигуру в центр прямоугольника, разделив его на две равных непересекающихся фигуры — а затем ему достаточно повторять ходы, сделанные вторым игроком в одной из фигур, симметрично в другой фигуре. Понятно, что если второй смог сделать ход, то первый тоже сможет.

\itB Выигрышная стратегия есть у второго игрока: пока он находится сзади, первый может делать только ходы по одному шагу вперёд. Второму надо следовать за ним — и когда тот шагнёт в клетку №301 (второй будет стоять в 229-ой), перепрыгнуть через него и выиграть.

\itC Первому надо вырезать свою букву `Г' по центру верхней стороны квадрата. То, что он выигрывает после этого, доказывается перебором возможных ходов второго игрока (смотреть рисунок): после каждого возможного хода первый игрок может походить так, что буква `L' второго игрока не может быть вписана никуда.

\vspace{0.4cm}
\def\mov#1#2{\begin{scope}[xshift = #1 cm] #2 \end{scope};}

\definecolor{hoda}{RGB}{165,165,165}
\definecolor{hodb}{RGB}{225,225,225}

\def\lshape#1#2{
	\filldraw[fill=hoda,draw=hoda] (#1, #2 - 1.4) rectangle (#1 + 0.7, #2);
	\filldraw[fill=hoda,draw=hoda] (#1, #2 - 1.4) rectangle (#1 + 1.4,#2 - 0.7);
	\draw[very thick] (#1,#2) -- ++(0,-1.4) -- ++(1.4,0) --
		++(0,0.7) -- ++(-0.7,0) -- ++(0,0.7) -- ++(-0.7,0);
}

\def\gshape#1#2{
	\filldraw[fill=hodb,draw=hodb] (#1, #2 - 1.4) rectangle (#1 + 0.7, #2);
	\filldraw[fill=hodb,draw=hodb] (#1, #2 - 0.7) rectangle (#1 + 1.4,#2);
	\draw[very thick] (#1,#2) -- ++(0,-1.4) -- ++(0.7,0) --
		++(0,0.7) -- ++(0.7,0) -- ++(0,0.7) -- ++(-1.4,0);
	\draw
}

\begin{center}
	\tikz{
		\foreach \x in {-3.5,0,3.5}
			\mov{\x}{
			  \draw[thick] (-1.4,-1.4) -- (-1.4,1.4) -- (1.4,1.4) -- (1.4,-1.4) -- cycle;
			  \foreach \t in {-0.7,0,0.7} {
			  	\draw[color=gray] (\t,-1.4) -- (\t,1.4);
			  	\draw[color=gray] (-1.4,\t) -- (1.4,\t);
			  } \gshape{-0.7}{1.4};
			};
		\mov{-3.5}{\lshape{-1.4}{0.7}; \gshape{0}{0};}
		\mov{0}{\lshape{-1.4}{0}; \gshape{0}{0};}
		\mov{3.5}{\lshape{-0.7}{0}; \gshape{0}{0.7};}
	} \\
\ \\
	\tikz{
		\foreach \x in {-1.75,1.75}
			\mov{\x}{
			  \draw[thick] (-1.4,-1.4) -- (-1.4,1.4) -- (1.4,1.4) -- (1.4,-1.4) -- cycle;
			  \foreach \t in {-0.7,0,0.7} {
			  	\draw[color=gray] (\t,-1.4) -- (\t,1.4);
			  	\draw[color=gray] (-1.4,\t) -- (1.4,\t);
			  } \gshape{-0.7}{1.4};
			};
		\mov{-1.75}{\lshape{0}{0.7}; \gshape{-1.4}{0};}
		\mov{1.75}{\lshape{0}{0}; \gshape{-1.4}{0};}
	}
\end{center}

\end{itemize}