\task{Одновременное вычитание}
\begin{itemize}
\itA Возьмём пять чисел: 0, 0, 0, 0, 6. Очевидно, для них мы не можем добиться того, чего просят в задаче, потому что по факту можем уменьшать только одно число.

\itB {\it Примечание автора:} эта задача на самом деле о том, что первая группа гомологий плоскости равна $\{0\}$. {\it :)}

Рассмотрим точку с весом, наибольшим по модулю. Не умаляя общности предположим, что её вес положителен. Так как сумма весов всех точек равна нулю, найдутся какие-то точки с отрицательным весом, суммарный вес которых «перевести» нашу по модулю. Соединим выбранную точку с найденными с помощью кривых так, чтобы (а) вес выбранной точки обратился после этого в ноль (б) модули весов найденных точек не увеличились.

Таким образом, (а) модули весов всех точек не увеличились (б) количество точек с нулевым весом увеличилось хотя бы на одну. На каждом шаге, при повторении процедуры, описанной в предыдущем абзаце, эти полуинварианты будут сохраняться — поэтому мы добьёмся ситуации, когда вес всех точек окажется нулевым (сумма весов всех точек сохраняется на каждом шаге).

\itC Возьмём дорогу с наименьшим весом и пустим по ней машину, на номере которой написан вес этой дороги. Когда машина въедет в какой-то город, она сможет из него выехать: её номер равен наименьшему среди всех весов дорог, а сумма входящих в город равна сумме исходящих — поэтому из города выходит дорога весом не меньше, чем число на номере машины.

Так машина будет ездить по городам, пока не окажется в городе, в котором она уже побывала. Тогда возьмём все дороги, по которым машина ездила между двумя посещениями этого города, и вычтем из их веса число на номере машины — и заставим машину ездить по кругу через эти города. При этом сумма весов всех дорог строго уменьшится.

Опять возьмём дорогу, вес которой на этот раз наименьший среди всех, и повторим описанную процедуру. Пока наименьший среди всех вес дороги не равен нулю (то есть, пока есть дороги с положительным весом), будем повторять эту процедуру. Очевидно, в итоге оставшиеся веса всех дорог обратятся в ноль.
\end{itemize}