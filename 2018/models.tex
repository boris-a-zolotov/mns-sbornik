\task{Модельки}
\begin{itemize}
\itA Свойства подобных фигур говорят нам, что объем фигур, подобных с коэффициентом $k$, различается в $k^3$ раз. Масса тела равна плотности вещества, умноженной на его объем — поэтому она также должна уменьшаться в $k^3$ раз при уменьшении тела в $k$ раз.

В свою очередь, $1200\,/\,43^3 \approx 0.015$. В реальность модельки несколько тяжелее, но это вполне объяснимо: сделаны они все-таки грубее, чем оригинальная машина, и металл в них сравнительно более толстый.

\itB Мы хотели бы отметить, что длина меридиана, \SI{40000}{\text{километров}}, это {\bfseries вся окружность} Земли, а не ее половина. То есть Парижский меридиан проходит через две долготы: \SI{2.33^\circ}{\text{в.\,д.}} и \SI{177.67^\circ}{\text{з.\,д.}}

Таким образом, самолету нужно пролететь \SI{40000}{\text{км}}, затрачивая на километр $0.54$ минуты. $40000 \cdot 0.54 \div 60 = \SI{360}{(\text{часов})}$.

\itC С одной стороны, если есть «классическая» плоская система из шестеренок, то в ней передача вращения симметрична. С другой — можно с применением некоторой креативности придумать «несимметричную» систему. Например, такую, как на рисунке:

\begin{center}
\tikz{
\draw[thick]
\foreach \i in {1,2,...,10} {%
[rotate=(\i-1)*36] 
(0:1) arc (0:18:1) {[rounded corners=2pt] -- ++(18: 0.15) arc (18:36:1.15) } -- ++(36: -0.15)
}; 
\draw[thick,xshift=-2.25cm,rotate=72](0:1) arc (0:18:1) 
  {[rounded corners=2pt] -- ++(18: 0.15) arc (18:36:1.15) } -- ++(36: -0.15) arc (36:360:1); 
\draw (-2.25,0) node {$A$}; 
\draw (0,0) node {$B$}; 
}

%% \includegraphics[natwidth=2560,natheight=2011,width=6cm]{figures/2018-models}
\end{center}

При вращении шестеренки $A$ она каждый оборот будет цепляться своим единственным зубом за шестеренку $B$, и та будет вращаться. При вращении же шестеренки $B$ в текущем положении шестеренок она не будет касаться $A$ и передавать ей вращение.

\end{itemize}