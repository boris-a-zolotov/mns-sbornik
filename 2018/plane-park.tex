\task{Мощения}

\begin{itemize}
\itA Из этой фигуры можно собрать горизонтальную полоску ширины 2, которой очевидно можно замостить плоскость (смотреть рисунок).

\begin{center}
	\tikz{
		\foreach \x in {-2,0,2}
			\draw[thick] (\x, 0) -- ++(1.5,0) -- ++(0,0.5) -- ++(-0.5,0)
				-- ++(0,0.5) -- ++(-0.5,0)
				-- ++(0,-0.5) -- ++(-0.5,0)
				-- ++(0,-0.5) -- ++(-0.5,0) -- ++(0,0.5)
				-- ++(-0.5,0) -- ++(0,0.5) -- ++(1.5,0);
			\draw (-3.7,0.5) node{$\ldots$};
			\draw (4.2,0.5) node{$\ldots$};
	}
\end{center}

\itB Из второй фигуры можно собрать горизонтальную полоску ширины 3, которой очевидно можно замостить плоскость. Из первой же фигуры соберём «лесенку» (смотреть рисунок выше): так как и верхний, и нижний её край имеет вид «на три клетки вправо–на клетку вверх», этой лесенкой можно замостить плоскость, прикладывая её к себе.

\itC Смотреть рисунок:
\begin{center}
	\tikz{
		\foreach \y in {2.5,1}
			\draw[very thick] (1, \y) -- ++(0,0.5) -- ++(-1,0) -- ++(0,-1) -- ++(0.5,0)
				-- ++(0,0.5) -- ++(0.5,0) -- ++(0,-1) -- ++(-1,0) -- +(0,0.5);
		\foreach \y in {0,1,2}
			\draw[very thick] (1.5, \y) -- ++(-0.5,0) -- ++(0,1) -- ++(1.5,0) -- ++(0,-0.5)
				-- ++(-1,0) -- ++(0,-0.5) -- ++(1.5,0) -- ++(0,1) -- ++(-0.5,0);
	}
\end{center}
\end{itemize}