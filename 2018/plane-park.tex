\task{Мощения}

\begin{itemize}
\itA Из этой фигуры можно собрать горизонтальную полоску ширины 2, которой очевидно можно замостить плоскость (смотреть рисунок).

\begin{center}
	\tikz{
		\foreach \x in {-2,0,2}
			\draw[thick] (\x, 0) -- ++(1.5,0) -- ++(0,0.5) -- ++(-0.5,0)
				-- ++(0,0.5) -- ++(-0.5,0)
				-- ++(0,-0.5) -- ++(-0.5,0)
				-- ++(0,-0.5) -- ++(-0.5,0) -- ++(0,0.5)
				-- ++(-0.5,0) -- ++(0,0.5) -- ++(1.5,0); 
			\draw (-3.7,0.5) node{$\ldots$}; 
			\draw (4.2,0.5) node{$\ldots$}; 
	}
\end{center}

\itB Из второй фигуры можно собрать горизонтальную полоску шириной 3, которой очевидно можно замостить плоскость. Из первой же фигуры соберем «лесенку» (смотреть рисунок ниже): так как и верхний, и нижний ее край имеет вид «на три клетки вправо–на клетку вверх», этой лесенкой можно замостить плоскость, прикладывая край следующего её экземпляра к краю предыдущего.

\begin{center}
	\tikz{ \foreach \x in {-2,0,2} {\begin{scope}[xshift = \x cm]
		\draw[thick] (0,0) -- ++(0.5,0) -- ++(0,1) -- ++(-2,0) -- ++(0,-1)
			-- ++(0.5,0) -- ++(0,0.5) -- ++(1,0) -- ++(0,-0.5); 
		\draw[thick,rotate=180,yshift=-0.5 cm,xshift=2 cm]
			(0,0) -- ++(0.5,0) -- ++(0,1) -- ++(-2,0) -- ++(0,-1)
			-- ++(0.5,0) -- ++(0,0.5) -- ++(1,0) -- ++(0,-0.5); 
	\end{scope}}
			\draw (-5,0.25) node{$\ldots$}; 
			\draw (3,0.25) node{$\ldots$}; 
	}
\end{center}

\medskip

\begin{center} \tikz{
\begin{scope}[rotate=-18.434]
		\draw[thick] (0,0) -- ++(0.5,0) -- ++(0,1) -- ++(-1.5,0) -- ++(0,-1)
			-- ++(0.5,0) -- ++(0,0.5) -- ++(0.5,0) -- ++(0,-0.5); 
		\draw[rotate=180,yshift=-0.5 cm,thick]
			(0,0) -- ++(0.5,0) -- ++(0,1) -- ++(-1.5,0) -- ++(0,-1)
			-- ++(0.5,0) -- ++(0,0.5) -- ++(0.5,0) -- ++(0,-0.5); 
\foreach \y in {-2,-1,1,2} {\begin{scope}[xshift = 1.5*\y cm, yshift = 0.5*\y cm]
		\draw[thick]
			(0.5,0.5) -- ++(0,0.5) -- ++(-1.5,0) -- ++(0,-1)
			-- ++(0.5,0) -- ++(0,-0.5) -- ++(1.5,0) -- ++(0,1)
			-- cycle; 
		\draw[thick,dashed] (0.5,0.5) -- ++(0,-0.5)
			-- ++(-0.5,0) -- ++(0,0.5) -- ++(-0.5,0) -- ++(0,-0.5); 
\end{scope}}
\end{scope}
	\draw (-4.6,0) node{$\ldots$}; 
	\draw (4.75,0) node{$\ldots$}; 
} \end{center}

\itC Смотреть рисунок:
\begin{center}
	\tikz{
		\foreach \y in {2.5,1}
			\draw[very thick] (1, \y) -- ++(0,0.5) -- ++(-1,0) -- ++(0,-1) -- ++(0.5,0)
				-- ++(0,0.5) -- ++(0.5,0) -- ++(0,-1) -- ++(-1,0) -- +(0,0.5); 
		\foreach \y in {0,1,2}
			\draw[very thick] (1.5, \y) -- ++(-0.5,0) -- ++(0,1) -- ++(1.5,0)
			-- ++(0,-0.5) -- ++(-1,0) -- ++(0,-0.5) -- ++(1.5,0)
			-- ++(0,1) -- ++(-0.5,0); 
	}
\end{center}
\end{itemize}