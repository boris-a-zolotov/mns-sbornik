\task{Напрасно называют север крайним}
\begin{itemize}
\itA Это задача–шутка: принималось большинство ответов, хотя наверняка на 10--градусном морозе туристическая группа отморозит себе половину ног, а на 20--градусном — все.

\itB Все долготы Земного шара оказываются очень близко друг к другу около полюсов. Так что, возможно, Мюнхгаузен просто обошёл по кругу (скажем, километровому) Северный или Южный полюс.

\itC Пусть четыре города — $B$, $C$, $D$ и $E$ расположены очень близко друг к другу — попарно на расстоянии в один километр. А пятый город~— $A$~— очень далеко, в 100 километрах. Пусть больше нет никаких городов. Тогда $A$ должен быть соединён дорогой с какими-то из  четырёх оставшихся городов, но ни один из тех городов не должен быть соединён с $A$.
\end{itemize}