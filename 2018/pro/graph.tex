\task{Изображения на плоскости}

Первый пункт задачи содержит простые технические утверждения, поэтому мы сразу начнём со второго.

\begin{enumerate}
\setcounter{enumi}{1}

\item Координаты $x$ и $y$ не могут одновременно быть по модулю меньше единицы, чтобы неравенство из условия было выполнено. Фигура, состоящая из точек, обе координаты которых по модулю меньше единицы, — это квадрат со стороной 2 и центром в начале координат, не включающий свою границу. Значит, искомое множество точек — дополнение этого квадрата (включая его границу).

\item Видно, что принадлежность точки множеству на первом рисунке зависит только от её второй координаты. Будем подбирать неравенство в виде $P(y) \le 0$, где $P$ — многочлен от одной переменной. Мы должны получить $P(y) \le 0$, когда $-1 \le y \le 1$.

Принимая в рассмотрение первый пункт этой же задачи, можно понять, что $(y+1)(y-1)$ — именно такой многочлен. Ответ:

\begin{enumerate}
	\item[а)] $(y+1)(y-1) \le 0$.
\end{enumerate}

Легко понять, что нижняя наклонная прямая, ограничивающая фигуру на втором рисунке, задаётся уравнением $y-x+1=0$\scolon верхняя же прямая — уравнением $y-x-1=0$. Между этими прямыми находится множество точек, для которых выражения $y-x+1$ и $y-x-1$ имеют разный знак: первое уже «успело» стать положительным, а второе ещё нет. Такое множество задаётся неравенством:

\begin{enumerate}
	\item[б)] $(y-x+1)(y-x-1) \le 0$.
\end{enumerate}

\item Неравенство для крестика удобно искать в виде $P(x) \cdot Q(y) \le 0$, где $P(x) \le 0$ — неравенство, задающее вертикальную полосу, а $Q(y) \le 0$ — горизонтальную полосу. Тогда в пересечении полос $P(x) \cdot Q(y)$ будет больше нуля, что исключит это пересечение из получаемой фигуры.

Как задавать полоску от $-1$ до $1$, мы знаем, поэтому сразу получаем ответ:

\begin{enumerate}
	\item[а)] $(x+1)(x-1) \cdot (y+1)(y-1) \le 0$.
\end{enumerate}

Для решения второго подпункта вспомним, что такое круг: это множество точек, расстояние от которых до выбранной меньше радиуса. Иными словами,
$$(x-x_0)^2+(y-y_0)^2 - R^2 \le 0.$$

В нашем случае $x_0 = y_0 = 0$, $R=1$. Верхняя полуплоскость, в свою очередь, задаётся уравнением $x-y \le 0$. Наша фигура тогда — множество точек, где ровно одно из двух выражений, указанных выше, не превосходит нуля. Ответ —

\begin{enumerate}
	\item[б)] $(x-y)(x^2 + y^2 -1) \le 0.$
\end{enumerate}

\item Мы уже знаем, каким неравенством задаются круги. Закрашенная фигура на рисунке — множество точек, где выполнено нечётное количество (1 или 3) неравенств, задающих круги. Значит, ответ:

\begin{align*}
	& \ll \ll x-1 \rr ^2 + \ll y+1 \rr ^2 - 2.25 \rr \cdot \\
	\cdot & \ll \ll x+1 \rr ^2 + \ll y+1 \rr ^2 - 2.25 \rr \cdot \\
	\cdot & \ll x^2 + \ll y-0.5 \rr ^2 - 2.25 \rr \le 0.
\end{align*}

\item Окружность задаётся уравнением
$$(x-x_0)^2+(y-y_0)^2 - R^2 = 0.$$

Фигура из трёх окружностей — это множество точек, для которых хотя бы одно из выражений, задающих окружность, обращается в ноль. Отсюда ответ:

\begin{enumerate}
	\item[а)]
	\begin{align*}
		& \ll x^2 + y^2 - 4 \rr \cdot \\
		\cdot & \ll \ll x+1 \rr ^2 + y^2 - 1 \rr \cdot \\
		\cdot & \ll \ll x-1 \rr ^2 + y^2 - 1 \rr = 0.
	\end{align*}
\end{enumerate}

\vspace{-0.4cm}
Для фигуры из трёх лучей заметим, что два из них — горизонтальный и направленный вверх — образуют график функции $y = 0.5 |x| + 0.5x$. Соответственно, горизонтальный и направленный вниз образуют график функции $y = -0.5 |x| - 0.5x$. Нам достаточно, чтобы для точки $(x,y)$ было выполнено хотя бы одно из этих условий. Отсюда ответ —

\begin{enumerate}
	\item[б)] $(0.5 |x| + 0.5 x - y)(- 0.5 |x| - 0.5 x - y) = 0.$
\end{enumerate}

\item При решении этого задания мы уже много раз пересекали и объединяли фигуры, поэтому ответ понятен без пояснений:

\begin{enumerate}
	\item[а)] $\max \ll \left| P_1 (x,y) \right|, \left| P_2 (x,y) \right|\rr = 0$\scolon
	\item[б)] $P_1(x,y) \cdot P_2(x,y) = 0$.
\end{enumerate}

\item \begin{enumerate}
	\item[а)] $\max \ll P_1(x,y), P_2(x,y) \rr < 0$\scolon 
	\item[б)] $\min \ll P_1(x,y), P_2(x,y) \rr < 0$\scolon
	\item[в)] $P_1(x,y) \cdot P_2(x,y) < 0$.
\end{enumerate}

\end{enumerate}