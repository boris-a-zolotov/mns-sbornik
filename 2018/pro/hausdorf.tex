\task{Расстояние между множествами}
\def\dist{\mathrm{dist}\,} \def\l#1{\limits_{#1}}
\def\Dist{\mathrm{DIST}\,}

\def\Dfst{\mathrm{DIST}_1\,}
\def\Dsnd{\mathrm{DIST}_2\,}

\begin{enumerate}

\item Несложно убедиться в том, что $\max\,\min$ равен 1 (вершина квадрата, ближайшая к данной, всегда находится на расстоянии 1 от неё). С другой стороны, $\min\,\max$ равен $\sqrt{2}$: дальняя вершина от данной находится «по диагонали», на рассоянии $\sqrt{2}$.

\item То, что $\max_{x \in A} F(x) \le T$, равносильно тому, что для всякого $x_0 \in A$ выполнено $F(x_0) \le T$. В нашем случае нужно доказать, что
$$\min\l{y \in B} \dist(x_0,y) \le \min\l{y \in B}
	\ll \max\l{x \in A}\ \dist (x,y) \rr.$$

Это неравенство очевидно, посколько в левой части мы максимизируем расстояние до {\itshape какой-то} точки из множества $A$, а в правой части — до {\itshape дальней} точки множества $A$.

\item Воспользуемся там, что уже было нами получено в первом пункте: для квадрата со стороной 1 разность указанных величин была равна $\sqrt 2 -1$. Тогда для квадрата со стороной $T$, в силу свойств подобных фигур, эта разность будет равна $(\sqrt 2 -1) \cdot T$ — это число, изменяя $T$, можно сделать больше наперёд заданного $r$.

\item Например, подойдут следующие множества: $B$ состоит из двух точек $t_1$, $t_2$ на расстоянии 5 друг от друга, а $A = \{ t_1 \}$.

\item Рассмотрим точку $c \in C$. Так как $C$ лежит в $\rho_2$–окрестности $B$, найдётся точка из $b \in B$ такая, что $\dist (b,c) \le \rho_2$. Далее, так как $B$ \linebreak  лежит в $\rho_1$--окрестности $A$, найдётся точка $a \in A$: $\dist (a,b) \le \rho_1$.

В силу неравенства треугольника, $\dist (a,c) \le \rho_1 + \rho_2$. Поэтому для (произвольной!) точки $c$ нашёлся круг нужного радиуса с центром в точке из $A$, в котором она лежит. Что и требовалось.

\item Положим
$$F(x) := \min\l{y \in A}\ \dist (x,y), \qquad x \in A.$$

Оказывается, $F(x)$ всегда равно нулю: dist всегда не меньше нуля, а если взять $y=x$ — получится как раз ноль, и минимум обратится в ноль. Теперь уж
$$\max\l{x\in A}\ F(x) = \max\l{x \in A}\ 0 = 0,$$

что и требовалось.

\item Фиксируем точку $x_0 \in A$. Мы знаем, что $\min_{y \in B}\ \dist (x_0,y) \le R$ (так как максимум подобных величин по всем точкам из $A$ не превосходит $R$. Тогда найдётся $y_0 \in A$ такая, что $\dist(x_0, y_0) \le R$. Это значит, что $x_0$ лежит в $R$--круге с центром в точке $y_0$ — а, значит, в $R$--окрестности множества $B$.

Если же $A$ целиком лежит в $R$--окрестности $B$, то для каждой точки из $x_0 \in A$ найдётся точка $y_0 \in B$ (центр круга, в который она попала), такая что $\dist (x_0, y_0) \le R$. Отсюда $\min_{y \in B}\ \dist (x_0,y) \le R$, и условие задачи выполнено.

\item $\phantom{x}$

\begin{enumerate}
	\item[\bfseries d=0\,:] Если два множества совпадают, то DIST равен нулю: это доказано в пункте 6. Если множества не равны — $A \ne B$ — то найдётся точка, скажем $a \in A$, которая не лежит в $B$. Для неё $\min_{y \in B}\ \dist(a,y) >0$ — значит, и DIST, получаемый взятием максимума из этой величины и каких-то других величин, будет строго больше нуля.
	\item[\bfseries d(a,b) + ...\,:] Введём
	\begin{align*}
		& \Dfst(A,B) := \max\l{x \in A}\ \ll\min\l{y \in B}\ \dist (x,y)\rr \\
		& \Dsnd(A,B) := \Dfst (B,A)
			\vphantom{\min\l{y \in B}}\\
		& \Dist(A,B) = \max\,\ll\Dfst(A,B), \Dsnd(A,B)\rr
	\end{align*}

Согласно пункту 7, $\Dfst (A,B)$ — наибольший радиус окрестности множества $B$, в которой лежит множество $A$.

Тогда $\Dfst (A,B)$ $+$ $\Dfst (B,C)$ $\le$ $\Dfst(A,C)$ на основании пункта 5. То же самое верно для $\Dsnd$. Теперь
\begin{align*}
	\Dist(A,B) & + \Dist(B,C) = \\
	=& \max\,\ll\Dfst(A,B), \Dsnd(A,B) \rr + \\
	&+ \max\,\ll\Dfst(B,C), \Dsnd(B,C) \rr \ge \\
	& \text{\footnotesize ($\max(a,b) + \max(c,d) \ge \max(a+b,c+d)$)} \\
	\ge \max & \Bigl( \Dfst(A,B) + \Dfst(B,C),\\
		& \Dsnd(A,B) + \Dsnd(B,C)\Bigr) \ge \\
	\ge \max & \Bigl( \Dfst(A,C), \Dsnd(A,C) \Bigr) = \Dist(A,C).
\end{align*}
\end{enumerate}

\item Если $A$ и $B$ — подмножества одного круга радиуса $R$, то расстояния между любыми двумя их точками вообще не превосходят $2R$. Как операции $\max$ и $\min$ к ним ни применяй, всегда будет получаться величина, не превосходящая $2R$.
\end{enumerate}