\task{Расстояние между множествами}
\def\dist{\mathrm{dist}\,} \def\l#1{\limits_{#1}}
\def\Dist{\mathrm{DIST}\,}

\begin{enumerate}

\item Несложно убедиться в том, что $\max\,\min$ равен 1 — вершина квадрата, ближайшая к данной, всегда находится на расстоянии 1 от неё. С другой стороны, $\min\,\max$ равен $\sqrt{2}$: дальняя вершина от данной находится «по диагонали», на рассоянии $\sqrt{2}$.

\item То, что $\max_{x \in A} F(x) \le T$, равносильно тому, что для всякого $x_0 \in A$ выполнено $F(x_0) \le T$. В нашем случае нужно доказать, что
$$\min\l{y \in B} \dist(x_0,y) \le \min\l{y \in B}
	\ll \max\l{x \in A}\ \dist (x,y) \rr.$$

Это неравенство очевидно, поскольку в левой части мы максимизируем расстояние до {\itshape какой-то} точки из множества $A$, а в правой части — до {\itshape дальней} точки множества $A$.

\item Воспользуемся тем, что уже было нами получено в первом пункте: для квадрата со стороной 1 разность указанных величин была равна $\sqrt 2 -1$. Тогда для квадрата со стороной $T$, в силу свойств подобных фигур, эта разность будет равна $(\sqrt 2 -1) \cdot T$ — это число, изменяя $T$, можно сделать больше наперёд заданного $r$.

\item Например, подойдут следующие множества: $B$ состоит из двух точек $t_1$, $t_2$ на расстоянии 5 друг от друга, а $A = \{ t_1 \}$.

\item Рассмотрим точку $c \in C$. Так как $C$ лежит в $\rho_2$–окрестности $B$, найдётся точка из $b \in B$ такая, что $\dist (b,c) \le \rho_2$. Далее, так как $B$ \linebreak  лежит в $\rho_1$--окрестности $A$, найдётся точка $a \in A$: $\dist (a,b) \le \rho_1$.

В силу неравенства треугольника, $\dist (a,c) \le \rho_1 + \rho_2$. Поэтому для (произвольной!) точки $c$ нашёлся круг нужного радиуса с центром в точке из $A$, в котором она лежит. Что и требовалось.

\item Положим
$$F(x) := \min\l{y \in A}\ \dist (x,y), \qquad x \in A.$$

Оказывается, $F(x)$ всегда равно нулю: dist всегда не меньше нуля, а если взять $y=x$ — получится как раз ноль, и минимум обратится в ноль. Теперь уж
$$\max\l{x\in A}\ F(x) = \max\l{x \in A}\ 0 = 0,$$

что и требовалось.

\item Фиксируем точку $x_0 \in A$. Мы знаем, что $\min_{y \in B}\ \dist (x_0,y) \le R$ (так как максимум подобных величин по всем точкам из $A$ не превосходит $R$). Тогда найдётся $y_0 \in A$ такая, что $\dist(x_0, y_0) \le R$. Это значит, что $x_0$ лежит в $R$--круге с центром в точке $y_0$, а, значит, в $R$--окрестности множества $B$.

Если же $A$ целиком лежит в $R$--окрестности $B$, то для каждой точки из $x_0 \in A$ найдётся точка $y_0 \in B$ (центр круга, в который она попала), такая что $\dist (x_0, y_0) \le R$. Отсюда $\min_{y \in B}\ \dist (x_0,y) \le R$, и условие задачи выполнено.

\item $\phantom{x}$

\begin{enumerate}
	\item[\bfseries d=0\,:] Если два множества совпадают, то DIST равен нулю: это доказано в пункте 6. Если множества не равны — $A \ne B$ — то найдётся точка, скажем $a \in A$, которая не лежит в $B$. Для неё $\min_{y \in B}\ \dist(a,y) >0$ — значит, и DIST, получаемый взятием максимума из этой величины и каких-то других величин, будет строго больше нуля.
	
	\item[\bfseries (a,b)=(b,a)\,:] Симметричность введённого нами расстояния очевидна, потому что при замене в формуле для него $A$ на $B$ и $B$ на $A$ формула остаётся дословно такой же.
	
	\item[\bfseries d(a,b) + ...\,:] Определим
	\begin{align*}
		& D(A,B) := \max\l{x \in A}\ \ll\min\l{y \in B}\ \dist (x,y)\rr. \\
		\text{Тогда\quad} & \Dist(A,B) = \max \left\{ D(A,B), D(B,A) \right\}.
	\end{align*}
	
	Шаг первый: $D(A,C) \le D(A,B) + D(B,C)$.
	\begin{align*}
		& D(A,C) = \max\l{x \in A}\ \ll \min\l{z \in C}\ \dist (x,z) \rr. \\
	\end{align*}
	
	Нам нужно доказать, что этот максимум не превосходит выражения в правой части. Для этого надо доказать, что для любого элемента $a_0 \in A$, взятого произвольным образом, максимизируемая величина не превосходит правой части. Итак,
	\begin{align*}
		& \min\l{z \in C}\ \dist(a_0, z) \le \\
		& \text{\scriptsize Пусть $b_0$ — точка из $B$, ближайшая к $a_0$.} \\
		\le\ & \min\l{z \in C}\ \Bigl( \dist(a_0, b_0) + \dist (b_0, z) \Bigr) = \\
		=\ & \min\l{z \in C}\ \ll \ll \min\l{y \in B}\ \dist(a_0, y) \rr + \dist (b_0, z) \rr = \\
		& \text{\scriptsize Первое слагаемое не зависит от $z$.} \\
		=\ & \min\l{y \in B}\ \dist(a_0, y) + \min\l{z \in C}\ \dist (b_0, z) \le \\
		\le\ & \max\l{x \in A}\ \ll \min\l{y \in B}\ \dist(x, y) \rr +
			\max\l{y \in B}\ \ll \min\l{z \in C}\ \dist(y, z) \rr = \\
		& \\
		& \qquad\qquad = D(A,B) + D(B,C). 
	\end{align*}

	Шаг второй: $D(C,A) \le D(C,B) + D(B,A)$. Получается заменой в формулах выше $A$ на $C$ и наоборот. Таким образом,
	\begin{align*}
		& \Dist (A,C) = \max\left\{ D(A,C),\ D(C,A) \right\} \le \\
		\le\ & \max \Bigl\{ D(A,B)+D(B,C),\ \ D(C,B)+D(B,A) \Bigr\}.
	\end{align*}
	
	Предыдущее верно, потому что при переходе от первой строчки ко второй мы не уменьшили каждую из двух величин в скобках.
	
	Наконец, $\max\{ p+q, u+v \} \le \max\{ p, u \} + \max\{ q, v \}$:
	\begin{align*}
		& \max\{ p+q, u+v \} = \\
		=\ & \frac{1}{2} \Bigl( p+q+u+v + \left| p+q-u-v \right| \Bigr) \le \\
		\le\ & \frac{1}{2} \Bigl( p+u + |p-u| + q+v + |q-v| \Bigr) = \\
		=\ & \max\{ p, u \} + \max\{ q, v \}.
	\end{align*}
	
	Отсюда
	\begin{align*}
		& \Dist (A,C) \le \\
		\le\ & \max \Bigl\{ D(A,B)+D(B,C),\ \ D(C,B)+D(B,A) \Bigr\} \le \\
		\le\ & \max \Bigl\{ D(A,B),\ D(B,A) \Bigr\}
			+ \max \Bigl\{ D(B,C),\ D(C,B) \Bigr\} = \\
		=\ & \Dist(A,B) + \Dist(B,C).
	\end{align*}
	
	Что и требовалось доказать.

\end{enumerate}

\item Если $A$ и $B$ — подмножества одного круга радиуса $R$, то расстояния между любыми двумя их точками вообще не превосходят $2R$. Как операции $\max$ и $\min$ к ним ни применяй, всегда будет получаться величина, не превосходящая $2R$.
\end{enumerate}