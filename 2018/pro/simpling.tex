\task{Простеющие числа}
\def\lne{\medskip \\ \rule{0.65\textwidth}{0.028cm}}


\noindent Перечислим вообще все простеющие числа. Вот они:

\begin{center}
	2, 3, 4, 6, 8, 12, 18, 24, 30.
\end{center}

\begin{enumerate} \setcounter{enumi}{1}
	\item Нечетное простеющее число не может быть больше 4 — так как тогда 4 взаимно просто с ним ($4 = 2 \cdot 2$) и не является при этом простым числом.
	\item[\bfseries 3–4.] \setcounter{enumi}{4} Вообще, простеющее число, не делящееся на простое $p$, не может быть больше $p^2$, так как тогда составное $p^2$ будет взаимно просто с ним. Поэтому простеющие числа, не делящиеся на 3, все не превосходят 9, а не делящиеся на 5 — не превосходят 25.
	\item[\bfseries 5–7.] \setcounter{enumi}{7} Числа $N$ и $N-1$ всегда взаимно просты, поэтому если $N$ простеющее, $N-1$ обязано быть простым (и никак не может быть квадратом простого).

	В шестом же пункте $n$ будет взаимно просто с $p_1 \cdot p_2$ (так как оно взаимно просто с $p_1$ и $p_2$), которое также является составным числом.
	
	\item Докажем, что не бывает простеющих чисел, больших 210. Для этого докажем, что число $N >210$ больше квадрата наименьшего простого числа, на которое оно не делится. Тогда оно окажется взаимно просто с этим квадратом, который является составным числом.
	
	Пронумеруем простые числа по возрастанию: $2 = p_1$, $3 = p_2$, $\ldots$. Пусть $p_k$ — наименьшее простое, такое что $N$ не делится на $p_k$. Тогда $N \geq p_1 \cdot \ldots \cdot p_{k-1}$.
	
	В силу постулата Бертрана $p_{k-1} \ge \tfrac{p_k}{2}$, а $p_{k-2} \ge \tfrac{p_{k-1}}{2} \ge \tfrac{p_k}{4}$. Отсюда $p_{k-2} \cdot p_{k-1} \ge p_k^2 / 8$.
	
	Если мы хотим, чтобы число $N$ было простеющим, то оно должно быть меньше, чем $p_k^2$. То есть, $p_1 \cdot \ldots \cdot p_{k-3} \le 8$. Отсюда уж точно $k-3 \le 2$, то есть, $k \le 5$. В свою очередь, $N \le 2 \cdot 3 \cdot 5 \cdot 7 = 210$.

	Что и требовалось доказать.
\end{enumerate}