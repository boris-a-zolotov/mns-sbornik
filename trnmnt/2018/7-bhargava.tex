\task{Факториалы Бхаргавы}

В 2000 году лауреат Филдсовской премии Манжул Бхаргава нашёл об-\linebreak общение целочисленного факториала, в котором для каждого подмножества $S\subseteq \text{\bfseries Z}$ определяется $n!_S$, причём, $n!_\text{\bfseries Z} = n!$. Оказалось, что его конструкция естественным образом обобщает наиболее интересные свойства обычного факториала. Ссылка на статью: \href{https://goo.gl/zF3p5N}{$goo.gl/zF3p5N$} (The Factor-\linebreak ial Function and Generalizations, Manjul Bhargava). В этой задаче предлагается продолжить исследование Бхаргавы в конкретном направлении. Одной из основ этого продолжения служит следующее утверждение.

\begin{enumerate}
\item Докажите, что любое положительное рациональное число может\linebreak  быть представлено в виде частного произведений факториалов (не обязательно различных) простых чисел. Например, 
\begin{align*}
\frac{10}{9} = \dfrac{2! \cdot 5!}{3! \cdot 3! \cdot 3!}.
\end{align*}
\end{enumerate}

Напомним конструкцию Бхаргавы. Зафиксируем подмножество $S \subseteq \text{\bfseries Z}$ и простое число $p \in \text{\bfseries P}$. Построим последовательность $a_0, a_1, \ldots \in S$ следующим образом: выберем произвольное $a_0 \in S$; выберем элемент $a_1 \in S$ так, чтобы разность $a_1 - a_0$ делилась на наименьшую возможную степень числа $p$; выберем элемент $a_2 \in S$ так, чтобы разность $(a_2 - a_0)(a_2 - a_1)$ делилась на наименьшую возможную степень числа $p$, и так далее. На шаге $k$ выберем $a_k \in S$ так, чтобы разность $(a_k - a_0)\cdot \ldots \cdot (a_k - a_{k-1})$ делилась на наименьшую возможную степень числа $p$. Вместе с построенной последовательностью $a_n$ мы получаем также монотонно возрастающую последовательность соответствующих степеней $p$

\begin{align*}
\nu_k(S,p) := p^{\text{ord}_p\left( \text{П}_{i = 0}^{k-1} (a_k - a_i)\right)},
\end{align*}
где $\nu_0(S,p) = 1.$ Теперь обобщённый факториал на $S$ для $k \geq 0$ определяется по формуле
\begin{align*}
k!_S := \prod_{p \in \text{\bfseries P}} \nu_k(S,p).
\end{align*}
\begin{enumerate} \setcounter{enumi}{1}
\item Проверьте, что последовательность $\nu_k(S,p)$ не зависит от выбора $a_k$. Кроме того, докажите, что для каждого $S \subseteq \text{\bfseries Z}$ в произведении выше лишь конечное число множителей не равно единице.
\item Пусть $S = a\text{\bfseries Z} + b := \{an + b \mid n \in \text{\bfseries Z}\}$ или $S = \{q^k \mid k \in \text{\bfseries Z}_{\geq 0}\}$. 
\begin{enumerate}
\item Какие значения может принимать отношение факториалов
	$$n!_S/m!_S$$
в зависимости от $a,b,q$?
\item Ответьте на предыдущий вопрос, если числа $n = p, m = q$ предполагаются простыми.
\item Опишите возможные значения, которые может принимать биномиальный коэффициент Бхаргавы
\begin{align*}
\binom{n}{k}_S := \dfrac{n!_S}{k!_S (n-k)!_S}.
\end{align*}
\item Ответьте на предыдущий вопрос, если числа $n = p$ предполагаются простыми.
\end{enumerate}

\item Ответьте на вопросы предыдущего пункта для произвольного $S.$

\item Исследуйте вопросы п. $3$ (а,б) и $4$, описав возможные отношения произведений факториалов Бхаргавы.
\end{enumerate}