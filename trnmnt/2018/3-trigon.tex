\task{Обобщенные гиперболические и тригонометрические системы}
\newcommand{\mmH}{\text{\bf H}}
\newcommand{\mmT}{\text{\bf T}}

Пусть $n \geq 2$. Набор бесконечно--дифференцируемых функций $\mmH_n = (H_1, \ldots, H_n)$ из $\text{\bf R}$ в $\text{\bf R}$ назовём {\it гиперболической системой} порядка $n$, если $H_i^\prime = H_{i+1}$ для $1 \leq i < n$, $H_n^\prime = H_1$ и $H_i(0) = \delta_{i,n}$. Аналогично, набор бесконечно--дифференцируемых функций $\mmT_n = (T_1, \ldots, T_n)$ из $\text{\bf R}$ в $\text{\bf R}$ называется {\it тригонометрической системой} порядка $n$, если $T_i^\prime = T_{i+1}$ для $1 \leq i < n$, $T_n^\prime = - T_1$ и $T_i(0) = \delta_{i,n}$. 

\begin{enumerate}
\item Докажите, что пары $\left(\sh(t), \ch(t)  \right)$ и $\left(\sin(t), \cos(t)\right)$ задают, соответственно, единственные возможные гиперболические и тригонометрические системы порядка $2$. В явном виде опишите гиперболические и тригонометрические системы порядка $n$. 

\item Докажите, что для гиперболических и тригонометрических систем порядка $2$ выполнены
формулы Эйлера: $H_2(t) + H_1(t) = e^t$ и $T_2(t) + i T_1(t) = e^{it}.$ Обобщите их на произвольные системы $\mmH_n, \mmT_n$.

\item При всех $1 \leq i \leq n$ выразите $H_i(s+t)$ через $H_j(s), H_k(t)$ и $T_i(s+t)$ через $T_j(s), T_k(t)$. Кроме того, исследуйте функции из наборов $\mmH_n$ и $\mmT_n$ на четность и нечетность.

Пусть $\lambda \in \text{\bf R}_+$. Изучите предыдущие пункты для модифицированных систем $\mmH_n^\lambda$ и $\mmT_n^\lambda$, в определении которых фигурируют равенства $H_n'(t) = \lambda H_1(t)$ и $T_n'(t) = -\lambda T_1(t)$, соответственно.
\item Докажите, что для гиперболических и тригонометрических систем порядка $2$ выполнены соотношения $H_2^2(t) - H_1^2(t) = 1$ и $T_2^2(t) + T_1^2(t) = 1.$ Иными словами, образы при отображениях $L_2: t \mapsto (H_1(t), H_2(t))$ и $M_2: t \mapsto (T_1(t), T_2(t))$ из $\text{\bf R}$ в $\text{\bf R}^2$ могут быть заданы алгебраическими уравнениями. Выясните, можно ли задать образы при отображениях 
\begin{align*}
L_n: t &\longmapsto (H_1(t), \ldots, H_n(t)) \\
M_n: t &\longmapsto (T_1(t), \ldots, T_n(t))
\end{align*}
из $\text{\bf R}$ в $\text{\bf R}^n$ алгебраическими уравнениями. В частности, найдите однородные многочлены $p_n, q_n$ степени $n$ такие, что для $f_n:=p_n - 1$ и $g_n := q_n-1$ выполнено
\begin{align*}
f_n(H_1(t), \ldots, H_n(t)) &= 0, \\
g_n(T_1(t), \ldots, T_n(t)) &= 0
\end{align*} 

при любом $t$. Например, $f_2(x,y) = y^2 - x^2 - 1$ и $g_2(x,y) = y^2 + x^2 - 1$. Верно ли, что многочлены $f_n, g_n \in \text{\bf Q}[x_1, \ldots, x_n]$ неприводимы?

\item Опишите все перестановки $\sigma \in S_n$, для которых $p_n(x_{\sigma(1)}, \ldots, x_{\sigma(n)}) = p_n(x_1, \ldots, x_n)$ при всех $x_1, \ldots, x_n$. Аналогичный вопрос для $q_n$.

\item 
Опишите все точки образов $L_n$ и $M_n$ в $\text{\bf R}^n$ с целыми координатами. Опишите все точки с рациональными координатами.

Придумайте, как по двум целым / рациональным точкам из образов $L_n, M_n$ получить новую целую / рациональную точку на $L_n$ и $M_n$ соответственно. Изучите получающееся «сложение точек».
\end{enumerate}