\task{Экстремальные тетраэдры}

\noindent Задачи на плоскости.
\begin{enumerate}
\item Среди всех треугольников, вписанных в единичную окружность, найдите треугольник с
\begin{enumerate}
\item максимальным периметром,
\item максимальной площадью,
\item максимальным радиусом вписанной окружности.
\end{enumerate}
\item Шириной треугольника в направлении $\alpha$, где $\alpha\in[0,\text{π}]$, называется величина $w(\alpha)$, равная длине проекции треугольника на прямую, образующую с осью абсцисс угол $\alpha$. Средней шириной треугольника называется величина
	\[
	\frac{1}{\text{π}}\int_{0}^{\text{π}}w(\alpha)\,\rm d \alpha.
	\]
\begin{enumerate}
\item Придумайте, как по периметру треугольника найти его среднюю ширину.	
\item Среди всех треугольников, вписанных в единичную окружность, найдите треугольник с максимальной средней шириной.
\end{enumerate}
\end{enumerate}
Задачи в $\text{\bf R}^3$.
\begin{enumerate} \setcounter{enumi}{2}
\item
\begin{enumerate}
\item Среди всех тетраэдров, вписанных в единичную сферу, найдите тетраэдр с максимальным объемом.
\item Докажите следующее обобщенное тождество параллелограмма: если $X_1,\dots, X_n$ --- векторы в $\mathbb R^3$, где $n$ --- натуральное число, то
$$
\sum_{1\leq i<j\leq n}\|X_i-X_j\|^2+\|\sum_{i=1}^n X_i\|^2=n\sum_{i=1}^n\| X_i\|^2.
$$
\item Среди всех тетраэдров, вписанных в единичную сферу, найдите тетраэдр с максимальной суммой длин ребер.
\item Пусть дан тетраэдр в $\mathbb R^3$. Известно, что любое его ребро ортогонально плоскости, проходящей через середину этого ребра и оставшиеся две вершины тетраэдра. Докажете, что тетраэдр правильный.
\item Среди всех тетраэдров, вписанных в единичную сферу, найдите тетраэдр  с максимальной площадью поверхности (суммой площадей граней).
\item Среди всех тетраэдров, вписанных в единичную сферу, найдите тетраэдр  с максимальным радиусом вписанной сферы.
\end{enumerate}
\item Пусть тетраэр $ABCD$ вписан в единичную сферу с центром $O$. Суммой углов обзора тетраэдра называется величина
\[
\sphericalangle AOB+\sphericalangle AOC+\sphericalangle AOD+\sphericalangle BOC+\sphericalangle BOD+\sphericalangle COD.
\]
Среди всех тетраэдров, вписанных в единичную сферу, найдите тетраэдр с максимальной суммой углов обзора.
\item
\begin{enumerate}
\item Используя полярные координаты в $\mathbb R^3$, обобщите понятние средней ширины треугольника на трехмерный случай (для тетраэдра).
\item Среди всех тетраэдров, вписанных в единичную окружность, попробуйте найти тетраэдр  с максимальной средней шириной.
\end{enumerate}
\item Многомерным обобщением треугольника и тетраэдра в $\mathbb R^n$ является симплекс --- многогранник, у которого $n+1$ вершина. Попытайтесь пункты $3)$ -- $5)$ обобщить на многомерный случай.
\end{enumerate}

\task{Динамические системы}

Зафиксируем натуральное число $n$ и рассмотрим множество $\text{\bf Z}\slash n\text{\bf Z}$ возможных остатков при делении на $n$. В этой задаче предлагается изучить некоторые разбиения $\text{\bf Z}\slash n\text{\bf Z}$ на подмножества (классы эквивалентности) и исследовать динамику этих разбиений при малых изменениях задающих их параметров. Изменения при этом будут контролироваться некоторой функцией $f:\text{\bf Z} \to \text{\bf Z}$. Каждое разбиение $\text{\bf Z}\slash n\text{\bf Z}$ задаёт представление числа $n$ в виде суммы неотрицательных слагаемых, а следовательно, задаёт диаграмму Юнга соответствующего порядка. Интерес вызывает как количество строчек в этой диаграмме, так и их длина.

Рассмотрим наименьшее отношение эквивалентности на $\text{\bf Z}\slash n\text{\bf Z},$ при котором каждый остаток $[x]$ эквивалентен остатку $[f(x)]$. В зависимости от $f$ и $n$ найдите число всех классов, на которые полученное отношение делит $\text{\bf Z}\slash n\text{\bf Z}$. Например, при $n = 7$ функция $f(x) = 4x + 1$ задаёт разбиение $\text{\bf Z}\slash 7\text{\bf Z} = \{[0],[1],[5]\}\cup \{[2]\}\cup \{[3],[6],[4]\}$ и диаграмму
\begin{center}
\begin{ytableau}
\ & & \cr
  & & \cr
      \cr
\end{ytableau}
\end{center}
Интерес в представляют как гипотезы и наблюдения, связанные с динамикой ответов, так и строгие доказательства. Какая трансформация $f$ вносит большее изменение: умножение на два $f(x) \mapsto 2f(x)$ или прибавление единицы $f(x) \mapsto f(x) + 1$? Предлагается проводить исследование в следующем порядке: 
\begin{enumerate}
\item Изучите случаи $f_a(x)=ax$, где a --- фиксированное целое число, и $f(x)=x^2$.
\item Изучите случай $f_{a, b}(x) = ax+b$, начиная с совсем малых целых b. Опробуйте оба подхода: фиксируйте $a,b$ и меняйте $n$ или фиксируйте $n$ и меняйте $a,b$, а затем изучите форму получающихся диаграмм.
\begin{enumerate}
\item Траекторией $x$ называется последовательность $x,$ $f_{a,b}(x),$ $f_{a,b}(f_{a,b}(x)), \ldots$. В предположении $(a,n) > 1$, опишите остатки $[x]$, трактории которых образуют цикл.
\item Постарайтесь описать траекторию остатка $[0]$.
\item Постарайтесь найти те характеристики получающихся диаграмм Юнга, которые поддаются вычислению в зависимости от $n,a,b$.
\item Фиксируйте $a,b$ и опишите чезаровские средние (по $n$) мощностей получающихся классов эквивалентности. Затем попробуйте брать средние по другой переменной ($a$ или $b$).
\end{enumerate}
\item Изучите случаи $f_m(x) = x^m$ и $f(x) = x^2+1$.
\item Выясните, для каких полиномиальных функций $f$ искомые числа классов эквивалентности и их размеров поддаются явному вычислению, и проведите соответствующее исследование. Например, выясните, какие замены функции $f \mapsto g$ приводят к незначительным изменениям ответов.
\end{enumerate}

\bigskip

\task{ПОЗ–коды}

\rightline{\it Сотрите мне память (Н.В.~Гоголь.~Вий)}
\rightline{\it Стереть нельзя исправить (крылатое выражение)}
\medskip

Возможно, вам знакома перфокарта --- картонка, в которой можно пробивать отверстия. С помощью перфокарты удобно хранить информацию в машиночитаемом виде: есть отверстие --- 1, нет --- 0. У перфокарты есть важное свойство: любой 0 легко меняется на 1, но обратная замена крайне затруднена. Память с таким свойством называют \emph{памятью с однократной записью (ПОЗ)}, или по-английски \emph{Write-Once Memory (WOM)}. Один бит такой памяти называется \emph{витом}.

Ограничение казалось бы не позволяет такую память перезаписывать несколько
раз, но в 1982 году Р.Ривест и А.Шамир в статье <<How to Reuse a ``Write-Once'' Memory>> предложили кодировку, позволяющую ценой некоторого увеличения объёма носителя предоставить возможность перезаписи информации.
                                                     
Например, с помощью трёх витов оказывается возможно записать некоторое двухбитовое
число, а потом однократно заменить его на другое:
\begin{center}
\begin{tabular}{ccc}
{\it число}    & {\it кодировка для первой записи  }   & {\it кодировка для повторной записи}\\
00       & 000                             & 111\\
01       & 100                             & 011\\
10       & 010                             & 101\\
11       & 001                             & 110
\end{tabular}
\end{center}
Допустим, можно сперва записать 10, используя код 010, а потом записать число 01
на его место, заменив третий вит на 1 и получив код 011.

Недавно эта область исследований получила второе дыхание в связи с распространением
флэш-памяти (обладающей очень похожими свойствами). 
Ниже мы предлагаем вам задачи, связанные с данной областью:

\begin{enumerate}
\item Представим себе проездной билет, стоимость которого (целое число от 0 до $n-1$) 
запоминается с помощью $k$ витов --- скажем, компостируется при покупке. Предложите кодировку, 
позволяющую исключить увеличивающее стоимость изменение витов (докомпостирование билета)
после покупки. Приведите по возможности точные верхние и нижние оценки на число $k$ для вашей кодировки. Также предложите 
теоретические верхние и нижние оценки на количество витов в кодировках с такими свойствами.
\item В условиях п.1 предложите кодировку, исключающую любое изменение стоимости после покупки.
Иными словами, кодировку, в которой любые дополнительные изменения витов делают код любого
числа некорректным (не соответствующим никакому числу). Также приведите оценки для предложенной 
кодировки и теоретические оценки.
\item Представим себе проездной билет, в котором используется ПОЗ для хранения числа поездок.
После каждой поездки число уменьшается на 1, пока не достигнет нуля. Предложите кодировку, 
позволяющую хранить эту информацию по возможности максимально эффективно по памяти. Дайте 
как теоретические, так и достигаемые вашей кодировкой верхние и нижние оценки необходимого количества 
витов.
Убедитесь, что предложенный вами код не позволяет увеличить количество поездок в процессе
перезаписи значений.
\item Пусть в ПОЗ хранится не число поездок, а оплаченная стоимость в рублях, при этом 
при поездке со счёта снимается либо 40, либо 45 рублей (в зависимости, например, от вида
транспорта). Возможно ли с учётом этого ограничения сделать кодировку более эффективной по
памяти и улучшить оценки из п.3?
\item Обобщите результат из п.4 на случай произвольного набора стоимостей поездки.
\item Рассмотрим ситуацию, когда мы записываем события на длинную ленту. Скажем, речь может
идти о показаниях скорости --- увеличилась ли она на 1 от предыдущего наблюдения, 
уменьшилась ли на 1 или осталась прежней. Сравнительно легко иметь дело с ситуациями, когда
каждое следующее событие имеет ровно $2^k$ значений --- тогда мы про каждое событие будем 
дописывать к ленте ровно $k$ витов и сразу переходить к следующему. 
Однако, можете ли вы предложить более эффективную по памяти кодировку в той ситуации, 
когда количество вариантов, например, равно 3? Естественно, вы можете, помимо добавления 
новых витов, исправлять какие-то из предыдущих.
\item Рассмотрите п.6 для произвольного количества вариантов значений, добавляемых на каждом шаге.
\item Предложите какие-нибудь свои аналогичные задачи и кодировки, подходящие для их решения. Этот пункт также подходит для изложения кодировок, придуманных вами в процессе решения задачи, но не подошедших под условия.
\end{enumerate}