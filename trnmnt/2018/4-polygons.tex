\task{Многоугольники}

\begin{enumerate}
\item Пусть $F_1$ и $F_2$ - два выпуклых многоугольника, множества вершин которых совпадают. Покажите, что $F_1$ равен $F_2$.
\item Пусть $F_1$ и $F_2$ не обязательно выпуклы, но удовлетворяют условию про вершины. Выясните, что можно сказать про их расположение.
\item Пусть $F_1$ и $F_2$ выпуклы, а середины сторон $F_1$ лежат в $F_2$. Как связаны площади $F_1$ и $F_2$?
\item Пусть $F_1$ и $F_2$ выпуклы, а середины сторон $F_1$ лежат в $F_2$ и середины сторон $F_2$ лежат в $F_1$. Верно ли, что такие многоугольники равны?
\item Зафиксируем $0\leq\alpha\leq 1$. Точка на стороне многоугольника называется его $\alpha$--серединой, если она делит эту сторону в отношении $\alpha:(1-\alpha)$. Известно, что для каждой стороны $F_1$ какая-то $\alpha$--середина лежит в $F_2$. Что можно сказать про площади $F_1$ и $F_2$?
\item Исследуйте предыдущие вопросы в случае, когда один из многоугольников не обязательно выпуклый. 
\end{enumerate}