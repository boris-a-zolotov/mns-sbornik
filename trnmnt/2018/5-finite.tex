\task{Конечные вычисления}

Основная идея этой задачи --- исследование дискретных аналогов дифференцирования и интегрирования. Интерес представляют явные сравнения непрерывных и дискретных конструкций между собой. 

Обозначим через $\text{\bfseries Seq}$ множество всех вещественных числовых последовательностей. Сами последовательности будем обозначать символами $x_n \in \text{\bfseries Seq},$ а их соответствующие элементы (значения) под номером $k$ через $x_n[k]$ (начиная с 0). Таким образом, $x_n = y_n \ \Longleftrightarrow \ x_n[k] = y_n[k], \forall k \geq 0$. 

Определим функции $\text{Δ}, \int$ из $\text{\bfseries Seq}$ в $\text{\bfseries Seq}$ по правилам 
\begin{align*}
(\text{Δ} x_n)[k] &:= x_n[k+1] - x_n[k] \\
\left(\int x_n dn\right) [k] &:= \sum_{i=0}^{k-1} x_n[i].
\end{align*}
Они называются разностным оператором и оператором суммирования, а последовательности $\text{Δ} x_n, \int x_n dn$ --- производной и интегралом $x_n$ соответственно. Если $\text{Δ} F_n = x_n$, то $F_n$ называется первообразной $x_n$. Если $A$ --- некоторый оператор (т.е. функция из $\text{\bfseries Seq}$ в $\text{\bfseries Seq}$), то $A^n$ --- это новый оператор, являющийся композицией $A$ с собой $n$ раз. 

\begin{enumerate}
\setcounter{enumi}{0}
\item Опишите связи между производной, интегралом, первообразной и сдвигом $x_n$, где под сдвигом понимается $(\text{\bfseries E}x_n)[k] = x_n[k+1]$. Кроме того, найдите явную формулу для $\text{Δ}^m x_n$.
\end{enumerate}

Константы $c \in \text{\bfseries R}$ задают постоянные и показательные последовательности $\text{c}, \text{c}^n \in \text{\bfseries Seq}$ по формулам $\text{c}[k] := c$ и $\text{c}^n[k] := c^k$. Кроме того, определим для $m = 0,1,2,\ldots$ последовательности $n^m$ и $n^{\underline{m}}$ степеней и падающих степеней по формулам $n^m[k] := k^m$ и $n^{\underline{m}}[k] := k(k-1)(k-2)\cdot \ldots \cdot (k-(m-1)).$ Набор всех последовательностей из $\text{\bfseries Seq}$, которые могут быть выражены как линейные комбинации последовательностей $1, n, n^2, \ldots, n^{m-1}$, обозначим через $\text{\bfseries\itshape P}_m$. Аналогично, через $\text{\bfseries\itshape FP}_m$ обозначим последовательности, являющиеся линейными комбинациями падающих степеней $1$, $n$, $n^{\underline{2}}$, $\ldots$, $n^{\underline{m-1}}$. Последовательности из $\text{\bfseries\itshape P}_m$ или $\text{\bfseries\itshape FP}_m$ будем называть полиномиальными.
\begin{enumerate}
\setcounter{enumi}{1}
\item Выясните, как связаны между собой $\text{\bfseries\itshape P}_m$ и $\text{\bfseries\itshape FP}_m$ и предложите какие--нибудь интересные эквивалентные описания последовательностей из этих множеств. Точнее, сравните $\text{\bfseries\itshape P}_i$ и $\text{Δ}^{j}$. А как связаны между собой $\int x_n dn$ и $\text{\bfseries\itshape P}_i$? 
\item Найдите дискретные аналоги формул Ньютона--Лейбница и интегрирования по частям. Затем найдите в явном виде $\int n\ dn$, $\int n^2 dn,$
$\int \text{\bfseries\itshape F}_n^{(3)} dn$,
$\int \text{λ}^n dn$,
$\int n \text{λ}^{n-1} dn$,
$\int n^2 \textrm{2}^{n-1} dn$,
$\int \text{\bfseries\itshape F}_n^{(m)} dn$. Существует ли ком- бинаторное / геометрическое решение? Предложите свои собственные числовые последовательности и опишите их производные и интегралы. Например, рассмотрите известные числовые последовательности такие, как числа Фибоначчи.

\item Рассмотрим последовательность $S_n^{(m)}$, которая определяется по формуле $S_n^{(m)} := \int n^m dn$. Покажите, что $S_n^{(m)}$ является полиномиальной последовательностью и в явном виде выразите её через последовательности степеней или падающих степеней. 

\item Для каждой последовательности $x_n \in \text{\bfseries Seq}$ определим её последовательность Тейлора $\langle x_n \rangle$ по правилу $\langle x_n \rangle [k] := (\text{Δ}^k x_n)[0]$. Найдите последовательности Тейлора ваших любимых последовательностей. Например, последовательностей падающих степеней. Кроме того, докажите, что функция $x_n \mapsto \langle x_n \rangle$ обратима и в явном виде найдите её обратную.

Обозначим через $\text{\bfseries\itshape P}_\infty$ множество всех формальных бесконечных комбинаций вида
\begin{align*}
x_n = \alpha_0 + \alpha_1 n^{\underline{1}} + \alpha_2 n^{\underline{2}} + \ldots,
\end{align*}
где $\alpha_k \in \text{\bfseries R}$. Каждая такая комбинация, в действительности, определяет числовую последовательность $x_n \in \text{\bfseries Seq}$, потому что при каждом $k$ сумма $x_n[k]$ будет конечна.

Например, $\text{\bfseries\itshape P}_m \subset \text{\bfseries\itshape P}_\infty$ при всех $m \geq 1$. Докажите, что каждая последовательность в $\text{\bfseries Seq}$ может быть представлена в таком виде и явно найдите соответствующие коэффициенты $\alpha_k$.
\item Придумайте свои обобщения полученных результатов. Например, изучите периодические или рекуррентные последовательности, их производные, интегралы и связанные с ними тождества --- рассмотрите функцию $\text{Δ}_T(x_n)[k] := x_n[k+T] - x_n[k]$, придумайте аналог биномиальной теоремы для падающих степеней, изучите «дифференциальные уравнения» вида $\sum_{i=1}^m \alpha_i \text{Δ}^i(x_n) = 0$, исследуйте частные производные последовательностей от двух индексов $x_{n,m}$ и кратные интегралы $\int \int x_{n,m} dn\ dm$ или привлеките иные методы дискретной математики для изучения разных классов последовательностей.

\end{enumerate}
