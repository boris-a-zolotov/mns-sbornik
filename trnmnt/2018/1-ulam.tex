\task{Узоры на скатерти Улама}

Расположим натуральные числа в виде спирали, как показано на рисунке ниже. Полученная картинка называется скатертью Улама.

\begin{center}
\begin{tabular}{ c c c c c c c }
37 & 36 & 35 & 34 & 33 & 32 & 31 \\
38 & 17 & 16 & 15 & 14 & 13 & 30 \\
39 & 18 & 5  & 4  & 3  & 12 & 29 \\
40 & 19 & 6  & 1  & 2  & 11 & 28 \\
41 & 20 & 7  & 8  & 9  & 10 & 27 \\
42 & 21 & 22 & 23 & 24 & 25 & 26 \\
43 & 44 & 45 & 46 & 47 & 48 & \ldots
\end{tabular}
\end{center}

В этой задаче требуется изучить закономерности и узоры, связанные с расположением разных числовых последовательности на скатерти. Интерес представляют не только гипотезы, полученные, например, на компьютере, но и явные соотношения.

\begin{enumerate}
\item Выясните, как на скатерти Улама описываются числовые после-\linebreak довательности на вертикальных, горизонтальных и диагональных\linebreak прямых.
\item Исследуйте на скатерти расположения
\begin{enumerate}
\item квадратных чисел $\mathcal{F}_n^{(4)}$,
\item треугольных чисел $\mathcal{F}_n^{(3)}$ (найдите количество ветвей на получающейся картинке),
\item фигурных чисел разных порядков $\mathcal{F}_n^{(m)}$,
\item многомерных фигурных чисел.
\end{enumerate}
\item Исследуйте на скатерти Улама расположения арифметических и\linebreak геометрических прогрессий. Попробуйте также описать расположения известных числовых последовательностей, возникающих в комбинаторике: чисел Фибоначчи, чисел Каталана и т.п.
\item Рассмотрите другие нумерации целых точек всей плоскости или ка-\linebreak ких-то фигур на этой плоскости. Например, занумеруем сектор следующим образом:

\begin{center}
\begin{tabular}{ c c c c c c }
1 &  &  &  &   & \\
2 & 3 &  &  &  &  \\
4 & 5 & 6 &  & & \\
7 & 8  & 9 & 10& &  \\
11& 12 & 13& 14& 15 & \\
16& 17 & 18& 19& 20 & 21  
\end{tabular}
\end{center}
Ответьте на сформулированные выше вопросы для новых нумераций.
\end{enumerate}