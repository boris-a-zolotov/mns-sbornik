\renewcommand{\mb}[1]{\mathbb{#1}}
\renewcommand{\ol}[1]{\overline{#1}}
\renewcommand{\tbf}[1]{\textbf{#1}}

\task{Определители}
Напомним, что на множестве квадратных матриц размера $n$ есть функция $\text{Δ}$, сопоставляющая матрице некоторое число, которое называется определителем этой матрицы.
Эта функция однозначно задаётся следующими условиями:
если матрица $A$ представлена в виде $(u_1, \dots, u_n)$, где $u_i$ столбцы чисел, то тогда

\begin{enumerate}
\item  Если случилось так, что столбец $u_i=v + \lambda v'$, где $v$ и $v'$ столбцы, а $\lambda$ --- некоторое число, то
$$\text{Δ}(A)=\text{Δ}(u_1, \dots, u_{i-1}, v, u_{i+1}, \dots, u_n) + \lambda\text{Δ}(u_1, \dots, u_{i-1}, v', u_{i+1}, \dots, u_n).$$
\item  Для любых $1\leq i<j\leq n$ выполнено
$$\text{Δ}(A)=-\text{Δ}(u_1, \dots, u_{i-1}, u_j, u_{i+1}, \dots, u_{j-1}, u_{i}, u_{j+1}, \dots, u_n).$$
\item  $\text{Δ}(E)=1$, где $E$ матрица, такая что $E_{ij}=0, $ для $i\neq j$ и $E_{ii}=1$.
\end{enumerate}
Так, например, определитель для матрицы $A=\left(\begin{smallmatrix}A_{11} & A_{21}\\ A_{12} & A_{22}\end{smallmatrix}\right)$ размера 2 может быть вычислен по формуле

$$\text{Δ}(A)= A_{11}A_{22}-A_{12}A_{21} .$$
Матрица $A$ называется симметричной, если $A_{i j}=A_{j i}$ для всех возможных $i$ и $j$.

Главным минором порядка $k$, или просто $k$-ым главным минором матрицы $A$, называется число, равное определителю матрицы $C$ размера $k$, где $C_{i, j}= A_{i, j}$ $(1 \leq i, j \leq k)$. Будем обозначаеть это число $\text{Δ}_k(A)$. Последовательностью главных миноров матрицы $A$ называется строка
$(\text{Δ}_1(A), \dots , \text{Δ}_n(A))$.


\begin{enumerate}
\item Покажите, что если $A$ симметричная матрица размера $2$, составленная из вещественных чисел и её первый главный минор равен $0$, то её определитель отрицателен.
\item Докажите, что для комплексных симметричных матриц $2\times 2$ в качестве последовательности главных миноров реализуется любая строка комплексных чисел.
\item Исследуйте эти же вопросы для матриц $3\times 3$.
\item \label{targ} Для любого натурального $n$ найдите все упорядоченные наборы $(B_1, \dots , B_n ) \in F^n$, для каждого из которых найдется симметричная матрица A размера $n$ с элементами из $F$, у которой последовательность главных миноров совпадает с $(B_1, \dots , B_n )$, а $F$ – одно из следующих множеств
\begin{description}
\item[а) ] $\mb{R}$,
\item[б) ] $\mb{C}$,
\item[в) ] $\mb{Q}$,
\item[г) ] любое другое поле.
\end{description}
\item Исследуйте вопрос пункта 4 для целочисленных матриц, матриц с коэффициентами в целых гауссовых числах и т.д.
\item Предложите свои обобщения этой задачи и решите их.
\end{enumerate}




\task{Короткие дороги}

В некоторой стране идёт активное строительство дорог. Основная задача состоит в том, чтобы соединить между собой все города наименьшей по общей длине системой дорог.
В данном случае будем считать, что города - это точки на плоскости, а система дорог - это набор отрезков, не пересекающихся между собой нигде, за исключением, возможно, своих концов. Назовём точку --- точкой разветвления дорог, если в этой точке встречаются три или более дороги. Стоит отметить, что концом отрезка не обязательно является город.
\begin{enumerate}
\item Определите, как выглядит оптимальная система дорог, если в стране всего три города, находящихся на равном расстоянии; на разных расстояниях друг от друга. Найдите длину этой сети дорог.
\item Покажите, что для любой конфигурации городов оптимальная сеть дорог образует дерево с вершинами в городах и точках разветвления дорог.
\item Выясните, какие возможны конфигурации дорог в точках разветвления.
\item Оцените число рёбер в этом графе.
\item Найдите оптимальную конфигурацию для страны, чьи города расположены в вершинах прямоугольника; в вершинах других многоугольников.
\item Оцените длину оптимальной системы дорог для произвольной конфигурации; для городов, находящихся в вершинах выпуклого многоугольника. Оптимальна ли Ваша оценка?
\item Верно ли, что Ваши необходимые условия  реализации графа в качестве оптимальной системы дорог являются достаточными.
\item Обобщите и решите задачу, когда точки лежат на сфере, а дороги проходят по дугам больших окружностей. Рассмотрите случай других метрических пространств.
\end{enumerate}



\task{Различные расстояния}
Рассмотрим $M$ некоторое множество точек в $k$-мерном пространстве. Пусть $D(M)=\left| \left\{ r=dist\left( x_i, x_j \right) | x_i, x_j \in M; \, \,x_i\neq x_j\right\} \right|$ - количество различных расстояний между точками множества $M$. Определим теперь
$$D_k(n)= \min\limits_{\substack{|M|=n\\ M\subset \mb{R}^k}} D(M).$$
К примеру, $D_2(3)=1$.
\begin{enumerate}
\item Найдите $D_2(4), D_2(5), D_2(6)$.
\item Оцените последовательность $D_2(n)$ сверху и снизу.
\item Решите пункты 1 и 2 в трёхмерном пространстве.
\item Найдите $D_n(n+2), D_n(n+3), D_n(n+4)$.
\item Верно ли, что существует предел $\lim\limits_{n\to \infty} D_n(n+c)$ для любого натурального $c$. Если да, то чему он равен?
\item Рассмотрите предыдущий вопрос для последовательности $D_n(cn^k)$ при фиксированных $c$ и $k$.
\item Верно ли, что $D_k(n)$ и $D_k(n+1)$ обязаны отличаются не более чем на 1?
\item Предложите верхнюю и нижнюю оценки для $D_k(n)$ при фиксированных $k$.
\item Обобщите задачу на другие пространства. Попробуйте оценить число различных конфигураций, при которых доcтигается минимум (с точностью до движений и подобия).
\end{enumerate}




\task{Циркуляции}
Пусть $G$ - неориентированный граф со множеством рёбер $E$ и множеством вершин $V$. При этом будем допускать в графе $G$ кратные рёбра и петли. Введём множество $\overline{E}=\left\{(e, x, y)| e\in E; \,\, x,y\in V; \,\, \hbox{$x$ и $y$ концы ребра $e$} \right\}$, каждый элемент которого задаёт ребро с выбранной ориентацией. Целочисленной циркуляцией  на графе $G$ назовём функцию $f\colon \overline{E} \to \mb{Z}$, удовлетворяющую двум условиям
\begin{enumerate}
\item[а)] $f(e, x, y)=-f(e,y,x)$.
\item[б)] Для любой вершины $x$ $\sum\limits_{\substack{e \colon \exists y\\ (e,x,y) \in \overline{E}}} f(e,x,y)=0$ (Закон Кирхгофа).
\end{enumerate}

$k$-циркуляцией для $k\geq 2$ называется циркуляция $f$ , такая что $0<|f(x)|<k$, $x\in \overline{E}$.

\begin{enumerate}
\item Покажите, что если из связного графа $G$ можно убрать одно ребро $e$, так что граф $G-e$ окажется несвязным(такое ребро будем называть мостом), то на этом графе не существует ни одной $k$-циркуляции ни для какого $k$.
\item Покажите, что 2-циркуляция на графе без мостов существует тогда и только тогда, когда степень любой вершины чётна.
\item Назовём потоковым числом графа $G$ наименьшее такое $k$, что на $G$ есть $k$-циркуляция. Если такого $k$ нет, будем говорить, что потоковое число равно $\infty$. Будем обозначать это число как $\eta(G)$. Верно ли, что если в графе нет мостов, то $\eta(G)<\infty$?
\item Найдите $\eta(K_{2n+1})$, для различных $n$, где $K_{2n+1}$ - полный граф на $2n+1$ вершине.
\item Найдите $\eta(K_{4})$. Посчитайте, сколько различных $k$-циркуляций на $K_4$.
\item Найдите $\eta(K_{2n})$.
\item Пусть $P$ - граф Петерсена. Покажите, что на этом графе нет 4-циркуляции. Верно ли, что любой граф, на котором нет 4-циркуляции содержит подразбиение графа $P$.

\item Пусть $H$ - абелева группа, например группа остатков $\mb{Z}/n\mb{Z}$. $H$-циркуляцией называется отображение $f\colon \overline{E} \to H$, удовлетворяющее условиям а) и б). Исследуйте количество $H$-циркуляций на различных графах. Напишите оценку количества $H$-циркуляций для конечной группы $H$.

\end{enumerate}


\task{Календарь}

Рассмотрим окружность радиуса $n \in  \mb{N}$ с центром в начале некоторой фиксированной системы координат. Число $n$ назвается календарным, если на этой окружности есть в точности 12 точек с целочисленными координатами.
\begin{enumerate}
\item Приведите пример календарных чисел.
\item Бесконечно ли множество календарных чисел?
\item Чему равна плотность множества календарных чисел, то есть предел $$\lim\limits_{x \to \infty} \frac{|\left\{ 0<n \leq x \,|\, n - \hbox{ календарное } \right\}|}{x} .$$
\item Рассмотрите вместо окружностей эллипсы, заданные уравнением $x^2 + qy^2=n$, где $q$ - натуральное число без квадратов. При каких $q$ есть такое $n$, что у этого уравнения есть ровно 12 решений.
\item Рассмотрите  <<циферблатные>> числа, где каждой минуте соответствовала бы точка на окружности с целыми координатами.
\item Какое количество целых решений может быть у уравнения $x^2+qy^2=n$?
\end{enumerate}


\task{Обобщение теоремы Штейнера-Лемуса}
\begin{enumerate}
\item Пусть задано вещественное положительное число $n$. На сторонах AB, BC треугольника ABC отметим точки $C_n$ и $A_n$ соответственно так, что ${\frac{|\widehat{BAA_n}|}{|\widehat{CAA_n}|}} = {\frac{|\widehat{BCC_n}}{|\widehat{ACC_n}|}} = n$, где $|\widehat{CAA_n}|$ обозначает градусную меру угла $CAA_n$. Известная теорема Штейнера-Лемуса утверждает, что равенство длин биссектрисс $|AA_1|=|CC_1|$ влечет равенство длин сторон $|AB|=|BC|$. Проверьте истинность утверждения: «Отрезки $AA_n$ и $CC_n$ имеют равные длины тогда и только тогда, когда стороны AB и BC имеют равные длины» в каждом из следующих случаев:\\
    \begin{description} \vspace{-0.4cm}
\item [а) ] $n=2$
\item [б) ] $n$ - произвольное натуральное число.
\item [в) ] $n$ - произвольное положительное рациональное число.
\item [г) ] $n$ - произвольное положительное вещественное число.
\end{description}
\item Сформулируйте и исследуйте аналогичную задачу, если точки $A_n$ и $C_n$ выбираются на прямых AB, BC соответственно так, что лучи $AA_n$, $C_n$ делят внешние углы при вершинах A и C треугольника ABC в равных отношениях.
\item Предложите свои обобщения или направления исследования в этой задаче и изучите их.
\end{enumerate}

\task{Иррациональные корни рациональных уравнений}
\begin{enumerate}
\item Известно, что уравнение $x^4+ax^3+29x^2+bx+4=0$ с рациональными коэффициентами имеет корнем число $2+\sqrt[] 3 $. Найдите остальные корни этого уравнения.
\item Обоснуйте следующее утверждение про рациональные корни уравнения вида $p(x)=\sum_{i=0}^n a_i \cdot x^i$ с целыми коэффициентами (если они, конечно, существуют): если $x_0$ – рациональный корень такого уравнения, то он обязательно равен $x_0={\frac{p}{q}}$, где $p$ – делитель свободного члена (т.е. $a_0$), а $q$ – делитель $a_n$. На основании этого соображения постройте алгоритм нахождения таких корней. Распространите этот алгоритм на уравнения такого же типа с рациональными коэффициентами.
\item Попробуйте предложить алгоритм определения (с обоснованием) корней вида $a+b\cdot \sqrt[] 2, a+b\cdot \sqrt[] 3,\dots$ где $a, b \in \mb{Q}$, для таких уравнений (по крайней мере, постройте алгоритмы определения таких корней).
\item Может, вы сможете определять корни более сложного вида $a+b\cdot \sqrt[]2 +c\cdot \sqrt[] 3$ или $a+b\cdot \sqrt[3] 2$?
\item Предложите алгоритм определения корней исходя из их общего вида, такого как $a+b \cdot \sqrt[] m$, $a+b\cdot \sqrt[] m + c\cdot \sqrt[] k$ и т.п., где $m, k,\dots$ – заранее неизвестные натуральные числа.
\item Попробуйте оценить сложность предлагаемых алгоритмов.
\item Рассмотрите корни уравнений еще более сложного вида (с корнями различных степеней или с «композицией» корней и т.п.).
\item Предложите свои обобщения или направления исследования в этой задаче и изучите их (например, попробуйте рассмотреть подобные задачи для систем уравнений с двумя и более переменными, а также уравнения с коэффициентами из множества
$$\mb{Q}(\sqrt[]2) = \left\{x+y \cdot \sqrt[]2|x,y \in \mb{Q} \right\}.$$
\end{enumerate}


\task{Функция Эйлера}
Пусть $n$ - натуральное число, большее единицы. Обозначим за $\phi(n)$ количество таких целых $0<x<n$, что $x$ взаимно просто с $n$. $$\phi(n)=|\left\{0<x<n | (x,n)=1 \right\}|$$
\begin{enumerate}
\item Покажите, что для любого $n\geq 3$ есть такое натуральное число $k(n)$, что
$$\underbrace{\phi(\phi(\cdots\phi(n)))}_{k(n) \hbox{ раз}}=\phi^{\circ k(n)}(n)=2 .$$
\item Оцените число $k(n)$ сверху и снизу, где
\begin{description}
\item [а) ] $n$ - число вида $\left\{3^s2^t\right\}_{s,t\in \mb{N}}$.
\item [б) ] $n$ - есть произведение всех различных простых меньших заданного числа.
\item [в) ] $n$ - произвольное натуральное число.
\end{description}
\item Рассмотрим уравнение $\phi(n)=m$ относительно $n$. Оцените число его решений
\begin{description}
\item [а) ] сверху.
\item [б) ] снизу.
\end{description}
\item Обобщите предыдущий пункт на случай уравнений $\phi(\phi(\cdots\phi(n)))=m$. При каких $m$ они разрешимы? Какова плотность множества значений функции $\phi^{\circ k}$, где плотность понимается в смысле задачи 5.
\item Число $n$ назовём совершенным, если $n=\sum\limits_{i=1}^{k(n)+1} \phi^{\circ i}(n)$. Докажите, что числа вида $3^k$ являются совершенными.
\item Постройте другие примеры совершенных чисел. Существуют ли совершенные числа, не делящиеся на 3? Какие числа не являются совершенными?
\end{enumerate}



\task{Игры с карточками}
\begin{enumerate}
\item Есть три автомата: первый по карточке с числами $(a, b)$, $a, b \in \mb{Z}$ выдаёт карточку с числами $(a - b, b)$; второй – карточку $(a + b, b)$; третий – карточку $(b, a)$. Все автоматы возвращают заложенные в них изначально карточки.
    \begin{description}
    \item [а) ]Пусть у вас в начале на руках имеется карточка $(19, 86)$. Можно ли получить карточку а) $(31,13)$; б) $(12, 21)$?
    \item [б) ]Попробуйте найти все карточки $(x, y)$, которые можно получить из карточки $(19, 86)$. Докажите, что других карточек получить нельзя.
    \item [в) ]Пусть у вас имеется карточка с числами $(a, b)$. Попробуйте найти все карточки $(x, y)$, которые можно получить.
    \end{description}
 \item Есть три автомата: первый по карточке $(a, b)$, $a, b \in \mb{N}$ выдаёт карточку с числами $(a + 1, b + 1)$; второй – карточку $(a/2,b/2)$ (он работает только тогда, когда $a$ и $b$ чётные); третий – по двум карточкам с числами $(a, b)$ и $(b, c)$ печатает карточку с числами $(a, c)$. Все автоматы возвращают заложенные в них карточки.
     \begin{description}
    \item [а) ]Можно ли с помощью этих операций из карточки $(5, 19)$ получить карточку а) $(1,50)$; б) $(1, 100)$?
    \item [б) ]Найдите все натуральные n, такие, что можно из карточки $(5, 19)$ получить карточку $(1, n)$. Докажите, что при остальных натуральных $n$ это сделать не получится.
    \item [в) ]Определите множество всех карточек $(m, n)$, $m, n \in \mb{N}$, которые можно получить из карточки $(5, 19)$.
    \end{description}
 \item Пусть первоначально имеется карточка с числами $(a, b)$, $a, b \in \mb{N}$, $a < b$, и автоматы такие же, как в пункте 2.
 \begin{description}
    \item [а) ]Для различных пар $a$, $b$ определите, при каких $n$ можно из заданной карточки $(a, b)$ получить карточку с числами $(1, n)$? Докажите, что при остальных натуральных n это сделать не получится.
    \item [б) ]Для различных пар $a$, $b$ определите множество всех карточек $(m, n)$, $m, n \in \mb{N}$, которые можно получить из карточки $(a, b)$.
    \item [в) ]Пусть первоначально имеется набор из $k$ карточек с числами $(a_1,b_1),...,(a_k, b_k)$. При каких натуральных $m$ и $n$ можно получить карточку с числами $(m, n)$ (конечно, в зависимости от исходного набора карточек)?
    \end{description}
\item Придумайте свои обобщения или направления исследования этой задачи и изучите их. Например, рассмотрите систему автоматов, способных выполнять над карточками какие-нибудь другие операции.
\end{enumerate}

\task{Числовые квадраты}
Возьмем 9 девятиклеточных квадратов. \\
\begin{enumerate}
\item Можно ли разместить в клетках этих квадратов натуральные числа от 1 до 9 так, чтобы затем было возможно соединить все 9 квадратов в один квадратный коврик  $9\times9$ таким образом, что одновременно выполняются условия:
     \begin{description}
      \item [а) ]Сумма чисел по каждой диагонали в любом девятиклеточном квадрате равнялась 15.
      \item [б) ]Сумма чисел в каждом из четырёх квадратов $2\times2$, входящих в состав девятиклеточного квадрата, а также сумма чисел, расположенных в клетках,         прилегающих к сторонам центрального квадратика, равнялась 16 в первом девятиклеточном квадрате коврика, 17 - во втором, 18 - в третьем и далее последовательно 19, 20, 21, 22, 23, 24.
      \item [в) ]В каждом столбце и в каждой строке полного квадрата $9\times9$ содержались бы все числа от 1 до 9 в произвольной последовательности.\\
    \end{description}
\item Можно ли расположить числа так, чтобы сумма чисел по углам каждого из центральных $3\times3$, $5\times5$, $7\times7$, $9\times9$ квадратов окажется равной 20?
\item Можно ли расположить числа так, чтобы суммы, вдоль прямых, симметричных относительно одной из диагоналей квадрата $9\times9$, оказались одинаковыми, причём суммы эти уменьшались бы регулярно на 5 единиц по мере удаления прямых от другой диагонали большого квадрата?
\item Можно ли расположить числа так, чтобы кроме предыдущих условий оказалось, что суммы квадратов чисел вдоль прямых, симметричных относительно той же диагонали полного квадрата, оказались одинаковы?
\item Найдите как можно больше дополнительных числовых свойств у образовавшегося полного квадрата и докажите их.
\item Предложите наиболее экономный алгоритм составления требуемого числового квадрата $9\times 9$ и обоснуйте его корректность. Сформулируйте аналогичные задачи для квадратов произвольного размера.
\end{enumerate}


\task{Почти арифметические прогрессии}
Попробуйте построить теорию «почти арифметических прогрессий». В качестве исходных направлений исследования могут быть следующие.\\
Пусть $a_1, d_1, d_2, n$ – фиксированные натуральные числа. Конечную последовательность чисел $a_1, a_2, …, a_n$, будем называть почти арифметической прогрессией, если для любого $k$, $2 \leq k \leq n$, $a_k=a_{k-1}+d_1$ или $a_k=a_{k-1}+d_2$. Множество всех таких почти арифметических прогрессий длины $n$ обозначим через $P_n(a_1,d_1,d_2)$.\\
\begin{enumerate}
\item Укажите последовательность $a_1, a_2, \dots , a_n$ из $P_n(a_1,d_1,d_2)$, у которой наименьшее количество членов равняется полусумме своих соседей.
\item Укажите последовательность из $P_n(a_1,d_1,d_2)$, у которой среди чисел $a_1+a_n, a_2+a_{n-1},...$ наименьшее количество равных между собой.
\item Сколько различных последовательностей содержит множество
$$P_n(a_1,d_1,d_2)?$$
\item Сколько различных сумм может быть у последовательностей из множества $P_n(a_1,d_1,d_2)$?
\item Какое наибольшее количество последовательностей из $P_n(a_1,d_1,d_2)$ имеет одинаковую сумму всех своих членов?
\item Пусть $P_{3n+1}(a_1,1,2,3)$ – множество всех последовательностей $a_1$, $\dots$, $a_{3n+1}$ таких, что при любом k, $2 \leq k \leq 3n+1$ имеет место одно из равенств $a_k=a_{k-1}+1$, $a_k=a_{k-1}+2$, $a_k=a_{k-1}+3$. У какого наибольшего количества последовательностей из $P_{3n+1}(a_1,1,2,3)$ одинаковая сумма всех членов?
\item Сколько различных последовательностей содержит множество
$$P_{3n+1}(a_1,1,2,3)?$$
\item Предложите свои направления исследования или обобщения этой задачи и изучите их.
\end{enumerate}

\task{Периодические дифференциальные уравнения}
\begin{enumerate}
\item Дана функция $f(x,y)\colon\mb{R}^2 \rightarrow \mb{R}, f \in C(\mb{R}^2), f'_y \in C(\mb{R}^2)$ и $f(x+T,y)=f(x,y)$ для любых $(x,y) \in \mb{R}^2$. Далее, существуют такие числа a, b, что $f(x,a)\cdot f(x,b)<0$ для любого вещественного $x$.\\
    \begin{description}
    \item [а) ]Докажите, что дифференциальное уравнение $y'=f(x,y)$ имеет $T$-периодическое решение.
    \item [б) ]Докажите, что если ${f'}_y>0$, то это периодическое решение единственно.
    \end{description}
\item Дано уравнение $y'=-y^{2k+1}+f(x)$, $f(x+T)=f(x)$, $f$ - непрерывна на вещественной прямой.
\begin{description}
    \item [а) ]Докажите, что существует $T$-периодическое решение.
    \item [б) ]Докажите, что это решение единственно.
    \end{description}
\item Найдите все периодические решения уравнения $y'=(y-a)(y-b)$, где $a$, $b$ - вещественные числа.\\
Средним за период для периодической функции $f(x)$ называется величина ${\frac{1}{T}} \int\limits_0^T f(x)\, dx$. Ниже везде предполагается, что функции $f, f_1, f_2 \in C(\mb{R})$ и $T$-периодические.
\item Дано уравнение $y'=(y-a)(y-f(x))$.
\begin{description}
    \item [а) ]Найдите необходимое и достаточное условие на среднее за период функции $f$, при котором это уравнение имеет $T$-периодическое решение(отличное от константы).
    \item [б) ]Сколько вообще периодических решений может иметь это уравнение?
    \end{description}
\item Исследуйте те же вопросы для уравнения $y'=(y-a)^2(y-f(x))$
\item Проведите исследование уравнения $y'=(y-a)^m(y-f(x))^n$ на предмет существования периодических решений в зависимости от натуральных параметров $n, m$ и величины ${\frac{1}{T}} \int\limits_0^T f(x) dx$.
\end{enumerate}